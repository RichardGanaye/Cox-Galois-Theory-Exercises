%&LaTeX
\documentclass[11pt,a4paper]{article}
\usepackage[frenchb,english]{babel}
\usepackage[applemac]{inputenc}
\usepackage[OT1]{fontenc}
\usepackage[]{graphicx}
\usepackage{amsmath}
\usepackage{amsfonts}
\usepackage{amsthm}
\usepackage{amssymb}
%\input{8bitdefs}

% marges
\topmargin 10pt
\headsep 10pt
\headheight 10pt
\marginparwidth 30pt
\oddsidemargin 40pt
\evensidemargin 40pt
\footskip 30pt
\textheight 670pt
\textwidth 420pt

\def\imp{\Rightarrow}
\def\gcro{\mbox{[\hspace{-.15em}[}}% intervalles d'entiers 
\def\dcro{\mbox{]\hspace{-.15em}]}}

\newcommand{\be} {\begin{enumerate}}
\newcommand{\ee} {\end{enumerate}}
\newcommand{\deb}{\begin{eqnarray*}}
\newcommand{\fin}{\end{eqnarray*}}
\newcommand{\ssi} {si et seulement si }
\newcommand{\D}{\mathrm{d}}
\newcommand{\Q}{\mathbb{Q}}
\newcommand{\Z}{\mathbb{Z}}
\newcommand{\N}{\mathbb{N}}
\newcommand{\R}{\mathbb{R}}
\newcommand{\C}{\mathbb{C}}
\newcommand{\F}{\mathbb{F}}
\newcommand{\re}{\,\mathrm{Re}\,}
\newcommand{\ord}{\mathrm{ord}}
\newcommand{\legendre}[2]{\genfrac{(}{)}{}{}{#1}{#2}}

\title{Solutions to David A.Cox  "Galois Theory''}
\author{Richard Ganaye}
\refstepcounter{section} \refstepcounter{section}
\refstepcounter{section} \refstepcounter{section}
\refstepcounter{section}\refstepcounter{section}\refstepcounter{section}\refstepcounter{section}
\refstepcounter{section}\refstepcounter{section}\refstepcounter{section}\refstepcounter{section}\refstepcounter{section}


\begin{document}



\section{Chapter 14 : SOLVABLE PERMUTATION GROUPS}

\subsection{POLYNOMIAL OF PRIME DEGREE}

\paragraph{Ex. 14.1.1}

{\it This exercise is concerned with the proof of part (a) of Lemma 14.1.2. Let $\theta =(12...p)\in S_p$.
\begin{enumerate}
\item[(a)] Prove that $\tau\in S_p$ lies in the normalizer of $\langle\theta\rangle$ if and only if $\tau\theta =\theta^l\tau$ for some $1\leq l\leq p-1$.
\item[(b)] Prove that (14.1) implies that $\tau(i+j)=\tau(i)+jl$ for all positive integers j.
\end{enumerate}
}
\begin{proof}
\begin{enumerate}
\item[(a)] The normalizer of $\langle\theta\rangle$ in $S_p$ consists of all $\tau\in S_p$ such that $\tau\langle\theta\rangle\tau^{-1}=\langle\theta\rangle$. It means that for any $\tau\in S_p$: 
\begin{align*}
    \tau\langle\theta\rangle\tau^{-1}=\langle\theta\rangle &\iff \exists m,n\nmid p: \tau\theta^m\tau^{-1}=\theta^n~\rm(since~\langle\theta\rangle=\{\theta^l\mid l\in\mathbb{Z}\})\\
    &\iff \exists k\nmid p,e: mk+pe=1~\rm(since~m\nmid p)\\
    &\iff (\tau\theta^m\tau^{-1})^k=\theta^{nk}=\theta^l,~l=nk~ (mod~ p) \\
    &\iff\tau\theta^{1-pe}\tau^{-1}=\tau\theta\tau^{-1}=\theta^l~\rm(since~\it \theta^p=e)\\
    &\iff\tau\theta=\theta^l\tau~\rm for~ some~\it 1\leq l\leq p-1.
\end{align*}

\item[(b)] By induction suppose that $\tau(i+j)=\tau(i)+jl$, then $\tau(i+j+1)=\tau(i+j)+l=\tau(i)+(j+1)l$. Case $j=1$ is valid by the identity (14.1). Hence, $\tau(i+j)=\tau(i)+jl$ for all positive integers j.  
\end{enumerate}
\end{proof}

\paragraph{Ex. 14.1.2}

{\it Let $H$ be a normal subgroup of a finite group $G$ and let $g\in G$. The goal of this exercise is to prove Lemma 14.1.3.
\begin{enumerate}
\item[(a)] Explain why $(gH)^{o(g)}=(gH)^{[G:H]}=H$ in the quotient group G/H.
\item[(b)] Now assume that $gcd(o(g),[G:H])=1$. Prove that $g\in H$.
\end{enumerate}
}
\begin{proof}
\begin{enumerate}
\item[(a)] Since $(gH)^2=gHgH=g^2H$ and $g^{o(g)}=e$, $(gH)^{o(g)}=g^{o(g)}H=H$.

Since $gH\in G/H$, exists some minimal $l$ such that $(gH)^l=H$ and $l\mid [G:H]$, i.e. $[G:H]=ql$. Then $(gH)^{[G:H]}=(gH)^{ql}=H^q=H$.
\item[(b)] The assumption $gcd(o(g),[G:H])=1$ means that $o(g)q+[G:H]l)=1$ for some $q,l\in\mathbb{Z}$. Then $gH=(gH)^{o(g)q+[G:H]l}=((gH)^{o(g)})^q((gH)^{[G:H]})^l=H^qH^l=H$, i.e. $g\in H$.
\end{enumerate}

\end{proof}

\paragraph{Ex. 14.1.3}

{\it Let G satisfy (14.2). Use (14.2) and the Third Sylow Theorem to prove that G has a unique p-Sylow subgroup H of order p. Then conclude that H is normal in G.
}

\begin{proof}
According to (14.2) $|G|=pm$ where $1<=m<=p-1$, hence $p$ is the highest power of $p$ dividing $|G|$ and, by the First Sylow Theorem, $G$ has a $p$-Sylow subgroup $H\subset G$ with $|H|=p$.

By the Second Sylow Theorem any two $p$-Sylow subgroups of $G$ are conjugate in $G$. Therefore, group $G$ acts on the set of $p$-Sylow subgroups by conjugation and this action is transitive. Due to transitivity of the action of $G$ on the set of $p$-Sylow subgroups, the orbit of a fixed $p$-Sylow subgroup $H$ is the whole set of $p$-Sylow subgroups, i.e., the order of such orbit is equal to the number of $p$-Sylow subgroups $N$ and, by the Fundamental Theorem of Group Actions, $N$ divides $|G|=pm$.

By the Third Sylow Theorem the number $N$ of p-Sylow subgroups of G is equal to one by modulo $p$, $N\equiv 1$ mod $p$. We have $N\nmid p$, hence $N \mid m$, which is possible only if $N=1$.

Suppose that there is exactly one $p$-Sylow subgroup $H$ of $G$.
For all $g \in G$, $gHg^{-1}$ is another subgroup of $G$ of order equal to $|H|$, hence $gHg^{-1}$
is also a $p$-Sylow subgroup and so $gHg^{-1}=H$ for all
$g \in G$. This says that $H$ is normal in $G$.

\end{proof}

\paragraph{Ex. 14.1.4}

{\it The definition of Frobenius group given in the Mathematical Notes involves a group G acting transitively on a set X. Prove that a group G is a Frobenius group if and only if G has a subgroup H such that $1<|H|<|G|$ and $H\cap gHg^{-1}=\{e\}$ for all $g \notin H$.
}

\begin{proof}
Suppose that $G$ has a Frobenius action on the set X. Let $x\in X$ be an element with a nontrivial isotropy subgroup $G_x=\{g\in G\mid g\cdot x=x\}$. Since $G$ acts transitively on $X$ and $1<|X|<|G|$, this element exists. Let fix such element and denote subgroup $H=G_x$ called a Frobenius complement. Due to transitivity, $1<|H|<|G|$. 

For any $g\in G$ an isotropy group of the element $g\cdot x$ is $G_{g\cdot x}=\{\hat g\in G\mid \hat gg\cdot x=g\cdot x\}$ and, since $gHg^{-1}\cdot g\cdot x=gH\cdot x=g\cdot x$, $gHg^{-1}=G_{g\cdot x}$.

If $g\in H$, then $g\cdot x=x$ and $G_{g\cdot x}=G_x$, hence $gHg^{-1}=H$.

If $g\notin H$, then $g\cdot x\neq x$ and $G_{g\cdot x}\cap G_x=\{e\}$, hence $gHg^{-1}\cap H=\{e\}$.

Now suppose that $G$ has a subgroup $H$ such that $1<|H|<|G|$ and $H\cap gHg^{-1}=\{e\}$ for all $g \notin H$. 

Let define the set $X$ as the full set of left cosets $gH$, i.e., $X=\{H,g_1H,g_2H,...,g_nH\}$. Since $1<|H|<|G|$, then $1<|X|<|G|$.
It is obvious that $G$ acts transitively on $X$ - for any two elements $g_jg_i^{-1}$ moves $g_iH$ to $g_jH$.

Let $N_G(H)=\{g\in G\mid gHg^{-1}=H$\} is normalizer of $H$. Then $H\subset N_G(H)$ and, since $H\cap gHg^{-1}=\{e\}$ for all $g \notin H$, for any $g\in N_G(H)$ we have $g\in H$, hence $N_G(H)=H$, i.e., $H$ is normal subgroup. 

An isotropy group of the element $g_iH$ is $G_{g_iH}=\{g\in G\mid  g\cdot g_iH=g_iH\}$ and, since $g_iHg_i^{-1}\cdot g_iH= g_iHH= g_iH$, $g_iHg_i^{-1}=G_{g_iH}$.

Let by contradiction exists $g\ne \{e\}$ fixing two different elements from X, i.e., $g\cdot g_iH=g_iH$ and $g\cdot g_jH=g_jH$. Then $g\in G_{g_iH}\cap G_{g_jH}$, $g\in g_iHg_i^{-1}\cap g_jHg_j^{-1}$ and $\{e\}\ne g_j^{-1}gg_j\in g_j^{-1}g_iH(g_j^{-1}g_i)^{-1}\cap H$, where $g_j^{-1}g_i\notin H$. This contradicts with the assumption that $H\cap gHg^{-1}=\{e\}$ for all $g \notin H$ and proves that group $G$ and set $X$ are corresponding to the Frobenius group definition given in the Mathematical Notes.  

\end{proof}

\paragraph{Ex. 14.1.5}

{\it Let F be a subfield of the real numbers, and let $f\in F[x]$ be irreducible of prime degree $p>2$. Assume that f is solvable by radicals. Prove that f has either a single real root or p real roots. 
}

\begin{proof}
This is the direct corollary from the Theorem 14.1.1. If $\alpha\neq\beta$ are two real roots of solvable by radicals polynomial $f$ of prime degree, then $F(\alpha,\beta)$ extension of subfield $F$ is the splitting field of $f$ over $F$. Hence all other $p$ roots are in  $F(\alpha,\beta)$ extension. 

Since any rational extension of a subfield of the real numbers by addition of real numbers is the subfield of real numbers, all $p$ roots are real.

Since the degree of $f$ is odd, at least one real roots always exists. Therefore $f$ has either a single real root or $p$ real roots.
\end{proof}

\paragraph{Ex. 14.1.6}

{\it By Example 8.5.5, $f=x^5-6x+3$ is not solvable by radicals over $\mathbb{Q}$. Give a new proof of this fact using the previous exercise together with the irreducibility of f and part (b) of Exercise 6 from Section 6.4.
}

\begin{proof}
The given polynomial $f$ has prime degree 5 and only three real roots, according to part (b) of Exercise 6.4.6. Since $f$ has more than one but less than 5 real roots, it is not solvable by radicals by Exercise 14.1.5. 

\end{proof}

\paragraph{Ex. 14.1.7}

{\it Use Lemma 14.1.3 and part (a) of Lemma 14.1.2 to give a proof of part (b) of Lemma 14.1.2 that doesn't use the Sylow Theorems.
}

\begin{proof}
Assume that $\tau\in S_p$ satisfies $\tau\theta\tau^{-1}\in AGL(1,\mathbb{F}_p)$. Then, since $\langle\theta\rangle$ is a group of order $p$, $\langle\tau\theta\tau^{-1}\rangle=\tau\langle\theta\rangle\tau^{-1}$ is a subgroup of $AGL(1,\mathbb{F}_p)$ of order $p$ and each element of this subgroup has order $p$.

By part (a) of Lemma 14.1.2, $AGL(1,\mathbb{F}_p)$ is the normalizer of $\langle\theta\rangle$ in $S_p$, therefore $\langle\theta\rangle$ is normal in $AGL(1,\mathbb{F}_p)$ with $[AGL(1,\mathbb{F}_p):\langle\theta\rangle]=(p-1)$. Order of each element from $\tau\langle\theta\rangle\tau^{-1}$ is relatively prime to $(p-1)$, then, by Lemma 14.1.3, $\tau\langle\theta\rangle\tau^{-1}\subset\langle\theta\rangle$, i.e., $\tau\langle\theta\rangle\tau^{-1}=\langle\theta\rangle$, since both groups have the same order $p$.  

Thus $\tau$ normalizes $\langle\theta\rangle$ and hence $\tau\in AGL(1,\mathbb{F}_p)$.
\end{proof}


\end{document}