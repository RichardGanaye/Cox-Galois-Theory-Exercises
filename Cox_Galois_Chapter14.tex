%&LaTeX
\documentclass[11pt,a4paper]{article}
\usepackage[frenchb,english]{babel}
\usepackage[applemac]{inputenc}
\usepackage[OT1]{fontenc}
\usepackage[]{graphicx}
\usepackage{amsmath}
\usepackage{amsfonts}
\usepackage{amsthm}
\usepackage{amssymb}
\usepackage{tikz}
%\input{8bitdefs}

% marges
\topmargin 10pt
\headsep 10pt
\headheight 10pt
\marginparwidth 30pt
\oddsidemargin 40pt
\evensidemargin 40pt
\footskip 30pt
\textheight 670pt
\textwidth 420pt

\def\imp{\Rightarrow}
\def\gcro{\mbox{[\hspace{-.15em}[}}% intervalles d'entiers 
\def\dcro{\mbox{]\hspace{-.15em}]}}

\newcommand{\be} {\begin{enumerate}}
\newcommand{\ee} {\end{enumerate}}
\newcommand{\deb}{\begin{eqnarray*}}
\newcommand{\fin}{\end{eqnarray*}}
\newcommand{\ssi} {si et seulement si }
\newcommand{\D}{\mathrm{d}}
\newcommand{\Q}{\mathbb{Q}}
\newcommand{\Z}{\mathbb{Z}}
\newcommand{\N}{\mathbb{N}}
\newcommand{\R}{\mathbb{R}}
\newcommand{\C}{\mathbb{C}}
\newcommand{\F}{\mathbb{F}}
\newcommand{\U}{\mathbb{U}}
\newcommand{\re}{\,\mathrm{Re}\,}
\newcommand{\im}{\,\mathrm{Im}\,}
\newcommand{\ord}{\mathrm{ord}}
\newcommand{\Gal}{\mathrm{Gal}}
\newcommand{\legendre}[2]{\genfrac{(}{)}{}{}{#1}{#2}}

\title{Solutions to David A.Cox  "Galois Theory''}

\refstepcounter{section} \refstepcounter{section}
\refstepcounter{section} \refstepcounter{section}
\refstepcounter{section}\refstepcounter{section}\refstepcounter{section}\refstepcounter{section}
\refstepcounter{section}\refstepcounter{section}\refstepcounter{section}\refstepcounter{section}\refstepcounter{section}


\begin{document}



\section{Chapter 14 : SOLVABLE PERMUTATION GROUPS}

\subsection{POLYNOMIAL OF PRIME DEGREE}

\paragraph{Ex. 14.1.1}
{\it This exercise is concerned with the proof of part (a) of Lemma 14.1.2. Let $\theta =(1\, 2 \ldots p)\in S_p$.
\begin{enumerate}
\item[(a)] Prove that $\tau\in S_p$ lies in the normalizer of $\langle\theta\rangle$ if and only if $\tau\theta =\theta^l\tau$ for some $1\leq l\leq p-1$.
\item[(b)] Prove that (14.1) implies that $\tau(i+j)=\tau(i)+jl$ for all positive integers j.
\end{enumerate}
}
\begin{proof}
\begin{enumerate}
\item[(a)]
If $\theta$ lies in the normalizer of $\langle \theta\rangle = \{e,\theta,\theta^2,\ldots,\theta^{p-1}\}$, then
$$\tau \theta \tau^{-1} \in \tau \langle \theta \rangle \tau^{-1}= \langle \theta \rangle,$$
hence
$$\tau \theta \tau^{-1} =\theta^l \text{ for some } l = 0,1\ldots,p-1.$$
If $l = 0$, then $\tau \theta \tau^{-1} = e$, thus $\tau \theta = \tau$, and $\theta = e$, which is false. Therefore $l \ne 0$.
$$\tau \theta \tau^{-1} =\theta^l , \ 1\leq l \leq p-1.$$

 \item[(b)] By induction suppose that $\tau(i+j)=\tau(i)+jl$, then $\tau(i+j+1)=\tau(i+j)+l=\tau(i)+(j+1)l$. Case $j=1$ is valid by the identity (14.1). Hence, $\tau(i+j)=\tau(i)+jl$ for all positive integers $j$.  
\end{enumerate}
\end{proof}

\paragraph{Ex. 14.1.2}

{\it Let $H$ be a normal subgroup of a finite group $G$ and let $g\in G$. The goal of this exercise is to prove Lemma 14.1.3.
\begin{enumerate}
\item[(a)] Explain why $(gH)^{o(g)}=(gH)^{[G:H]}=H$ in the quotient group $G/H$.
\item[(b)] Now assume that $\gcd(o(g),[G:H])=1$. Prove that $g\in H$.
\end{enumerate}
}
\begin{proof}
\begin{enumerate}
\item[(a)] Since $(gH)^2=gHgH=g^2H$ and $g^{o(g)}=e$, $(gH)^{o(g)}=g^{o(g)}H=H$.

Since $gH\in G/H$, exists some minimal $l$ such that $(gH)^l=H$ and $l\mid [G:H]$, i.e. $[G:H]=ql$. Then $(gH)^{[G:H]}=(gH)^{ql}=H^q=H$.
\item[(b)] The assumption $\gcd(o(g),[G:H])=1$ means that $o(g)q+[G:H]l=1$ for some $q,l\in\mathbb{Z}$. Then $gH=(gH)^{o(g)q+[G:H]l}=((gH)^{o(g)})^q((gH)^{[G:H]})^l=H^qH^l=H$, i.e. $g\in H$.
\end{enumerate}

\end{proof}

\paragraph{Ex. 14.1.3}

{\it Let $G$ satisfy (14.2). Use (14.2) and the Third Sylow Theorem to prove that $G$ has a unique p-Sylow subgroup $H$ of order $p$. Then conclude that $H$ is normal in $G$.
}

\begin{proof}
By (14.2),
 $$|G|= |\Gal(L/F)| =pm, \qquad 1\leq m \leq p-1.$$

According the Third Sylow Theorem the number $N$ of p-Sylow subgroups of G satisfies
 $$N\equiv 1 \pmod p,  \qquad N\mid |G|,$$ 
so that $N = 1 + kp,\ k\geq 0$, thus $N \wedge p = 1$, and $N \mid pm$, therefore $N \mid m$. If $k\ne 0$, then $N>p$, but $N \mid m>0$, which implies $N \leq m <p$. This contradiction shows that $k = 0$, and $N = 1$, i.e. there is exactly one $p$-Sylow subgroup $H$ of $G$.

For all $g \in G$, $gHg^{-1}$ is also a $p$-Sylow subgroup of $G$, hence $gHg^{-1}=H$ for all
$g \in G$: $H$ is normal in $G$.

\end{proof}

\paragraph{Ex. 14.1.4}

{\it The definition of Frobenius group given in the Mathematical Notes involves a group $G$ acting transitively on a set $X$. Prove that a group $G$ is a Frobenius group if and only if $G$ has a subgroup $H$ such that $1<|H|<|G|$ and $H\cap gHg^{-1}=\{e\}$ for all $g \notin H$.
}

\begin{proof}

$(\Rightarrow)$ Assume that $G$  is a Frobenius group. Then $G$ acts transitively on a set $X$ such that $1< |X| < |G|$, and for every $(x,y) \in X \times X$ such that $x \ne y$, the identity is the only element of $G$ fixing $x$ and $y$.

First we show that every isotropy group $G_x$ is non trivial, i.e. $G_x \ne \{e\}$ and $G_x \ne G$, for all $x \in G$.

Since $G$ acts transitively on $X$, $X = G\cdot x$ is the orbit of $x$, thus
$$ |X| = |G\cdot x | = (G:G_x) = |G| / |G_x|,$$
and since $1 < |X| < |G|$, this proves $1 < |G_x| < |G|$, so $ G_x \ne \{e\}, G_x \ne G$.
Fix $x_0 \in G, x_0 \ne e$, and take $H = G_{x_0}$ the isotropy group of this chosen element $x_0$. Then $1<|H|<G$.

Assume that $g \in G, g \not \in H$, and $h \in H \cap gHg^{-1}$. Then $h$ and $g^{-1}hg$ are both in $H = G_{x_0}$, so that $h \cdot x_0 = x_0$, and 
$(g^{-1}h g) \cdot x_0 = x_0$,
that is
$$
\left\{
\begin{array}{lll}
h \cdot x_0 &= &x_0,\\
h \cdot (g\cdot x_0) &= & (g\cdot x_0).
\end{array}
\right.
$$
Since $g \not \in H = G_{x_0}$, $x_0 \ne g\cdot x_0$, thus $h$ fixes two distinct elements of $X$, and this shows that $h =e$. We have proved $H \cap gHg^{-1} = \{e\}$ for all $g\not \in H$.

\bigskip

$(\Leftarrow)$ Conversely, assume that $G$ has a subgroup $H$ such that $1<|H|<|G|$ and $H\cap gHg^{-1}=\{e\}$ for all $g \notin H$.

Take $X$ as the set of left cosets $hH, \ h\in G$ relative to $H$, and consider the action of $G$ on $X$ defined for all $h \in G$ by
$$g \cdot hH = (gh)H.$$
\be
\item[$\bullet$] This action is transitive: if $kH$ and $lH$ are left cosets, then $(lk)^{-1}\cdot kH = lH$.

\item[$\bullet$] Since $1 < |H| < |G|$, then $1 < |G|/|H| < |G|$, thus $1<|X| < |G|$.

\item[$\bullet$] Assume that $g$ fixes two distinct left cosets $hH \ne kH$:
\begin{align*}
g\cdot hH &= hH,\\
g\cdot kH &= k H.
\end{align*}
Then $l = h^{-1} g h \in H, m = k^{-1} g k \in H$, therefore $m = k^{-1}g k = k^{-1} h l h^{-1} k \in H$, so that
$$l \in H,\qquad(h^{-1}k)^{-1} l (h^{-1}k) \in H.$$
This proves $l \in H \cap gHg^{-1}$, where $g = h^{-1}k\not \in H$ (since $hH \ne k H$), and the hypothesis $H\cap gHg^{-1}=\{e\}$ gives $l = e$, and $g = hlh^{-1} =e$. The identity is the only element of $G$ fixing $hH$ and $kH$.
\ee
Therefore $G$ is a Frobenius group.
\end{proof}

\paragraph{Ex. 14.1.5}

{\it Let $F$ be a subfield of the real numbers, and let $f\in F[x]$ be irreducible of prime degree $p>2$. Assume that $f$ is solvable by radicals. Prove that $f$ has either a single real root or $p$ real roots. 
}

\begin{proof}
Since $\deg(f) = p$ is odd, $f$ has at least a real root. Suppose that $f$ has two distinct real roots $\alpha, \beta$.  By Theorem 14.1.1, since $f$ is solvable by radicals, the splitting field of $f$ over $F$ is $F(\alpha, \beta) \subset \R$. In this case all roots of $f$ are real, and these roots are distinct, since the characteristic of $F$ is $0$, thus the irreducible polynomial $f$ is separable.

We have proved that $f$ has either a single real root or $p$ real roots. 

\end{proof}

\paragraph{Ex. 14.1.6}

{\it By Example 8.5.5, $f=x^5-6x+3$ is not solvable by radicals over $\mathbb{Q}$. Give a new proof of this fact using the previous exercise together with the irreducibility of f and part (b) of Exercise 6 from Section 6.4.
}

\begin{proof}
The given polynomial $f$ has prime degree 5 and only three real roots, according to part (b) of Exercise 6.4.6. Since $f$ has more than one but less than 5 real roots, it is not solvable by radicals by Exercise 14.1.5. 

\end{proof}

\paragraph{Ex. 14.1.7}

{\it Use Lemma 14.1.3 and part (a) of Lemma 14.1.2 to give a proof of part (b) of Lemma 14.1.2 that doesn't use the Sylow Theorems.
}

\begin{proof}
Assume that $\tau\in S_p$ satisfies $\tau\theta\tau^{-1}\in \mathrm{AGL}(1,\mathbb{F}_p)$. Then, since $\langle\theta\rangle$ is a group of order $p$, $\langle\tau\theta\tau^{-1}\rangle=\tau\langle\theta\rangle\tau^{-1}$ is a subgroup of $\mathrm{AGL}(1,\mathbb{F}_p)$ of order $p$ and each element of this subgroup has order $p$ (or $1$).

By part (a) of Lemma 14.1.2, $\mathrm{AGL}(1,\mathbb{F}_p)$ is the normalizer of $\langle\theta\rangle$ in $S_p$, therefore $\langle\theta\rangle$ is normal in $\mathrm{AGL}(1,\mathbb{F}_p)$ with $[\mathrm{AGL}(1,\mathbb{F}_p):\langle\theta\rangle]=p-1$. The order of each element of $\tau\langle\theta\rangle\tau^{-1}$ is relatively prime to $p-1$, then, by Lemma 14.1.3, $\tau\langle\theta\rangle\tau^{-1} \in \langle \theta \rangle$, thus $\tau\langle\theta\rangle\tau^{-1}\subset\langle\theta\rangle$, therefore $\tau\langle\theta\rangle\tau^{-1}=\langle\theta\rangle$, since both groups have the same order $p$.  

Thus $\tau$ normalizes $\langle\theta\rangle$, hence $\tau\in \mathrm{AGL}(1,\mathbb{F}_p)$.
\end{proof}
\paragraph{Ex. 14.1.8}

{\it Let $f\in F[x]$ be irreducible of prime degree $p\geq 5$, where $F$ has characteristic 0, and let $\alpha\ne\beta$ be roots of f in some splitting field. If $F(\alpha,\beta)$ contains all other roots of $f$, then $f$ is solvable by radicals by Theorem 14.1.1. But suppose that there is some third root $\gamma$ such that $\gamma\in F(\alpha,\beta)$. Is this enough to force $f$ to be solvable by radicals?
\begin{enumerate}
\item[(a)] Use the classification of transitive subgroups of $S_5$ from Section 13.2 to show that the answer is ``yes'' when p=5.
\item[(b)] Use the polynomial $x^7-154\,x+99$ from Example 13.3.10 to show that the answer is ``no'' when p=7.
\end{enumerate}
}
\begin{proof}
\begin{enumerate}
\item[(a)]   
By hypothesis, $\deg(f) = p = 5$, and $\alpha \ne \beta$ are roots of $f$ in some splitting field.

Since $\alpha$ is a root of $f$, which is irreducible over $F$,
$$[F(\alpha) : F] = \deg(f) = p = 5.$$
Then $\beta$ is a root of $\frac{f(x)}{x-\alpha} \in F(\alpha)[x]$, so that the minimal polynomial of $\beta$ over $F(\alpha)$ has degree $d\leq p-1$. Thus
$$[F(\alpha,\beta) : F(\alpha] \leq p-1 = 4.$$
By the Tower Theorem,
$$[F(\alpha,\beta):F]  =   [F(\alpha,\beta):F(\alpha)] \ [F(\alpha):F] \leq p(p-1) = 20.$$
Now, suppose that there is some third root $\gamma$ such that $\gamma\in F(\alpha,\beta)$. Then $F(\alpha,\beta,\gamma) = F(\alpha,\beta)$. Let $\delta, \varepsilon$ be the remaining roots of $f$. Since the characteristic is $0$, the irreducible polynomial $f$ is separable.Then $\delta$ is a root of $\frac{f(x)}{(x-\alpha)(x-\beta)(x-\gamma)} \in F(\alpha,\beta,\gamma)[x]$, so that
$$[F(\alpha,\beta,\gamma,\delta):F(\alpha,\beta,\gamma)] \leq 2.$$
Since $F(\alpha,\beta,\gamma) = F(\alpha,\beta)$, the tower theorem gives
$$[F(\alpha,\beta,\gamma, \delta):F] \leq 40.$$
Moreover $\alpha + \beta +\gamma + \delta + \varepsilon = \sigma_1(\alpha,\beta,\gamma, \delta,\varepsilon) \in F$, thus $F(\alpha,\beta,\gamma, \delta,\varepsilon) = F(\alpha,\beta,\gamma, \delta)$. Write $L = F(\alpha,\beta,\gamma, \delta,\varepsilon)$ the splitting field of $f$ over $F$. We have proved
$$[L:F]\leq 40.$$
The classification of transitive subgroups of $S_5$ from Section 13.2 shows that any transitive subgroup of $S_5$ with cardinality $\le 40$ is a subgroup of $\mathrm{AGL}(1,\F_5)$, thus is solvable. So $\Gal(L/F)$ is a solvable group, where $F$ has characteristic $0$, therefore $f$ is solvable (Theorem 8.5.3).

To conclude, the answer is ``yes''  when $p = \deg(f) = 5$.
\item[(b)] To prove that the answer is ``no''  when $p = \deg(f) = 7$, we use the counterexample $f = x^7-154\, x+99$ from Example 13.3.10.

The polynomial $f$ is not solvable, since its Galois group is $\mathrm{GL}(3,\F_2)$, which is simple (Section 14.3) and not commutative, thus non solvable.

We prove that there are roots $\alpha,\beta, \gamma$ of $f$ such that $\gamma \in F(\alpha,\beta)$.

As in Example 13.3.10, consider the resolvant
$$\Theta_f(y) = \prod_{1\leq i < j < k \leq 7} \left(y - (\alpha_i + \alpha_j + \alpha_k)\right) \in \Q[y].$$
Then the factorization of $\Theta_f(y)$ over $\Q$ is 
$$\Theta_f(y) = g(y) h(y),$$
where the polynomials $g,h$, given in Example 13.3.10, are irreducible factors of degrees $7$ and $28$.

Take three roots $\alpha,\beta,\gamma$ of $f$ such that $y-(\alpha+\beta+ \gamma)$ is any linear factor of $g$, so that the minimal polynomial of $\alpha+\beta+ \gamma$ is $g$, with $\deg(g) = 7$, thus
$$[\Q(\alpha+\beta+\gamma) : \Q] = 7.$$
Now we prove that $\gamma \in F(\alpha,\beta)$. Consider the chain of extensions
$$\Q\subset \Q(\alpha) \subset \Q(\alpha,\beta) \subset \Q(\alpha,\beta,\gamma) \subset L,$$
where $L$ is the splitting field of $f$ over $\Q$.

The minimal polynomial of $\alpha$ over $\Q$ is $f$, thus $[\Q(\alpha) : \Q] = 7$, and 
$$[L:\Q] = |\Gal(L/\Q)| = |\mathrm{GL}(3,\F_2)| = 168 = 2^3 \times 3 \times 7.$$
By the Tower Theorem,
$$[L : \Q(\alpha)] = \frac{ [L : \Q]}{[ \Q(\alpha) : \Q]} = 2^3 \times 3$$
is not divisible by 7.

Since $\gamma$ is a root of $f$, the minimal polynomial of $\gamma$ over $f$ divides $f$. Thus
$$[\Q(\alpha,\beta,\gamma) : \Q(\alpha,\beta)] = 1\text{ or } 7.$$
If $[\Q(\alpha,\beta,\gamma) : \Q(\alpha,\beta)]  = 7$, by the Tower Theorem, $7$ divides $[L : \Q(\alpha)] = 2^3\times 3$. This contradiction proves that 
$$[\Q(\alpha,\beta,\gamma) : \Q(\alpha,\beta)] = 1,$$
therefore $\gamma \in \Q(\alpha,\beta).$

In this example, there exist roots $\alpha \ne \beta$ of $f$, and some third root $\gamma$ such that $\gamma \in F(\alpha,\beta)$, but $f$ is not solvable.

This shows that the answer is ``no'' when $p =\deg(f) = 7$.
\end{enumerate}

\end{proof}
\qquad

Note: In the proof of the Proposition 13.3.9, we saw that $G_f$ must be conjugate to $\mathrm{GL}(3,\F_2)$. This means that there is some numbering of the roots
$$
\left\{
\begin{array}{ccc}
\F_2^3\setminus\{(0,0,0\} & \to &\{\alpha \in L \, |\, f(\alpha) = 0\}\\
(\nu_1,\nu_2,\nu_3) &\to &\alpha_{\nu_1,\nu_2,\nu_3}
\end{array}
\right.
$$
which verify that, for all $\sigma \in\Gal(L/F)$, there is some $g \in \mathrm{GL}(3,\F_2)$ such that
$$\sigma(\alpha_{\nu_1,\nu_2,\nu_3}) = \alpha_{g \cdot (\nu_1,\nu_2,\nu_3)}.$$

In this correspondance, the roots of $f$ are seen as nonzero vectors in $\F_2^3$, and the seven roots of $g$ correspond to the seven (unordered)  triples of  linearly dependent  nonzero vectors in $\F_2^3$. So the roots $\alpha,\beta, \gamma$ where chosen in the preceding proof such that the corresponding vectors $u,v,w$ verify $w= u+v$ (but not $\gamma = \alpha + \beta$) .

This is what we understand in the hint of D.A. Cox ``Regard the roots as the nonzero vectors of $\F_2^3$ and pick roots $\alpha,\beta, \gamma$ such that $\gamma = \alpha+\beta$''.

This last equality is not true in $L$, but true for the corresponding vectors in $\F_2^3$.

Moreover, let $\alpha\ne \beta$ be {\it any} pair of roots. The corresponding vectors $u,v$ are such that $u,v,u+v = -u-v$ is not a base, so that the root $\gamma$ corresponding to $u+v$ is such that $ y - (\alpha+\beta+\gamma)$ is a factor of $g$, and the preceding proof shows that $\gamma \in \Q(\alpha,\beta)$. For each pair $\alpha\ne \beta$ of roots of $f = x^7-154\, x+99$, there exists a third root $\gamma \not \in \{\alpha,\beta\}$ such that $\gamma \in F(\alpha,\beta)$.







\subsection{IMPRIMITIVE POLYNOMIALS OF PRIME-SQUARED DEGREE}
\paragraph{Ex. 14.2.1}

{\it Prove (14.7).
}

\begin{proof}
Given $\sigma'=(\tau';\mu_1',...,\mu_k'),\sigma=(\tau;\mu_1,...,\mu_k)\in A\wr B$. Since $\sigma'$ maps $R_i$ to $R_{\tau'(i)}$ via $\mu_i'$, if we set $j=\tau'(i)$, then $\sigma$ maps $R_j$ to $R_{\tau(j)}=R_{\tau(\tau'(i))}=R_{\tau\tau'(i)}$ via $\mu_j=\mu_{\tau'(i)}$.

Hence $\sigma\sigma'$ maps $R_i$ to $R_{\tau\tau'(i)}$ via $\mu_{\tau'(i)}\mu_i'$.

More explicitly, by the definition of $(\tau; \mu_1,\ldots,\mu_k)$, for all $(i,j) \in \{1,\ldots,k\} \times \{1,\ldots,l\}$,
$$(\tau;\mu_1,\ldots,\mu_k)(i,j) = (\tau(i), \mu_i(j)).$$
Applying three times this definition, we obtain
\begin{align*}
(\tau; \mu_1,\ldots,\mu_k)(\tau'; \mu'_1,\ldots,\mu'_k)&= (\tau; \mu_1,\ldots,\mu_k)(\tau'(i),\mu'_i(j))\\
&= (\tau( \tau'(i)), \mu_{\tau'(i)}(\mu'_i(j))\\
&= ((\tau \tau')(i), (\mu_{\tau'(i)} \mu'_i)(j)\\
&=(\tau\tau';\mu_{\tau'(1)}\mu_1',...,\mu_{\tau'(k)}\mu_k')(i,j)
\end{align*}
Since this equality is true for all $(i,j) \in \{1,\ldots,k\} \times \{1,\ldots,l\}$,
$$(\tau;\mu_1,...,\mu_k)(\tau';\mu_1',...,\mu_k')=(\tau\tau';\mu_{\tau'(1)}\mu_1',...,\mu_{\tau'(k)}\mu_k').$$
\end{proof}

\paragraph{Ex. 14.2.2}
{\it The wreath product $S_3 \wr S_2 \subset S_6$ can be thought of as the subgroup of all permutations that preserve the blocs $R_1 = \{1,2\}, R_2 = \{3,4\},R_3 = \{5,6\}$. As noted in Example 14.2.11, $S_3 \wr S_2$ has order $6\cdot3^3 = 48$.
\be
\item[(a)] Show that $(S_3 \wr S_2) \cap A_6$ has order $24$.
\item[(b)] Show that $S_3 \wr S_2$ is the centralizer of $(1\,2)(3\,4)(5\,6)$ in $S_6$ (meaning that $S_3 \wr S_2$ consists of all permutations in $S_6$ that commute with $(1\,2)(3\,4)(5\,6)$).
\item[(c)] Use part (b) to show that $S_3 \wr S_2$ is isomorphic to $((S_3 \wr S_2) \cap A_6) \times S_2$.
\ee
See the next exercise for more on $S_3 \wr S_2$ and $(S_3 \wr S_2) \cap A_6$.
}
\begin{proof}
\item[(a)] Let $\varphi$ the restriction of the sign $\mathrm{sgn}$ to $(S_3 \wr S_2) \cap A_6$:
$$
\varphi \left\{
\begin{array}{ccc}
S_3 \wr S_2 & \to & \{-1,1\}\\
\sigma & \mapsto & \mathrm{sgn}(\sigma)
\end{array}
\right.
$$
Since $\mathrm{sgn}$ is a morphism, its restriction $\varphi$ is also a morphism, and $\varphi$ is surjective (onto), because $\varphi(e) = 1$, and $\varphi((1\,2)) = -1$, where $(1\,2) \in S_3\wr S_2$. Moreover the kernel of $\varphi$ is $\ker(\varphi) =  (S_3 \wr S_2) \cap A_6$.

Therefore $\mathrm{im}(\varphi)  \{-1,1\} \simeq (S_3 \wr S_2)/ ((S_3 \wr S_2) \cap A_6)$. This shows that $$|(S_3 \wr S_2) \cap A_6| = \frac{1}{2} |S_3 \wr S_2| = 24.$$

\item[(b)] Let $\tau \in S_n$. Then $\tau$ is in the centralizer of $\sigma = (1\,2)(3\,4)(5\,6)$ if and only if
$$\tau (1\,2)(3\,4)(5\,6) \tau^{-1} = (1\,2)(3\,4)(5\,6),$$
which is equivalent to
$$(\tau(1)\, \tau(2))(\tau(3)\, \tau(4))(\tau(5)\, \tau(6)) = (1\,2)(3\,4)(5\,6).$$
Write $R_1 = \{1,2\}, R_2 = \{3,4\},R_3 = \{4,5\}$. Then $R_1, R_2,R_3$ are the three orbits  of $\sigma$ acting on $\{1,\ldots,6\}$, the supports of the decomposition of $\sigma$ in disjoint cycles.

Since $\tau$ is a bijection, the 6 values $\tau(1),\tau(2),\tau(3), \tau(4),\tau(5), \tau(6)$ are distinct, so $(\tau(1)\, \tau(2)),(\tau(3)\, \tau(4)),(\tau(5)\, \tau(6))$ are disjoint $2$-cycles. 

If $\tau$ is the centralizer of $\sigma$, the equality $(\tau(1)\, \tau(2))(\tau(3)\, \tau(4))(\tau(5)\, \tau(6)) = (1\,2)(3\,4)(5\,6)$ shows that $\tau(R_1),\tau(R_2),\tau(R_3)$ are also the three orbits of $\sigma$, so that
$$\{\{1,2\},\{3,4\},\{5,6\}\} =\{\{ \tau(1), \tau(2)\},\{\tau(3), \tau(4)\},\{\tau(5), \tau(6)\}\},$$
that is
$$\{R_1,R_2,R_3\} = \{\tau(R_1),\tau(R_2),\tau(R_3)\},$$
which means that there is some permutation $\tau'$ of $\{1,2,3\}$ such that $\tau(R_i) = R_{\tau'(i)}, \ i=1,2,3$. In other words, $\sigma$ preserves the blocks $R_1,R_2,R_3$, so that $\sigma \in S_3 \wr S_2$.

To prove the converse, it is more convenient to use the other usual representation of $S_3\wr S_2$. Then $\sigma = (e; \mu,\mu,\mu)$, where $\mu = (1\, 2) \in S_2$. Let $\tau = (\lambda; \mu_1,\mu_2,\mu_3)$ be any element of $S_3 \wr S_2$ (then $\mu_i = ()$ or $\mu_i = \mu)$. Then (14.7) gives
\begin{align*}
\tau \sigma &= (\lambda; \mu_1,\mu_2,\mu_3)(e; \mu,\mu,\mu)\\
&=(\lambda;\mu_1\mu, \mu_2 \mu,\mu_3\mu)\\
\sigma \tau &= (e;\mu,\mu;\mu)(\lambda,\mu_2,\mu_2,\mu_3)\\
&= (\lambda; \mu \mu_1, \mu \mu_2,\mu \mu_3)
\end{align*}
Since $S_2 = \{e,\mu\}$ is commutative, $\mu \mu_i = \mu_i \mu,\ i=1,2,3$, thus $\tau \sigma = \sigma \tau$.

The centralizer of $(1\,2)(3\,4)(5\,6)$ in $S_n$ is $S_3 \wr S_2$.

\item[(c)] Since the order of $\sigma = (1\,2)(3\,4)(5\,6)$ is $2$, $\langle \sigma \rangle = \{e,\sigma\} \simeq S_2$ and we can write $\sigma^{\varepsilon }, \varepsilon \in \{0,1\}$ the two elements of $\langle \sigma \rangle$.
Let
$$
\varphi
\left \{
\begin{array}{ccc}
(S_3 \wr S_2) \cap A_6 \times \langle \sigma \rangle & \to & S_3\wr S_2\\
(\tau, \sigma^{\varepsilon}) & \mapsto &\tau \sigma^{\varepsilon}.
\end{array}
\right.
$$
$\bullet$ $\varphi$ is a morphism: For all $\tau, \tau' \in (S_3 \wr S_2) \cap A_6$ and $\sigma^{\varepsilon}, \sigma^{\varepsilon'} \in \langle \sigma \rangle $, $\sigma \tau' = \tau' \sigma$ by part (b), thus
\begin{align*}
\varphi(\tau \sigma^{\varepsilon}) \varphi(\tau' \sigma^{\varepsilon'})&= \tau \sigma^{\varepsilon} \tau' \sigma^{\varepsilon'}\\
&= \tau \tau' \sigma^\varepsilon \sigma^{\varepsilon'}\\
&= \varphi((\tau,\sigma^\varepsilon) (\tau', \sigma^{\varepsilon'}))
\end{align*}

$\bullet$ $\ker \varphi$ is trivial: if $\varphi(\tau, \sigma^{\varepsilon}) = e$, then $\tau \sigma^{\varepsilon} = e$, so that $\tau  = \sigma^{-\varepsilon} \in \{e, \sigma\}$. $\tau = \sigma$ is impossible, since $\tau$ is an even permutation, and $\sigma$ is odd. Therefore $\tau = e$, and $\sigma^{\varepsilon} = e$. Thus $\varphi$ is injective (one to one).

$\bullet$ Since $|((S_3 \wr S_2) \cap A_6) \times \langle \sigma \rangle | = |S_3\wr S_2|$, $\varphi$ is a bijection, thus $\varphi$ is a group isomorphism.

$$S_3 \wr S_2 \simeq ((S_3 \wr S_2) \cap A_6) \times \langle \sigma \rangle  \simeq  ((S_3 \wr S_2) \cap A_6) \times S_2 .$$
\end{proof}

\paragraph{Ex. 14.2.3}
{\it One of the challenges of group theory is that the same group can have radically differnet descriptions. For instance, $S_4$ and the group $G = (S_3 \wr S_2) \cap A_6$ appearing in Example 14.2.11 both have order $24$. In this exercise, you will prove that they are isomorphic. We will use the notation of Exercise 2.
\be
\item[(a)] There is a natural homomorphism $G \to S_3$ given by how elements of $G$ permute the blocks $R_1,R_2,R_3$. Show that this map is onto, and express the elements of the kernel as products of disjoints cycles.
\item[(b)] Use the Sylow Theorems to show that $G$ has one or four $3$-Sylow subgroups.
\item[(c)] Show that $A_6$ has no element of order $6$.
\item[(d)] Use part (c) and the kernel of the map $G \to S_3$ from part (a) to show that $G$ has four $3$-Sylow subgroups.
\item[(e)] $G$ acts by conjugation on its four $3$-Sylow subgroups. Use this to prove that $G \simeq S_4$.
\item[(f)] Using Exercise 2, conclude that $S_3 \wr S_2 \simeq S_4 \times S_2$.

We note without proof that $S_3 \wr S_2 \simeq S_4 \times S_2$ is also isomorphic to the full symmetry group (rotations and reflexions) of the octahedron.
\ee
}

\begin{proof}
\item[(a)] Let $\varphi : G \to S_3$ defined by $\tau = \varphi(\sigma)$ iff $\sigma(R_i) = R_{\tau(i)}$. In other notations, this is the restriction to $G$ of the homomorphism of part (b) of Lemma 14.2.8, thus $\varphi$ is an homomorphism.

$\bullet$ $\varphi$ is surjective: Let $\tau$ be any permutation in $S_3$. 

If $\tau$ is even, $\tau = (1\,2\,3)^k,\ k = 0,1,2$. Let
$$\sigma =
\left(
\begin{array}{cccccccc}
1 & 2 & & 3 & 4 & & 5 & 6\\
3 & 4 & & 5 & 6 & & 1& 2\\
\end{array}
\right)
= (1\,3\,5)(2\,4\,6).
$$
$\sigma$ preserves the block structure defined by $R_1,R_2,R_3$, and $\sigma \in A_6$, so that $\sigma \in G = (S_3 \wr S_2) \cap A_6$. Moreover $\sigma(R_1) = R_2, \sigma(R_2) = R_3, \sigma(R_3) = R_1$, thus $\varphi(\sigma) = (1\,2\,3)$, and $\varphi(\sigma^k )= (1\,2\,3)^k = \tau$.

If $\tau$ is odd, then $\tau \in \{(1\,2),(2\,3),(1\,3)\}$, and
\begin{align*}
(1\,2) = \varphi(\sigma_1), \qquad \sigma_1 = 
\left(
\begin{array}{cccccccc}
1 & 2 & & 3 & 4 & & 5 & 6\\
3 & 4 & & 1 & 2 & & 5& 6\\
\end{array}
\right) = (1\,3)(2\,4) \in G,\\
(2\,3) = \varphi(\sigma_2), \qquad \sigma_2 = 
\left(
\begin{array}{cccccccc}
1 & 2 & & 3 & 4 & & 5 & 6\\
1 & 2 & & 5 & 6 & & 3& 4\\
\end{array}
\right) = (3\,5)(4\,6) \in G,\\
(1\,3) = \varphi(\sigma_3), \qquad \sigma_3 = 
\left(
\begin{array}{cccccccc}
1 & 2 & & 3 & 4 & & 5 & 6\\
5 & 6 & & 3 & 4 & & 1& 2\\
\end{array}
\right) = (1\,5)(2\,6) \in G.\\
\end{align*}
Therefore $\varphi$ is surjective.

$\bullet$ Let $\sigma \in S_6$. Then $\sigma \in \ker \varphi$ iff $\sigma \in A_6$ and $\sigma(R_1) = R_1, \sigma(R_2) = R_2, \sigma(R_3) = R_3$.

Morerover, for all $\sigma \in A_6$,
\begin{align*}
&\sigma(R_1) = R_1, \sigma(R_2) = R_2, \sigma(R_3) = R_3\\
\iff& \{\sigma(1), \sigma(2)\} = \{1,2\}, \{\sigma(3),\sigma(4)\} = \{3,4\}, \{\sigma(5),\sigma(6)\} = \{5,6\}\\
\iff &\sigma \in \{e,(1\,2)(3\,4), (1\,2)(5\,6), (3\,4)(5\,6)\}.
\end{align*}
$$\ker \varphi = \{e,(1\,2)(3\,4), (1\,2)(5\,6), (3\,4)(5\,6)\}.$$
Verification: $6 = |S_3| = |G/\ker(\varphi)| = 24/4$.

\item[(b)] Let $N$ be the number of $3$-Sylow subgroups of $G$. By the third Slow Theorem,
$$N \mid 24 = |G|, \qquad N \equiv 1 \pmod 3.$$
Therefore $N = 1$ or $N = 4$.

\item[(c)] Let $\tau \in S_6$ be a permutation of order 6. If $\tau = \tau_1\cdots \tau_k$ is the decomposition of $\tau$ in disjoint cycles, then the order of $\tau$ is the lcm of the order of$ \tau_1,\ldots,\tau_k$. Therefore $\tau$ is a $6$-cycle or a product of a $2$-cycle by a $3$-cycle. In both cases $\tau$ is odd. Therefore $A_6$ has no element of order $6$.


\item[(d)] Reasoning by contradiction, suppose that $G$ has only one $3$-Sylow subgroup $H$. Then, for all $g \in G$, $gH g^{-1}$ is a $3$-Sylow, thus $gH g^{-1} = H$, and $H$ is a normal subgroup of $G$. 

Moreover $K = \ker \varphi = \{e,(1\,2)(3\,4), (1\,2)(5\,6), (3\,4)(5\,6)\}$ is normal in $G$, and has order 4. Therefore $H \cap K = \{e\}$.

The usual characterization of direct products (see Ex. 14.3.7) shows that, for all $h \in H$, all $k \in K$, $hk = kh$, and $HK$ is a normal subgroup of $G$ isomorphic to $H \times K$. 

Take $h$ an element of order $3$ in $H,$ and $k$ and element of order 2 in $K$. Since $kh = hk$, the order of $hk \in A_6$ is 6, which is impossible by part (c).

Therefore $G$ has exactly four $3$-Sylow subgroups.

\item[(e)] Write $X = \{H_1,H_2,H_3,H_4\}$ the set of $3$-Sylow subgroups of $G$, and $S(X)$ the set of permutations of $X$. Then $S(X) \simeq S_4$, and $g\cdot H = g H g^{-1}$ defines a left action of $G$ on $X$, so that
$$
\psi 
\left\{
\begin{array}{ccc}
G & \to & S(X)\\
g & \mapsto & 
\sigma = \left(
    \begin{array}{cccc}
      H_1&H_2&H_3&H_4\\
      gH_1g^{-1} & gH_2 g^{-1} & gH_3g^{-1} & gH_4g^{-1}
    \end{array}
\right)
\end{array}
\right.
$$
is a group homomorphism.

It is not obvious that $\psi$ is bijective. We prove first that $\psi$ is surjective (onto). We give explicitly the $3$-Sylow subgroups. Let
\begin{align*}
&\lambda_1 = 
\left(
\begin{array}{cccccccc}
1 & 2 & & 3 & 4 & & 5 & 6\\
 3&4&&5&6&&1&2 \\
\end{array}
\right) = (1\,3\,5)(2\,4\,6),\\
&\lambda_2 = 
\left(
\begin{array}{cccccccc}
1 & 2 & & 3 & 4 & & 5 & 6\\
 3&4&&6&5&&2&1 \\
\end{array}
\right) = (1\,3\,6)(4\,5\,2),\\
&\lambda_3 = 
\left(
\begin{array}{cccccccc}
1 & 2 & & 3 & 4 & & 5 & 6\\
 6&5&&2&1&&3&4 \\
\end{array}
\right) = (1\,6\,4)(5\,3\,2),\\
&\lambda_4 =
\left(
\begin{array}{cccccccc}
1 & 2 & & 3 & 4 & & 5 & 6\\
 4&3&&6&5&&1&2 \\
\end{array}
\right) = (1\,4\,5)(3\,6\,2).
\end{align*}
Then $\lambda_1,\ldots,\lambda_4 \in G$ have order $3$, and $H_1 = \langle \lambda_1 \rangle = \{e,\lambda_1,\lambda_1^2\},\ldots,H_4 = \langle \lambda_1 \rangle = \{e,\lambda_4,\lambda_4^2\}$ are distinct, thus they are the four $3$-Sylow of $G$.

Now take 
\begin{align*}
&g = 
\left(
\begin{array}{cccccccc}
1 & 2 & & 3 & 4 & & 5 & 6\\
 4&3&&2&1&&5&6 \\
\end{array}
\right) = (1\,4)(2\,3)\\
&h=
\left(
\begin{array}{cccccccc}
1 & 2 & & 3 & 4 & & 5 & 6\\
 2&1&&5&6&&4&3 \\
\end{array}
\right) = (1\,2)(3\,5\,4\,6)
\end{align*}
(We give a geometrical explanation of this choice in the final note.)

Then 
\begin{align*}
g \lambda_1 g^{-1} &= (1\,4)(2\,3)(1\,3\,5)(2\,4\,6)(1\,4)(2\,3)\\
&=
\left(
\begin{array}{cccccccc}
1 & 2 & & 3 & 4 & & 5 & 6\\
 6&5&&1&2&&4&3 \\
\end{array}
\right) = (1\,6\,3)(2\,5\,4) = \lambda_2^2,
\end{align*}
thus $gH_1g^{-1} = H_2$, and since $g = g^{-1}$, $g H_2 g^{-1} = H_1$.
Moreover
\begin{align*}
g \lambda_3g^{-1} &=(1\,4)(2\,3)(1\,6\,4)(5\,3\,2)(1\,4)(2\,3)\\
&= 
\left(
\begin{array}{cccccccc}
1 & 2 & & 3 & 4 & & 5 & 6\\
 4&3&&5&6&&2&1 \\
  \end{array}
  \right)
 =(1\,4\,6)(2\,3\,5)
 =\lambda_3^2,
\end{align*}
thus $gH_3g^{-1} = H_3$, and since $\psi(g)$ is a permutation, $gH_4 g^{-1} = H_4$.

Therefore $\psi(g) \in S(X)$ is the permutation 
$ 
\left(
\begin{array}{cccc}
H_1&H_2&H_3&H_4\\
H_2&H_1&H_3&H_4
\end{array}
\right)
$, which corresponds to the transposition $(1\,2) \in S_4$.
Similarly
\begin{align*}
h\lambda_1h^{-1}&= (1\,2)(3\,4\,5\,6)(1\,3\,5)(2\,4\,6)(3\,6\,4\,5)(1\,2)\\
&=
\left(
\begin{array}{cccccccc}
1 & 2 & & 3 & 4 & & 5 & 6\\
 6&5&&1&2&&4&3 \\
  \end{array}
  \right) =(1\,6\,3)(2\,5\,4) = \lambda_2^2,\\
 h\lambda_2h^{-1} &= (1\,2)(3\,4\,5\,6)(1\,3\,6)(4\,5\,2)(3\,6\,4\,5)(1\,2)\\
 &=
 \left(
\begin{array}{cccccccc}
1 & 2 & & 3 & 4 & & 5 & 6\\
 6&5&&2&1&&3&4 \\
  \end{array}
  \right) = (1\,6\,4)(2\,5\,3) = \lambda_3,\\
   h\lambda_3h^{-1} &= (1\,2)(3\,4\,5\,6)(1\,6\,4)(5\,3\,2)(3\,6\,4\,5)(1\,2)\\
 &=
 \left(
\begin{array}{cccccccc}
1 & 2 & & 3 & 4 & & 5 & 6\\
 4&3&&6&5&&1&2 \\
  \end{array}
  \right) = (1\,4\,5)(2\,6\,4) = \lambda_4,
\end{align*}
thus $hH_1h^{-1} = H_2, hH_2h^{-1} = H_3, hH_1h^{-1} = H_4$, and since $\psi(g)$ is a permutation, $h H_4h^{-1} = H_1$. Therefore 
$\psi(g) = \left(
\begin{array}{cccc}
H_1&H_2&H_3&H_4\\
H_2&H_3&H_4&H_1
\end{array}
\right)
$
corresponds to the $4$-cycle $(1\,2 \,3 \,4)$. 

Since $\{(1\,2), (1\,2\,3\,4)\}$ is a set of generators of $S_4$, $S(X)$ is generated by $\psi(g), \psi(h)$, so that $S(X) = \psi(G)$, and $\psi$ is surjective. Moreover, $|G| = |S(X)| = 24$, thus $\psi$ is a bijection, and a group isomorphism:
$$G \simeq S(X) \simeq S_4.$$

\item[(f)] To conclude, using Exercise 2, we obtain
$$S_3\wr S_2 \simeq ((S_3 \wr S_2) \cap A_6) \times S_2  = G \times S_2 \simeq S_4 \times S_2.$$
\bigskip

Note: We have proved in Exercise 7.5.10 that the symmetry group $G_0$ of the cube (or octahedron), is isomorphic to $S_4$. By composition with the indirect isometry $\sigma : v\mapsto -v$, which commutes with all elements in the group, we obtain the full symmetry group, isomorphic to $S_4 \times S_2$.

We have a geometrical description of $G = (S_3\wr S_2) \cap A_6$ by regrouping the opposite faces of a cube in blocs: stick $1$ on a face of a dice, $2$ on the opposite face, and so on (I sticked labels on my Rubik's cube). Then the $24$ rotations of the cube send opposite faces on opposite faces, so that the bloc structure $\{\{1,2\},\{3,4\},\{5,6\}\}$ is preserved by rotations.

We have proved in Exercise 7.5.10 that $G_0$ acts on the 4 long diagonals $D_1,D_2,D_3,D_4$ of the cube, so that $G_0 \simeq S_4$. Each of the four $3$-Sylow of $G_0$ is generated by the rotation with angle $\frac{2\pi}{3}$ around such a  long diagonal. They correspond to the $3$-Sylow $H_1,\ldots,H_4$ of $G$: this was useful for the above description of the $H_i$. Each $3$-Sylow corresponds to a long diagonal, so that $gH_ig^{-1} = H_j$ is equivalent to $\sigma(D_i) = D_j$, where $\sigma$ corresponds to $g$. It remains to find a rotation which acts on these diagonals as some given permutation in $S_4$, such that $(1\,2)$ or $(1\,2\,3\,4)$. The corresponding permutations $g,h \in G $ are given in the text.

\end{proof}


\end{document}
