%&LaTeX
\documentclass[11pt,a4paper]{article}
\usepackage[frenchb,english]{babel}
\usepackage[applemac]{inputenc}
\usepackage[OT1]{fontenc}
\usepackage[]{graphicx}
\usepackage{amsmath}
\usepackage{amsfonts}
\usepackage{amsthm}
\usepackage{amssymb}
\usepackage{tikz}
%\input{8bitdefs}

% marges
\topmargin 10pt
\headsep 10pt
\headheight 10pt
\marginparwidth 30pt
\oddsidemargin 40pt
\evensidemargin 40pt
\footskip 30pt
\textheight 670pt
\textwidth 420pt

\def\imp{\Rightarrow}
\def\gcro{\mbox{[\hspace{-.15em}[}}% intervalles d'entiers 
\def\dcro{\mbox{]\hspace{-.15em}]}}

\newcommand{\be} {\begin{enumerate}}
\newcommand{\ee} {\end{enumerate}}
\newcommand{\deb}{\begin{eqnarray*}}
\newcommand{\fin}{\end{eqnarray*}}
\newcommand{\ssi} {si et seulement si }
\newcommand{\D}{\mathrm{d}}
\newcommand{\Q}{\mathbb{Q}}
\newcommand{\Z}{\mathbb{Z}}
\newcommand{\N}{\mathbb{N}}
\newcommand{\R}{\mathbb{R}}
\newcommand{\C}{\mathbb{C}}
\newcommand{\F}{\mathbb{F}}
\newcommand{\U}{\mathbb{U}}
\newcommand{\re}{\,\mathrm{Re}\,}
\newcommand{\im}{\,\mathrm{Im}\,}
\newcommand{\ord}{\mathrm{ord}}
\newcommand{\Gal}{\mathrm{Gal}}
\newcommand{\legendre}[2]{\genfrac{(}{)}{}{}{#1}{#2}}

\title{Solutions to David A.Cox  "Galois Theory''}

\refstepcounter{section} \refstepcounter{section}
\refstepcounter{section} \refstepcounter{section}
\refstepcounter{section}\refstepcounter{section}\refstepcounter{section}\refstepcounter{section}
\refstepcounter{section}\refstepcounter{section}\refstepcounter{section}\refstepcounter{section}\refstepcounter{section}


\begin{document}



\section{Chapter 14 : SOLVABLE PERMUTATION GROUPS}

\subsection{POLYNOMIAL OF PRIME DEGREE}

\paragraph{Ex. 14.1.1}
{\it This exercise is concerned with the proof of part (a) of Lemma 14.1.2. Let $\theta =(1\, 2 \ldots p)\in S_p$.
\begin{enumerate}
\item[(a)] Prove that $\tau\in S_p$ lies in the normalizer of $\langle\theta\rangle$ if and only if $\tau\theta =\theta^l\tau$ for some $1\leq l\leq p-1$.
\item[(b)] Prove that (14.1) implies that $\tau(i+j)=\tau(i)+jl$ for all positive integers j.
\end{enumerate}
}
\begin{proof}
\begin{enumerate}
\item[(a)]
If $\theta$ lies in the normalizer of $\langle \theta\rangle = \{e,\theta,\theta^2,\ldots,\theta^{p-1}\}$, then
$$\tau \theta \tau^{-1} \in \tau \langle \theta \rangle \tau^{-1}= \langle \theta \rangle,$$
hence
$$\tau \theta \tau^{-1} =\theta^l \text{ for some } l = 0,1\ldots,p-1.$$
If $l = 0$, then $\tau \theta \tau^{-1} = e$, thus $\tau \theta = \tau$, and $\theta = e$, which is false. Therefore $l \ne 0$.
$$\tau \theta \tau^{-1} =\theta^l , \ 1\leq l \leq p-1.$$

 \item[(b)] By induction suppose that $\tau(i+j)=\tau(i)+jl$, then $\tau(i+j+1)=\tau(i+j)+l=\tau(i)+(j+1)l$. Case $j=1$ is valid by the identity (14.1). Hence, $\tau(i+j)=\tau(i)+jl$ for all positive integers $j$.  
\end{enumerate}
\end{proof}

\paragraph{Ex. 14.1.2}

{\it Let $H$ be a normal subgroup of a finite group $G$ and let $g\in G$. The goal of this exercise is to prove Lemma 14.1.3.
\begin{enumerate}
\item[(a)] Explain why $(gH)^{o(g)}=(gH)^{[G:H]}=H$ in the quotient group $G/H$.
\item[(b)] Now assume that $\gcd(o(g),[G:H])=1$. Prove that $g\in H$.
\end{enumerate}
}
\begin{proof}
\begin{enumerate}
\item[(a)] Since $(gH)^2=gHgH=g^2H$ and $g^{o(g)}=e$, $(gH)^{o(g)}=g^{o(g)}H=H$.

Since $gH\in G/H$, exists some minimal $l$ such that $(gH)^l=H$ and $l\mid [G:H]$, i.e. $[G:H]=ql$. Then $(gH)^{[G:H]}=(gH)^{ql}=H^q=H$.
\item[(b)] The assumption $\gcd(o(g),[G:H])=1$ means that $o(g)q+[G:H]l=1$ for some $q,l\in\mathbb{Z}$. Then $gH=(gH)^{o(g)q+[G:H]l}=((gH)^{o(g)})^q((gH)^{[G:H]})^l=H^qH^l=H$, i.e. $g\in H$.
\end{enumerate}

\end{proof}

\paragraph{Ex. 14.1.3}

{\it Let $G$ satisfy (14.2). Use (14.2) and the Third Sylow Theorem to prove that $G$ has a unique p-Sylow subgroup $H$ of order $p$. Then conclude that $H$ is normal in $G$.
}

\begin{proof}
By (14.2),
 $$|G|= |\Gal(L/F)| =pm, \qquad 1\leq m \leq p-1.$$

According the Third Sylow Theorem the number $N$ of p-Sylow subgroups of G satisfies
 $$N\equiv 1 \pmod p,  \qquad N\mid |G|,$$ 
so that $N = 1 + kp,\ k\geq 0$, thus $N \wedge p = 1$, and $N \mid pm$, therefore $N \mid m$. If $k\ne 0$, then $N>p$, but $N \mid m>0$, which implies $N \leq m <p$. This contradiction shows that $k = 0$, and $N = 1$, i.e. there is exactly one $p$-Sylow subgroup $H$ of $G$.

For all $g \in G$, $gHg^{-1}$ is also a $p$-Sylow subgroup of $G$, hence $gHg^{-1}=H$ for all
$g \in G$: $H$ is normal in $G$.

\end{proof}

\paragraph{Ex. 14.1.4}

{\it The definition of Frobenius group given in the Mathematical Notes involves a group $G$ acting transitively on a set $X$. Prove that a group $G$ is a Frobenius group if and only if $G$ has a subgroup $H$ such that $1<|H|<|G|$ and $H\cap gHg^{-1}=\{e\}$ for all $g \notin H$.
}

\begin{proof}

$(\Rightarrow)$ Assume that $G$  is a Frobenius group. Then $G$ acts transitively on a set $X$ such that $1< |X| < |G|$, and for every $(x,y) \in X \times X$ such that $x \ne y$, the identity is the only element of $G$ fixing $x$ and $y$.

First we show that every isotropy group $G_x$ is non trivial, i.e. $G_x \ne \{e\}$ and $G_x \ne G$, for all $x \in G$.

Since $G$ acts transitively on $X$, $X = G\cdot x$ is the orbit of $x$, thus
$$ |X| = |G\cdot x | = (G:G_x) = |G| / |G_x|,$$
and since $1 < |X| < |G|$, this proves $1 < |G_x| < |G|$, so $ G_x \ne \{e\}, G_x \ne G$.
Fix $x_0 \in G, x_0 \ne e$, and take $H = G_{x_0}$ the isotropy group of this chosen element $x_0$. Then $1<|H|<G$.

Assume that $g \in G, g \not \in H$, and $h \in H \cap gHg^{-1}$. Then $h$ and $g^{-1}hg$ are both in $H = G_{x_0}$, so that $h \cdot x_0 = x_0$, and 
$(g^{-1}h g) \cdot x_0 = x_0$,
that is
$$
\left\{
\begin{array}{lll}
h \cdot x_0 &= &x_0,\\
h \cdot (g\cdot x_0) &= & (g\cdot x_0).
\end{array}
\right.
$$
Since $g \not \in H = G_{x_0}$, $x_0 \ne g\cdot x_0$, thus $h$ fixes two distinct elements of $X$, and this shows that $h =e$. We have proved $H \cap gHg^{-1} = \{e\}$ for all $g\not \in H$.

\bigskip

$(\Leftarrow)$ Conversely, assume that $G$ has a subgroup $H$ such that $1<|H|<|G|$ and $H\cap gHg^{-1}=\{e\}$ for all $g \notin H$.

Take $X$ as the set of left cosets $hH, \ h\in G$ relative to $H$, and consider the action of $G$ on $X$ defined for all $h \in G$ by
$$g \cdot hH = (gh)H.$$
\be
\item[$\bullet$] This action is transitive: if $kH$ and $lH$ are left cosets, then $(lk)^{-1}\cdot kH = lH$.

\item[$\bullet$] Since $1 < |H| < |G|$, then $1 < |G|/|H| < |G|$, thus $1<|X| < |G|$.

\item[$\bullet$] Assume that $g$ fixes two distinct left cosets $hH \ne kH$:
\begin{align*}
g\cdot hH &= hH,\\
g\cdot kH &= k H.
\end{align*}
Then $l = h^{-1} g h \in H, m = k^{-1} g k \in H$, therefore $m = k^{-1}g k = k^{-1} h l h^{-1} k \in H$, so that
$$l \in H,\qquad(h^{-1}k)^{-1} l (h^{-1}k) \in H.$$
This proves $l \in H \cap gHg^{-1}$, where $g = h^{-1}k\not \in H$ (since $hH \ne k H$), and the hypothesis $H\cap gHg^{-1}=\{e\}$ gives $l = e$, and $g = hlh^{-1} =e$. The identity is the only element of $G$ fixing $hH$ and $kH$.
\ee
Therefore $G$ is a Frobenius group.
\end{proof}

\paragraph{Ex. 14.1.5}

{\it Let $F$ be a subfield of the real numbers, and let $f\in F[x]$ be irreducible of prime degree $p>2$. Assume that $f$ is solvable by radicals. Prove that $f$ has either a single real root or $p$ real roots. 
}

\begin{proof}
Since $\deg(f) = p$ is odd, $f$ has at least a real root. Suppose that $f$ has two distinct real roots $\alpha, \beta$.  By Theorem 14.1.1, since $f$ is solvable by radicals, the splitting field of $f$ over $F$ is $F(\alpha, \beta) \subset \R$. In this case all roots of $f$ are real, and these roots are distinct, since the characteristic of $F$ is $0$, thus the irreducible polynomial $f$ is separable.

We have proved that $f$ has either a single real root or $p$ real roots. 

\end{proof}

\paragraph{Ex. 14.1.6}

{\it By Example 8.5.5, $f=x^5-6x+3$ is not solvable by radicals over $\mathbb{Q}$. Give a new proof of this fact using the previous exercise together with the irreducibility of f and part (b) of Exercise 6 from Section 6.4.
}

\begin{proof}
The given polynomial $f$ has prime degree 5 and only three real roots, according to part (b) of Exercise 6.4.6. Since $f$ has more than one but less than 5 real roots, it is not solvable by radicals by Exercise 14.1.5. 

\end{proof}

\paragraph{Ex. 14.1.7}

{\it Use Lemma 14.1.3 and part (a) of Lemma 14.1.2 to give a proof of part (b) of Lemma 14.1.2 that doesn't use the Sylow Theorems.
}

\begin{proof}
Assume that $\tau\in S_p$ satisfies $\tau\theta\tau^{-1}\in \mathrm{AGL}(1,\mathbb{F}_p)$. Then, since $\langle\theta\rangle$ is a group of order $p$, $\langle\tau\theta\tau^{-1}\rangle=\tau\langle\theta\rangle\tau^{-1}$ is a subgroup of $\mathrm{AGL}(1,\mathbb{F}_p)$ of order $p$ and each element of this subgroup has order $p$ (or $1$).

By part (a) of Lemma 14.1.2, $\mathrm{AGL}(1,\mathbb{F}_p)$ is the normalizer of $\langle\theta\rangle$ in $S_p$, therefore $\langle\theta\rangle$ is normal in $\mathrm{AGL}(1,\mathbb{F}_p)$, with $[\mathrm{AGL}(1,\mathbb{F}_p):\langle\theta\rangle]=p-1$. The order of each element of $\tau\langle\theta\rangle\tau^{-1}$ is relatively prime to $p-1$, then, by Lemma 14.1.3, $\tau\langle\theta\rangle\tau^{-1}\subset\langle\theta\rangle$, therefore $\tau\langle\theta\rangle\tau^{-1}=\langle\theta\rangle$, since both groups have the same order $p$.  

Thus $\tau$ normalizes $\langle\theta\rangle$, hence $\tau\in \mathrm{AGL}(1,\mathbb{F}_p)$.
\end{proof}
\paragraph{Ex. 14.1.8}

{\it Let $f\in F[x]$ be irreducible of prime degree $p\geq 5$, where $F$ has characteristic 0, and let $\alpha\ne\beta$ be roots of f in some splitting field. If $F(\alpha,\beta)$ contains all other roots of $f$, then $f$ is solvable by radicals by Theorem 14.1.1. But suppose that there is some third root $\gamma$ such that $\gamma\in F(\alpha,\beta)$. Is this enough to force $f$ to be solvable by radicals?
\begin{enumerate}
\item[(a)] Use the classification of transitive subgroups of $S_5$ from Section 13.2 to show that the answer is ``yes'' when p=5.
\item[(b)] Use the polynomial $x^7-154\,x+99$ from Example 13.3.10 to show that the answer is ``no'' when p=7.
\end{enumerate}
}
\begin{proof}
\begin{enumerate}
\item[(a)]   
By hypothesis, $\deg(f) = p = 5$, and $\alpha \ne \beta$ are roots of $f$ in some splitting field.

Since $\alpha$ is a root of $f$, which is irreducible over $F$,
$$[F(\alpha) : F] = \deg(f) = p = 5.$$
Then $\beta$ is a root of $\frac{f(x)}{x-\alpha} \in F(\alpha)[x]$, so that the minimal polynomial of $\beta$ over $F(\alpha)$ has degree $d\leq p-1$. Thus
$$[F(\alpha,\beta) : F(\alpha] \leq p-1 = 4.$$
By the Tower Theorem,
$$[F(\alpha,\beta):F]  =   [F(\alpha,\beta):F(\alpha)] \ [F(\alpha):F] \leq p(p-1) = 20.$$
Now, suppose that there is some third root $\gamma$ such that $\gamma\in F(\alpha,\beta)$. Then $F(\alpha,\beta,\gamma) = F(\alpha,\beta)$. Let $\delta, \varepsilon$ be the remaining roots of $f$. Since the characteristic is $0$, the irreducible polynomial $f$ is separable.Then $\delta$ is a root of $\frac{f(x)}{(x-\alpha)(x-\beta)(x-\gamma)} \in F(\alpha,\beta,\gamma)[x]$, so that
$$[F(\alpha,\beta,\gamma,\delta):F(\alpha,\beta,\gamma)] \leq 2.$$
Since $F(\alpha,\beta,\gamma) = F(\alpha,\beta)$, the tower theorem gives
$$[F(\alpha,\beta,\gamma, \delta):F] \leq 40.$$
Moreover $\alpha + \beta +\gamma + \delta + \varepsilon = \sigma_1(\alpha,\beta,\gamma, \delta,\varepsilon) \in F$, thus $F(\alpha,\beta,\gamma, \delta,\varepsilon) = F(\alpha,\beta,\gamma, \delta)$. Write $L = F(\alpha,\beta,\gamma, \delta,\varepsilon)$ the splitting field of $f$ over $F$. We have proved
$$[L:F]\leq 40.$$
The classification of transitive subgroups of $S_5$ from Section 13.2 shows that any transitive subgroup of $S_5$ with cardinality $\le 40$ is a subgroup of $\mathrm{AGL}(1,\F_5)$, thus is solvable. So $\Gal(L/F)$ is a solvable group, where $F$ has characteristic $0$, therefore $f$ is solvable (Theorem 8.5.3).

To conclude, the answer is ``yes''  when $p = \deg(f) = 5$.
\item[(b)] To prove that the answer is ``no''  when $p = \deg(f) = 7$, we use the counterexample $f = x^7-154\, x+99$ from Example 13.3.10.

The polynomial $f$ is not solvable, since its Galois group is $\mathrm{GL}(3,\F_2)$, which is simple (Section 14.3) and not commutative, thus non solvable.

We prove that there are roots $\alpha,\beta, \gamma$ of $f$ such that $\gamma \in F(\alpha,\beta)$.

As in Example 13.3.10, consider the resolvant
$$\Theta_f(y) = \prod_{1\leq i < j < k \leq 7} \left(y - (\alpha_i + \alpha_j + \alpha_k)\right) \in \Q[y].$$
Then the factorization of $\Theta_f(y)$ over $\Q$ is 
$$\Theta_f(y) = g(y) h(y),$$
where the polynomials $g,h$, given in Example 13.3.10, are irreducible factors of degrees $7$ and $28$.

Take three roots $\alpha,\beta,\gamma$ of $f$ such that $y-(\alpha+\beta+ \gamma)$ is any linear factor of $g$, so that the minimal polynomial of $\alpha+\beta+ \gamma$ is $g$, with $\deg(g) = 7$, thus
$$[\Q(\alpha+\beta+\gamma) : \Q] = 7.$$
Now we prove that $\gamma \in F(\alpha,\beta)$. Consider the chain of extensions
$$\Q\subset \Q(\alpha) \subset \Q(\alpha,\beta) \subset \Q(\alpha,\beta,\gamma) \subset L,$$
where $L$ is the splitting field of $f$ over $\Q$.

The minimal polynomial of $\alpha$ over $\Q$ is $f$, thus $[\Q(\alpha) : \Q] = 7$, and 
$$[L:\Q] = |\Gal(L/\Q)| = |\mathrm{GL}(3,\F_2)| = 168 = 2^3 \times 3 \times 7.$$
By the Tower Theorem,
$$[L : \Q(\alpha)] = \frac{ [L : \Q]}{[ \Q(\alpha) : \Q]} = 2^3 \times 3$$
is not divisible by 7.

Since $\gamma$ is a root of $f$, the minimal polynomial of $\gamma$ over $f$ divides $f$. Thus
$$[\Q(\alpha,\beta,\gamma) : \Q(\alpha,\beta)] = 1\text{ or } 7.$$
If $[\Q(\alpha,\beta,\gamma) : \Q(\alpha,\beta)]  = 7$, by the Tower Theorem, $7$ divides $[L : \Q(\alpha)] = 2^3\times 3$. This contradiction proves that 
$$[\Q(\alpha,\beta,\gamma) : \Q(\alpha,\beta)] = 1,$$
therefore $\gamma \in \Q(\alpha,\beta).$

In this example, there exist roots $\alpha \ne \beta$ of $f$, and some third root $\gamma$ such that $\gamma \in F(\alpha,\beta)$, but $f$ is not solvable.

This shows that the answer is ``no'' when $p =\deg(f) = 7$.
\end{enumerate}

\end{proof}
\qquad

Note: In the proof of the Proposition 13.3.9, we saw that $G_f$ must be conjugate to $\mathrm{GL}(3,\F_2)$. This means that there is some numbering of the roots
$$
\left\{
\begin{array}{ccc}
\F_2^3\setminus\{(0,0,0\} & \to &\{\alpha \in L \, |\, f(\alpha) = 0\}\\
(\nu_1,\nu_2,\nu_3) &\to &\alpha_{\nu_1,\nu_2,\nu_3}
\end{array}
\right.
$$
which verify that, for all $\sigma \in\Gal(L/F)$, there is some $g \in \mathrm{GL}(3,\F_2)$ such that
$$\sigma(\alpha_{\nu_1,\nu_2,\nu_3}) = \alpha_{g \cdot (\nu_1,\nu_2,\nu_3)}.$$

In this correspondance, the roots of $f$ are seen as nonzero vectors in $\F_2^3$, and the seven roots of $g$ correspond to the seven (unordered)  triples of  linearly dependent  nonzero vectors in $\F_2^3$. So the roots $\alpha,\beta, \gamma$ where chosen in the preceding proof such that the corresponding vectors $u,v,w$ verify $w= u+v$ (but not $\gamma = \alpha + \beta$) .

This is what we understand in the hint of D.A. Cox ``Regard the roots as the nonzero vectors of $\F_2^3$ and pick roots $\alpha,\beta, \gamma$ such that $\gamma = \alpha+\beta$''.

This last equality is not true in $L$, but true for the corresponding vectors in $\F_2^3$.

Moreover, let $\alpha\ne \beta$ be {\it any} pair of roots. The corresponding vectors $u,v$ are such that $u,v,u+v = -u-v$ is not a base, so that the root $\gamma$ corresponding to $u+v$ is such that $ y - (\alpha+\beta+\gamma)$ is a factor of $g$, and the preceding proof shows that $\gamma \in \Q(\alpha,\beta)$. For each pair $\alpha\ne \beta$ of roots of $f = x^7-154\, x+99$, there exists a third root $\gamma \not \in \{\alpha,\beta\}$ such that $\gamma \in F(\alpha,\beta)$.







\subsection{IMPRIMITIVE POLYNOMIALS OF PRIME-SQUARED DEGREE}
\paragraph{Ex. 14.2.1}

{\it Prove (14.7).
}

\begin{proof}
Given $\sigma'=(\tau';\mu_1',...,\mu_k'),\sigma=(\tau;\mu_1,...,\mu_k)\in A\wr B$. Since $\sigma'$ maps $R_i$ to $R_{\tau'(i)}$ via $\mu_i'$, if we set $j=\tau'(i)$, then $\sigma$ maps $R_j$ to $R_{\tau(j)}=R_{\tau(\tau'(i))}=R_{\tau\tau'(i)}$ via $\mu_j=\mu_{\tau'(i)}$.

Hence $\sigma\sigma'$ maps $R_i$ to $R_{\tau\tau'(i)}$ via $\mu_{\tau'(i)}\mu_i'$.

More explicitly, by the definition of $(\tau; \mu_1,\ldots,\mu_k)$, for all $(i,j) \in \{1,\ldots,k\} \times \{1,\ldots,l\}$,
$$(\tau;\mu_1,\ldots,\mu_k)(i,j) = (\tau(i), \mu_i(j)).$$
Applying three times this definition, we obtain
\begin{align*}
(\tau; \mu_1,\ldots,\mu_k)(\tau'; \mu'_1,\ldots,\mu'_k)&= (\tau; \mu_1,\ldots,\mu_k)(\tau'(i),\mu'_i(j))\\
&= (\tau( \tau'(i)), \mu_{\tau'(i)}(\mu'_i(j))\\
&= ((\tau \tau')(i), (\mu_{\tau'(i)} \mu'_i)(j)\\
&=(\tau\tau';\mu_{\tau'(1)}\mu_1',...,\mu_{\tau'(k)}\mu_k')(i,j)
\end{align*}
Since this equality is true for all $(i,j) \in \{1,\ldots,k\} \times \{1,\ldots,l\}$,
$$(\tau;\mu_1,...,\mu_k)(\tau';\mu_1',...,\mu_k')=(\tau\tau';\mu_{\tau'(1)}\mu_1',...,\mu_{\tau'(k)}\mu_k').$$
\end{proof}

\paragraph{Ex. 14.2.2}
{\it The wreath product $S_3 \wr S_2 \subset S_6$ can be thought of as the subgroup of all permutations that preserve the blocs $R_1 = \{1,2\}, R_2 = \{3,4\},R_3 = \{5,6\}$. As noted in Example 14.2.11, $S_3 \wr S_2$ has order $6\cdot3^3 = 48$.
\be
\item[(a)] Show that $(S_3 \wr S_2) \cap A_6$ has order $24$.
\item[(b)] Show that $S_3 \wr S_2$ is the centralizer of $(1\,2)(3\,4)(5\,6)$ in $S_6$ (meaning that $S_3 \wr S_2$ consists of all permutations in $S_6$ that commute with $(1\,2)(3\,4)(5\,6)$).
\item[(c)] Use part (b) to show that $S_3 \wr S_2$ is isomorphic to $((S_3 \wr S_2) \cap A_6) \times S_2$.
\ee
See the next exercise for more on $S_3 \wr S_2$ and $(S_3 \wr S_2) \cap A_6$.
}
\begin{proof}
\item[(a)] Let $\varphi$ the restriction of the sign $\mathrm{sgn}$ to $(S_3 \wr S_2) \cap A_6$:
$$
\varphi \left\{
\begin{array}{ccc}
S_3 \wr S_2 & \to & \{-1,1\}\\
\sigma & \mapsto & \mathrm{sgn}(\sigma)
\end{array}
\right.
$$
Since $\mathrm{sgn}$ is a morphism, its restriction $\varphi$ is also a morphism, and $\varphi$ is surjective (onto), because $\varphi(e) = 1$, and $\varphi((1\,2)) = -1$, where $(1\,2) \in S_3\wr S_2$. Moreover the kernel of $\varphi$ is $\ker(\varphi) =  (S_3 \wr S_2) \cap A_6$.

Therefore $\mathrm{im}(\varphi) =  \{-1,1\} \simeq (S_3 \wr S_2)/ ((S_3 \wr S_2) \cap A_6)$. This shows that $$|(S_3 \wr S_2) \cap A_6| = \frac{1}{2} |S_3 \wr S_2| = 24.$$

\item[(b)] Let $\tau \in S_n$. Then $\tau$ is in the centralizer of $\sigma = (1\,2)(3\,4)(5\,6)$ if and only if
$$\tau (1\,2)(3\,4)(5\,6) \tau^{-1} = (1\,2)(3\,4)(5\,6),$$
which is equivalent to
$$(\tau(1)\, \tau(2))(\tau(3)\, \tau(4))(\tau(5)\, \tau(6)) = (1\,2)(3\,4)(5\,6).$$
Write $R_1 = \{1,2\}, R_2 = \{3,4\},R_3 = \{4,5\}$. Then $R_1, R_2,R_3$ are the three orbits  of $\sigma$ acting on $\{1,\ldots,6\}$, the supports of the decomposition of $\sigma$ in disjoint cycles.

Since $\tau$ is a bijection, the 6 values $\tau(1),\tau(2),\tau(3), \tau(4),\tau(5), \tau(6)$ are distinct, so $(\tau(1)\, \tau(2)),(\tau(3)\, \tau(4)),(\tau(5)\, \tau(6))$ are disjoint $2$-cycles. 

If $\tau$ is the centralizer of $\sigma$, the equality $(\tau(1)\, \tau(2))(\tau(3)\, \tau(4))(\tau(5)\, \tau(6)) = (1\,2)(3\,4)(5\,6)$ shows that $\tau(R_1),\tau(R_2),\tau(R_3)$ are also the three orbits of $\sigma$, so that
$$\{\{1,2\},\{3,4\},\{5,6\}\} =\{\{ \tau(1), \tau(2)\},\{\tau(3), \tau(4)\},\{\tau(5), \tau(6)\}\},$$
that is
$$\{R_1,R_2,R_3\} = \{\tau(R_1),\tau(R_2),\tau(R_3)\},$$
which means that there is some permutation $\tau'$ of $\{1,2,3\}$ such that $\tau(R_i) = R_{\tau'(i)}, \ i=1,2,3$. In other words, $\sigma$ preserves the blocks $R_1,R_2,R_3$, so that $\sigma \in S_3 \wr S_2$.

To prove the converse, it is more convenient to use the other usual representation of $S_3\wr S_2$. Then $\sigma = (e; \mu,\mu,\mu)$, where $\mu = (1\, 2) \in S_2$. Let $\tau = (\lambda; \mu_1,\mu_2,\mu_3)$ be any element of $S_3 \wr S_2$ (then $\mu_i = ()$ or $\mu_i = \mu)$. Then (14.7) gives
\begin{align*}
\tau \sigma &= (\lambda; \mu_1,\mu_2,\mu_3)(e; \mu,\mu,\mu)\\
&=(\lambda;\mu_1\mu, \mu_2 \mu,\mu_3\mu)\\
\sigma \tau &= (e;\mu,\mu;\mu)(\lambda,\mu_2,\mu_2,\mu_3)\\
&= (\lambda; \mu \mu_1, \mu \mu_2,\mu \mu_3)
\end{align*}
Since $S_2 = \{e,\mu\}$ is commutative, $\mu \mu_i = \mu_i \mu,\ i=1,2,3$, thus $\tau \sigma = \sigma \tau$.

The centralizer of $(1\,2)(3\,4)(5\,6)$ in $S_n$ is $S_3 \wr S_2$.

\item[(c)] Since the order of $\sigma = (1\,2)(3\,4)(5\,6)$ is $2$, $\langle \sigma \rangle = \{e,\sigma\} \simeq S_2$ and we can write $\sigma^{\varepsilon }, \varepsilon \in \{0,1\}$ the two elements of $\langle \sigma \rangle$.
Let
$$
\varphi
\left \{
\begin{array}{ccc}
(S_3 \wr S_2) \cap A_6 \times \langle \sigma \rangle & \to & S_3\wr S_2\\
(\tau, \sigma^{\varepsilon}) & \mapsto &\tau \sigma^{\varepsilon}.
\end{array}
\right.
$$
$\bullet$ $\varphi$ is a morphism: For all $\tau, \tau' \in (S_3 \wr S_2) \cap A_6$ and $\sigma^{\varepsilon}, \sigma^{\varepsilon'} \in \langle \sigma \rangle $, $\sigma \tau' = \tau' \sigma$ by part (b), thus
\begin{align*}
\varphi(\tau \sigma^{\varepsilon}) \varphi(\tau' \sigma^{\varepsilon'})&= \tau \sigma^{\varepsilon} \tau' \sigma^{\varepsilon'}\\
&= \tau \tau' \sigma^\varepsilon \sigma^{\varepsilon'}\\
&= \varphi((\tau,\sigma^\varepsilon) (\tau', \sigma^{\varepsilon'}))
\end{align*}

$\bullet$ $\ker \varphi$ is trivial: if $\varphi(\tau, \sigma^{\varepsilon}) = e$, then $\tau \sigma^{\varepsilon} = e$, so that $\tau  = \sigma^{-\varepsilon} \in \{e, \sigma\}$. $\tau = \sigma$ is impossible, since $\tau$ is an even permutation, and $\sigma$ is odd. Therefore $\tau = e$, and $\sigma^{\varepsilon} = e$. Thus $\varphi$ is injective (one to one).

$\bullet$ Since $|((S_3 \wr S_2) \cap A_6) \times \langle \sigma \rangle | = |S_3\wr S_2|$, $\varphi$ is a bijection, thus $\varphi$ is a group isomorphism.

$$S_3 \wr S_2 \simeq ((S_3 \wr S_2) \cap A_6) \times \langle \sigma \rangle  \simeq  ((S_3 \wr S_2) \cap A_6) \times S_2 .$$
\end{proof}

\paragraph{Ex. 14.2.3}
{\it One of the challenges of group theory is that the same group can have radically differnet descriptions. For instance, $S_4$ and the group $G = (S_3 \wr S_2) \cap A_6$ appearing in Example 14.2.11 both have order $24$. In this exercise, you will prove that they are isomorphic. We will use the notation of Exercise 2.
\be
\item[(a)] There is a natural homomorphism $G \to S_3$ given by how elements of $G$ permute the blocks $R_1,R_2,R_3$. Show that this map is onto, and express the elements of the kernel as products of disjoints cycles.
\item[(b)] Use the Sylow Theorems to show that $G$ has one or four $3$-Sylow subgroups.
\item[(c)] Show that $A_6$ has no element of order $6$.
\item[(d)] Use part (c) and the kernel of the map $G \to S_3$ from part (a) to show that $G$ has four $3$-Sylow subgroups.
\item[(e)] $G$ acts by conjugation on its four $3$-Sylow subgroups. Use this to prove that $G \simeq S_4$.
\item[(f)] Using Exercise 2, conclude that $S_3 \wr S_2 \simeq S_4 \times S_2$.

We note without proof that $S_3 \wr S_2 \simeq S_4 \times S_2$ is also isomorphic to the full symmetry group (rotations and reflexions) of the octahedron.
\ee
}

\begin{proof}
\item[(a)] Let $\varphi : G \to S_3$ defined by $\tau = \varphi(\sigma)$ iff $\sigma(R_i) = R_{\tau(i)}$. In other notations, this is the restriction to $G$ of the homomorphism of part (b) of Lemma 14.2.8, thus $\varphi$ is an homomorphism.

$\bullet$ $\varphi$ is surjective: Let $\tau$ be any permutation in $S_3$. 

If $\tau$ is even, $\tau = (1\,2\,3)^k,\ k = 0,1,2$. Let
$$\sigma =
\left(
\begin{array}{cccccccc}
1 & 2 & & 3 & 4 & & 5 & 6\\
3 & 4 & & 5 & 6 & & 1& 2\\
\end{array}
\right)
= (1\,3\,5)(2\,4\,6).
$$
$\sigma$ preserves the block structure defined by $R_1,R_2,R_3$, and $\sigma \in A_6$, so that $\sigma \in G = (S_3 \wr S_2) \cap A_6$. Moreover $\sigma(R_1) = R_2, \sigma(R_2) = R_3, \sigma(R_3) = R_1$, thus $\varphi(\sigma) = (1\,2\,3)$, and $\varphi(\sigma^k )= (1\,2\,3)^k = \tau$.

If $\tau$ is odd, then $\tau \in \{(1\,2),(2\,3),(1\,3)\}$, and
\begin{align*}
(1\,2) = \varphi(\sigma_1), \qquad \sigma_1 = 
\left(
\begin{array}{cccccccc}
1 & 2 & & 3 & 4 & & 5 & 6\\
3 & 4 & & 1 & 2 & & 5& 6\\
\end{array}
\right) = (1\,3)(2\,4) \in G,\\
(2\,3) = \varphi(\sigma_2), \qquad \sigma_2 = 
\left(
\begin{array}{cccccccc}
1 & 2 & & 3 & 4 & & 5 & 6\\
1 & 2 & & 5 & 6 & & 3& 4\\
\end{array}
\right) = (3\,5)(4\,6) \in G,\\
(1\,3) = \varphi(\sigma_3), \qquad \sigma_3 = 
\left(
\begin{array}{cccccccc}
1 & 2 & & 3 & 4 & & 5 & 6\\
5 & 6 & & 3 & 4 & & 1& 2\\
\end{array}
\right) = (1\,5)(2\,6) \in G.\\
\end{align*}
Therefore $\varphi$ is surjective.

$\bullet$ Let $\sigma \in S_6$. Then $\sigma \in \ker \varphi$ iff $\sigma \in A_6$ and $\sigma(R_1) = R_1, \sigma(R_2) = R_2, \sigma(R_3) = R_3$.

Morerover, for all $\sigma \in A_6$,
\begin{align*}
&\sigma(R_1) = R_1, \sigma(R_2) = R_2, \sigma(R_3) = R_3\\
\iff& \{\sigma(1), \sigma(2)\} = \{1,2\}, \{\sigma(3),\sigma(4)\} = \{3,4\}, \{\sigma(5),\sigma(6)\} = \{5,6\}\\
\iff &\sigma \in \{e,(1\,2)(3\,4), (1\,2)(5\,6), (3\,4)(5\,6)\}.
\end{align*}
$$\ker \varphi = \{e,(1\,2)(3\,4), (1\,2)(5\,6), (3\,4)(5\,6)\}.$$
Verification: $6 = |S_3| = |G/\ker(\varphi)| = 24/4$.

\item[(b)] Let $N$ be the number of $3$-Sylow subgroups of $G$. By the third Slow Theorem,
$$N \mid 24 = |G|, \qquad N \equiv 1 \pmod 3.$$
Therefore $N = 1$ or $N = 4$.

\item[(c)] Let $\tau \in S_6$ be a permutation of order 6. If $\tau = \tau_1\cdots \tau_k$ is the decomposition of $\tau$ in disjoint cycles, then the order of $\tau$ is the lcm of the order of$ \tau_1,\ldots,\tau_k$. Therefore $\tau$ is a $6$-cycle or a product of a $2$-cycle by a $3$-cycle. In both cases $\tau$ is odd. Therefore $A_6$ has no element of order $6$.


\item[(d)] Reasoning by contradiction, suppose that $G$ has only one $3$-Sylow subgroup $H$. Then, for all $g \in G$, $gH g^{-1}$ is a $3$-Sylow, thus $gH g^{-1} = H$, and $H$ is a normal subgroup of $G$. 

Moreover $K = \ker \varphi = \{e,(1\,2)(3\,4), (1\,2)(5\,6), (3\,4)(5\,6)\}$ is normal in $G$, and has order 4. Therefore $H \cap K = \{e\}$.

The usual characterization of direct products (see Ex. 14.3.7) shows that, for all $h \in H$, all $k \in K$, $hk = kh$, and $HK$ is a normal subgroup of $G$ isomorphic to $H \times K$. 

Take $h$ an element of order $3$ in $H,$ and $k$ and element of order 2 in $K$. Since $kh = hk$, the order of $hk \in A_6$ is 6, which is impossible by part (c).

Therefore $G$ has exactly four $3$-Sylow subgroups.

\item[(e)] Write $X = \{H_1,H_2,H_3,H_4\}$ the set of $3$-Sylow subgroups of $G$, and $S(X)$ the set of permutations of $X$. Then $S(X) \simeq S_4$, and $g\cdot H = g H g^{-1}$ defines a left action of $G$ on $X$, so that
$$
\psi 
\left\{
\begin{array}{ccc}
G & \to & S(X)\\
g & \mapsto & 
\sigma = \left(
    \begin{array}{cccc}
      H_1&H_2&H_3&H_4\\
      gH_1g^{-1} & gH_2 g^{-1} & gH_3g^{-1} & gH_4g^{-1}
    \end{array}
\right)
\end{array}
\right.
$$
is a group homomorphism.

It is not obvious that $\psi$ is bijective. We prove first that $\psi$ is surjective (onto). We give explicitly the $3$-Sylow subgroups. Let
\begin{align*}
&\lambda_1 = 
\left(
\begin{array}{cccccccc}
1 & 2 & & 3 & 4 & & 5 & 6\\
 3&4&&5&6&&1&2 \\
\end{array}
\right) = (1\,3\,5)(2\,4\,6),\\
&\lambda_2 = 
\left(
\begin{array}{cccccccc}
1 & 2 & & 3 & 4 & & 5 & 6\\
 3&4&&6&5&&2&1 \\
\end{array}
\right) = (1\,3\,6)(4\,5\,2),\\
&\lambda_3 = 
\left(
\begin{array}{cccccccc}
1 & 2 & & 3 & 4 & & 5 & 6\\
 6&5&&2&1&&3&4 \\
\end{array}
\right) = (1\,6\,4)(5\,3\,2),\\
&\lambda_4 =
\left(
\begin{array}{cccccccc}
1 & 2 & & 3 & 4 & & 5 & 6\\
 4&3&&6&5&&1&2 \\
\end{array}
\right) = (1\,4\,5)(3\,6\,2).
\end{align*}
Then $\lambda_1,\ldots,\lambda_4 \in G$ have order $3$, and $H_1 = \langle \lambda_1 \rangle = \{e,\lambda_1,\lambda_1^2\},\ldots,H_4 = \langle \lambda_1 \rangle = \{e,\lambda_4,\lambda_4^2\}$ are distinct, thus they are the four $3$-Sylow of $G$.

Now take 
\begin{align*}
&g = 
\left(
\begin{array}{cccccccc}
1 & 2 & & 3 & 4 & & 5 & 6\\
 4&3&&2&1&&5&6 \\
\end{array}
\right) = (1\,4)(2\,3)\\
&h=
\left(
\begin{array}{cccccccc}
1 & 2 & & 3 & 4 & & 5 & 6\\
 2&1&&5&6&&4&3 \\
\end{array}
\right) = (1\,2)(3\,5\,4\,6)
\end{align*}
(We give a geometrical explanation of this choice in the final note.)

Then 
\begin{align*}
g \lambda_1 g^{-1} &= (1\,4)(2\,3)(1\,3\,5)(2\,4\,6)(1\,4)(2\,3)\\
&=
\left(
\begin{array}{cccccccc}
1 & 2 & & 3 & 4 & & 5 & 6\\
 6&5&&1&2&&4&3 \\
\end{array}
\right) = (1\,6\,3)(2\,5\,4) = \lambda_2^2,
\end{align*}
thus $gH_1g^{-1} = H_2$, and since $g = g^{-1}$, $g H_2 g^{-1} = H_1$.
Moreover
\begin{align*}
g \lambda_3g^{-1} &=(1\,4)(2\,3)(1\,6\,4)(5\,3\,2)(1\,4)(2\,3)\\
&= 
\left(
\begin{array}{cccccccc}
1 & 2 & & 3 & 4 & & 5 & 6\\
 4&3&&5&6&&2&1 \\
  \end{array}
  \right)
 =(1\,4\,6)(2\,3\,5)
 =\lambda_3^2,
\end{align*}
thus $gH_3g^{-1} = H_3$, and since $\psi(g)$ is a permutation, $gH_4 g^{-1} = H_4$.

Therefore $\psi(g) \in S(X)$ is the permutation 
$ 
\left(
\begin{array}{cccc}
H_1&H_2&H_3&H_4\\
H_2&H_1&H_3&H_4
\end{array}
\right)
$, which corresponds to the transposition $(1\,2) \in S_4$.
Similarly,
\begin{align*}
h\lambda_1h^{-1}&= (1\,2)(3\,5\,4\,6)(1\,3\,5)(2\,4\,6)(3\,6\,4\,5)(1\,2)\\
&=
\left(
\begin{array}{cccccccc}
1 & 2 & & 3 & 4 & & 5 & 6\\
 6&5&&1&2&&4&3 \\
  \end{array}
  \right) =(1\,6\,3)(2\,5\,4) = \lambda_2^2,\\
 h\lambda_2h^{-1} &= (1\,2)(3\,5\,4\,6)(1\,3\,6)(4\,5\,2)(3\,6\,4\,5)(1\,2)\\
 &=
 \left(
\begin{array}{cccccccc}
1 & 2 & & 3 & 4 & & 5 & 6\\
 6&5&&2&1&&3&4 \\
  \end{array}
  \right) = (1\,6\,4)(2\,5\,3) = \lambda_3,\\
   h\lambda_3h^{-1} &= (1\,2)(3\,5\,4\,6)(1\,6\,4)(5\,3\,2)(3\,6\,4\,5)(1\,2)\\
 &=
 \left(
\begin{array}{cccccccc}
1 & 2 & & 3 & 4 & & 5 & 6\\
 4&3&&6&5&&1&2 \\
  \end{array}
  \right) = (1\,4\,5)(2\,3\,6) = \lambda_4,
\end{align*}
thus $hH_1h^{-1} = H_2, hH_2h^{-1} = H_3, hH_1h^{-1} = H_4$, and since $\psi(g)$ is a permutation, $h H_4h^{-1} = H_1$. Therefore 
$\psi(g) = \left(
\begin{array}{cccc}
H_1&H_2&H_3&H_4\\
H_2&H_3&H_4&H_1
\end{array}
\right)
$
corresponds to the $4$-cycle $(1\,2 \,3 \,4)$. 

Since $\{(1\,2), (1\,2\,3\,4)\}$ is a set of generators of $S_4$, $S(X)$ is generated by $\psi(g), \psi(h)$, so that $S(X) = \psi(G)$, and $\psi$ is surjective. Moreover, $|G| = |S(X)| = 24$, thus $\psi$ is a bijection, and a group isomorphism:
$$G \simeq S(X) \simeq S_4.$$

\item[(f)] To conclude, using Exercise 2, we obtain
$$S_3\wr S_2 \simeq ((S_3 \wr S_2) \cap A_6) \times S_2  = G \times S_2 \simeq S_4 \times S_2.$$
\bigskip

Note: We have proved in Exercise 7.5.10 that the symmetry group $G_0$ of the cube (or octahedron), is isomorphic to $S_4$. By composition with the indirect isometry $\sigma : v\mapsto -v$, which commutes with all elements in the group, we obtain the full symmetry group, isomorphic to $S_4 \times S_2$.

We have a geometrical description of $G = (S_3\wr S_2) \cap A_6$ by regrouping the opposite faces of a cube in blocs: stick $1$ on a face of a dice, $2$ on the opposite face, and so on (I stuck labels on my Rubik's cube). Then the $24$ rotations of the cube send opposite faces on opposite faces, so that the bloc structure $\{\{1,2\},\{3,4\},\{5,6\}\}$ is preserved by rotations.

We have proved in Exercise 7.5.10 that $G_0$ acts on the 4 long diagonals $D_1,D_2,D_3,D_4$ of the cube, so that $G_0 \simeq S_4$. Each of the four $3$-Sylow of $G_0$ is generated by the rotation with angle $\frac{2\pi}{3}$ around such a  long diagonal. They correspond to the $3$-Sylow $H_1,\ldots,H_4$ of $G$: this was useful for the above description of the $H_i$. Each $3$-Sylow corresponds to a long diagonal, so that $gH_ig^{-1} = H_j$ is equivalent to $\sigma(D_i) = D_j$, where $\sigma$ corresponds to $g$. It remains to find a rotation which acts on these diagonals as some given permutation in $S_4$, such that $(1\,2)$ or $(1\,2\,3\,4)$. The corresponding permutations $g,h \in G $ are given in the text.

\end{proof}

\paragraph{Ex. 14.2.4}
{ \it
Let $A$ and $B$ be solvable permutation groups. Prove that their wreath product $A\wr B$ is also solvable.
}

\medskip

We first proof a lemma, which is not given in Chapter 8.

\medskip

{\bf Lemma.} {\it If $G,H$ are solvable groups, then $G \times H$ is solvable.}

\medskip

{\it Proof of Lemma.} We have subgroups 
\begin{align*}
&\{e\} \subset G_n \subset \cdots \subset G_1 \subset G_0 = G\\
&\{e'\} \subset H_m \subset \cdots \subset H_1 \subset H_0 = H
\end{align*}
such that $G_i$ is normal in $G_{i-1}$ and $G_{i-1}/G_i$ is Abelian for $i=1,\ldots,n$, and $H_i$ is normal in $H_{i-1}$ and $H_{i-1}/H_i$ is Abelian for $i=1,\ldots,m$.

If $n >m$, we can define $H_{m+1} = H_{m+2} =\cdots=H_n=\{e'\}$,  and proceed similarly if $n<m$, so we can assume that $n=m$: 
\begin{align*}
&\{e\} \subset G_n \subset \cdots \subset G_1 \subset G_0 = G\\
&\{e'\} \subset H_n \subset \cdots \subset H_1 \subset H_0 = H
\end{align*}
Then
$$
\{(e,e')\} =G_n\times H_n \subset \cdots \subset G_1\times H_1 \subset G_0\times H_0 = G \times H.
$$
We prove 
$$(G_{i-1}\times H_{i-1}) / (G_i \times H_i) \simeq G_{i-1}/G_i \times H_{i-1}/H_i.$$
Indeed, 
$$
\psi
\left\{
\begin{array}{ccc}
G_{i-1} \times H_{i-1} & \to  & G_{i-1}/G_i \times H_{i-1}/H_i\\
(g,h) & \mapsto & (gG_i, hH_i)
\end{array}
\right.
$$
is surjective, and its kernel is $G_i \times H_i$. This proves our assertion.

Therefore $(G_{i-1}\times H_{i-1}) / (G_i \times H_i)$ is Abelian. Then Exercise 8.1.8 shows that $G\times H $ is solvable.
\qed
\begin{proof} {\it (of Ex.14.2.4.)}
Let 
$$
\varphi
\left\{
\begin{array}{ccc}
A \wr B & \to & A\\
(\tau; \mu_1,\ldots,\mu_k) & \mapsto &\tau.
\end{array}
\right.
$$
By Lemma 14.2.8, $\varphi$ is onto, and its kernel $H = \ker(\varphi)$ is isomorphic to $B^k$. Then $B^k$ is solvable by induction with the above Lemma, so that $H$ is solvable, and $(A \wr B) / H  = (A \wr B)/\ker(\varphi) \simeq A$ is solvable. By Theorem 8.1.4, $A\wr B$ is solvable.
\end{proof}

\paragraph{Ex. 14.2.5}
{\it This exercise will complete the proof of Theorem 14.2.15.
\be
\item[(a)] Let $G_i \to S_p$ be the map defined in (14.9). Prove that it is a group homomorphism and that its image $G'_i \subset S_p$ is transitive and solvable.
\item[(b)] Let $\sigma = (\tau; \mu_1,\ldots,\mu_p)$ and $(\rho;\nu_1,\ldots,\nu_p)$ be as in the proof of Theorem 14.2.15. Thus we have a fixed $j$ such that $i=\tau(j), \nu_i = \theta$, and $\rho(i) = i$. Now let $\gamma = (\tau^{-1} \rho \tau; \lambda_1,\ldots,\lambda_p)$ be as in (14.11). Prove carefully that $\lambda_j = \mu_j^{-1} \theta \mu_j$.
\ee
}

\begin{proof}
\item[(a)] The map $\varphi_i$ defined in  (14.9) is
$$
\varphi_i 
\left\{
\begin{array}{ccc}
G_i & \to &S_p\\
(\tau; \mu_1,\ldots,\mu_p) & \mapsto &\mu_i.
\end{array}
\right.
$$
Let $\lambda = (\tau; \mu_1,\ldots,\mu_p),\lambda' =  (\tau'; \mu_1,\ldots,\mu_p)$ be elements of $G_i$. The definition of $G_i$ shows that $\lambda(R_i) = \lambda'(R_i) = R_i$, so that $\tau(i) =\tau'(i)=i$.  

By (14.7) (see Exercise 1),
$$\lambda \lambda' =(\tau;\mu_1,...,\mu_k)(\tau';\mu_1',...,\mu_k')=(\tau\tau';\mu_{\tau'(1)}\mu_1',...,\mu_{\tau'(k)}\mu_k'),$$ 
therefore, using $\tau'(i) = i$,
\begin{align*}
\varphi_i( \lambda \lambda') &= \mu_{\tau'(i)} \mu'_i\\
&=\mu_i \mu'_i\\
&= \varphi_i(\lambda) \varphi_i(\lambda'),
\end{align*}
thus $\varphi_i$ is a group homomorphism.

\bigskip

Write $G'_i = \varphi_i(G_i) \subset S_p$. We prove first that $G'_i$ is transitive.

Take any $k$ and $l$ in $\{1,\ldots,p\}$. Since $G$ is transitive, there exists some $\lambda = (\tau;\mu_1,...,\mu_k) \in G$ which sends $(i,j)$ on $(i,k)$:
$$(\tau;\mu_1,\ldots,\mu_k)(i,j) = (\tau(i), \mu_i(j)) = (i,k).$$
Then $\tau(i) = i$, so that $\lambda \in G_i$ and $\mu_i = \varphi_i(\lambda) \in G'_i$. Moreover $\mu_i(j) = k$. This proves that $G'_i$ is a transitive subgroup of $S_p$.

Moreover, $G_i$ is a subgroup of the solvable group $G$, thus $G_i$ is solvable. Then $G'_i = \varphi_i(G_i)$ is isomorphic to $G_i/\ker(\varphi_i)$, which is a quotient of a solvable group, thus $G'_i$ is solvable.
\item[(b)] As in the proof of Theorem 14.2.15, let $\sigma = (\tau; \mu_1,\ldots,\mu_p) \in G$ be arbitrary, and fix $j$ between $1$ and $p$. By (14.10) with  $i = \tau(j)$, $\theta \in G'_i = \varphi_i(G_i)$, thus there exists $\lambda = (\rho; \nu_1,\ldots,\nu_p) \in G_i$ such that $\theta = \varphi_i(\lambda)$, thus $\theta = \nu_i$ and $\rho(i) = i$.

Now consider the element $\gamma = \sigma^{-1} \lambda \sigma \in G$. Using (14.6) and (14.7), we obtain
\begin{align*}
\gamma &=  (\tau; \mu_1,\ldots,\mu_p)^{-1} (\rho; \nu_1,\ldots,\nu_p)(\tau; \mu_1,\ldots,\mu_p)\\
&=(\tau^{-1};\mu^{-1}_{\tau^{-1}(1)},\ldots,\mu^{-1}_{\tau^{-1}(p) })(\rho \tau; \nu_{\tau(1)}\mu_1,\ldots,\nu_{\tau(p)} \mu_p)\\
&=(\tau^{-1};\xi_1,\ldots,\xi_p)(\rho \tau; \nu_{\tau(1)}\mu_1,\ldots,\nu_{\tau(p)} \mu_p)\qquad (\text{where }\xi_1 =\mu^{-1}_{\tau^{-1}(1)},\ldots, \xi_p =\mu^{-1}_{\tau^{-1}(p)}) \\
&= (\tau^{-1} \rho \tau; \xi_{(\rho \tau)(1)}\nu_{\tau(1)}\mu_1,\ldots,\xi_{(\rho \tau)(p)}\nu_{\tau(p)}\mu_p)\\
&= (\tau^{-1} \rho \tau; \mu^{-1}_{\tau^{-1}((\rho \tau)(1))}\nu_{\tau(1)}\mu_1,\ldots,\mu^{-1}_{\tau^{-1}((\rho \tau)(p))}\nu_{\tau(p)}\mu_p)\\
&=(\tau^{-1} \rho \tau; \mu^{-1}_{(\tau^{-1}\rho \tau)(1)}\nu_{\tau(1)}\mu_1,\ldots,\mu^{-1}_{(\tau^{-1}\rho \tau)(p)}\nu_{\tau(p)}\mu_p)\\
\end{align*}
If we write $\gamma = (\tau^{-1} \rho \tau; \lambda_1,\ldots,\lambda_p)$, we obtain
$$\lambda_k = \mu^{-1}_{(\tau^{-1}\rho \tau)(k)}\nu_{\tau(k)}\mu_k,\qquad k = 1,\ldots,p,$$
and at the index $j$, using $\theta = \nu_i = \nu_{\tau(j)}$,
\begin{align*}
\lambda_j &= \mu^{-1}_{(\tau^{-1}\rho \tau)(j)}\nu_{\tau(j)}\mu_j\\
&=\mu^{-1}_{(\tau^{-1}\rho \tau)(j)}\theta\mu_j.\\
\end{align*}
Since $i = \tau(j)$ and $\rho(i) = i$, 
$$(\tau^{-1} \rho\tau)(j) = (\tau^{-1} \rho)(i) = \tau^{-1} (i) = j,$$
thus
$$\lambda_j = \mu_j^{-1} \theta \mu_j.$$
\end{proof}



\paragraph{Ex. 14.2.6}
{\it Let $A$ be a subgroup of $S_n$, and let $G$ be any group. Then define $A \wr G$ as in the Mathematical Notes.
\be
\item[(a)] Prove that $A \wr G$ is a group under the multiplication defined in the Mathematical Notes.
\item[(b)] State and prove a version of part (b) of Lemma 14.2.8 for $A \wr G$.
\item[(c)] Prove that $|A \wr G | = |A| |G|^n$ when $G$ is finite.
\ee
}

\begin{proof}
\be
\item[(a)]
Let $G$ be any group and let $A \subset S_n$ be a permutation group. Then set
$$A \wr G = \{ (\tau; g_1,\ldots,g_n)\, | \, \tau \in A,\  g_1,\ldots,g_n \in G\},$$
with an operation on this set defined by
$$(\tau; g_1,\ldots,g_n)(\tau'; g'_1,\ldots,g'_n) = (\tau \tau'; g_{\tau'(1)}g'_1,\ldots,g_{\tau'(n)} g'_n) \in A \wr G.$$
We write $e$ the identity of $G$, and $()$ the identity of $S_n$.

\be
\item[$\bullet$] Let $\lambda = (\tau; g_1,\ldots,g_n), \lambda' = (\tau';g'_1,\ldots,g'_n), \lambda''=(\tau'';g''_1,\ldots,g''_n)$ be elements of $A \wr G$. Then
\begin{align*}
\lambda(\lambda'\lambda'' )&=  (\tau; g_1,\ldots,g_n)(\tau' \tau''; g_{\tau''(1)}g''_1,\ldots,g_{\tau''(n)} g''_n) \\
&= (\tau\tau'\tau''; g_{(\tau'\tau'')(1)} g'_{\tau''(1)} g''_1,\ldots,g_{(\tau'\tau'')(n)} g'_{\tau''(n)} g''_n)\\
(\lambda \lambda')\lambda'' &= (\tau \tau'; g_{\tau'(1)}g'_1,\ldots,g_{\tau'(n)} g'_n)(\tau'';g''_1,\ldots,g''_n)\\
&=(\tau\tau';h_1,\ldots,h_n)(\tau'';g''_1,\ldots,g''_n)\qquad (\text{where } h_k = g_{\tau'(k)}g'_k)\\
&=(\tau\tau'\tau''; h_{\tau''(1)} g''_1,\ldots,h_{\tau''(n)} g''_n)\\
&=(\tau\tau'\tau''; g_{\tau'(\tau''(1))}g'_{\tau''(1)}g''_1,\cdots,g_{\tau'(\tau''(n))}g'_{\tau''(n)}g''_n)\\
&=(\tau\tau'\tau''; g_{(\tau'\tau'')(1)} g'_{\tau''(1)} g''_1,\ldots,g_{(\tau'\tau'')(n)} g'_{\tau''(n)} g''_n)
\end{align*}
thus $\lambda(\lambda'\lambda'' ) = (\lambda \lambda')\lambda'' $, and the law is associative.

\item[$\bullet$] Write $\varepsilon = (();e,\ldots,e) = (\iota; e_1,\ldots, e_n)$, where $\iota = ()$, and $e_k = e, k=1,\ldots,n$. Then
\begin{align*}
\varepsilon \lambda &= (\iota;e_1,\ldots,e_n)(\tau; g_1,\ldots,g_n)\\
&=(\tau; e_{\tau'(1)}g_1,\ldots,e_{\tau'(n)} g_n)\\
&= (\tau; g_1,\ldots,g_n) = \lambda\qquad (\text{since } e_{\tau'(k)} = e)\\
\lambda \varepsilon &= (\tau; g_{\iota(1)} e_1,\ldots,g_{\iota(n)} e_n)\\
&= (\tau; g_1,\ldots,g_n) = \lambda \qquad(\text{since } \iota(k) = k, e_k = e).
\end{align*}
Therefore $\varepsilon =(();e,\ldots,e) $ is the identity of $A \wr G$.

\item[$\bullet$] Set $\mu = (\tau^{-1}; h_1,\ldots,h_n) = (\tau^{-1}; g_{\tau^{-1}(1)}^{-1}, \ldots,g_{\tau^{-1}(n)}^{-1})$, with $h_k = g_{\tau^{-1}(k)}^{-1},\ k=1,\ldots,n$. Then
\begin{align*}
\lambda \mu &= (\tau; g_1,\ldots,g_n)(\tau^{-1}; h_1,\ldots,h_n)\\
&=((); g_{\tau^{-1}(1)} h_1,\ldots,g_{\tau^{-1}(n)} h_n)\\
&=((); g_{\tau^{-1}(1)} g_{\tau^{-1}(1)}^{-1}, \ldots,g_{\tau^{-1}(n)} g_{\tau^{-1}(n)}^{-1})\\
&=((); e,\ldots,e) =\varepsilon\\
\mu \lambda &= (\tau^{-1}; h_1,\ldots,h_n)(\tau; g_1,\ldots,g_n)\\
&=((); h_{\tau(1)} g_1,\ldots,h_{\tau(n)} g_n\\
&=((); g_{\tau^{-1}(\tau(1))}^{-1} g_1,\ldots,g_{\tau^{-1}(\tau(n))}^{-1} g_n)\\
&=((); g_1^{-1}g_1,\ldots, g_n^{-1}g_n) = (();e,\ldots,e) = \varepsilon.
\end{align*}
Therefore every element in $A \wr G$ is invertible.

$A \wr G$ is a group under the multiplication defined in the Mathematical Notes.
\ee
\item[(b)] For the group $A \wr G$ of part (a), where $A \subset S_n$ and $G$ is a group, we show the following lemma:

{\bf Lemma.}
{\it The map
$$
\varphi 
\left\{
\begin{array}{ccc}
A \wr G & \to A\\
(\tau; g_1,\ldots,g_n) & \mapsto \tau
\end{array}
\right.
$$
is a group homomorphism that is surjective and whose kernel is isomorphic to $G^n$.
}

\bigskip

Let $\lambda = (\tau; g_1,\ldots,g_n) , \lambda' = (\tau'; g'_1,\ldots,g'_n)$ be any elements of $A \wr G$. By definition,
$\lambda \lambda' = (\tau \tau'; g_{\tau'(1)}g'_1,\ldots,g_{\tau'(n)} g'_n),$
so that
$$\varphi(\lambda \lambda') = \tau \tau' = \varphi(\lambda) \varphi(\lambda').$$
$\varphi$ is a group homormphism.

If $\tau$ is any element of $A$, then $\varphi(\tau; e,\ldots,e) = \tau$, where $(\tau; e,\ldots,e) \in A \wr G$. Therefore $\varphi$ is surjective.

Moreover $(\tau; g_1,\ldots,g_n) \in \ker \varphi$ if and only if $\tau = ()$, therefore
$$\ker \varphi = \{\iota; g_1,\ldots, g_n)\, |\, (g_1,\ldots,g_n) \in G^n\}, \qquad \text{where } \iota = ().$$

Consider
$$
\psi
\left\{
\begin{array}{ccc}
\ker \varphi & \to & G^n\\
(\iota; g_1,\ldots,g_n) & \mapsto &(g_1,\ldots, g_n)
\end{array}
\right.
$$
Then $\psi$ is bijective (with inverse map $(g_1,\ldots,g_n) \mapsto (\iota,g_1,\ldots,g_n)$). We verify that $\psi$ is a group homomorphism: if $\lambda = (\iota; g_1,\ldots,g_n) , \lambda' = (\iota; g'_1,\ldots,g'_n)$ are elements of $\ker \varphi$, then
\begin{align*}
\psi (\lambda \lambda') &= \psi((\iota; g_1,\ldots,g_n)(\iota; g'_1,\ldots,g'_n))\\
&=\psi(\iota; g_{\iota(1)} g'_1,\ldots, g_{\iota(n)} g'_n)\\
&=\psi(\iota; g_1 g'_1,\ldots, g_n g'_n)\qquad (\text{since } \iota(k) = k)\\
&=(g_1 g'_1,\ldots, g_n g'_n)\\
&=(g_1\ldots,g_n)(g'_1,\ldots,g'_n)\\
&= \psi(\lambda) \psi(\lambda').
\end{align*}
So $\psi$ is an group isomorphism, and $\ker \varphi \simeq G^n$.

\item[(c)] By part (b), since $\varphi$ is a surjective homomorphism,
$$
(A\wr G )/ \ker \varphi \simeq A,
$$
and $\ker \varphi \simeq G^n$. Therefore
$$
|A| = |A \wr G| / |\ker \varphi| = |A\wr G|/ |G|^n,
$$
which proves
$$|A \wr G| = |A| |G|^n.$$
\ee
\end{proof}

\paragraph{Ex. 14.2.7}
{\it Let $A \wr G$ be as in Exercise 6, and let $H$ be the set of all functions
$$\phi:\{1,\ldots,n\} \to G.$$
\be
\item[(a)] Given $\phi,\chi \in H$, define $\phi \chi \in H$ by $(\phi \chi)(i) = \phi(i)\chi(i)$. Prove that this makes $H$ into a group isomorphic to the product group $G^n$.
\item[(b)] Elements of $A \wr G$ can be written $(\tau,\phi)$, where $\phi \in H$. Prove that in this notation, (14.7) becomes
$$(\tau,\phi)(\tau',\phi') = (\tau \tau',((\tau')^{-1} \cdot \varphi) \phi').$$
\item[(c)] $A \subset S_n$ acts on $\{1,\ldots,n\}$. Show that this induces an action of $A$ on $H$ via $(\tau\cdot \phi)(i) = \phi(\tau^{-1}(i))$. Be sure you understand why the inverse is necessary.
\item[(d)] The action of part (c) enable us to define the semidirect product $H \rtimes A$. Using the description of $A \wr G$ given in part (b), prove that the map
$$(\tau, \phi) \mapsto (\tau \cdot \phi, \tau)$$
defines a group isomorphism $A \wr G \simeq H \rtimes A$. This shows that wreath products can be represented as semidirect products.

\ee
}
\begin{proof}
\be
\item[(a)] Consider the two maps
$$
\varphi
\left\{
\begin{array}{ccc}
H & \to &G^n\\
\phi & \mapsto &(\phi(1),\ldots,\phi(n)),
\end{array}
\right.
\qquad
\psi
\left\{
\begin{array}{ccc}
G^n & \to &H\\
(x_1,\ldots,x_n)& \mapsto &
\xi
\left\{
   \begin{array}{ccc}
   \{1,\ldots,n\} & \to & G\\
   i& \mapsto & x_i.
   \end{array}
   \right.
\end{array}
\right.
$$
Then $\psi \circ \varphi = 1_H$ and $\varphi \circ \psi = 1_{G^n}$, therefore $\varphi$ is bijective.

Moreover, for all $(\phi,\chi) \in H$,
\begin{align*}
\varphi(\phi \chi) &= ((\phi\chi)(1),\ldots, (\phi \chi)(n))\\
&=(\phi(1) \chi(1), \ldots, \phi(n) \chi(n))\\
&= (\phi(1),\ldots,\phi(n)) (\chi(1),\ldots,\chi(n))\\
&= \varphi(\phi) \varphi(\chi)).
\end{align*}
Therefore $H \simeq G^n$ via $\varphi$.

\item[(b,c)] If we define $\phi^\tau$, for $\tau \in S_n$ and $\phi \in H$, by $(\phi^\tau)(i) = \phi(\tau(i)),\ i=1,\ldots,n$, we obtain a right action: if $\tau, \tau' \in S_n$, for all $i\in\{1,\ldots,n\}$,
$$((\phi^{\tau})^{\tau'})(i) =  (\phi^\tau)(\tau'(i)) = \phi(\tau(\tau'(i))) = \phi((\tau \tau')(i)) = \phi^{\tau \tau'}(i)),$$
thus $(\phi^{\tau})^{\tau'} = \phi^{\tau \tau'}$. To obtain a left action, we must define, as in part (c),
$$(\tau\cdot \phi)(i) = \phi(\tau^{-1}(i)),\ i=1,\ldots,n.$$
Then
$$(\tau' \cdot (\tau\cdot \phi))(i) = (\tau\cdot \phi)(\tau'^{-1}(i)) = \phi(\tau^{-1}(\tau'^{-1}(i))) = \phi(\tau'\tau)^{-1}(i) = ((\tau' \tau)\cdot \phi)(i),$$
so that $\tau' \cdot (\tau\cdot \phi) = (\tau' \tau)\cdot \phi)$ (and $e \cdot \tau = \tau$).

This is a proof of part (c), and this explains the recurrent and stressful injonction from D.A.Cox ``{\bf Be sure you understand} why the inverse is necessary''.

Using this action for part (b), we define $(\tau, \phi)$ for $\tau \in S_n, \phi \in H = G^{\{1,\cdots,n\}}$,by
$$(\tau, \phi) = (\tau; \phi(1),\ldots,\phi(n)),$$
so that
$$(\tau, \phi) = (\tau; g_1,\ldots,g_n) \iff  \phi(1) = g_1,\ldots,\phi(n) = g_n.$$
If $(\tau, \phi) = (\tau; g_1,\ldots,g_n), (\tau', \phi') = (\tau; g'_1,\ldots,g'_n)$, then
\begin{align*}
(\tau, \phi) (\tau', \phi') &= (\tau; g_1,\ldots,g_n)(\tau; g'_1,\ldots,g'_n)\\
&=(\tau \tau'; g_{\tau'(1)}g'_1,\ldots,g_{\tau'(n)} g'_n) \\
&=(\tau \tau'; \phi(\tau'(1))\phi'(1),\ldots,\phi(\tau'(n)) \phi'(n)\\
&=(\tau \tau'; ((\tau')^{-1}\cdot\phi)(1)\phi'(1),\ldots,((\tau')^{-1}\cdot\phi)(n))\phi'(n)\\
&= (\tau \tau', ((\tau')^{-1}\cdot\phi)\phi').
\end{align*}

\item[(d)] Consider the map
$$
\varphi
\left\{
\begin{array}{ccc}
A \wr G & \to & H \rtimes A\\
(\tau,\phi) & \mapsto & (\tau\cdot \phi, \tau).
\end{array}
\right.
$$
If $\psi : H \rtimes A \to A \wr G$ is defined by $\psi(\chi, \tau) = (\tau, \tau^{-1} \cdot \chi)$, then, for all $\tau \in S_n, \phi,\chi \in H$,
\begin{align*}
(\psi \circ \varphi)(\tau, \phi) &= \psi(\tau\cdot \phi, \tau) = (\tau, \tau^{-1}\cdot (\tau \cdot \phi) = (\tau, \phi),\\
(\varphi \circ \psi)(\chi, \tau) &= \varphi(\tau, \tau^{-1} \cdot \chi) = (\tau \cdot (\tau^{-1} \cdot \chi), \tau) = (\chi, \tau). 
\end{align*}
Thus $\psi \circ \varphi = 1_{A \wr G},\  \varphi \circ \psi = 1_{H \rtimes A}$. This proves that $\varphi$ is bijective.

Recall that the binary operation in $ H \rtimes A$ is defined by (6.9):
$$(\phi,\tau) (\phi',\tau') = (\phi(\tau\cdot \phi'), \tau \tau').
$$
We verify that $\varphi$ is a group homomorphism. Note first that, for $\tau \in S_n, \phi \chi \in H$, 
$$\tau \cdot (\phi \chi) = (\tau \cdot \phi)(\tau \cdot \chi).$$
Indeed, for all $i \in \{1,\ldots,n\}$,
\begin{align*}(\tau \cdot (\phi \chi))(i) &= (\phi\chi)(\tau^{-1}(i))\\
&=\phi(\tau^{-1}(i)) \chi(\tau^{-1}(i))\\
&=(\tau \cdot \phi)(i) (\tau \cdot \chi)(i)\\
&=((\tau \cdot \phi)(\tau \cdot \chi))(i).
\end{align*}
Using this rule, we obtain
\begin{align*}
\varphi((\tau,\phi) (\tau',\phi')) &= \varphi(\tau \tau'; ((\tau')^{-1}\cdot\phi)\phi')\\
&= ((\tau\tau')\cdot((\tau')^{-1}\cdot\phi)\phi'),\tau \tau')\\
&=  ((\tau\tau')\cdot((\tau')^{-1}\cdot\phi) ((\tau\tau')\cdot\phi'), \tau \tau')\\
&=((\tau \cdot\phi) ((\tau\tau' )\cdot \phi'), \tau \tau'),\\
\end{align*}
and using the binary operation in $H\rtimes A$,
\begin{align*}
\varphi(\tau, \phi) \varphi(\tau',\phi') &=(\tau\cdot \phi, \tau)((\tau'\cdot \phi', \tau')\\
&= ((\tau \cdot \phi) (\tau \cdot (\tau' \cdot \phi')), \tau \tau')\\
&=((\tau \cdot\phi) ((\tau\tau' )\cdot \phi'), \tau \tau'),
\end{align*}
thus $\varphi((\tau,\phi) (\tau',\phi')) = \varphi(\tau, \phi) \varphi(\tau',\phi') $.
We have proved that $\varphi$ is a group isomorphism, so
$$A \wr G \simeq H \rtimes A = G^{\{1,\ldots,n\} }\rtimes A.$$ 
Wreath products can be represented by semidirect products.
\ee
\end{proof}

\paragraph{Ex. 14.2.8}
{\it
The goal of this exercise is to relate Definition 14.2.2 to Galois's definition of not primitive. Let $f \in F[x]$ be monic, separable, and irreducible with splitting field $F \subset L$. Also assume that $f$ is imprimitive with blocks of roots given by $R_1,\ldots,R_m$, where each block has $n$ elements (thus $\deg(f) = mn$). Let $f_i$ be the monic polynomial whose roots are the elements of $R_i$, and let $K \subset L$ be the fixed field of $$\{\sigma \in \Gal(L/F)\, | \, \sigma(R_i) = R_i \text{ for all } i\}.$$
\be
\item[(a)] Show that $f = \prod_{i=1}^m f_i$ and that $f_i \in K[x]$ for all $i$.
\item[(b)] In Galois' definition, $K$ is obtained by adjoining the roots of a separable polynomial of degree $m$. In modern terms, Galois wants $F \subset K$ to be Galois extension such that $\Gal(K/F)$ (*) is isomorphic to a subgroup of $S_m$. Prove that the field $K$ defined in part (a) has these properties.
See Exercise 14 for some examples.

[(*) misprint in Cox.]
\ee
}

\begin{proof}
\be
\item[(a)] By Definition 14.2.2, $R = R_1 \cup \cdots \cup R_m$ (disjoint union) is the set of roots of $f$. Since $f$ is separable, by definition of $f_i$,
$$f = \prod_{\alpha \in R} (x - \alpha) = \prod_{i=1}^m \prod_{\alpha \in R_i} (x - \alpha) = \prod_{i=1}^m f_i.$$
Let 
$$G = \{\sigma \in \Gal(L/F)\, | \, \forall i \in \gcro 1, m \dcro, \ \sigma(R_i) = R_i \}.$$
If $G_i = \{\sigma \in \Gal(L/F)\, | \, \sigma(R_i) = R_i \}$ for $i = 1,\ldots,n$, then $G = \bigcap\limits_{i=1}^m G_i$.

Each $G_i$ is a subgroup of $\Gal(L/F)$: $e(R_i) = R_i$, and if $\sigma, \tau \in R_i$, then  $(\sigma \tau)(R_i) = \sigma(R_i) = R_i$ and $R_i = \sigma^{-1}(R_i)$. Therefore $G = \bigcap\limits_{i=1}^m G_i$ is a subgroup of $\Gal(L/F)$.

Let $K = L_G$ be the fixed field of $G$. By the Galois correspondence, $G = \Gal(L/K)$.

If $\sigma \in G_i$, since $\sigma(R_i) = R_i$, where le restriction of $\sigma$ to $R_i$ is bijective, then 
$$\sigma \cdot f_i = \prod_{\alpha \in R_i} (x - \sigma(\alpha)) = \prod_{\beta \in R_i} (x - \beta) = f_i, \qquad (\beta = \sigma(\alpha)).$$
Therefore, if $\sigma \in G = \bigcap\limits_{i=1}^m G_i$, then for all $i \in \{1,\ldots,m\}$, $\sigma\cdot f_i = f_i$. The coefficients of $f_i$ are in the fixed field $K$ of $G$, so that
$$f_i\in K[x],\ i=1,\ldots,m.$$
To give a first example, $f = x^4 - 2$ is imprimitive with blocks $$R_1 =\{\sqrt[4]{2},-\sqrt[4]{2}\},R_2 = \{i \sqrt[4]{2},-i\sqrt[4]{2}\}.$$

If $\tau,\sigma$ are defined by $\tau(\sqrt[4]{2}) = \sqrt[4]{2}, \tau(i) = -i$, and $\sigma(\sqrt[4]{2}) = i \sqrt[4]{2}, \sigma(i) = i$, then
$$\Gal(L/F) = \{e,\sigma,\sigma^2,\sigma^3, \tau, \sigma \tau, \sigma^2 \tau, \sigma^3\tau\} \simeq D_8.$$
Here $G = G_1 = G_2 = \{e,\sigma^2,\tau,\sigma^2 \tau\}$, and $K = L_G = \Q(\sqrt{2})$ (see Ex. 6.3.2 and Ex. 7.3.3).

We verify $f_1(x) = x^2 -\sqrt{2}, f_2(x) = x^2 + \sqrt{2} \in K[x]$.

\item[(b)] We prove that $G$ is a normal subgroup of $\Gal(L/F)$. 

Let $\lambda \in \Gal(L/F)$, and $\sigma \in G$. Since $f$ is imprimitive, $\lambda \in \Gal(L/F)$ permutes the blocks $R_i$: there exists $\tau \in S_m$ such that
$$\lambda(R_i) = R_{\tau(i)},\ i=1,\ldots,m.$$
Let $j$ be any fixed index in $\{1,\ldots,m\}$, and $i$ such that $\tau(i) = j$. Since $\sigma \in G \supset G_i$, $\sigma(R_i) = R_i$, thus
$$(\lambda \sigma \lambda^{-1})(R_j) = (\lambda \sigma)(R_i) = \lambda(R_i) = R_j.$$
Since this is true for all $j \in \{1,\ldots,m\}$, $\lambda \sigma \lambda^{-1} \in G$. This proves that $G$ is a normal subgroup of $\Gal(L/F)$. Therefore $F \subset K$ is a Galois extension (Theorem 7.2.5). 

Now we prove that $\Gal(K/F)$ is isomorphic to a subgroup of $S_m$.

Since $f$ is imprimitive with blocks of roots given by $R_1,\ldots,R_m$, for each $\sigma \in \Gal(L/F)$, there exists $\tau \in S_m$ such that $\sigma(R_i) = R_{\tau(i)},\ i=1,\ldots,m$.
Consider the map $\varphi$ sending $\sigma$ to $\tau$:
$$
\varphi
\left\{
\begin{array}{ccl}
\Gal(L/F) & \to & S_m\\
\sigma & \mapsto &\tau : \forall i \in \gcro 1,m \dcro,\ \sigma(R_i) = R_{\tau(i)}.
\end{array}
\right.
$$
Then , for all $\sigma \in \Gal(L/F)$,
$$\varphi(\sigma) = ()  \iff \forall i \in \gcro 1,m \dcro,\ \sigma(R_i) = R_i \iff \sigma \in G.$$
Therefore, $\ker(\varphi) = G$, and by the Galois correspondence (see part (a)) $G = \Gal(L/K)$, so that
$$\ker(\varphi) = G = \Gal(L/K).$$
Then, by Theorem 7.3.2, since $K$ is Galois over $F$,
$$ S_m \supset \mathrm{im}(\varphi) \simeq \Gal(L/F)/ \ker(\varphi) = \Gal(L/F)/\Gal(L/K) \simeq \Gal(K/F).$$
Therefore $\Gal(K/F)$ is isomorphic to the subgroup $\mathrm{im}(\varphi)$ of $S_m$.
\ee
\end{proof}

\paragraph{Ex.14.2.9}
Assume that $G \subset S_n$ is transitive and Abelian.
\be
\item[(a)] Prove that $|G| = n$ by considering the isotropy subgroups of $G$.
\item[(b)] Prove that $G$ is primitive if and only if $|G|$ is prime.
\ee
Thus a transitive Abelian permutation group is imprimitive unless it is cyclic of prime order.

\begin{proof}
\item[(a)] $G \subset S_n$ acts on $\{1,\ldots,n\}$ by the action defined by $\sigma\cdot k = \sigma(k),\ \sigma \in G, k \in \{1,\ldots,n\}$.

Consider the isotropy group $G_1$ of $1$: $G_1 = \{\sigma \in G\, |\, \sigma(1) = 1\}$. Let $\sigma$ be any permutation in $G_1$, and let $i$ be any element in $\{1,\ldots,n\}$. Since $G$ is transitive, there exists $\tau \in G$ such that $\tau(1) = i$. By hypothesis, $G$ is Abelian, therefore
$$\sigma(i) = (\sigma \tau)(1) = (\tau \sigma)(1) = \tau(1) = i.$$
Since this is true for all $i\in \{1,\ldots,n\}$, $\sigma = e$. This proves $G_1 = \{e\}$.

Moreover, since $G$ is transitive, the orbit of $1$ is $G\cdot 1 =\{1,\ldots,n\}$, thus $|G\cdot 1| = n$.

By the Fundamental Theorem of Group Actions,
$$|G\cdot 1 | = (G:G_1) = |G|,$$
thus $|G| = n$.
\item[(b)] By Lemma 14.2.7, if $G \subset S_n$ is transitive and imprimitive, then $n =kl,\ k>1,l>1$, is composite. Thus, if $n$ is prime, a transitive subgroup of $S_n$ is primitive.

Conversely, let $G$ be a transitive Abelian subgroup of $S_n$, where $n>1$ is composite. By part (a), $|G| = n$. We must prove that $G$ is imprimitive.

By the Kronecker's Theorem on the structure of Abelian groups, 
$$G \simeq C_{n_1} \times \cdots \times C_{n_r}$$
is a product of cyclic groups.

Therefore, either $G$ is cyclic of order $n$, or $G = HK \simeq H\times K, H\ne\{e\},K \ne \{e\}$ is a direct product of two non trivial subgroups (take for instance $H \simeq C_{n_1}, K \simeq C_{n_2} \times \cdots \times C_{n_r}$). We will deal with these two cases.

$\bullet$ Case 1.  We assume that $G \simeq C_n$ is cyclic, where $n = ml, \ m>1, l>1$. Then $G = \langle \sigma \rangle$, where the permutation $\sigma$ has order $n$. Take
\begin{align*}
R_1 &= \{1,\sigma^m(1),\ldots, \sigma^{(l-1)m}(1)\},\\
R_2 &= \{\sigma(1), \sigma^{m+1}(1), \ldots,\sigma^{(l-1)m + 1}(1)\} = \sigma(R_1),\\
\cdots\\
R_m&=\{\sigma^{m-1}(1),\sigma^{m + m-1}(1), \ldots,\sigma^{lm - 1}(1)\} = \sigma^{m-1}(R_1).
\end{align*}
Since $G$ is transitive, $$R_1\cup\cdots\cup R_m =\{1,\sigma(1),\ldots,\sigma^{n-1}(1)\} = G\cdot 1 = \{1,\ldots,n\}.$$ Moreover, if $\tau \in G$, then $\tau = \sigma^{j},\ j=0,\ldots,n-1$, thus, using $\sigma^n = e$, if $k$ is the remainder of $i+j-1$ modulo $n$,
$$\tau(R_i) = (\sigma^j \sigma^{i-1})(R_1) = \sigma^{i+j-1}(R_1) = \sigma^{k}(R_1) = R_{k+1}.$$

This proves that $G$ is imprimitive, with blocks $R_1,\ldots,R_m$.

\bigskip

$\bullet$ Case 2. Now, assume that $G = HK \simeq H \times K, |H| = l>1, |K| = m>1$. Then $n = ml$. Write
\begin{align*}
H &= \{\sigma_1 =e,\ldots,\sigma_l\},\\
K &= \{\tau_1 = e,\ldots,\tau_m\},
\end{align*}
and take
\begin{align*}
R_1 &=\{ (\sigma_1 \tau_1)(1), \ldots, (\sigma_l \tau_1)(1) \},\\
R_2 &=\{ (\sigma_1 \tau_2)(1),\ldots, (\sigma_l \tau_2)(1)\},\\
\cdots\\
R_m&= \{ (\sigma_1 \tau_m)(1), \ldots, (\sigma_l \tau_m)(1)\}.
\end{align*}
Since $G = HK$, every permutation $\lambda \in G$ is a product $\lambda = \sigma_i \tau_j,\ 1\leq i\leq l, 1 \leq j \leq m$, and since $G$ is transitive,
$$R_1 \cup \cdots \cup R_m = G\cdot 1 = \{1,\ldots,n\}.$$
Take $\lambda = \sigma_i \tau_j \in G$, and $R_k = \{\sigma_u \tau_k, 1\leq u \leq n\}$. Then $\tau_j \tau_k = \tau_r$ for some fixed $r \in \{1,\ldots,m\}$. Since $G$ is Abelian,
$$\lambda(R_k) = \{(\sigma_i \sigma_u \tau_j \tau_k)(1) , 1 \leq u \leq n\} = \{(\sigma_i \sigma_u \tau_r)(1) , 1 \leq u \leq n\} =  \{(\sigma_v \tau_r)(1) , 1 \leq v \leq n\} = R_r,$$
because the map $H \to H$, $\sigma_u \mapsto \sigma_i \sigma_u$ is bijective.

This proves that $G$ is imprimitive, with blocks $R_1,\ldots,R_m$. 

To conclude, $G$ is primitive if and only if $|G|$ is prime (or $|G| = 1$). Thus a non trivial transitive Abelian permutation group is imprimitive unless it is cyclic of prime order.

\bigskip

Examples: $G = \{(), (1\,2)(3\,4), (1\,3)(2\,4),(1\,4)(2\,3)\}$ is a transitive Abelian subgroup of $S_4$, and an example of Case 2. If $H, K$ are two distinct subgroups of $G$ with order 2, then $G = HK \simeq C_2 \times C_2$. We can take $R_1 = \{1,2\}, R_2 = \{3,4\}$, but we can also take $R'_1 = \{1,3\},R'_2 = \{2,4\}$. G is imprimitive with blocks $R_1,R_2$, or with blocks $R'_1,R'_2$.

$G' = \langle (1\,2\,3\,4) \rangle$ is a another transitive Abelian subgroup of $S_4$, and an example of Case 1. This times, we can take only $R_1 = \{1,3\},R_2 = \{2,4\}$.
\end{proof}

\paragraph{Ex.14.2.10} {\it Let $\Phi_p(x)$ be the cyclotomic polynomial whose roots are the primitive $p$th roots of unity, where $p$ is prime. We know that $\Phi_p(x)$ is irreducible of degree $p-1$. In the quotation given in the Historical Notes, Galois asserts that $\Phi_p(x)$ is imprimitive.
\be
\item[(a)]
Prove Galois's claim for $p>3$ using Exercise 9.
\item[(b)] Explain why we need to assume that $p>3$ in part (a).
\ee
}

\begin{proof}
\item[(a)]
We know that the splitting field of $\Phi_p(x)$ over $\Q$ is $L = \Q(\zeta_p)$ (where $\zeta_p = e^{\frac{2i\pi}{p}}$), and that $\Gal(L/\Q) \simeq (\Z/p\Z)^*$, via the isomorphism 
$$
\left\{
\begin{array}{ccl}
\Gal(L/\Q) & \to & (\Z/p\Z)^*\\
\sigma & \mapsto & a : \sigma(\zeta_p) = \zeta_p^a.
\end{array}
\right.
$$
Therefore, $\Gal(L/\Q)$ is Abelian, and even cyclic with order $p-1$. Let $\Gal(L/\Q) \simeq G  \subset S_{p-1}$. If $p>3$, then $p-1$ is not prime, and Exercise 9 prove that  $G_n$ is imprimitive, so that $\Phi_p(x)$ is imprimitive.

\item[(b)] If $p = 3$, $p-1 = 2$ is prime, and $(\Z/3\Z)^* = \{1 ,-1 \}$ is cyclic of prime order. Moreover $\Phi_3(x) = x^2+x + 1$ is not imprimitive, as every polynomial of degree 2: by Definition 14.2.2, if $f$ is imprimitive, since $k>1,l>1$, $\deg(f) \geq |R_1|+|R_2| >2$.

If $p = 2$, $(\Z/2\Z)^* = \{1\}$, and $\Phi_2(x) = x+1$. Since $\deg(\Phi_2) = 1$, $\Phi_2(x)$  is not imprimitive.
 
\end{proof}


\paragraph{Ex. 14.2.11}{\it
Given a prime $p$, let $C_p \subset S_p$ be the cyclic subgroup generated by the $p$-cycle $(1\,2\cdots p)$. As explained in the text, this gives the wreath product $C_p \wr C_p \subset S_{p^2}$. Prove that $C_p \wr C_p$ is a $p$-Sylow subgroup of $S_{p^2}$.
}
\begin{proof}
By Lemma 14.2.8, we know that $C_p \wr C_p$ is a subgroup of $S_p \wr S_p$, which may be viewed as $S_{p^2}$.

Exercise 6 shows that $|C_p \wr C_p| = |C_p| |C_p|^p = p^{p+1}$.

Moreover $|S_{p^2}| = (p^2)!$, and, if $\nu_p(n)$ is the exponent of $p$ in the prime factorization of $n!$, then
$$\nu_p(n!) =\left \lfloor \frac{n}{p} \right \rfloor +  \left \lfloor \frac{n}{p^2}\right  \rfloor + \cdots + \left \lfloor \frac{n}{p^k} \right \rfloor + \cdots.$$
(see for instance Ex. 2.6 in Ireland and Rosen)
 
Therefore\begin{align*}
\nu_p((p^2)!) &=\left \lfloor \frac{p^2}{p} \right \rfloor +  \left \lfloor \frac{p^2 }{p^2}\right  \rfloor + \cdots + \left \lfloor \frac{p^2}{p^k} \right \rfloor + \cdots\\
&=\left \lfloor \frac{p^2}{p} \right \rfloor +  \left \lfloor \frac{p^2 }{p^2}\right  \rfloor\\
&=p + 1.
\end{align*}
Therefore $p^{p+1}$ is the maximal power of $p$ which divides $(p^2)! = S_{p^2}$, so that $C_p \wr C_p$ is a $p$-Sylow subgroup of $S_{p^2}$.
\end{proof}


\paragraph{Ex. 14.2.12}{\it Let $f$ be an irreducible imprimitive polynomial of degree 6,8 or 9 over a field $F$ of characteristic 0. Prove that $f$ is solvable by radicals over $F$.
}

\begin{proof}
Write $L$ the splitting field of $f$. Since the characteristic of $F$ is 0, the irreducible polynomial $f$ is separable, so $f$ is separable, irreducible and imprimitive.  By Corollary 14.2.10, $G = \Gal(L/F)$ is isomorphic to a subgroup of $S_k \wr S_l$, where $n = kl$ is a nontrivial factorization. The only nontrivial factorizations of 6,8 or 9 are
$$6 = 2 \times 3 = 3 \times 2, \qquad 8 = 2 \times 4 = 4 \times 2, \qquad 9 = 3 \times 3.$$
Thus $G$ is isomorphic to a subgroup of the list
$$S_2 \wr S_3, S_3\wr S_2,S_4 \wr S_2,S_2 \wr S_4, S_3\wr S_3,$$
whose cardinalities are
\begin{align*}
|S_2 \wr S_3| &= 2!3^2 = 2 \times 3^2,\\
 |S_3\wr S_2| &= 3!2^3 = 2^4 \times 3,\\
 |S_4 \wr S_2| &= 4!2^4 = 2^7 \times 3,\\
 |S_2 \wr S_4| &= 2!4^2 = 2^5,\\
 |S_3\wr S_3| &=3!3^3 = 2 \times 3^4. 
\end{align*}
So $S_k \wr S_l$ has only two prime factors 2 and 3. By Burnside's Theorem (Theorem 8.1.8), $S_k \wr S_l$ solvable for these values of $k,r$, thus the subgroup $G$ is solvable. Since the characteristic of $F$ is 0, this proves that $f$ is solvable by radicals over $f$.
\end{proof}





\paragraph{Ex. 14.2.13}{\it Let $f = x^6 + bx^3 + c \in F[x]$ be irreducible, where $F$ has characteristic different from 2 or 3. We will study the size of the Galois group of $f$ over $F$.
\be
\item[(a)] Show that $f$ is separable. So we can think of the Galois group as a subgroup of $S_6$.
\item[(b)] Show that $x^6+bx^3 + c$ is imprimitive and that its Galois group lies in $S_2 \wr S_3$. Also show that $|S_2 \wr S_3| = 72$. Thus the Galois group has order $\leq 72$.
\item[(c)] Let $F \subset L $ be the splitting field of $f$ over $F$. Use the Tower Theorem to show that $[L:F] \leq 36$. Hence the Galois group has order at most $36$.
\ee
Using Maple or Sage, one can show that the Galois group of $x^6 + 2x^3 - 2$ over $\Q$ has order $36$ and hence is as large as possible.
}
\begin{proof}
\item[(a)] Since $F$ has characteristic different from 2 or 3, $f' = 6x^5 + 3bx^2 \ne 0$. By hypothesis, $f$ is irreducible, thus any factor of $f$ is associate to $f$ or $1$. Since $f' \ne 0$, $\gcd(f,f')$ divides $f'$, thus $\deg(\gcd(f,f')) \leq \deg(f') = 5$, and $\gcd(f,f')$ is a factor of $f$, which cannot be associate to $f$, therefore $\gcd(f,f') = 1$. This proves that $f$ is separable.
\item[(b)]  If $\alpha$ is a root of $f$ in $L$, then $ \lambda = \alpha^3 \in L $ is a root of $g = x^2 +bx+c$. Let $\mu = -b-\lambda \in L$. Then $\mu$ is a root of $g$: 
 $$\mu^2+b\mu+c = (-b-\lambda)^2 + b(-b-\lambda)+c = \lambda^2 + b\lambda + c = 0.$$
Moreover $\lambda \ne \mu$, otherwise $b = -2\lambda, c = -\lambda^2 - b\lambda = \lambda^2$, and $g = (x-\lambda)^2$, where $\lambda = -b/2 \in F$, so that  $f =g(x^3) = (x^3-\lambda)^2$ would not be irreducible over $f$. Therefore
$$g = (x- \lambda)(x-\mu),\qquad \lambda\ne \mu.$$
Write $K = F(\lambda,\mu)$. Then $F\subset K \subset L$ is an intermediate field which is a splitting field of $g$ over $F$.
If $\lambda \in F$, then $\mu \in F$ and $f = g(x^3) = (x^3 - \lambda)(x^3 - \mu)$ would not be irreducible over $F$. Therefore $\lambda \not \in F,\mu \not \in F$, $g$ is irreducible over $F$. $K$ is the splitting field of the separable polynomial $g$, thus $F \subset K$ is a Galois extension, where $F \subsetneq K \subset L$.

Moreover,
$$f = g(x^2) = (x^3 - \lambda)(x^3 - \mu) = f_1 f_2,$$
where $f_1 = x^3 - \lambda, f_2 = x^3 - \mu \in K[x]$.
Therefore three roots $\alpha,\alpha',\alpha''$ of $f$ are the roots of $x^3 -\lambda$, and three other roots $\beta,\beta',\beta''$ of $f$ are the roots of $x^3 - \mu$ :
$$\alpha^3 = \alpha'^3 = \alpha''^3 = \lambda,\qquad \beta^3 = \beta'^3 = \beta''^3 = \mu.$$
This gives the blocks
$$R_1 = \{\alpha, \alpha', \alpha''\},\qquad R_2 = \{\beta,\beta',\beta''\}.$$
If $\sigma \in \Gal(L/F)$, then $g = \sigma\cdot g = (x -\sigma(\lambda))(x - \sigma(\mu))$, thus $\{\lambda, \mu\} = \{\sigma(\lambda), \sigma(\mu)\}$: $\sigma$ fixes $\lambda, \mu$, or exchanges $\lambda, \mu$. Since $\alpha,\beta,\gamma$ are the roots of $x^3 - \lambda$, $\sigma(\alpha), \sigma(\beta), \sigma(\gamma)$ are the roots of $x^3 - \sigma(\lambda)$, thus $\sigma(R_1) = R_1$ or $\sigma(R_1) = R_2$, and similarly $\sigma(R_2) = R_1$ or $R_2$. This proves that $f$ is imprimitive, with blocks $R_1,R_2$, and $\Gal(L/F)$ is isomorphic to a subgroup of $S_2 \wr S_3$(Corollary 14.2.10), whose order is $2(3!)^2 = 72$ (see Ex. 6 (c)).

\item[(c)] We don't know if $F$ or $K$ contains the cubic roots of unity, but $L$ does: since $\alpha, \alpha'$ are two distinct roots of $x^3 - \lambda$ (where $\alpha \ne 0$, otherwise $\lambda = 0 \in F$), then $(\frac{\alpha'}{\alpha})^3 = 1$. If we write $\omega = \frac{\alpha'}{\alpha}$, then $\omega \in L$ and $\omega^3 = 1, \omega\ne 1$, thus $1,\omega, \omega^2$ are three distinct roots of $x^3 - 1$, and $1 + \omega + \omega^2 = 0$, so that $[K(\omega) : K ] = 1$ or $2$. 

Then the six roots of $f$ are 
$$\alpha_1 = \alpha, \alpha_2 = \omega \alpha, \alpha_3 = \omega^2 \alpha, \alpha_4 = \alpha', \alpha_5 = \omega \alpha', \alpha_6 = \omega^2 \alpha'.$$
Therefore
$$L = F(\alpha,\omega \alpha, \omega^2 \alpha, \beta, \omega \beta, \beta^2) = F(\omega,\alpha,\beta) = K(\omega, \alpha, \beta). $$

Consider the chain of fields
$$F \subset K = F(\lambda, \mu) \subset K(\omega,\alpha) \subset L = K(\omega, \alpha, \beta).$$

\begin{center}
\begin{tikzpicture}
    \node (L) at (4,8) {$L = K(\omega,\alpha,\beta)$};
    \node (A3) at (1,6) {$K(\omega,\alpha)$};
    \node (t23) at (7,6) {$K(\omega,\beta)$};
    \node (id) at (4,4) {$K(\omega)$};
    \node (K) at (4,2) {$K = F(\lambda,\mu)$};
    \node (F) at (4,0) {$F$};
    \draw[<-] (L) edge (A3)   edge (t23);
    \draw[->] (id) edge (A3)  edge (t23);
    \draw[->] (K) edge (id);
     \draw[->] (F) edge (K);
\end{tikzpicture}
\end{center}


$\bullet$ Since $\lambda,\mu$ are the roots of the irreducible polynomial $f$, $K=F(\lambda,\mu)$ is a quadratic extension of $F$, so $[K:F] = 2$.

$\bullet$ Since $\omega^2 + \omega + 1 = 0$, $[K(\omega) : K] \leq 2$.

$\bullet$ Since $\alpha^3 - \lambda = 0$, where $\lambda \in K \subset K(\omega)$, $[K(\omega,\alpha) : K(\omega)] \leq 3$.

$\bullet$ Since $\beta^3 - \mu = 0$, where $\mu \in K \subset K(\omega, \alpha)$,  $[K(\omega, \alpha,\beta) : K(\omega, \alpha)] \leq 3$.

By the Tower Theorem,
$$[L:K] = [K(\omega, \alpha,\beta) : K] = [K(\omega, \alpha,\beta) : K(\omega, \alpha)] [K(\omega,\alpha) : K(\omega)][K(\omega) : K]  \leq 3\times3 \times2 \times 2 = 36.$$
Therefore,
$$|\Gal(L/K) | = [L:K] \leq 36.$$
\end{proof}
Note : Some permutations of $S_2\wr S_3$ can't lie in the Galois group $G \subset S_6$ of $f$, for instance if the transposition $(4\,5)$ corresponds to $\sigma \in \Gal(L/F)$, given by
$$
\left(
\begin{array}{cccccc}
\alpha & \omega\alpha &\omega^2\alpha & \beta & \omega \beta & \omega^2 \beta\\
\alpha & \omega\alpha &\omega^2\alpha & \omega\beta &  \beta & \omega^2 \beta
\end{array}
\right),
$$
then $\sigma(\alpha) = \alpha$ and $\sigma(\omega \alpha) = \omega \alpha$ show that $\sigma(\omega) = \omega$, and $\sigma(\beta) = \omega \beta$ implies $\sigma(\omega \beta) = \omega^2 \beta \ne \beta$. This contradiction proves that $(4\,5) \not \in G$ (but $(4\, 5) \in S_3 \wr S_2 \subset S_6$). More generally, all odd permutations are impossible, so that $G$ is a subgroup of $(S_2\wr S_3) \cap A_6$.












\paragraph{Ex. 14.2.14} {\it Here are some examples to illustrate Galois's definition of imprimitive. We will use the notation of Exercise 8. Let $F$ be a field of characteristic different from 2 or 3.
\be
\item[(a)] Let $f = x^6 + bx^4+cx^2 +d \in F[x]$ be irreducible with splitting field $F \subset L$. Show that the splitting field of $x^3+bx^2+cx+d$ gives an intermediate field $F\subset K \subset L$ such that $F \subset K$ is Galois and $f=f_1f_2f_3$, where $f_i \in K[x]$ has degree 2 for $i=1,2,3$. Also explain how $K$ relates to the field $K$ constructed in Exercise 8.
\item[(b)] Work out the analogous theory when $f = x^6 + bx^3+c \in F[x]$ is irreducible.
\ee
}
\begin{proof}

\item[(a)] By hypothesis, $f = x^6 + bx^4+cx^2 +d$ is irreducible. Since the characteristic of $F$ is different from 2 or 3, $f' = 6x^5+\cdots \ne 0$, thus $\gcd(f,f') =1$, and $f$ is separable.

If $\alpha$ is a root of $f$, then $-\alpha$ is a root of $f$. Moreover $\alpha \ne 0$, otherwise $d= 0$ and $f$ would not be irreducible, therefore $\alpha \ne -\alpha$. Since $f$ is separable, the roots of $f$ can be partitioned into three blocks
$$R_1 = \{\alpha,-\alpha\},\qquad R_2 = \{\beta,-\beta\},\qquad R_3 = \{\gamma, -\gamma\}.$$
If $\sigma \in \Gal(L/F)$ and if $\lambda \in R_i$ is a root of $f$, then $\sigma(\lambda) \in R_j$ for some index $\lambda$, and $\sigma(-\lambda) = - \sigma(\lambda) \in R_j$, thus $\sigma(R_i) = R_j$. Therefore $f$ is imprimitive, with blocks $R_1,R_2,R_3$. By Corollary 14.2.10, $\Gal(L/F)$ is isomorphic to a subgroup of $S_3 \wr S_2$.

Since $\alpha,-\alpha, \beta,-\beta,\gamma,-\gamma$ are the distinct roots of $f$, then $\alpha^2,\beta^2,\gamma^2$ are distinct and they are the roots of $g = x^3 +bx^2+cx+d$. Therefore
$$g =  x^3 +bx^2+cx+d = (x-\alpha^2)(x-\beta^2)(x-\gamma^2),$$
so that a splitting field $K$ of $g$ over $F$ is 
$$K = F(\alpha^2,\beta^2,\gamma^2),\qquad F \subset K \subset L.$$
Since $K$ is the splitting field of the separable polynomial $g$, $F \subset K$ is a Galois extension. Note that $g$ is irreducible over $F$, otherwise any non trivial factorization of $g$ over $F$ gives a factorisation of $f$ over $F$. Therefore $K \ne F$.

Moreover,
$$f = g(x^2) = (x^2-\alpha^2)(x^2-\beta^2)(x^2-\gamma^2) = f_1f_2f_3,$$
where $f_1 = x^2 - \alpha^2, f_2 = x^2 - \beta^2,f_3 = x^2 -\gamma^2 \in K[x]$. This proves the first assertion of part (a).

\bigskip

It remains to prove that $K$ is the fixed field of the subgroup $G$ of $\Gal(L/F)$ defined in Exercise 8:
$$G = \{\sigma \in \Gal(L/F)\, | \, \forall i \in \gcro 1, 3 \dcro, \ \sigma(R_i) = R_i \}.$$
To give an explicit description of $G$, note that
\begin{align*}
\sigma \in G &\iff \sigma(\alpha) = \pm \alpha,\ \sigma(\beta) = \pm \beta, \ \sigma(\gamma) = \pm \gamma\\
&\iff \sigma(\alpha^2) = \alpha^2,\ \sigma(\beta^2) = \beta^2, \sigma(\gamma^2) = \gamma^2\\
&\iff \forall \lambda \in K,\ \sigma(\lambda) = \lambda,
\end{align*}
where the last equivalence is explained by the fact that every $\lambda \in K = F(\alpha^2,\beta^2,\gamma^2)$ is a polynomial in $\alpha^2,\beta^2,\gamma^2$.

This proves that every element of $K$ is fixed by every $\sigma \in G$, thus $K \subset L_G$, where $L_G$ is the fixed field of $G$.

Since the Galois correspondence is order reversing, $K \subset L_G$ implies $\Gal(L/K) \supset G$. To prove the inverse inclusion, take $\sigma \in \Gal(L/K)$. Then $\sigma(\lambda) = \lambda$ for all $\lambda \in K$, and the preceding equivalence shows that $\sigma \in G$. Thus $\Gal(L/K) = G$. Applying the Galois correspondence once more, the fixed fields of $\Gal(L/K)$ and $G$ are equal, that is
$$K = L_G.$$
$K$ is the fixed field of $G =  \{\sigma \in \Gal(L/F)\, | \, \sigma(R_1) = R_1,\sigma(R_2) = R_2, \sigma(R_3) = R_3 \}$.

\item[(b)] We have proved in Exercise 13 that $f$ is imprimitive, with blocks
$$R_1 = \{\alpha, \beta, \gamma\},\qquad R_2 = \{\alpha',\beta',\gamma'\}.$$
With the same notations as in Exercise 13 and 8, we have 
$$G = \{\sigma \in \Gal(L/F) \,|\, \sigma(R_1) = R_1,\ \sigma(R_2) = R_2\}.$$
Since $\alpha,\beta,\gamma$ are the roots of $x^3 -\lambda$, and $\alpha',\beta',\gamma'$ the roots of $x^3 -\mu$, where $\{\lambda, \mu\} = \{\sigma(\lambda), \sigma(\mu)\}$, then, for all $\sigma \in \Gal(L/F)$,
\begin{align*}
\sigma \in G &\iff \sigma(\{\alpha,\beta, \gamma\} = \{\alpha,\beta,\gamma\},\  \sigma(\{\alpha',\beta', \gamma'\}) = \{\alpha',\beta',\gamma'\}\\
&\iff \sigma(\lambda) = \lambda,\ \sigma(\mu) = \mu\\
&\iff \forall \xi \in K,\ \sigma(\xi) \in K.
\end{align*}
This proves as in part (a) that $K = L_G$.
\end{proof}


\paragraph{Ex. 14.2.15}{\it
Let $G \subset S_n$ be transitive. Prove that $G$ is primitive if and only if the isotropy subgroups of $G$ are maximal with respect to inclusion.
}

\begin{proof}

Let $i$ be a fixed integer in $\{1,\ldots,n\}$. Since $G$ is transitive, $G\cdot i = \{1,\ldots,n\}$. The Fundamental Theorem of group actions gives
$n = |G\cdot i| = (G:G_i),$
thus $|G_i| n = |G|$.

Given a subgroup $G_i \subset H \subset G$, let $\{\tau_1=e,\ldots,\tau_m\}$ be a complete system of representatives of the left cosets $\tau H, \tau \in G$, where $m = (G:H)$, so that $\tau_1H,\ldots,\tau_mH$ partition $G$.

Consider
$$R_1 = (\tau_1H)\cdot i,\ldots,R_m = (\tau_m H)\cdot i.$$
As $G =\tau_1H \cup \cdots \cup \tau_m H$, then
$$R_1 \cup \cdots\cup R_m = (\tau_1H)\cdot i \cup \cdots \cup (\tau_mH)\cdot I =  G\cdot i = \{1,\ldots,n\}.$$
Now we show that this union is disjoint.

If $u \in R_j \cap R_k = (\tau_j H)\cdot i \cap (\tau_kH)\cdot i$, then
$$u = (\tau_j h)(i) = (\tau_k h')(i),\quad h,h'\in H.$$
Then $h'^{-1}\tau_k^{-1} \tau_j h)(i) = i$, thus
$$h'' = h'^{-1} \tau_k^{-1} \tau_j h \in G_i \subset H,$$
hence $\tau_j h = \tau_k h'h'',\ h,h',h'' \in H$. This shows that $\tau_j H =\tau_k H$, thus $j=k$. This proves
$$j \ne k \Rightarrow R_j \cap R_k = (\tau_j H)\cdot i \cap (\tau_kH)\cdot i = \varnothing.$$

Now we prove that every $\sigma \in G$ preserves the block structure:

If $\sigma \in G$ and $R_j = (\tau_j H)\cdot i$, then $\sigma \tau_j H$ is a left coset, thus $\sigma \tau_j H = \tau_k H$ for some index $k$, and
$$\sigma(R_j) =  (\sigma (\tau_j H \cdot i)) = (\sigma \tau_j H)\cdot i = (\tau_kH)\cdot i = R_k.$$
Since $G$ is transitive, all $R_j$ have same cardinality $l$, and $n = l m$.

To conclude, $R_1,\ldots,R_m$ have same cardinality, partition $\{1,\ldots,n\}$, and every $\sigma \in G$ preserves the block structure. If $l>1$ and $m>1$, then $G$ is imprimitive.

Hence, if we assume that $G$ is primitive, either $l= 1$ or $m=1$.

$\bullet$ If $l=1$, then for all indices $k$,  $(\tau_k H) \cdot i = \{\tau_k(i)\}$. With $k=1$ and $\tau_k = e$, we obtain $H\cdot i = \{i\}$, which shows that $H \subset G_i$, thus $H = G_i$.

$\bullet$ If $m=1$, then $(G:H) = m = 1$, thus $H = G$.

This proves that there is no subgroup $H$ such that $G_i \subsetneq H \subsetneq G$ : $G_i$ is maximal with respect to inclusion.

\bigskip

Conversely, suppose that $G$ is imprimitive, with respect to the blocks $R_1,\ldots,R_m$, where $m>1$. Since $G$ is transitive, all $R_j$ have the same cardinality $|R_j| = l >1$. 

If $i \in \{1,\ldots,n\}$ is some fixed integer, there is some index $j, 1\leq j \leq m$ such that $i \in R_j$. Now, consider the subgroup
$$H = \{\sigma\in G\, |\, \sigma(R_j) = R_j\}.$$
Then $G_i \subset H$: if $\sigma \in G_i$, then $\sigma(i) = i \in R_j$, thus $\sigma(R_j) = R_j$. Moreover

$\bullet$ $H\ne G$: Since $m>1$, there is some $R_k \ne R_j$, and some $w \in R_k$. Since $G$ is transitive, there is some $\sigma \in G$ such that $\sigma(i) = w$, so that $\sigma(R_j) = R_k$. Then $\sigma \not \in H$, and $H \ne G$.

$\bullet$ $G_i \ne H$: Since $|R_j|>1$, there is some $i'\ne i$ in the same block $R_j$. Since $G$ is transitive, there is some $\sigma' \in G$ such that $\sigma'(i) = i'$, so that $\sigma(R_j) = R_j$. Then $\sigma \in H$, but $\sigma \not \in G_i$, so $G_i \ne H$.

To conclude, the subgroup $H$ satisfies
$$G_i \subsetneq H \subsetneq G.$$
This proves that $G_i$ is not maximal with respect to inclusion.

We have proved that $G$ is primitive if and only if the isotropy subgroups of $G$ are maximal with respect to inclusion.

\end{proof}


\end{document}
