%&LaTeX
\documentclass[11pt,a4paper]{article}
\usepackage[frenchb,english]{babel}
\usepackage[applemac]{inputenc}
\usepackage[OT1]{fontenc}
\usepackage[]{graphicx}
\usepackage{amsmath}
\usepackage{amsfonts}
\usepackage{amsthm}
\usepackage{amssymb}
\usepackage{tikz}
%\input{8bitdefs}

% marges
\topmargin 10pt
\headsep 10pt
\headheight 10pt
\marginparwidth 30pt
\oddsidemargin 40pt
\evensidemargin 40pt
\footskip 30pt
\textheight 670pt
\textwidth 420pt

\def\imp{\Rightarrow}
\def\gcro{\mbox{[\hspace{-.15em}[}}% intervalles d'entiers 
\def\dcro{\mbox{]\hspace{-.15em}]}}

\newcommand{\be} {\begin{enumerate}}
\newcommand{\ee} {\end{enumerate}}
\newcommand{\deb}{\begin{eqnarray*}}
\newcommand{\fin}{\end{eqnarray*}}
\newcommand{\ssi} {si et seulement si }
\newcommand{\D}{\mathrm{d}}
\newcommand{\Q}{\mathbb{Q}}
\newcommand{\Z}{\mathbb{Z}}
\newcommand{\N}{\mathbb{N}}
\newcommand{\R}{\mathbb{R}}
\newcommand{\C}{\mathbb{C}}
\newcommand{\F}{\mathbb{F}}
\newcommand{\U}{\mathbb{U}}
\newcommand{\re}{\,\mathrm{Re}\,}
\newcommand{\im}{\,\mathrm{Im}\,}
\newcommand{\ord}{\mathrm{ord}}
\newcommand{\Gal}{\mathrm{Gal}}
\newcommand{\legendre}[2]{\genfrac{(}{)}{}{}{#1}{#2}}

\title{Solutions to David A.Cox  "Galois Theory''}

\refstepcounter{section} \refstepcounter{section}
\refstepcounter{section} \refstepcounter{section}
\refstepcounter{section}\refstepcounter{section}\refstepcounter{section}\refstepcounter{section}
\refstepcounter{section}\refstepcounter{section}\refstepcounter{section}\refstepcounter{section}\refstepcounter{section}


\begin{document}



\section{Chapter 14 : SOLVABLE PERMUTATION GROUPS}

\subsection{POLYNOMIAL OF PRIME DEGREE}

\paragraph{Ex. 14.1.1}
{\it This exercise is concerned with the proof of part (a) of Lemma 14.1.2. Let $\theta =(1\, 2 \ldots p)\in S_p$.
\begin{enumerate}
\item[(a)] Prove that $\tau\in S_p$ lies in the normalizer of $\langle\theta\rangle$ if and only if $\tau\theta =\theta^l\tau$ for some $1\leq l\leq p-1$.
\item[(b)] Prove that (14.1) implies that $\tau(i+j)=\tau(i)+jl$ for all positive integers j.
\end{enumerate}
}
\begin{proof}
\begin{enumerate}
\item[(a)]
If $\tau$ lies in the normalizer of $\langle \theta\rangle = \{e,\theta,\theta^2,\ldots,\theta^{p-1}\}$, then
$$\tau \theta \tau^{-1} \in \tau \langle \theta \rangle \tau^{-1}= \langle \theta \rangle,$$
hence
$$\tau \theta \tau^{-1} =\theta^l \text{ for some } l = 0,1\ldots,p-1.$$
If $l = 0$, then $\tau \theta \tau^{-1} = e$, thus $\tau \theta = \tau$, and $\theta = e$, which is false. Therefore $l \ne 0$.
$$\tau \theta \tau^{-1} =\theta^l , \ 1\leq l \leq p-1.$$

 \item[(b)] By induction suppose that $\tau(i+j)=\tau(i)+jl$, then $\tau(i+j+1)=\tau(i+j)+l=\tau(i)+(j+1)l$. Case $j=1$ is valid by the identity (14.1). Hence, $\tau(i+j)=\tau(i)+jl$ for all positive integers $j$.  
\end{enumerate}
\end{proof}

\paragraph{Ex. 14.1.2}

{\it Let $H$ be a normal subgroup of a finite group $G$ and let $g\in G$. The goal of this exercise is to prove Lemma 14.1.3.
\begin{enumerate}
\item[(a)] Explain why $(gH)^{o(g)}=(gH)^{[G:H]}=H$ in the quotient group $G/H$.
\item[(b)] Now assume that $\gcd(o(g),[G:H])=1$. Prove that $g\in H$.
\end{enumerate}
}
\begin{proof}
\begin{enumerate}
\item[(a)] Since $(gH)^2=gHgH=g^2H$ and $g^{o(g)}=e$, $(gH)^{o(g)}=g^{o(g)}H=H$.

Since $gH\in G/H$, exists some minimal $l$ such that $(gH)^l=H$ and $l\mid [G:H]$, i.e. $[G:H]=ql$. Then $(gH)^{[G:H]}=(gH)^{ql}=H^q=H$.
\item[(b)] The assumption $\gcd(o(g),[G:H])=1$ means that $o(g)q+[G:H]l=1$ for some $q,l\in\mathbb{Z}$. Then $gH=(gH)^{o(g)q+[G:H]l}=((gH)^{o(g)})^q((gH)^{[G:H]})^l=H^qH^l=H$, i.e. $g\in H$.
\end{enumerate}

\end{proof}

\paragraph{Ex. 14.1.3}

{\it Let $G$ satisfy (14.2). Use (14.2) and the Third Sylow Theorem to prove that $G$ has a unique p-Sylow subgroup $H$ of order $p$. Then conclude that $H$ is normal in $G$.
}

\begin{proof}
By (14.2),
 $$|G|= |\Gal(L/F)| =pm, \qquad 1\leq m \leq p-1.$$

According the Third Sylow Theorem the number $N$ of p-Sylow subgroups of G satisfies
 $$N\equiv 1 \pmod p,  \qquad N\mid |G|,$$ 
so that $N = 1 + kp,\ k\geq 0$, thus $N \wedge p = 1$, and $N \mid pm$, therefore $N \mid m$. If $k\ne 0$, then $N>p$, but $N \mid m>0$, which implies $N \leq m <p$. This contradiction shows that $k = 0$, and $N = 1$, i.e. there is exactly one $p$-Sylow subgroup $H$ of $G$.

For all $g \in G$, $gHg^{-1}$ is also a $p$-Sylow subgroup of $G$, hence $gHg^{-1}=H$ for all
$g \in G$: $H$ is normal in $G$.

\end{proof}

\paragraph{Ex. 14.1.4}

{\it The definition of Frobenius group given in the Mathematical Notes involves a group $G$ acting transitively on a set $X$. Prove that a group $G$ is a Frobenius group if and only if $G$ has a subgroup $H$ such that $1<|H|<|G|$ and $H\cap gHg^{-1}=\{e\}$ for all $g \notin H$.
}

\begin{proof}

$(\Rightarrow)$ Assume that $G$  is a Frobenius group. Then $G$ acts transitively on a set $X$ such that $1< |X| < |G|$, and for every $(x,y) \in X \times X$ such that $x \ne y$, the identity is the only element of $G$ fixing $x$ and $y$.

First we show that every isotropy group $G_x$ is non trivial, i.e. $G_x \ne \{e\}$ and $G_x \ne G$, for all $x \in G$.

Since $G$ acts transitively on $X$, $X = G\cdot x$ is the orbit of $x$, thus
$$ |X| = |G\cdot x | = (G:G_x) = |G| / |G_x|,$$
and since $1 < |X| < |G|$, this proves $1 < |G_x| < |G|$, so $ G_x \ne \{e\}, G_x \ne G$.
Fix $x_0 \in G, x_0 \ne e$, and take $H = G_{x_0}$ the isotropy group of this chosen element $x_0$. Then $1<|H|<G$.

Assume that $g \in G, g \not \in H$, and $h \in H \cap gHg^{-1}$. Then $h$ and $g^{-1}hg$ are both in $H = G_{x_0}$, so that $h \cdot x_0 = x_0$, and 
$(g^{-1}h g) \cdot x_0 = x_0$,
that is
$$
\left\{
\begin{array}{lll}
h \cdot x_0 &= &x_0,\\
h \cdot (g\cdot x_0) &= & g\cdot x_0.
\end{array}
\right.
$$
Since $g \not \in H = G_{x_0}$, $x_0 \ne g\cdot x_0$, thus $h$ fixes two distinct elements of $X$, and this shows that $h =e$. We have proved $H \cap gHg^{-1} = \{e\}$ for all $g\not \in H$.

\bigskip

$(\Leftarrow)$ Conversely, assume that $G$ has a subgroup $H$ such that $1<|H|<|G|$ and $H\cap gHg^{-1}=\{e\}$ for all $g \notin H$.

Take $X$ as the set of left cosets $hH, \ h\in G$ relative to $H$, and consider the action of $G$ on $X$ defined for all $h \in G$ by
$$g \cdot hH = (gh)H.$$
\be
\item[$\bullet$] This action is transitive: if $kH$ and $lH$ are left cosets, then $(lk)^{-1}\cdot kH = lH$.

\item[$\bullet$] Since $1 < |H| < |G|$, then $1 < |G|/|H| < |G|$, thus $1<|X| < |G|$.

\item[$\bullet$] Assume that $g$ fixes two distinct left cosets $hH \ne kH$:
\begin{align*}
g\cdot hH &= hH,\\
g\cdot kH &= k H.
\end{align*}
Then $l = h^{-1} g h \in H, m = k^{-1} g k \in H$, therefore $m = k^{-1}g k = k^{-1} h l h^{-1} k \in H$, so that
$$l \in H,\qquad(h^{-1}k)^{-1} l (h^{-1}k) \in H.$$
This proves $l \in H \cap gHg^{-1}$, where $g = h^{-1}k\not \in H$ (since $hH \ne k H$), and the hypothesis $H\cap gHg^{-1}=\{e\}$ gives $l = e$, and $g = hlh^{-1} =e$. The identity is the only element of $G$ fixing $hH$ and $kH$.
\ee
Therefore $G$ is a Frobenius group.
\end{proof}

\paragraph{Ex. 14.1.5}

{\it Let $F$ be a subfield of the real numbers, and let $f\in F[x]$ be irreducible of prime degree $p>2$. Assume that $f$ is solvable by radicals. Prove that $f$ has either a single real root or $p$ real roots. 
}

\begin{proof}
Since $\deg(f) = p$ is odd, $f$ has at least a real root. Suppose that $f$ has two distinct real roots $\alpha, \beta$.  By Theorem 14.1.1, since $f$ is solvable by radicals, the splitting field of $f$ over $F$ is $F(\alpha, \beta) \subset \R$. In this case all roots of $f$ are real, and these roots are distinct, since the characteristic of $F$ is $0$, thus the irreducible polynomial $f$ is separable.

We have proved that $f$ has either a single real root or $p$ real roots. 

\end{proof}

\paragraph{Ex. 14.1.6}

{\it By Example 8.5.5, $f=x^5-6x+3$ is not solvable by radicals over $\mathbb{Q}$. Give a new proof of this fact using the previous exercise together with the irreducibility of f and part (b) of Exercise 6 from Section 6.4.
}

\begin{proof}
The given polynomial $f$ has prime degree 5 and only three real roots, according to part (b) of Exercise 6.4.6. Since $f$ has more than one but less than 5 real roots, it is not solvable by radicals by Exercise 14.1.5. 

\end{proof}

\paragraph{Ex. 14.1.7}

{\it Use Lemma 14.1.3 and part (a) of Lemma 14.1.2 to give a proof of part (b) of Lemma 14.1.2 that doesn't use the Sylow Theorems.
}

\begin{proof}
Assume that $\tau\in S_p$ satisfies $\tau\theta\tau^{-1}\in \mathrm{AGL}(1,\mathbb{F}_p)$. Then, since $\langle\theta\rangle$ is a group of order $p$, $\langle\tau\theta\tau^{-1}\rangle=\tau\langle\theta\rangle\tau^{-1}$ is a subgroup of $\mathrm{AGL}(1,\mathbb{F}_p)$ of order $p$ and each element of this subgroup has order $p$ (or $1$).

By part (a) of Lemma 14.1.2, $\mathrm{AGL}(1,\mathbb{F}_p)$ is the normalizer of $\langle\theta\rangle$ in $S_p$, therefore $\langle\theta\rangle$ is normal in $\mathrm{AGL}(1,\mathbb{F}_p)$, with $[\mathrm{AGL}(1,\mathbb{F}_p):\langle\theta\rangle]=p-1$. The order of each element of $\tau\langle\theta\rangle\tau^{-1}$ is relatively prime to $p-1$, then, by Lemma 14.1.3, $\tau\langle\theta\rangle\tau^{-1}\subset\langle\theta\rangle$, therefore $\tau\langle\theta\rangle\tau^{-1}=\langle\theta\rangle$, since both groups have the same order $p$.  

Thus $\tau$ normalizes $\langle\theta\rangle$, hence $\tau\in \mathrm{AGL}(1,\mathbb{F}_p)$.
\end{proof}
\paragraph{Ex. 14.1.8}

{\it Let $f\in F[x]$ be irreducible of prime degree $p\geq 5$, where $F$ has characteristic 0, and let $\alpha\ne\beta$ be roots of f in some splitting field. If $F(\alpha,\beta)$ contains all other roots of $f$, then $f$ is solvable by radicals by Theorem 14.1.1. But suppose that there is some third root $\gamma$ such that $\gamma\in F(\alpha,\beta)$. Is this enough to force $f$ to be solvable by radicals?
\begin{enumerate}
\item[(a)] Use the classification of transitive subgroups of $S_5$ from Section 13.2 to show that the answer is ``yes'' when p=5.
\item[(b)] Use the polynomial $x^7-154\,x+99$ from Example 13.3.10 to show that the answer is ``no'' when p=7.
\end{enumerate}
}
\begin{proof}
\begin{enumerate}
\item[(a)]   
By hypothesis, $\deg(f) = p = 5$, and $\alpha \ne \beta$ are roots of $f$ in some splitting field.

Since $\alpha$ is a root of $f$, which is irreducible over $F$,
$$[F(\alpha) : F] = \deg(f) = p = 5.$$
Then $\beta$ is a root of $\frac{f(x)}{x-\alpha} \in F(\alpha)[x]$, so that the minimal polynomial of $\beta$ over $F(\alpha)$ has degree $d\leq p-1$. Thus
$$[F(\alpha,\beta) : F(\alpha] \leq p-1 = 4.$$
By the Tower Theorem,
$$[F(\alpha,\beta):F]  =   [F(\alpha,\beta):F(\alpha)] \ [F(\alpha):F] \leq p(p-1) = 20.$$
Now, suppose that there is some third root $\gamma$ such that $\gamma\in F(\alpha,\beta)$. Then $F(\alpha,\beta,\gamma) = F(\alpha,\beta)$. Let $\delta, \varepsilon$ be the remaining roots of $f$. Since the characteristic is $0$, the irreducible polynomial $f$ is separable.Then $\delta$ is a root of $\frac{f(x)}{(x-\alpha)(x-\beta)(x-\gamma)} \in F(\alpha,\beta,\gamma)[x]$, so that
$$[F(\alpha,\beta,\gamma,\delta):F(\alpha,\beta,\gamma)] \leq 2.$$
Since $F(\alpha,\beta,\gamma) = F(\alpha,\beta)$, the tower theorem gives
$$[F(\alpha,\beta,\gamma, \delta):F] \leq 40.$$
Moreover $\alpha + \beta +\gamma + \delta + \varepsilon = \sigma_1(\alpha,\beta,\gamma, \delta,\varepsilon) \in F$, thus $F(\alpha,\beta,\gamma, \delta,\varepsilon) = F(\alpha,\beta,\gamma, \delta)$. Write $L = F(\alpha,\beta,\gamma, \delta,\varepsilon)$ the splitting field of $f$ over $F$. We have proved
$$[L:F]\leq 40.$$
The classification of transitive subgroups of $S_5$ from Section 13.2 shows that any transitive subgroup of $S_5$ with cardinality $\le 40$ is a subgroup of $\mathrm{AGL}(1,\F_5)$, thus is solvable. So $\Gal(L/F)$ is a solvable group, where $F$ has characteristic $0$, therefore $f$ is solvable (Theorem 8.5.3).

To conclude, the answer is ``yes''  when $p = \deg(f) = 5$.
\item[(b)] To prove that the answer is ``no''  when $p = \deg(f) = 7$, we use the counterexample $f = x^7-154\, x+99$ from Example 13.3.10.

The polynomial $f$ is not solvable, since its Galois group is $\mathrm{GL}(3,\F_2)$, which is simple (Section 14.3) and not commutative, thus non solvable.

We prove that there are roots $\alpha,\beta, \gamma$ of $f$ such that $\gamma \in F(\alpha,\beta)$.

As in Example 13.3.10, consider the resolvant
$$\Theta_f(y) = \prod_{1\leq i < j < k \leq 7} \left(y - (\alpha_i + \alpha_j + \alpha_k)\right) \in \Q[y].$$
Then the factorization of $\Theta_f(y)$ over $\Q$ is 
$$\Theta_f(y) = g(y) h(y),$$
where the polynomials $g,h$, given in Example 13.3.10, are irreducible factors of degrees $7$ and $28$.

Take three roots $\alpha,\beta,\gamma$ of $f$ such that $y-(\alpha+\beta+ \gamma)$ is any linear factor of $g$, so that the minimal polynomial of $\alpha+\beta+ \gamma$ is $g$, with $\deg(g) = 7$, thus
$$[\Q(\alpha+\beta+\gamma) : \Q] = 7.$$
Now we prove that $\gamma \in F(\alpha,\beta)$. Consider the chain of extensions
$$\Q\subset \Q(\alpha) \subset \Q(\alpha,\beta) \subset \Q(\alpha,\beta,\gamma) \subset L,$$
where $L$ is the splitting field of $f$ over $\Q$.

The minimal polynomial of $\alpha$ over $\Q$ is $f$, thus $[\Q(\alpha) : \Q] = 7$, and 
$$[L:\Q] = |\Gal(L/\Q)| = |\mathrm{GL}(3,\F_2)| = 168 = 2^3 \times 3 \times 7.$$
By the Tower Theorem,
$$[L : \Q(\alpha)] = \frac{ [L : \Q]}{[ \Q(\alpha) : \Q]} = 2^3 \times 3$$
is not divisible by 7.

Since $\gamma$ is a root of $f$, the minimal polynomial of $\gamma$ over $f$ divides $f$. Thus
$$[\Q(\alpha,\beta,\gamma) : \Q(\alpha,\beta)] = 1\text{ or } 7.$$
If $[\Q(\alpha,\beta,\gamma) : \Q(\alpha,\beta)]  = 7$, by the Tower Theorem, $7$ divides $[L : \Q(\alpha)] = 2^3\times 3$. This contradiction proves that 
$$[\Q(\alpha,\beta,\gamma) : \Q(\alpha,\beta)] = 1,$$
therefore $\gamma \in \Q(\alpha,\beta).$

In this example, there exist roots $\alpha \ne \beta$ of $f$, and some third root $\gamma$ such that $\gamma \in F(\alpha,\beta)$, but $f$ is not solvable.

This shows that the answer is ``no'' when $p =\deg(f) = 7$.
\end{enumerate}

\end{proof}
\qquad

Note: In the proof of the Proposition 13.3.9, we saw that $G_f$ must be conjugate to $\mathrm{GL}(3,\F_2)$. This means that there is some numbering of the roots
$$
\left\{
\begin{array}{ccc}
\F_2^3\setminus\{(0,0,0\} & \to &\{\alpha \in L \, |\, f(\alpha) = 0\}\\
(\nu_1,\nu_2,\nu_3) &\to &\alpha_{\nu_1,\nu_2,\nu_3}
\end{array}
\right.
$$
which verify that, for all $\sigma \in\Gal(L/F)$, there is some $g \in \mathrm{GL}(3,\F_2)$ such that
$$\sigma(\alpha_{\nu_1,\nu_2,\nu_3}) = \alpha_{g \cdot (\nu_1,\nu_2,\nu_3)}.$$

In this correspondance, the roots of $f$ are seen as nonzero vectors in $\F_2^3$, and the seven roots of $g$ correspond to the seven (unordered)  triples of  linearly dependent  nonzero vectors in $\F_2^3$. So the roots $\alpha,\beta, \gamma$ where chosen in the preceding proof such that the corresponding vectors $u,v,w$ verify $w= u+v$ (but not $\gamma = \alpha + \beta$) .

This is what we understand in the hint of D.A. Cox ``Regard the roots as the nonzero vectors of $\F_2^3$ and pick roots $\alpha,\beta, \gamma$ such that $\gamma = \alpha+\beta$''.

This last equality is not true in $L$, but true for the corresponding vectors in $\F_2^3$.

Moreover, let $\alpha\ne \beta$ be {\it any} pair of roots. The corresponding vectors $u,v$ are such that $u,v,u+v = -u-v$ is not a base, so that the root $\gamma$ corresponding to $u+v$ is such that $ y - (\alpha+\beta+\gamma)$ is a factor of $g$, and the preceding proof shows that $\gamma \in \Q(\alpha,\beta)$. For each pair $\alpha\ne \beta$ of roots of $f = x^7-154\, x+99$, there exists a third root $\gamma \not \in \{\alpha,\beta\}$ such that $\gamma \in F(\alpha,\beta)$.







\subsection{IMPRIMITIVE POLYNOMIALS OF PRIME-SQUARED DEGREE}
\paragraph{Ex. 14.2.1}

{\it Prove (14.7).
}

\begin{proof}
Given $\sigma'=(\tau';\mu_1',...,\mu_k'),\sigma=(\tau;\mu_1,...,\mu_k)\in A\wr B$. Since $\sigma'$ maps $R_i$ to $R_{\tau'(i)}$ via $\mu_i'$, if we set $j=\tau'(i)$, then $\sigma$ maps $R_j$ to $R_{\tau(j)}=R_{\tau(\tau'(i))}=R_{\tau\tau'(i)}$ via $\mu_j=\mu_{\tau'(i)}$.

Hence $\sigma\sigma'$ maps $R_i$ to $R_{\tau\tau'(i)}$ via $\mu_{\tau'(i)}\mu_i'$.

More explicitly, by the definition of $(\tau; \mu_1,\ldots,\mu_k)$, for all $(i,j) \in \{1,\ldots,k\} \times \{1,\ldots,l\}$,
$$(\tau;\mu_1,\ldots,\mu_k)(i,j) = (\tau(i), \mu_i(j)).$$
Applying three times this definition, we obtain
\begin{align*}
(\tau; \mu_1,\ldots,\mu_k)(\tau'; \mu'_1,\ldots,\mu'_k)&= (\tau; \mu_1,\ldots,\mu_k)(\tau'(i),\mu'_i(j))\\
&= (\tau( \tau'(i)), \mu_{\tau'(i)}(\mu'_i(j))\\
&= ((\tau \tau')(i), (\mu_{\tau'(i)} \mu'_i)(j)\\
&=(\tau\tau';\mu_{\tau'(1)}\mu_1',...,\mu_{\tau'(k)}\mu_k')(i,j)
\end{align*}
Since this equality is true for all $(i,j) \in \{1,\ldots,k\} \times \{1,\ldots,l\}$,
$$(\tau;\mu_1,...,\mu_k)(\tau';\mu_1',...,\mu_k')=(\tau\tau';\mu_{\tau'(1)}\mu_1',...,\mu_{\tau'(k)}\mu_k').$$
\end{proof}

\paragraph{Ex. 14.2.2}
{\it The wreath product $S_3 \wr S_2 \subset S_6$ can be thought of as the subgroup of all permutations that preserve the blocs $R_1 = \{1,2\}, R_2 = \{3,4\},R_3 = \{5,6\}$. As noted in Example 14.2.11, $S_3 \wr S_2$ has order $6\cdot3^3 = 48$.
\be
\item[(a)] Show that $(S_3 \wr S_2) \cap A_6$ has order $24$.
\item[(b)] Show that $S_3 \wr S_2$ is the centralizer of $(1\,2)(3\,4)(5\,6)$ in $S_6$ (meaning that $S_3 \wr S_2$ consists of all permutations in $S_6$ that commute with $(1\,2)(3\,4)(5\,6)$).
\item[(c)] Use part (b) to show that $S_3 \wr S_2$ is isomorphic to $((S_3 \wr S_2) \cap A_6) \times S_2$.
\ee
See the next exercise for more on $S_3 \wr S_2$ and $(S_3 \wr S_2) \cap A_6$.
}
\begin{proof}
\item[(a)] Let $\varphi$ the restriction of the sign $\mathrm{sgn}$ to $(S_3 \wr S_2) \cap A_6$:
$$
\varphi \left\{
\begin{array}{ccc}
S_3 \wr S_2 & \to & \{-1,1\}\\
\sigma & \mapsto & \mathrm{sgn}(\sigma)
\end{array}
\right.
$$
Since $\mathrm{sgn}$ is a morphism, its restriction $\varphi$ is also a morphism, and $\varphi$ is surjective (onto), because $\varphi(e) = 1$, and $\varphi((1\,2)) = -1$, where $(1\,2) \in S_3\wr S_2$. Moreover the kernel of $\varphi$ is $\ker(\varphi) =  (S_3 \wr S_2) \cap A_6$.

Therefore $\mathrm{im}(\varphi) =  \{-1,1\} \simeq (S_3 \wr S_2)/ ((S_3 \wr S_2) \cap A_6)$. This shows that $$|(S_3 \wr S_2) \cap A_6| = \frac{1}{2} |S_3 \wr S_2| = 24.$$

\item[(b)] Let $\tau \in S_n$. Then $\tau$ is in the centralizer of $\sigma = (1\,2)(3\,4)(5\,6)$ if and only if
$$\tau (1\,2)(3\,4)(5\,6) \tau^{-1} = (1\,2)(3\,4)(5\,6),$$
which is equivalent to
$$(\tau(1)\, \tau(2))(\tau(3)\, \tau(4))(\tau(5)\, \tau(6)) = (1\,2)(3\,4)(5\,6).$$
Write $R_1 = \{1,2\}, R_2 = \{3,4\},R_3 = \{4,5\}$. Then $R_1, R_2,R_3$ are the three orbits  of $\sigma$ acting on $\{1,\ldots,6\}$, the supports of the decomposition of $\sigma$ in disjoint cycles.

Since $\tau$ is a bijection, the 6 values $\tau(1),\tau(2),\tau(3), \tau(4),\tau(5), \tau(6)$ are distinct, so $(\tau(1)\, \tau(2)),(\tau(3)\, \tau(4)),(\tau(5)\, \tau(6))$ are disjoint $2$-cycles. 

If $\tau$ is the centralizer of $\sigma$, the equality $(\tau(1)\, \tau(2))(\tau(3)\, \tau(4))(\tau(5)\, \tau(6)) = (1\,2)(3\,4)(5\,6)$ shows that $\tau(R_1),\tau(R_2),\tau(R_3)$ are also the three orbits of $\sigma$, so that
$$\{\{1,2\},\{3,4\},\{5,6\}\} =\{\{ \tau(1), \tau(2)\},\{\tau(3), \tau(4)\},\{\tau(5), \tau(6)\}\},$$
that is
$$\{R_1,R_2,R_3\} = \{\tau(R_1),\tau(R_2),\tau(R_3)\},$$
which means that there is some permutation $\tau'$ of $\{1,2,3\}$ such that $\tau(R_i) = R_{\tau'(i)}, \ i=1,2,3$. In other words, $\sigma$ preserves the blocks $R_1,R_2,R_3$, so that $\sigma \in S_3 \wr S_2$.

To prove the converse, it is more convenient to use the other usual representation of $S_3\wr S_2$. Then $\sigma = (e; \mu,\mu,\mu)$, where $\mu = (1\, 2) \in S_2$. Let $\tau = (\lambda; \mu_1,\mu_2,\mu_3)$ be any element of $S_3 \wr S_2$ (then $\mu_i = ()$ or $\mu_i = \mu)$. Then (14.7) gives
\begin{align*}
\tau \sigma &= (\lambda; \mu_1,\mu_2,\mu_3)(e; \mu,\mu,\mu)\\
&=(\lambda;\mu_1\mu, \mu_2 \mu,\mu_3\mu)\\
\sigma \tau &= (e;\mu,\mu;\mu)(\lambda,\mu_2,\mu_2,\mu_3)\\
&= (\lambda; \mu \mu_1, \mu \mu_2,\mu \mu_3)
\end{align*}
Since $S_2 = \{e,\mu\}$ is commutative, $\mu \mu_i = \mu_i \mu,\ i=1,2,3$, thus $\tau \sigma = \sigma \tau$.

The centralizer of $(1\,2)(3\,4)(5\,6)$ in $S_n$ is $S_3 \wr S_2$.

\item[(c)] Since the order of $\sigma = (1\,2)(3\,4)(5\,6)$ is $2$, $\langle \sigma \rangle = \{e,\sigma\} \simeq S_2$ and we can write $\sigma^{\varepsilon }, \varepsilon \in \{0,1\}$ the two elements of $\langle \sigma \rangle$.
Let
$$
\varphi
\left \{
\begin{array}{ccc}
(S_3 \wr S_2) \cap A_6 \times \langle \sigma \rangle & \to & S_3\wr S_2\\
(\tau, \sigma^{\varepsilon}) & \mapsto &\tau \sigma^{\varepsilon}.
\end{array}
\right.
$$
$\bullet$ $\varphi$ is a morphism: For all $\tau, \tau' \in (S_3 \wr S_2) \cap A_6$ and $\sigma^{\varepsilon}, \sigma^{\varepsilon'} \in \langle \sigma \rangle $, $\sigma \tau' = \tau' \sigma$ by part (b), thus
\begin{align*}
\varphi(\tau \sigma^{\varepsilon}) \varphi(\tau' \sigma^{\varepsilon'})&= \tau \sigma^{\varepsilon} \tau' \sigma^{\varepsilon'}\\
&= \tau \tau' \sigma^\varepsilon \sigma^{\varepsilon'}\\
&= \varphi((\tau,\sigma^\varepsilon) (\tau', \sigma^{\varepsilon'}))
\end{align*}

$\bullet$ $\ker \varphi$ is trivial: if $\varphi(\tau, \sigma^{\varepsilon}) = e$, then $\tau \sigma^{\varepsilon} = e$, so that $\tau  = \sigma^{-\varepsilon} \in \{e, \sigma\}$. $\tau = \sigma$ is impossible, since $\tau$ is an even permutation, and $\sigma$ is odd. Therefore $\tau = e$, and $\sigma^{\varepsilon} = e$. Thus $\varphi$ is injective (one to one).

$\bullet$ Since $|((S_3 \wr S_2) \cap A_6) \times \langle \sigma \rangle | = |S_3\wr S_2|$, $\varphi$ is a bijection, thus $\varphi$ is a group isomorphism.

$$S_3 \wr S_2 \simeq ((S_3 \wr S_2) \cap A_6) \times \langle \sigma \rangle  \simeq  ((S_3 \wr S_2) \cap A_6) \times S_2 .$$
\end{proof}

\paragraph{Ex. 14.2.3}
{\it One of the challenges of group theory is that the same group can have radically differnet descriptions. For instance, $S_4$ and the group $G = (S_3 \wr S_2) \cap A_6$ appearing in Example 14.2.11 both have order $24$. In this exercise, you will prove that they are isomorphic. We will use the notation of Exercise 2.
\be
\item[(a)] There is a natural homomorphism $G \to S_3$ given by how elements of $G$ permute the blocks $R_1,R_2,R_3$. Show that this map is onto, and express the elements of the kernel as products of disjoints cycles.
\item[(b)] Use the Sylow Theorems to show that $G$ has one or four $3$-Sylow subgroups.
\item[(c)] Show that $A_6$ has no element of order $6$.
\item[(d)] Use part (c) and the kernel of the map $G \to S_3$ from part (a) to show that $G$ has four $3$-Sylow subgroups.
\item[(e)] $G$ acts by conjugation on its four $3$-Sylow subgroups. Use this to prove that $G \simeq S_4$.
\item[(f)] Using Exercise 2, conclude that $S_3 \wr S_2 \simeq S_4 \times S_2$.

We note without proof that $S_3 \wr S_2 \simeq S_4 \times S_2$ is also isomorphic to the full symmetry group (rotations and reflexions) of the octahedron.
\ee
}

\begin{proof}
\item[(a)] Let $\varphi : G \to S_3$ defined by $\tau = \varphi(\sigma)$ iff $\sigma(R_i) = R_{\tau(i)}$. In other notations, this is the restriction to $G$ of the homomorphism of part (b) of Lemma 14.2.8, thus $\varphi$ is an homomorphism.

$\bullet$ $\varphi$ is surjective: Let $\tau$ be any permutation in $S_3$. 

If $\tau$ is even, $\tau = (1\,2\,3)^k,\ k = 0,1,2$. Let
$$\sigma =
\left(
\begin{array}{cccccccc}
1 & 2 & & 3 & 4 & & 5 & 6\\
3 & 4 & & 5 & 6 & & 1& 2\\
\end{array}
\right)
= (1\,3\,5)(2\,4\,6).
$$
$\sigma$ preserves the block structure defined by $R_1,R_2,R_3$, and $\sigma \in A_6$, so that $\sigma \in G = (S_3 \wr S_2) \cap A_6$. Moreover $\sigma(R_1) = R_2, \sigma(R_2) = R_3, \sigma(R_3) = R_1$, thus $\varphi(\sigma) = (1\,2\,3)$, and $\varphi(\sigma^k )= (1\,2\,3)^k = \tau$.

If $\tau$ is odd, then $\tau \in \{(1\,2),(2\,3),(1\,3)\}$, and
\begin{align*}
(1\,2) = \varphi(\sigma_1), \qquad \sigma_1 = 
\left(
\begin{array}{cccccccc}
1 & 2 & & 3 & 4 & & 5 & 6\\
3 & 4 & & 1 & 2 & & 5& 6\\
\end{array}
\right) = (1\,3)(2\,4) \in G,\\
(2\,3) = \varphi(\sigma_2), \qquad \sigma_2 = 
\left(
\begin{array}{cccccccc}
1 & 2 & & 3 & 4 & & 5 & 6\\
1 & 2 & & 5 & 6 & & 3& 4\\
\end{array}
\right) = (3\,5)(4\,6) \in G,\\
(1\,3) = \varphi(\sigma_3), \qquad \sigma_3 = 
\left(
\begin{array}{cccccccc}
1 & 2 & & 3 & 4 & & 5 & 6\\
5 & 6 & & 3 & 4 & & 1& 2\\
\end{array}
\right) = (1\,5)(2\,6) \in G.\\
\end{align*}
Therefore $\varphi$ is surjective.

$\bullet$ Let $\sigma \in S_6$. Then $\sigma \in \ker \varphi$ iff $\sigma \in A_6$ and $\sigma(R_1) = R_1, \sigma(R_2) = R_2, \sigma(R_3) = R_3$.

Morerover, for all $\sigma \in A_6$,
\begin{align*}
&\sigma(R_1) = R_1, \sigma(R_2) = R_2, \sigma(R_3) = R_3\\
\iff& \{\sigma(1), \sigma(2)\} = \{1,2\}, \{\sigma(3),\sigma(4)\} = \{3,4\}, \{\sigma(5),\sigma(6)\} = \{5,6\}\\
\iff &\sigma \in \{e,(1\,2)(3\,4), (1\,2)(5\,6), (3\,4)(5\,6)\}.
\end{align*}
$$\ker \varphi = \{e,(1\,2)(3\,4), (1\,2)(5\,6), (3\,4)(5\,6)\}.$$
Verification: $6 = |S_3| = |G/\ker(\varphi)| = 24/4$.

\item[(b)] Let $N$ be the number of $3$-Sylow subgroups of $G$. By the third Sylow Theorem,
$$N \mid 24 = |G|, \qquad N \equiv 1 \pmod 3.$$
Therefore $N = 1$ or $N = 4$.

\item[(c)] Let $\tau \in S_6$ be a permutation of order 6. If $\tau = \tau_1\cdots \tau_k$ is the decomposition of $\tau$ in disjoint cycles, then the order of $\tau$ is the lcm of the order of $ \tau_1,\ldots,\tau_k$. Therefore $\tau$ is a $6$-cycle or a product of a $2$-cycle by a $3$-cycle. In both cases $\tau$ is odd. Therefore $A_6$ has no element of order $6$.


\item[(d)] Reasoning by contradiction, suppose that $G$ has only one $3$-Sylow subgroup $H$. Then, for all $g \in G$, $gH g^{-1}$ is a $3$-Sylow, thus $gH g^{-1} = H$, and $H$ is a normal subgroup of $G$. 

Moreover $K = \ker \varphi = \{e,(1\,2)(3\,4), (1\,2)(5\,6), (3\,4)(5\,6)\}$ is normal in $G$, and has order 4. Therefore $H \cap K = \{e\}$.

The usual characterization of direct products (see Ex. 14.3.7) shows that, for all $h \in H$, all $k \in K$, $hk = kh$, and $HK$ is a normal subgroup of $G$ isomorphic to $H \times K$. 

Take $h$ an element of order $3$ in $H,$ and $k$ and element of order 2 in $K$. Since $kh = hk$, the order of $hk \in A_6$ is 6, which is impossible by part (c).

Therefore $G$ has exactly four $3$-Sylow subgroups.

\item[(e)] Write $X = \{H_1,H_2,H_3,H_4\}$ the set of $3$-Sylow subgroups of $G$, and $S(X)$ the set of permutations of $X$. Then $S(X) \simeq S_4$, and $g\cdot H = g H g^{-1}$ defines a left action of $G$ on $X$, so that
$$
\psi 
\left\{
\begin{array}{ccc}
G & \to & S(X)\\
g & \mapsto & 
\sigma = \left(
    \begin{array}{cccc}
      H_1&H_2&H_3&H_4\\
      gH_1g^{-1} & gH_2 g^{-1} & gH_3g^{-1} & gH_4g^{-1}
    \end{array}
\right)
\end{array}
\right.
$$
is a group homomorphism.

It is not obvious that $\psi$ is bijective. We prove first that $\psi$ is surjective (onto). We give explicitly the $3$-Sylow subgroups. Let
\begin{align*}
&\lambda_1 = 
\left(
\begin{array}{cccccccc}
1 & 2 & & 3 & 4 & & 5 & 6\\
 3&4&&5&6&&1&2 \\
\end{array}
\right) = (1\,3\,5)(2\,4\,6),\\
&\lambda_2 = 
\left(
\begin{array}{cccccccc}
1 & 2 & & 3 & 4 & & 5 & 6\\
 3&4&&6&5&&2&1 \\
\end{array}
\right) = (1\,3\,6)(4\,5\,2),\\
&\lambda_3 = 
\left(
\begin{array}{cccccccc}
1 & 2 & & 3 & 4 & & 5 & 6\\
 6&5&&2&1&&3&4 \\
\end{array}
\right) = (1\,6\,4)(5\,3\,2),\\
&\lambda_4 =
\left(
\begin{array}{cccccccc}
1 & 2 & & 3 & 4 & & 5 & 6\\
 4&3&&6&5&&1&2 \\
\end{array}
\right) = (1\,4\,5)(3\,6\,2).
\end{align*}4
Then $\lambda_1,\ldots,\lambda_4 \in G$ have order $3$, and $H_1 = \langle \lambda_1 \rangle = \{e,\lambda_1,\lambda_1^2\},\ldots,H_4 = \langle \lambda_4 \rangle = \{e,\lambda_4,\lambda_4^2\}$ are distinct, thus they are the four $3$-Sylow of $G$.

Now take 
\begin{align*}
&g = 
\left(
\begin{array}{cccccccc}
1 & 2 & & 3 & 4 & & 5 & 6\\
 4&3&&2&1&&5&6 \\
\end{array}
\right) = (1\,4)(2\,3)\\
&h=
\left(
\begin{array}{cccccccc}
1 & 2 & & 3 & 4 & & 5 & 6\\
 2&1&&5&6&&4&3 \\
\end{array}
\right) = (1\,2)(3\,5\,4\,6)
\end{align*}
(We give a geometrical explanation of this choice in the final note.)

Then 
\begin{align*}
g \lambda_1 g^{-1} &= (1\,4)(2\,3)(1\,3\,5)(2\,4\,6)(1\,4)(2\,3)\\
&=
\left(
\begin{array}{cccccccc}
1 & 2 & & 3 & 4 & & 5 & 6\\
 6&5&&1&2&&4&3 \\
\end{array}
\right) = (1\,6\,3)(2\,5\,4) = \lambda_2^2,
\end{align*}
thus $gH_1g^{-1} = H_2$, and since $g = g^{-1}$, $g H_2 g^{-1} = H_1$.
Moreover
\begin{align*}
g \lambda_3g^{-1} &=(1\,4)(2\,3)(1\,6\,4)(5\,3\,2)(1\,4)(2\,3)\\
&= 
\left(
\begin{array}{cccccccc}
1 & 2 & & 3 & 4 & & 5 & 6\\
 4&3&&5&6&&2&1 \\
  \end{array}
  \right)
 =(1\,4\,6)(2\,3\,5)
 =\lambda_3^2,
\end{align*}
thus $gH_3g^{-1} = H_3$, and since $\psi(g)$ is a permutation, $gH_4 g^{-1} = H_4$.

Therefore $\psi(g) \in S(X)$ is the permutation 
$ 
\left(
\begin{array}{cccc}
H_1&H_2&H_3&H_4\\
H_2&H_1&H_3&H_4
\end{array}
\right)
$, which corresponds to the transposition $(1\,2) \in S_4$.
Similarly,
\begin{align*}
h\lambda_1h^{-1}&= (1\,2)(3\,5\,4\,6)(1\,3\,5)(2\,4\,6)(3\,6\,4\,5)(1\,2)\\
&=
\left(
\begin{array}{cccccccc}
1 & 2 & & 3 & 4 & & 5 & 6\\
 6&5&&1&2&&4&3 \\
  \end{array}
  \right) =(1\,6\,3)(2\,5\,4) = \lambda_2^2,\\
 h\lambda_2h^{-1} &= (1\,2)(3\,5\,4\,6)(1\,3\,6)(4\,5\,2)(3\,6\,4\,5)(1\,2)\\
 &=
 \left(
\begin{array}{cccccccc}
1 & 2 & & 3 & 4 & & 5 & 6\\
 6&5&&2&1&&3&4 \\
  \end{array}
  \right) = (1\,6\,4)(2\,5\,3) = \lambda_3,\\
   h\lambda_3h^{-1} &= (1\,2)(3\,5\,4\,6)(1\,6\,4)(5\,3\,2)(3\,6\,4\,5)(1\,2)\\
 &=
 \left(
\begin{array}{cccccccc}
1 & 2 & & 3 & 4 & & 5 & 6\\
 4&3&&6&5&&1&2 \\
  \end{array}
  \right) = (1\,4\,5)(2\,3\,6) = \lambda_4,
\end{align*}
thus $hH_1h^{-1} = H_2, hH_2h^{-1} = H_3, hH_1h^{-1} = H_4$, and since $\psi(g)$ is a permutation, $h H_4h^{-1} = H_1$. Therefore 
$\psi(g) = \left(
\begin{array}{cccc}
H_1&H_2&H_3&H_4\\
H_2&H_3&H_4&H_1
\end{array}
\right)
$
corresponds to the $4$-cycle $(1\,2 \,3 \,4)$. 

Since $\{(1\,2), (1\,2\,3\,4)\}$ is a set of generators of $S_4$, $S(X)$ is generated by $\psi(g), \psi(h)$, so that $S(X) = \psi(G)$, and $\psi$ is surjective. Moreover, $|G| = |S(X)| = 24$, thus $\psi$ is a bijection, and a group isomorphism:
$$G \simeq S(X) \simeq S_4.$$

\item[(f)] To conclude, using Exercise 2, we obtain
$$S_3\wr S_2 \simeq ((S_3 \wr S_2) \cap A_6) \times S_2  = G \times S_2 \simeq S_4 \times S_2.$$
\bigskip

Note: We have proved in Exercise 7.5.10 that the symmetry group $G_0$ of the cube (or octahedron), is isomorphic to $S_4$. By composition with the indirect isometry $\sigma : v\mapsto -v$, which commutes with all elements in the group, we obtain the full symmetry group, isomorphic to $S_4 \times S_2$.

We have a geometrical description of $G = (S_3\wr S_2) \cap A_6$ by regrouping the opposite faces of a cube in blocs: stick $1$ on a face of a dice, $2$ on the opposite face, and so on (I stuck labels on my Rubik's cube). Then the $24$ rotations of the cube send opposite faces on opposite faces, so that the bloc structure $\{\{1,2\},\{3,4\},\{5,6\}\}$ is preserved by rotations.

We have proved in Exercise 7.5.10 that $G_0$ acts on the 4 long diagonals $D_1,D_2,D_3,D_4$ of the cube, so that $G_0 \simeq S_4$. Each of the four $3$-Sylow of $G_0$ is generated by the rotation with angle $\frac{2\pi}{3}$ around such a  long diagonal. They correspond to the $3$-Sylow $H_1,\ldots,H_4$ of $G$: this was useful for the above description of the $H_i$. Each $3$-Sylow corresponds to a long diagonal, so that $gH_ig^{-1} = H_j$ is equivalent to $\sigma(D_i) = D_j$, where $\sigma$ corresponds to $g$. It remains to find a rotation which acts on these diagonals as some given permutation in $S_4$, such that $(1\,2)$ or $(1\,2\,3\,4)$. The corresponding permutations $g,h \in G $ are given in the text.

\end{proof}

\paragraph{Ex. 14.2.4}
{ \it
Let $A$ and $B$ be solvable permutation groups. Prove that their wreath product $A\wr B$ is also solvable.
}

\medskip

We first proof a lemma, which is not given in Chapter 8.

\medskip

{\bf Lemma.} {\it If $G,H$ are solvable groups, then $G \times H$ is solvable.}

\medskip

{\it Proof of Lemma.} We have subgroups 
\begin{align*}
&\{e\} \subset G_n \subset \cdots \subset G_1 \subset G_0 = G\\
&\{e'\} \subset H_m \subset \cdots \subset H_1 \subset H_0 = H
\end{align*}
such that $G_i$ is normal in $G_{i-1}$ and $G_{i-1}/G_i$ is Abelian for $i=1,\ldots,n$, and $H_i$ is normal in $H_{i-1}$ and $H_{i-1}/H_i$ is Abelian for $i=1,\ldots,m$.

If $n >m$, we can define $H_{m+1} = H_{m+2} =\cdots=H_n=\{e'\}$,  and proceed similarly if $n<m$, so we can assume that $n=m$: 
\begin{align*}
&\{e\} \subset G_n \subset \cdots \subset G_1 \subset G_0 = G\\
&\{e'\} \subset H_n \subset \cdots \subset H_1 \subset H_0 = H
\end{align*}
Then
$$
\{(e,e')\} =G_n\times H_n \subset \cdots \subset G_1\times H_1 \subset G_0\times H_0 = G \times H.
$$
We prove 
$$(G_{i-1}\times H_{i-1}) / (G_i \times H_i) \simeq G_{i-1}/G_i \times H_{i-1}/H_i.$$
Indeed, 
$$
\psi
\left\{
\begin{array}{ccc}
G_{i-1} \times H_{i-1} & \to  & G_{i-1}/G_i \times H_{i-1}/H_i\\
(g,h) & \mapsto & (gG_i, hH_i)
\end{array}
\right.
$$
is surjective, and its kernel is $G_i \times H_i$. This proves our assertion.

Therefore $(G_{i-1}\times H_{i-1}) / (G_i \times H_i)$ is Abelian. Then Exercise 8.1.8 shows that $G\times H $ is solvable.
\qed
\begin{proof} {\it (of Ex.14.2.4.)}
Let 
$$
\varphi
\left\{
\begin{array}{ccc}
A \wr B & \to & A\\
(\tau; \mu_1,\ldots,\mu_k) & \mapsto &\tau.
\end{array}
\right.
$$
By Lemma 14.2.8, $\varphi$ is onto, and its kernel $H = \ker(\varphi)$ is isomorphic to $B^k$. Then $B^k$ is solvable by induction with the above Lemma, so that $H$ is solvable, and $(A \wr B) / H  = (A \wr B)/\ker(\varphi) \simeq A$ is solvable. By Theorem 8.1.4, $A\wr B$ is solvable.
\end{proof}

\paragraph{Ex. 14.2.5}
{\it This exercise will complete the proof of Theorem 14.2.15.
\be
\item[(a)] Let $G_i \to S_p$ be the map defined in (14.9). Prove that it is a group homomorphism and that its image $G'_i \subset S_p$ is transitive and solvable.
\item[(b)] Let $\sigma = (\tau; \mu_1,\ldots,\mu_p)$ and $(\rho;\nu_1,\ldots,\nu_p)$ be as in the proof of Theorem 14.2.15. Thus we have a fixed $j$ such that $i=\tau(j), \nu_i = \theta$, and $\rho(i) = i$. Now let $\gamma = (\tau^{-1} \rho \tau; \lambda_1,\ldots,\lambda_p)$ be as in (14.11). Prove carefully that $\lambda_j = \mu_j^{-1} \theta \mu_j$.
\ee
}

\begin{proof}
\item[(a)] The map $\varphi_i$ defined in  (14.9) is
$$
\varphi_i 
\left\{
\begin{array}{ccc}
G_i & \to &S_p\\
(\tau; \mu_1,\ldots,\mu_p) & \mapsto &\mu_i.
\end{array}
\right.
$$
Let $\lambda = (\tau; \mu_1,\ldots,\mu_p),\lambda' =  (\tau'; \mu_1,\ldots,\mu_p)$ be elements of $G_i$. The definition of $G_i$ shows that $\lambda(R_i) = \lambda'(R_i) = R_i$, so that $\tau(i) =\tau'(i)=i$.  

By (14.7) (see Exercise 1),
$$\lambda \lambda' =(\tau;\mu_1,...,\mu_k)(\tau';\mu_1',...,\mu_k')=(\tau\tau';\mu_{\tau'(1)}\mu_1',...,\mu_{\tau'(k)}\mu_k'),$$ 
therefore, using $\tau'(i) = i$,
\begin{align*}
\varphi_i( \lambda \lambda') &= \mu_{\tau'(i)} \mu'_i\\
&=\mu_i \mu'_i\\
&= \varphi_i(\lambda) \varphi_i(\lambda'),
\end{align*}
thus $\varphi_i$ is a group homomorphism.

\bigskip

Write $G'_i = \varphi_i(G_i) \subset S_p$. We prove first that $G'_i$ is transitive.

Take any $k$ and $l$ in $\{1,\ldots,p\}$. Since $G$ is transitive, there exists some $\lambda = (\tau;\mu_1,...,\mu_k) \in G$ which sends $(i,j)$ on $(i,k)$:
$$(\tau;\mu_1,\ldots,\mu_k)(i,j) = (\tau(i), \mu_i(j)) = (i,k).$$
Then $\tau(i) = i$, so that $\lambda \in G_i$ and $\mu_i = \varphi_i(\lambda) \in G'_i$. Moreover $\mu_i(j) = k$. This proves that $G'_i$ is a transitive subgroup of $S_p$.

Moreover, $G_i$ is a subgroup of the solvable group $G$, thus $G_i$ is solvable. Then $G'_i = \varphi_i(G_i)$ is isomorphic to $G_i/\ker(\varphi_i)$, which is a quotient of a solvable group, thus $G'_i$ is solvable.
\item[(b)] As in the proof of Theorem 14.2.15, let $\sigma = (\tau; \mu_1,\ldots,\mu_p) \in G$ be arbitrary, and fix $j$ between $1$ and $p$. By (14.10) with  $i = \tau(j)$, $\theta \in G'_i = \varphi_i(G_i)$, thus there exists $\lambda = (\rho; \nu_1,\ldots,\nu_p) \in G_i$ such that $\theta = \varphi_i(\lambda)$, thus $\theta = \nu_i$ and $\rho(i) = i$.

Now consider the element $\gamma = \sigma^{-1} \lambda \sigma \in G$. Using (14.6) and (14.7), we obtain
\begin{align*}
\gamma &=  (\tau; \mu_1,\ldots,\mu_p)^{-1} (\rho; \nu_1,\ldots,\nu_p)(\tau; \mu_1,\ldots,\mu_p)\\
&=(\tau^{-1};\mu^{-1}_{\tau^{-1}(1)},\ldots,\mu^{-1}_{\tau^{-1}(p) })(\rho \tau; \nu_{\tau(1)}\mu_1,\ldots,\nu_{\tau(p)} \mu_p)\\
&=(\tau^{-1};\xi_1,\ldots,\xi_p)(\rho \tau; \nu_{\tau(1)}\mu_1,\ldots,\nu_{\tau(p)} \mu_p)\qquad (\text{where }\xi_1 =\mu^{-1}_{\tau^{-1}(1)},\ldots, \xi_p =\mu^{-1}_{\tau^{-1}(p)}) \\
&= (\tau^{-1} \rho \tau; \xi_{(\rho \tau)(1)}\nu_{\tau(1)}\mu_1,\ldots,\xi_{(\rho \tau)(p)}\nu_{\tau(p)}\mu_p)\\
&= (\tau^{-1} \rho \tau; \mu^{-1}_{\tau^{-1}((\rho \tau)(1))}\nu_{\tau(1)}\mu_1,\ldots,\mu^{-1}_{\tau^{-1}((\rho \tau)(p))}\nu_{\tau(p)}\mu_p)\\
&=(\tau^{-1} \rho \tau; \mu^{-1}_{(\tau^{-1}\rho \tau)(1)}\nu_{\tau(1)}\mu_1,\ldots,\mu^{-1}_{(\tau^{-1}\rho \tau)(p)}\nu_{\tau(p)}\mu_p)\\
\end{align*}
If we write $\gamma = (\tau^{-1} \rho \tau; \lambda_1,\ldots,\lambda_p)$, we obtain
$$\lambda_k = \mu^{-1}_{(\tau^{-1}\rho \tau)(k)}\nu_{\tau(k)}\mu_k,\qquad k = 1,\ldots,p,$$
and at the index $j$, using $\theta = \nu_i = \nu_{\tau(j)}$,
\begin{align*}
\lambda_j &= \mu^{-1}_{(\tau^{-1}\rho \tau)(j)}\nu_{\tau(j)}\mu_j\\
&=\mu^{-1}_{(\tau^{-1}\rho \tau)(j)}\theta\mu_j.\\
\end{align*}
Since $i = \tau(j)$ and $\rho(i) = i$, 
$$(\tau^{-1} \rho\tau)(j) = (\tau^{-1} \rho)(i) = \tau^{-1} (i) = j,$$
thus
$$\lambda_j = \mu_j^{-1} \theta \mu_j.$$
\end{proof}



\paragraph{Ex. 14.2.6}
{\it Let $A$ be a subgroup of $S_n$, and let $G$ be any group. Then define $A \wr G$ as in the Mathematical Notes.
\be
\item[(a)] Prove that $A \wr G$ is a group under the multiplication defined in the Mathematical Notes.
\item[(b)] State and prove a version of part (b) of Lemma 14.2.8 for $A \wr G$.
\item[(c)] Prove that $|A \wr G | = |A| |G|^n$ when $G$ is finite.
\ee
}

\begin{proof}
\be
\item[(a)]
Let $G$ be any group and let $A \subset S_n$ be a permutation group. Then set
$$A \wr G = \{ (\tau; g_1,\ldots,g_n)\, | \, \tau \in A,\  g_1,\ldots,g_n \in G\},$$
with an operation on this set defined by
$$(\tau; g_1,\ldots,g_n)(\tau'; g'_1,\ldots,g'_n) = (\tau \tau'; g_{\tau'(1)}g'_1,\ldots,g_{\tau'(n)} g'_n) \in A \wr G.$$
We write $e$ the identity of $G$, and $()$ the identity of $S_n$.

\be
\item[$\bullet$] Let $\lambda = (\tau; g_1,\ldots,g_n), \lambda' = (\tau';g'_1,\ldots,g'_n), \lambda''=(\tau'';g''_1,\ldots,g''_n)$ be elements of $A \wr G$. Then
\begin{align*}
\lambda(\lambda'\lambda'' )&=  (\tau; g_1,\ldots,g_n)(\tau' \tau''; g_{\tau''(1)}g''_1,\ldots,g_{\tau''(n)} g''_n) \\
&= (\tau\tau'\tau''; g_{(\tau'\tau'')(1)} g'_{\tau''(1)} g''_1,\ldots,g_{(\tau'\tau'')(n)} g'_{\tau''(n)} g''_n)\\
(\lambda \lambda')\lambda'' &= (\tau \tau'; g_{\tau'(1)}g'_1,\ldots,g_{\tau'(n)} g'_n)(\tau'';g''_1,\ldots,g''_n)\\
&=(\tau\tau';h_1,\ldots,h_n)(\tau'';g''_1,\ldots,g''_n)\qquad (\text{where } h_k = g_{\tau'(k)}g'_k)\\
&=(\tau\tau'\tau''; h_{\tau''(1)} g''_1,\ldots,h_{\tau''(n)} g''_n)\\
&=(\tau\tau'\tau''; g_{\tau'(\tau''(1))}g'_{\tau''(1)}g''_1,\cdots,g_{\tau'(\tau''(n))}g'_{\tau''(n)}g''_n)\\
&=(\tau\tau'\tau''; g_{(\tau'\tau'')(1)} g'_{\tau''(1)} g''_1,\ldots,g_{(\tau'\tau'')(n)} g'_{\tau''(n)} g''_n)
\end{align*}
thus $\lambda(\lambda'\lambda'' ) = (\lambda \lambda')\lambda'' $, and the law is associative.

\item[$\bullet$] Write $\varepsilon = (();e,\ldots,e) = (\iota; e_1,\ldots, e_n)$, where $\iota = ()$, and $e_k = e, k=1,\ldots,n$. Then
\begin{align*}
\varepsilon \lambda &= (\iota;e_1,\ldots,e_n)(\tau; g_1,\ldots,g_n)\\
&=(\tau; e_{\tau'(1)}g_1,\ldots,e_{\tau'(n)} g_n)\\
&= (\tau; g_1,\ldots,g_n) = \lambda\qquad (\text{since } e_{\tau'(k)} = e)\\
\lambda \varepsilon &= (\tau; g_{\iota(1)} e_1,\ldots,g_{\iota(n)} e_n)\\
&= (\tau; g_1,\ldots,g_n) = \lambda \qquad(\text{since } \iota(k) = k, e_k = e).
\end{align*}
Therefore $\varepsilon =(();e,\ldots,e) $ is the identity of $A \wr G$.

\item[$\bullet$] Set $\mu = (\tau^{-1}; h_1,\ldots,h_n) = (\tau^{-1}; g_{\tau^{-1}(1)}^{-1}, \ldots,g_{\tau^{-1}(n)}^{-1})$, with $h_k = g_{\tau^{-1}(k)}^{-1},\ k=1,\ldots,n$. Then
\begin{align*}
\lambda \mu &= (\tau; g_1,\ldots,g_n)(\tau^{-1}; h_1,\ldots,h_n)\\
&=((); g_{\tau^{-1}(1)} h_1,\ldots,g_{\tau^{-1}(n)} h_n)\\
&=((); g_{\tau^{-1}(1)} g_{\tau^{-1}(1)}^{-1}, \ldots,g_{\tau^{-1}(n)} g_{\tau^{-1}(n)}^{-1})\\
&=((); e,\ldots,e) =\varepsilon\\
\mu \lambda &= (\tau^{-1}; h_1,\ldots,h_n)(\tau; g_1,\ldots,g_n)\\
&=((); h_{\tau(1)} g_1,\ldots,h_{\tau(n)} g_n\\
&=((); g_{\tau^{-1}(\tau(1))}^{-1} g_1,\ldots,g_{\tau^{-1}(\tau(n))}^{-1} g_n)\\
&=((); g_1^{-1}g_1,\ldots, g_n^{-1}g_n) = (();e,\ldots,e) = \varepsilon.
\end{align*}
Therefore every element in $A \wr G$ is invertible.

$A \wr G$ is a group under the multiplication defined in the Mathematical Notes.
\ee
\item[(b)] For the group $A \wr G$ of part (a), where $A \subset S_n$ and $G$ is a group, we show the following lemma:

{\bf Lemma.}
{\it The map
$$
\varphi 
\left\{
\begin{array}{ccc}
A \wr G & \to A\\
(\tau; g_1,\ldots,g_n) & \mapsto \tau
\end{array}
\right.
$$
is a group homomorphism that is surjective and whose kernel is isomorphic to $G^n$.
}

\bigskip

Let $\lambda = (\tau; g_1,\ldots,g_n) , \lambda' = (\tau'; g'_1,\ldots,g'_n)$ be any elements of $A \wr G$. By definition,
$\lambda \lambda' = (\tau \tau'; g_{\tau'(1)}g'_1,\ldots,g_{\tau'(n)} g'_n),$
so that
$$\varphi(\lambda \lambda') = \tau \tau' = \varphi(\lambda) \varphi(\lambda').$$
$\varphi$ is a group homormphism.

If $\tau$ is any element of $A$, then $\varphi(\tau; e,\ldots,e) = \tau$, where $(\tau; e,\ldots,e) \in A \wr G$. Therefore $\varphi$ is surjective.

Moreover $(\tau; g_1,\ldots,g_n) \in \ker \varphi$ if and only if $\tau = ()$, therefore
$$\ker \varphi = \{\iota; g_1,\ldots, g_n)\, |\, (g_1,\ldots,g_n) \in G^n\}, \qquad \text{where } \iota = ().$$

Consider
$$
\psi
\left\{
\begin{array}{ccc}
\ker \varphi & \to & G^n\\
(\iota; g_1,\ldots,g_n) & \mapsto &(g_1,\ldots, g_n)
\end{array}
\right.
$$
Then $\psi$ is bijective (with inverse map $(g_1,\ldots,g_n) \mapsto (\iota,g_1,\ldots,g_n)$). We verify that $\psi$ is a group homomorphism: if $\lambda = (\iota; g_1,\ldots,g_n) , \lambda' = (\iota; g'_1,\ldots,g'_n)$ are elements of $\ker \varphi$, then
\begin{align*}
\psi (\lambda \lambda') &= \psi((\iota; g_1,\ldots,g_n)(\iota; g'_1,\ldots,g'_n))\\
&=\psi(\iota; g_{\iota(1)} g'_1,\ldots, g_{\iota(n)} g'_n)\\
&=\psi(\iota; g_1 g'_1,\ldots, g_n g'_n)\qquad (\text{since } \iota(k) = k)\\
&=(g_1 g'_1,\ldots, g_n g'_n)\\
&=(g_1\ldots,g_n)(g'_1,\ldots,g'_n)\\
&= \psi(\lambda) \psi(\lambda').
\end{align*}
So $\psi$ is an group isomorphism, and $\ker \varphi \simeq G^n$.

\item[(c)] By part (b), since $\varphi$ is a surjective homomorphism,
$$
(A\wr G )/ \ker \varphi \simeq A,
$$
and $\ker \varphi \simeq G^n$. Therefore
$$
|A| = |A \wr G| / |\ker \varphi| = |A\wr G|/ |G|^n,
$$
which proves
$$|A \wr G| = |A| |G|^n.$$
\ee
\end{proof}

\paragraph{Ex. 14.2.7}
{\it Let $A \wr G$ be as in Exercise 6, and let $H$ be the set of all functions
$$\phi:\{1,\ldots,n\} \to G.$$
\be
\item[(a)] Given $\phi,\chi \in H$, define $\phi \chi \in H$ by $(\phi \chi)(i) = \phi(i)\chi(i)$. Prove that this makes $H$ into a group isomorphic to the product group $G^n$.
\item[(b)] Elements of $A \wr G$ can be written $(\tau,\phi)$, where $\phi \in H$. Prove that in this notation, (14.7) becomes
$$(\tau,\phi)(\tau',\phi') = (\tau \tau',((\tau')^{-1} \cdot \varphi) \phi').$$
\item[(c)] $A \subset S_n$ acts on $\{1,\ldots,n\}$. Show that this induces an action of $A$ on $H$ via $(\tau\cdot \phi)(i) = \phi(\tau^{-1}(i))$. Be sure you understand why the inverse is necessary.
\item[(d)] The action of part (c) enable us to define the semidirect product $H \rtimes A$. Using the description of $A \wr G$ given in part (b), prove that the map
$$(\tau, \phi) \mapsto (\tau \cdot \phi, \tau)$$
defines a group isomorphism $A \wr G \simeq H \rtimes A$. This shows that wreath products can be represented as semidirect products.

\ee
}
\begin{proof}
\be
\item[(a)] Consider the two maps
$$
\varphi
\left\{
\begin{array}{ccc}
H & \to &G^n\\
\phi & \mapsto &(\phi(1),\ldots,\phi(n)),
\end{array}
\right.
\qquad
\psi
\left\{
\begin{array}{ccc}
G^n & \to &H\\
(x_1,\ldots,x_n)& \mapsto &
\xi
\left\{
   \begin{array}{ccc}
   \{1,\ldots,n\} & \to & G\\
   i& \mapsto & x_i.
   \end{array}
   \right.
\end{array}
\right.
$$
Then $\psi \circ \varphi = 1_H$ and $\varphi \circ \psi = 1_{G^n}$, therefore $\varphi$ is bijective.

Moreover, for all $(\phi,\chi) \in H$,
\begin{align*}
\varphi(\phi \chi) &= ((\phi\chi)(1),\ldots, (\phi \chi)(n))\\
&=(\phi(1) \chi(1), \ldots, \phi(n) \chi(n))\\
&= (\phi(1),\ldots,\phi(n)) (\chi(1),\ldots,\chi(n))\\
&= \varphi(\phi) \varphi(\chi)).
\end{align*}
Therefore $H \simeq G^n$ via $\varphi$.

\item[(b,c)] If we define $\phi^\tau$, for $\tau \in S_n$ and $\phi \in H$, by $(\phi^\tau)(i) = \phi(\tau(i)),\ i=1,\ldots,n$, we obtain a right action: if $\tau, \tau' \in S_n$, for all $i\in\{1,\ldots,n\}$,
$$((\phi^{\tau})^{\tau'})(i) =  (\phi^\tau)(\tau'(i)) = \phi(\tau(\tau'(i))) = \phi((\tau \tau')(i)) = \phi^{\tau \tau'}(i)),$$
thus $(\phi^{\tau})^{\tau'} = \phi^{\tau \tau'}$. To obtain a left action, we must define, as in part (c),
$$(\tau\cdot \phi)(i) = \phi(\tau^{-1}(i)),\ i=1,\ldots,n.$$
Then
$$(\tau' \cdot (\tau\cdot \phi))(i) = (\tau\cdot \phi)(\tau'^{-1}(i)) = \phi(\tau^{-1}(\tau'^{-1}(i))) = \phi(\tau'\tau)^{-1}(i) = ((\tau' \tau)\cdot \phi)(i),$$
so that $\tau' \cdot (\tau\cdot \phi) = (\tau' \tau)\cdot \phi)$ (and $e \cdot \tau = \tau$).

This is a proof of part (c), and this explains the recurrent and stressful injonction from D.A.Cox ``{\bf Be sure you understand} why the inverse is necessary''.

Using this action for part (b), we define $(\tau, \phi)$ for $\tau \in S_n, \phi \in H = G^{\{1,\cdots,n\}}$,by
$$(\tau, \phi) = (\tau; \phi(1),\ldots,\phi(n)),$$
so that
$$(\tau, \phi) = (\tau; g_1,\ldots,g_n) \iff  \phi(1) = g_1,\ldots,\phi(n) = g_n.$$
If $(\tau, \phi) = (\tau; g_1,\ldots,g_n), (\tau', \phi') = (\tau'; g'_1,\ldots,g'_n)$, then
\begin{align*}
(\tau, \phi) (\tau', \phi') &= (\tau; g_1,\ldots,g_n)(\tau'; g'_1,\ldots,g'_n)\\
&=(\tau \tau'; g_{\tau'(1)}g'_1,\ldots,g_{\tau'(n)} g'_n) \\
&=(\tau \tau'; \phi(\tau'(1))\phi'(1),\ldots,\phi(\tau'(n)) \phi'(n))\\
&=(\tau \tau'; ((\tau')^{-1}\cdot\phi)(1)\phi'(1),\ldots,((\tau')^{-1}\cdot\phi)(n)\phi'(n))\\
&= (\tau \tau', ((\tau')^{-1}\cdot\phi)\phi').
\end{align*}

\item[(d)] Consider the map
$$
\varphi
\left\{
\begin{array}{ccc}
A \wr G & \to & H \rtimes A\\
(\tau,\phi) & \mapsto & (\tau\cdot \phi, \tau).
\end{array}
\right.
$$
If $\psi : H \rtimes A \to A \wr G$ is defined by $\psi(\chi, \tau) = (\tau, \tau^{-1} \cdot \chi)$, then, for all $\tau \in S_n, \phi,\chi \in H$,
\begin{align*}
(\psi \circ \varphi)(\tau, \phi) &= \psi(\tau\cdot \phi, \tau) = (\tau, \tau^{-1}\cdot (\tau \cdot \phi) = (\tau, \phi),\\
(\varphi \circ \psi)(\chi, \tau) &= \varphi(\tau, \tau^{-1} \cdot \chi) = (\tau \cdot (\tau^{-1} \cdot \chi), \tau) = (\chi, \tau). 
\end{align*}
Thus $\psi \circ \varphi = 1_{A \wr G},\  \varphi \circ \psi = 1_{H \rtimes A}$. This proves that $\varphi$ is bijective.

Recall that the binary operation in $ H \rtimes A$ is defined by (6.9):
$$(\phi,\tau) (\phi',\tau') = (\phi(\tau\cdot \phi'), \tau \tau').
$$
We verify that $\varphi$ is a group homomorphism. Note first that, for $\tau \in S_n, \phi \chi \in H$, 
$$\tau \cdot (\phi \chi) = (\tau \cdot \phi)(\tau \cdot \chi).$$
Indeed, for all $i \in \{1,\ldots,n\}$,
\begin{align*}(\tau \cdot (\phi \chi))(i) &= (\phi\chi)(\tau^{-1}(i))\\
&=\phi(\tau^{-1}(i)) \chi(\tau^{-1}(i))\\
&=(\tau \cdot \phi)(i) (\tau \cdot \chi)(i)\\
&=((\tau \cdot \phi)(\tau \cdot \chi))(i).
\end{align*}
Using this rule, we obtain
\begin{align*}
\varphi((\tau,\phi) (\tau',\phi')) &= \varphi(\tau \tau'; ((\tau')^{-1}\cdot\phi)\phi')\\
&= ((\tau\tau')\cdot((\tau')^{-1}\cdot\phi)\phi'),\tau \tau')\\
&=  ((\tau\tau')\cdot((\tau')^{-1}\cdot\phi) ((\tau\tau')\cdot\phi'), \tau \tau')\\
&=((\tau \cdot\phi) ((\tau\tau' )\cdot \phi'), \tau \tau'),\\
\end{align*}
and using the binary operation in $H\rtimes A$,
\begin{align*}
\varphi(\tau, \phi) \varphi(\tau',\phi') &=(\tau\cdot \phi, \tau)((\tau'\cdot \phi', \tau')\\
&= ((\tau \cdot \phi) (\tau \cdot (\tau' \cdot \phi')), \tau \tau')\\
&=((\tau \cdot\phi) ((\tau\tau' )\cdot \phi'), \tau \tau'),
\end{align*}
thus $\varphi((\tau,\phi) (\tau',\phi')) = \varphi(\tau, \phi) \varphi(\tau',\phi') $.
We have proved that $\varphi$ is a group isomorphism, so
$$A \wr G \simeq H \rtimes A = G^{\{1,\ldots,n\} }\rtimes A.$$ 
Wreath products can be represented by semidirect products.
\ee
\end{proof}

\paragraph{Ex. 14.2.8}
{\it
The goal of this exercise is to relate Definition 14.2.2 to Galois's definition of not primitive. Let $f \in F[x]$ be monic, separable, and irreducible with splitting field $F \subset L$. Also assume that $f$ is imprimitive with blocks of roots given by $R_1,\ldots,R_m$, where each block has $n$ elements (thus $\deg(f) = mn$). Let $f_i$ be the monic polynomial whose roots are the elements of $R_i$, and let $K \subset L$ be the fixed field of $$\{\sigma \in \Gal(L/F)\, | \, \sigma(R_i) = R_i \text{ for all } i\}.$$
\be
\item[(a)] Show that $f = \prod_{i=1}^m f_i$ and that $f_i \in K[x]$ for all $i$.
\item[(b)] In Galois' definition, $K$ is obtained by adjoining the roots of a separable polynomial of degree $m$. In modern terms, Galois wants $F \subset K$ to be Galois extension such that $\Gal(K/F)$ (*) is isomorphic to a subgroup of $S_m$. Prove that the field $K$ defined in part (a) has these properties.
See Exercise 14 for some examples.

[(*) misprint in Cox.]
\ee
}

\begin{proof}
\be
\item[(a)] By Definition 14.2.2, $R = R_1 \cup \cdots \cup R_m$ (disjoint union) is the set of roots of $f$. Since $f$ is separable, by definition of $f_i$,
$$f = \prod_{\alpha \in R} (x - \alpha) = \prod_{i=1}^m \prod_{\alpha \in R_i} (x - \alpha) = \prod_{i=1}^m f_i.$$
Let 
$$G = \{\sigma \in \Gal(L/F)\, | \, \forall i \in \gcro 1, m \dcro, \ \sigma(R_i) = R_i \}.$$
If $G_i = \{\sigma \in \Gal(L/F)\, | \, \sigma(R_i) = R_i \}$ for $i = 1,\ldots,n$, then $G = \bigcap\limits_{i=1}^m G_i$.

Each $G_i$ is a subgroup of $\Gal(L/F)$: $e(R_i) = R_i$, and if $\sigma, \tau \in R_i$, then  $(\sigma \tau)(R_i) = \sigma(R_i) = R_i$ and $R_i = \sigma^{-1}(R_i)$. Therefore $G = \bigcap\limits_{i=1}^m G_i$ is a subgroup of $\Gal(L/F)$.

Let $K = L_G$ be the fixed field of $G$. By the Galois correspondence, $G = \Gal(L/K)$.

If $\sigma \in G_i$, since $\sigma(R_i) = R_i$, where le restriction of $\sigma$ to $R_i$ is bijective, then 
$$\sigma \cdot f_i = \prod_{\alpha \in R_i} (x - \sigma(\alpha)) = \prod_{\beta \in R_i} (x - \beta) = f_i, \qquad (\beta = \sigma(\alpha)).$$
Therefore, if $\sigma \in G = \bigcap\limits_{i=1}^m G_i$, then for all $i \in \{1,\ldots,m\}$, $\sigma\cdot f_i = f_i$. The coefficients of $f_i$ are in the fixed field $K$ of $G$, so that
$$f_i\in K[x],\ i=1,\ldots,m.$$
To give a first example, $f = x^4 - 2$ is imprimitive with blocks $$R_1 =\{\sqrt[4]{2},-\sqrt[4]{2}\},R_2 = \{i \sqrt[4]{2},-i\sqrt[4]{2}\}.$$

If $\tau,\sigma$ are defined by $\tau(\sqrt[4]{2}) = \sqrt[4]{2}, \tau(i) = -i$, and $\sigma(\sqrt[4]{2}) = i \sqrt[4]{2}, \sigma(i) = i$, then
$$\Gal(L/F) = \{e,\sigma,\sigma^2,\sigma^3, \tau, \sigma \tau, \sigma^2 \tau, \sigma^3\tau\} \simeq D_8.$$
Here $G = G_1 = G_2 = \{e,\sigma^2,\tau,\sigma^2 \tau\}$, and $K = L_G = \Q(\sqrt{2})$ (see Ex. 6.3.2 and Ex. 7.3.3).

We verify $f_1(x) = x^2 -\sqrt{2}, f_2(x) = x^2 + \sqrt{2} \in K[x]$.

\item[(b)] We prove that $G$ is a normal subgroup of $\Gal(L/F)$. 

Let $\lambda \in \Gal(L/F)$, and $\sigma \in G$. Since $f$ is imprimitive, $\lambda \in \Gal(L/F)$ permutes the blocks $R_i$: there exists $\tau \in S_m$ such that
$$\lambda(R_i) = R_{\tau(i)},\ i=1,\ldots,m.$$
Let $j$ be any fixed index in $\{1,\ldots,m\}$, and $i$ such that $\tau(i) = j$. Since $\sigma \in G \supset G_i$, $\sigma(R_i) = R_i$, thus
$$(\lambda \sigma \lambda^{-1})(R_j) = (\lambda \sigma)(R_i) = \lambda(R_i) = R_j.$$
Since this is true for all $j \in \{1,\ldots,m\}$, $\lambda \sigma \lambda^{-1} \in G$. This proves that $G$ is a normal subgroup of $\Gal(L/F)$. Therefore $F \subset K$ is a Galois extension (Theorem 7.2.5). 

Now we prove that $\Gal(K/F)$ is isomorphic to a subgroup of $S_m$.

Since $f$ is imprimitive with blocks of roots given by $R_1,\ldots,R_m$, for each $\sigma \in \Gal(L/F)$, there exists $\tau \in S_m$ such that $\sigma(R_i) = R_{\tau(i)},\ i=1,\ldots,m$.
Consider the map $\varphi$ sending $\sigma$ to $\tau$:
$$
\varphi
\left\{
\begin{array}{ccl}
\Gal(L/F) & \to & S_m\\
\sigma & \mapsto &\tau : \forall i \in \gcro 1,m \dcro,\ \sigma(R_i) = R_{\tau(i)}.
\end{array}
\right.
$$
Then , for all $\sigma \in \Gal(L/F)$,
$$\varphi(\sigma) = ()  \iff \forall i \in \gcro 1,m \dcro,\ \sigma(R_i) = R_i \iff \sigma \in G.$$
Therefore, $\ker(\varphi) = G$, and by the Galois correspondence (see part (a)) $G = \Gal(L/K)$, so that
$$\ker(\varphi) = G = \Gal(L/K).$$
Then, by Theorem 7.3.2, since $K$ is Galois over $F$,
$$ S_m \supset \mathrm{im}(\varphi) \simeq \Gal(L/F)/ \ker(\varphi) = \Gal(L/F)/\Gal(L/K) \simeq \Gal(K/F).$$
Therefore $\Gal(K/F)$ is isomorphic to the subgroup $\mathrm{im}(\varphi)$ of $S_m$.
\ee
\end{proof}

\paragraph{Ex.14.2.9}
Assume that $G \subset S_n$ is transitive and Abelian.
\be
\item[(a)] Prove that $|G| = n$ by considering the isotropy subgroups of $G$.
\item[(b)] Prove that $G$ is primitive if and only if $|G|$ is prime.
\ee
Thus a transitive Abelian permutation group is imprimitive unless it is cyclic of prime order.

\begin{proof}
\item[(a)] $G \subset S_n$ acts on $\{1,\ldots,n\}$ by the action defined by $\sigma\cdot k = \sigma(k),\ \sigma \in G, k \in \{1,\ldots,n\}$.

Consider the isotropy group $G_1$ of $1$: $G_1 = \{\sigma \in G\, |\, \sigma(1) = 1\}$. Let $\sigma$ be any permutation in $G_1$, and let $i$ be any element in $\{1,\ldots,n\}$. Since $G$ is transitive, there exists $\tau \in G$ such that $\tau(1) = i$. By hypothesis, $G$ is Abelian, therefore
$$\sigma(i) = (\sigma \tau)(1) = (\tau \sigma)(1) = \tau(1) = i.$$
Since this is true for all $i\in \{1,\ldots,n\}$, $\sigma = e$. This proves $G_1 = \{e\}$.

Moreover, since $G$ is transitive, the orbit of $1$ is $G\cdot 1 =\{1,\ldots,n\}$, thus $|G\cdot 1| = n$.

By the Fundamental Theorem of Group Actions,
$$|G\cdot 1 | = (G:G_1) = |G|,$$
thus $|G| = n$.
\item[(b)] By Lemma 14.2.7, if $G \subset S_n$ is transitive and imprimitive, then $n =kl,\ k>1,l>1$, is composite. Thus, if $n$ is prime, a transitive subgroup of $S_n$ is primitive.

Conversely, let $G$ be a transitive Abelian subgroup of $S_n$, where $n>1$ is composite. By part (a), $|G| = n$. We must prove that $G$ is imprimitive.

By the Kronecker's Theorem on the structure of Abelian groups, 
$$G \simeq C_{n_1} \times \cdots \times C_{n_r}$$
is a product of cyclic groups.

Therefore, either $G$ is cyclic of order $n$, or $G = HK \simeq H\times K, H\ne\{e\},K \ne \{e\}$ is a direct product of two non trivial subgroups (take for instance $H \simeq C_{n_1}, K \simeq C_{n_2} \times \cdots \times C_{n_r}$). We will deal with these two cases.

$\bullet$ Case 1.  We assume that $G \simeq C_n$ is cyclic, where $n = ml, \ m>1, l>1$. Then $G = \langle \sigma \rangle$, where the permutation $\sigma$ has order $n$. Take
\begin{align*}
R_1 &= \{1,\sigma^m(1),\ldots, \sigma^{(l-1)m}(1)\},\\
R_2 &= \{\sigma(1), \sigma^{m+1}(1), \ldots,\sigma^{(l-1)m + 1}(1)\} = \sigma(R_1),\\
\cdots\\
R_m&=\{\sigma^{m-1}(1),\sigma^{m + m-1}(1), \ldots,\sigma^{lm - 1}(1)\} = \sigma^{m-1}(R_1).
\end{align*}
Since $G$ is transitive, $$R_1\cup\cdots\cup R_m =\{1,\sigma(1),\ldots,\sigma^{n-1}(1)\} = G\cdot 1 = \{1,\ldots,n\}.$$ Moreover, if $\tau \in G$, then $\tau = \sigma^{j},\ j=0,\ldots,n-1$, thus, using $\sigma^n = e$, if $k$ is the remainder of $i+j-1$ modulo $n$,
$$\tau(R_i) = (\sigma^j \sigma^{i-1})(R_1) = \sigma^{i+j-1}(R_1) = \sigma^{k}(R_1) = R_{k+1}.$$

This proves that $G$ is imprimitive, with blocks $R_1,\ldots,R_m$.

\bigskip

$\bullet$ Case 2. Now, assume that $G = HK \simeq H \times K, |H| = l>1, |K| = m>1$. Then $n = ml$. Write
\begin{align*}
H &= \{\sigma_1 =e,\ldots,\sigma_l\},\\
K &= \{\tau_1 = e,\ldots,\tau_m\},
\end{align*}
and take
\begin{align*}
R_1 &=\{ (\sigma_1 \tau_1)(1), \ldots, (\sigma_l \tau_1)(1) \},\\
R_2 &=\{ (\sigma_1 \tau_2)(1),\ldots, (\sigma_l \tau_2)(1)\},\\
\cdots\\
R_m&= \{ (\sigma_1 \tau_m)(1), \ldots, (\sigma_l \tau_m)(1)\}.
\end{align*}
Since $G = HK$, every permutation $\lambda \in G$ is a product $\lambda = \sigma_i \tau_j,\ 1\leq i\leq l, 1 \leq j \leq m$, and since $G$ is transitive,
$$R_1 \cup \cdots \cup R_m = G\cdot 1 = \{1,\ldots,n\}.$$
Take $\lambda = \sigma_i \tau_j \in G$, and $R_k = \{\sigma_u \tau_k, 1\leq u \leq l\}$. Then $\tau_j \tau_k = \tau_r$ for some fixed $r \in \{1,\ldots,m\}$. Since $G$ is Abelian,
$$\lambda(R_k) = \{(\sigma_i \sigma_u \tau_j \tau_k)(1) , 1 \leq u \leq l\} = \{(\sigma_i \sigma_u \tau_r)(1) , 1 \leq u \leq l\} =  \{(\sigma_v \tau_r)(1) , 1 \leq v \leq l\} = R_r,$$
because the map $H \to H$, $\sigma_u \mapsto \sigma_i \sigma_u$ is bijective.

This proves that $G$ is imprimitive, with blocks $R_1,\ldots,R_m$. 

To conclude, $G$ is primitive if and only if $|G|$ is prime (or $|G| = 1$). Thus a non trivial transitive Abelian permutation group is imprimitive unless it is cyclic of prime order.

\bigskip

Examples: $G = \{(), (1\,2)(3\,4), (1\,3)(2\,4),(1\,4)(2\,3)\}$ is a transitive Abelian subgroup of $S_4$, and an example of Case 2. If $H, K$ are two distinct subgroups of $G$ with order 2, then $G = HK \simeq C_2 \times C_2$. We can take $R_1 = \{1,2\}, R_2 = \{3,4\}$, but we can also take $R'_1 = \{1,3\},R'_2 = \{2,4\}$. G is imprimitive with blocks $R_1,R_2$, or with blocks $R'_1,R'_2$.

$G' = \langle (1\,2\,3\,4) \rangle$ is a another transitive Abelian subgroup of $S_4$, and an example of Case 1. This times, we can take only $R_1 = \{1,3\},R_2 = \{2,4\}$.
\end{proof}

\paragraph{Ex.14.2.10} {\it Let $\Phi_p(x)$ be the cyclotomic polynomial whose roots are the primitive $p$th roots of unity, where $p$ is prime. We know that $\Phi_p(x)$ is irreducible of degree $p-1$. In the quotation given in the Historical Notes, Galois asserts that $\Phi_p(x)$ is imprimitive.
\be
\item[(a)]
Prove Galois's claim for $p>3$ using Exercise 9.
\item[(b)] Explain why we need to assume that $p>3$ in part (a).
\ee
}

\begin{proof}
\item[(a)]
We know that the splitting field of $\Phi_p(x)$ over $\Q$ is $L = \Q(\zeta_p)$ (where $\zeta_p = e^{\frac{2i\pi}{p}}$), and that $\Gal(L/\Q) \simeq (\Z/p\Z)^*$, via the isomorphism 
$$
\left\{
\begin{array}{ccl}
\Gal(L/\Q) & \to & (\Z/p\Z)^*\\
\sigma & \mapsto & a : \sigma(\zeta_p) = \zeta_p^a.
\end{array}
\right.
$$
Therefore, $\Gal(L/\Q)$ is Abelian, and even cyclic with order $p-1$. Let $\Gal(L/\Q) \simeq G  \subset S_{p-1}$. If $p>3$, then $p-1$ is not prime, and Exercise 9 prove that  $G$ is imprimitive, so that $\Phi_p(x)$ is imprimitive.

\item[(b)] If $p = 3$, $p-1 = 2$ is prime, and $(\Z/3\Z)^* = \{1 ,-1 \}$ is cyclic of prime order. Moreover $\Phi_3(x) = x^2+x + 1$ is not imprimitive, as every polynomial of degree 2: by Definition 14.2.2, if $f$ is imprimitive, since $k>1,l>1$, $\deg(f) \geq |R_1|+|R_2| >2$.

If $p = 2$, $(\Z/2\Z)^* = \{1\}$, and $\Phi_2(x) = x+1$. Since $\deg(\Phi_2) = 1$, $\Phi_2(x)$  is not imprimitive.
 
\end{proof}


\paragraph{Ex. 14.2.11}{\it
Given a prime $p$, let $C_p \subset S_p$ be the cyclic subgroup generated by the $p$-cycle $(1\,2\cdots p)$. As explained in the text, this gives the wreath product $C_p \wr C_p \subset S_{p^2}$. Prove that $C_p \wr C_p$ is a $p$-Sylow subgroup of $S_{p^2}$.
}
\begin{proof}
By Lemma 14.2.8, we know that $C_p \wr C_p$ is a subgroup of $S_p \wr S_p$, which may be viewed as $S_{p^2}$.

Exercise 6 shows that $|C_p \wr C_p| = |C_p| |C_p|^p = p^{p+1}$.

Moreover $|S_{p^2}| = (p^2)!$, and, if $\nu_p(n)$ is the exponent of $p$ in the prime factorization of $n!$, then
$$\nu_p(n!) =\left \lfloor \frac{n}{p} \right \rfloor +  \left \lfloor \frac{n}{p^2}\right  \rfloor + \cdots + \left \lfloor \frac{n}{p^k} \right \rfloor + \cdots.$$
(see for instance Ex. 2.6 in Ireland and Rosen)
 
Therefore\begin{align*}
\nu_p((p^2)!) &=\left \lfloor \frac{p^2}{p} \right \rfloor +  \left \lfloor \frac{p^2 }{p^2}\right  \rfloor + \cdots + \left \lfloor \frac{p^2}{p^k} \right \rfloor + \cdots\\
&=\left \lfloor \frac{p^2}{p} \right \rfloor +  \left \lfloor \frac{p^2 }{p^2}\right  \rfloor\\
&=p + 1.
\end{align*}
Therefore $p^{p+1}$ is the maximal power of $p$ which divides $(p^2)! = S_{p^2}$, so that $C_p \wr C_p$ is a $p$-Sylow subgroup of $S_{p^2}$.
\end{proof}


\paragraph{Ex. 14.2.12}{\it Let $f$ be an irreducible imprimitive polynomial of degree 6,8 or 9 over a field $F$ of characteristic 0. Prove that $f$ is solvable by radicals over $F$.
}

\begin{proof}
Write $L$ the splitting field of $f$. Since the characteristic of $F$ is 0, the irreducible polynomial $f$ is separable, so $f$ is separable, irreducible and imprimitive.  By Corollary 14.2.10, $G = \Gal(L/F)$ is isomorphic to a subgroup of $S_k \wr S_l$, where $n = kl$ is a nontrivial factorization. The only nontrivial factorizations of 6,8 or 9 are
$$6 = 2 \times 3 = 3 \times 2, \qquad 8 = 2 \times 4 = 4 \times 2, \qquad 9 = 3 \times 3.$$
Thus $G$ is isomorphic to a subgroup of the list
$$S_2 \wr S_3, S_3\wr S_2,S_4 \wr S_2,S_2 \wr S_4, S_3\wr S_3,$$
whose cardinalities are
\begin{align*}
|S_2 \wr S_3| &= 2!3^2 = 2 \times 3^2,\\
 |S_3\wr S_2| &= 3!2^3 = 2^4 \times 3,\\
 |S_4 \wr S_2| &= 4!2^4 = 2^7 \times 3,\\
 |S_2 \wr S_4| &= 2!4^2 = 2^5,\\
 |S_3\wr S_3| &=3!3^3 = 2 \times 3^4. 
\end{align*}
So $S_k \wr S_l$ has only two prime factors 2 and 3. By Burnside's Theorem (Theorem 8.1.8), $S_k \wr S_l$ solvable for these values of $k,l$, thus the subgroup $G$ is solvable. Since the characteristic of $F$ is 0, this proves that $f$ is solvable by radicals over $f$.
\end{proof}



\paragraph{Ex. 14.2.13}{\it Let $f = x^6 + bx^3 + c \in F[x]$ be irreducible, where $F$ has characteristic different from 2 or 3. We will study the size of the Galois group of $f$ over $F$.
\be
\item[(a)] Show that $f$ is separable. So we can think of the Galois group as a subgroup of $S_6$.
\item[(b)] Show that $x^6+bx^3 + c$ is imprimitive and that its Galois group lies in $S_2 \wr S_3$. Also show that $|S_2 \wr S_3| = 72$. Thus the Galois group has order $\leq 72$.
\item[(c)] Let $F \subset L $ be the splitting field of $f$ over $F$. Use the Tower Theorem to show that $[L:F] \leq 36$. Hence the Galois group has order at most $36$.
\ee
Using Maple or Sage, one can show that the Galois group of $x^6 + 2x^3 - 2$ over $\Q$ has order $36$ and hence is as large as possible.
}
\begin{proof}
\item[(a)] Since $F$ has characteristic different from 2 or 3, $f' = 6x^5 + 3bx^2 \ne 0$. By hypothesis, $f$ is irreducible, thus any factor of $f$ is associate to $f$ or $1$. Since $f' \ne 0$, $\gcd(f,f')$ divides $f'$, thus $\deg(\gcd(f,f')) \leq \deg(f') = 5$, and $\gcd(f,f')$ is a factor of $f$, which cannot be associate to $f$, therefore $\gcd(f,f') = 1$. This proves that $f$ is separable.
\item[(b)]  If $\alpha$ is a root of $f$ in $L$, then $ \lambda = \alpha^3 \in L $ is a root of $g = x^2 +bx+c$. Let $\mu = -b-\lambda \in L$. Then $\mu$ is a root of $g$: 
 $$\mu^2+b\mu+c = (-b-\lambda)^2 + b(-b-\lambda)+c = \lambda^2 + b\lambda + c = 0.$$
Moreover $\lambda \ne \mu$, otherwise $b = -2\lambda, c = -\lambda^2 - b\lambda = \lambda^2$, and $g = (x-\lambda)^2$, where $\lambda = -b/2 \in F$, so that  $f =g(x^3) = (x^3-\lambda)^2$ would not be irreducible over $f$. Therefore
$$g = (x- \lambda)(x-\mu),\qquad \lambda\ne \mu.$$
Write $K = F(\lambda,\mu)$. Then $F\subset K \subset L$ is an intermediate field which is a splitting field of $g$ over $F$.
If $\lambda \in F$, then $\mu \in F$ and $f = g(x^3) = (x^3 - \lambda)(x^3 - \mu)$ would not be irreducible over $F$. Therefore $\lambda \not \in F,\mu \not \in F$, $g$ is irreducible over $F$. $K$ is the splitting field of the separable polynomial $g$, thus $F \subset K$ is a Galois extension, where $F \subsetneq K \subset L$.

Moreover,
$$f = g(x^3) = (x^3 - \lambda)(x^3 - \mu) = f_1 f_2,$$
where $f_1 = x^3 - \lambda, f_2 = x^3 - \mu \in K[x]$.
Therefore three roots $\alpha,\alpha',\alpha''$ of $f$ are the roots of $x^3 -\lambda$, and three other roots $\beta,\beta',\beta''$ of $f$ are the roots of $x^3 - \mu$ :
$$\alpha^3 = \alpha'^3 = \alpha''^3 = \lambda,\qquad \beta^3 = \beta'^3 = \beta''^3 = \mu.$$
This gives the blocks
$$R_1 = \{\alpha, \alpha', \alpha''\},\qquad R_2 = \{\beta,\beta',\beta''\}.$$
If $\sigma \in \Gal(L/F)$, then $g = \sigma\cdot g = (x -\sigma(\lambda))(x - \sigma(\mu))$, thus $\{\lambda, \mu\} = \{\sigma(\lambda), \sigma(\mu)\}$: $\sigma$ fixes $\lambda, \mu$, or exchanges $\lambda, \mu$. Since $\alpha,\beta,\gamma$ are the roots of $x^3 - \lambda$, $\sigma(\alpha), \sigma(\beta), \sigma(\gamma)$ are the roots of $x^3 - \sigma(\lambda)$, thus $\sigma(R_1) = R_1$ or $\sigma(R_1) = R_2$, and similarly $\sigma(R_2) = R_1$ or $R_2$. This proves that $f$ is imprimitive, with blocks $R_1,R_2$, and $\Gal(L/F)$ is isomorphic to a subgroup of $S_2 \wr S_3$(Corollary 14.2.10), whose order is $2(3!)^2 = 72$ (see Ex. 6 (c)).

\item[(c)] We don't know if $F$ or $K$ contains the cubic roots of unity, but $L$ does: since $\alpha, \alpha'$ are two distinct roots of $x^3 - \lambda$ (where $\alpha \ne 0$, otherwise $\lambda = 0 \in F$), then $(\frac{\alpha'}{\alpha})^3 = 1$. If we write $\omega = \frac{\alpha'}{\alpha}$, then $\omega \in L$ and $\omega^3 = 1, \omega\ne 1$, thus $1,\omega, \omega^2$ are three distinct roots of $x^3 - 1$, and $1 + \omega + \omega^2 = 0$, so that $[K(\omega) : K ] = 1$ or $2$. 

Then the six roots of $f$ are 
$$\alpha_1 = \alpha, \alpha_2 = \omega \alpha, \alpha_3 = \omega^2 \alpha, \alpha_4 = \alpha', \alpha_5 = \omega \alpha', \alpha_6 = \omega^2 \alpha'.$$
Therefore
$$L = F(\alpha,\omega \alpha, \omega^2 \alpha, \beta, \omega \beta, \beta^2) = F(\omega,\alpha,\beta) = K(\omega, \alpha, \beta). $$

Consider the chain of fields
$$F \subset K = F(\lambda, \mu) \subset  K(\omega) \subset K(\omega,\alpha) \subset L = K(\omega, \alpha, \beta).$$

\begin{center}
\begin{tikzpicture}
    \node (L) at (4,8) {$L = K(\omega,\alpha,\beta)$};
    \node (A3) at (1,6) {$K(\omega,\alpha)$};
    \node (t23) at (7,6) {$K(\omega,\beta)$};
    \node (id) at (4,4) {$K(\omega)$};
    \node (K) at (4,2) {$K = F(\lambda,\mu)$};
    \node (F) at (4,0) {$F$};
    \draw[<-] (L) edge (A3)   edge (t23);
    \draw[->] (id) edge (A3)  edge (t23);
    \draw[->] (K) edge (id);
     \draw[->] (F) edge (K);
\end{tikzpicture}
\end{center}


$\bullet$ Since $\lambda,\mu$ are the roots of the irreducible polynomial $f$, $K=F(\lambda,\mu)$ is a quadratic extension of $F$, so $[K:F] = 2$.

$\bullet$ Since $\omega^2 + \omega + 1 = 0$, $[K(\omega) : K] \leq 2$.

$\bullet$ Since $\alpha^3 - \lambda = 0$, where $\lambda \in K \subset K(\omega)$, $[K(\omega,\alpha) : K(\omega)] \leq 3$.

$\bullet$ Since $\beta^3 - \mu = 0$, where $\mu \in K \subset K(\omega, \alpha)$,  $[K(\omega, \alpha,\beta) : K(\omega, \alpha)] \leq 3$.

By the Tower Theorem,
$$[L:K] = [K(\omega, \alpha,\beta) : K] = [K(\omega, \alpha,\beta) : K(\omega, \alpha)] [K(\omega,\alpha) : K(\omega)][K(\omega) : K]  \leq 3\times3 \times2 \times 2 = 36.$$
Therefore,
$$|\Gal(L/K) | = [L:K] \leq 36.$$
\end{proof}
Note : Some permutations of $S_2\wr S_3$ can't lie in the Galois group $G \subset S_6$ of $f$, for instance if the transposition $(4\,5)$ corresponds to $\sigma \in \Gal(L/F)$, given by
$$
\left(
\begin{array}{cccccc}
\alpha & \omega\alpha &\omega^2\alpha & \beta & \omega \beta & \omega^2 \beta\\
\alpha & \omega\alpha &\omega^2\alpha & \omega\beta &  \beta & \omega^2 \beta
\end{array}
\right),
$$
then $\sigma(\alpha) = \alpha$ and $\sigma(\omega \alpha) = \omega \alpha$ show that $\sigma(\omega) = \omega$, and $\sigma(\beta) = \omega \beta$ implies $\sigma(\omega \beta) = \omega^2 \beta \ne \beta$. This contradiction proves that $(4\,5) \not \in G$ (but $(4\, 5) \in S_3 \wr S_2 \subset S_6$). More generally, all odd permutations are impossible, so that $G$ is a subgroup of $(S_2\wr S_3) \cap A_6$.




\paragraph{Ex. 14.2.14} {\it Here are some examples to illustrate Galois's definition of imprimitive. We will use the notation of Exercise 8. Let $F$ be a field of characteristic different from 2 or 3.
\be
\item[(a)] Let $f = x^6 + bx^4+cx^2 +d \in F[x]$ be irreducible with splitting field $F \subset L$. Show that the splitting field of $x^3+bx^2+cx+d$ gives an intermediate field $F\subset K \subset L$ such that $F \subset K$ is Galois and $f=f_1f_2f_3$, where $f_i \in K[x]$ has degree 2 for $i=1,2,3$. Also explain how $K$ relates to the field $K$ constructed in Exercise 8.
\item[(b)] Work out the analogous theory when $f = x^6 + bx^3+c \in F[x]$ is irreducible.
\ee
}
\begin{proof}

\item[(a)] By hypothesis, $f = x^6 + bx^4+cx^2 +d$ is irreducible. Since the characteristic of $F$ is different from 2 or 3, $f' = 6x^5+\cdots \ne 0$, thus $\gcd(f,f') =1$, and $f$ is separable.

If $\alpha$ is a root of $f$, then $-\alpha$ is a root of $f$. Moreover $\alpha \ne 0$, otherwise $d= 0$ and $f$ would not be irreducible, therefore $\alpha \ne -\alpha$. Since $f$ is separable, the roots of $f$ can be partitioned into three blocks
$$R_1 = \{\alpha,-\alpha\},\qquad R_2 = \{\beta,-\beta\},\qquad R_3 = \{\gamma, -\gamma\}.$$
If $\sigma \in \Gal(L/F)$ and if $\lambda \in R_i$ is a root of $f$, then $\sigma(\lambda) \in R_j$ for some index $j$, and $\sigma(-\lambda) = - \sigma(\lambda) \in R_j$, thus $\sigma(R_i) = R_j$. Therefore $f$ is imprimitive, with blocks $R_1,R_2,R_3$. By Corollary 14.2.10, $\Gal(L/F)$ is isomorphic to a subgroup of $S_3 \wr S_2$.

Since $\alpha,-\alpha, \beta,-\beta,\gamma,-\gamma$ are the distinct roots of $f$, then $\alpha^2,\beta^2,\gamma^2$ are distinct and they are the roots of $g = x^3 +bx^2+cx+d$. Therefore
$$g =  x^3 +bx^2+cx+d = (x-\alpha^2)(x-\beta^2)(x-\gamma^2),$$
so that a splitting field $K$ of $g$ over $F$ is 
$$K = F(\alpha^2,\beta^2,\gamma^2),\qquad F \subset K \subset L.$$
Since $K$ is the splitting field of the separable polynomial $g$, $F \subset K$ is a Galois extension. Note that $g$ is irreducible over $F$, otherwise any non trivial factorization of $g$ over $F$ gives a factorisation of $f$ over $F$. Therefore $K \ne F$.

Moreover,
$$f = g(x^2) = (x^2-\alpha^2)(x^2-\beta^2)(x^2-\gamma^2) = f_1f_2f_3,$$
where $f_1 = x^2 - \alpha^2, f_2 = x^2 - \beta^2,f_3 = x^2 -\gamma^2 \in K[x]$. This proves the first assertion of part (a).

\bigskip

It remains to prove that $K$ is the fixed field of the subgroup $G$ of $\Gal(L/F)$ defined in Exercise 8:
$$G = \{\sigma \in \Gal(L/F)\, | \, \forall i \in \gcro 1, 3 \dcro, \ \sigma(R_i) = R_i \}.$$
To give an explicit description of $G$, note that
\begin{align*}
\sigma \in G &\iff \sigma(\alpha) = \pm \alpha,\ \sigma(\beta) = \pm \beta, \ \sigma(\gamma) = \pm \gamma\\
&\iff \sigma(\alpha^2) = \alpha^2,\ \sigma(\beta^2) = \beta^2, \sigma(\gamma^2) = \gamma^2\\
&\iff \forall \lambda \in K,\ \sigma(\lambda) = \lambda,
\end{align*}
where the last equivalence is explained by the fact that every $\lambda \in K = F(\alpha^2,\beta^2,\gamma^2)$ is a polynomial in $\alpha^2,\beta^2,\gamma^2$.

This proves that every element of $K$ is fixed by every $\sigma \in G$, thus $K \subset L_G$, where $L_G$ is the fixed field of $G$.

Since the Galois correspondence is order reversing, $K \subset L_G$ implies $\Gal(L/K) \supset G$. To prove the inverse inclusion, take $\sigma \in \Gal(L/K)$. Then $\sigma(\lambda) = \lambda$ for all $\lambda \in K$, and the preceding equivalence shows that $\sigma \in G$. Thus $\Gal(L/K) = G$. Applying the Galois correspondence once more, the fixed fields of $\Gal(L/K)$ and $G$ are equal, that is
$$K = L_G.$$
$K$ is the fixed field of $G =  \{\sigma \in \Gal(L/F)\, | \, \sigma(R_1) = R_1,\sigma(R_2) = R_2, \sigma(R_3) = R_3 \}$.

\item[(b)] We have proved in Exercise 13 that $f$ is imprimitive, with blocks
$$R_1 = \{\alpha, \beta, \gamma\},\qquad R_2 = \{\alpha',\beta',\gamma'\}.$$
With the same notations as in Exercise 13 and 8, we have 
$$G = \{\sigma \in \Gal(L/F) \,|\, \sigma(R_1) = R_1,\ \sigma(R_2) = R_2\}.$$
Since $\alpha,\beta,\gamma$ are the roots of $x^3 -\lambda$, and $\alpha',\beta',\gamma'$ the roots of $x^3 -\mu$, where $\{\lambda, \mu\} = \{\sigma(\lambda), \sigma(\mu)\}$, then, for all $\sigma \in \Gal(L/F)$,
\begin{align*}
\sigma \in G &\iff \sigma(\{\alpha,\beta, \gamma\} = \{\alpha,\beta,\gamma\},\  \sigma(\{\alpha',\beta', \gamma'\}) = \{\alpha',\beta',\gamma'\}\\
&\iff \sigma(\lambda) = \lambda,\ \sigma(\mu) = \mu\\
&\iff \forall \xi \in K,\ \sigma(\xi) \in K.
\end{align*}
This proves as in part (a) that $K = L_G$.
\end{proof}


\paragraph{Ex. 14.2.15}{\it
Let $G \subset S_n$ be transitive. Prove that $G$ is primitive if and only if the isotropy subgroups of $G$ are maximal with respect to inclusion.
}

\begin{proof}

Let $i$ be a fixed integer in $\{1,\ldots,n\}$. Since $G$ is transitive, $G\cdot i = \{1,\ldots,n\}$. The Fundamental Theorem of group actions gives
$n = |G\cdot i| = (G:G_i),$
thus $|G_i| n = |G|$.

Given a subgroup $G_i \subset H \subset G$, let $\{\tau_1=e,\ldots,\tau_m\}$ be a complete system of representatives of the left cosets $\tau H, \tau \in G$, where $m = (G:H)$, so that $\tau_1H,\ldots,\tau_mH$ partition $G$.

Consider
$$R_1 = (\tau_1H)\cdot i,\ldots,R_m = (\tau_m H)\cdot i.$$
As $G =\tau_1H \cup \cdots \cup \tau_m H$, then
$$R_1 \cup \cdots\cup R_m = (\tau_1H)\cdot i \cup \cdots \cup (\tau_mH)\cdot i =  G\cdot i = \{1,\ldots,n\}.$$
Now we show that this union is disjoint.

If $u \in R_j \cap R_k = (\tau_j H)\cdot i \cap (\tau_kH)\cdot i$, then
$$u = (\tau_j h)(i) = (\tau_k h')(i),\quad h,h'\in H.$$
Then $h'^{-1}\tau_k^{-1} \tau_j h)(i) = i$, thus
$$h'' = h'^{-1} \tau_k^{-1} \tau_j h \in G_i \subset H,$$
hence $\tau_j h = \tau_k h'h'',\ h,h',h'' \in H$. This shows that $\tau_j H =\tau_k H$, thus $j=k$. This proves
$$j \ne k \Rightarrow R_j \cap R_k = (\tau_j H)\cdot i \cap (\tau_kH)\cdot i = \varnothing.$$

Now we prove that every $\sigma \in G$ preserves the block structure:

If $\sigma \in G$ and $R_j = (\tau_j H)\cdot i$, then $\sigma \tau_j H$ is a left coset, thus $\sigma \tau_j H = \tau_k H$ for some index $k$, and
$$\sigma(R_j) =  (\sigma (\tau_j H \cdot i)) = (\sigma \tau_j H)\cdot i = (\tau_kH)\cdot i = R_k.$$
Since $G$ is transitive, all $R_j$ have same cardinality $l$, and $n = l m$.

To conclude, $R_1,\ldots,R_m$ have same cardinality, partition $\{1,\ldots,n\}$, and every $\sigma \in G$ preserves the block structure. If $l>1$ and $m>1$, then $G$ is imprimitive.

Hence, if we assume that $G$ is primitive, either $l= 1$ or $m=1$.

$\bullet$ If $l=1$, then for all indices $k$,  $(\tau_k H) \cdot i = \{\tau_k(i)\}$. With $k=1$ and $\tau_k = e$, we obtain $H\cdot i = \{i\}$, which shows that $H \subset G_i$, thus $H = G_i$.

$\bullet$ If $m=1$, then $(G:H) = m = 1$, thus $H = G$.

This proves that there is no subgroup $H$ such that $G_i \subsetneq H \subsetneq G$ : $G_i$ is maximal with respect to inclusion.

\bigskip

Conversely, suppose that $G$ is imprimitive, with respect to the blocks $R_1,\ldots,R_m$, where $m>1$. Since $G$ is transitive, all $R_j$ have the same cardinality $|R_j| = l >1$. 

If $i \in \{1,\ldots,n\}$ is some fixed integer, there is some index $j, 1\leq j \leq m$ such that $i \in R_j$. Now, consider the subgroup
$$H = \{\sigma\in G\, |\, \sigma(R_j) = R_j\}.$$
Then $G_i \subset H$: if $\sigma \in G_i$, then $\sigma(i) = i \in R_j$, thus $\sigma(R_j) = R_j$. Moreover

$\bullet$ $H\ne G$: Since $m>1$, there is some $R_k \ne R_j$, and some $w \in R_k$. Since $G$ is transitive, there is some $\sigma \in G$ such that $\sigma(i) = w$, so that $\sigma(R_j) = R_k$. Then $\sigma \not \in H$, and $H \ne G$.

$\bullet$ $G_i \ne H$: Since $|R_j|>1$, there is some $i'\ne i$ in the same block $R_j$. Since $G$ is transitive, there is some $\sigma' \in G$ such that $\sigma'(i) = i'$, so that $\sigma(R_j) = R_j$. Then $\sigma \in H$, but $\sigma \not \in G_i$, so $G_i \ne H$.

To conclude, the subgroup $H$ satisfies
$$G_i \subsetneq H \subsetneq G.$$
This proves that $G_i$ is not maximal with respect to inclusion.

We have proved that $G$ is primitive if and only if the isotropy subgroups of $G$ are maximal with respect to inclusion.

\end{proof}

 \paragraph{Ex. 14.2.16}{\it Let $p$ be prime. The ring $\Z/p^2\Z$ is not a field, but one can still define the group $\mathrm{AGL}(1,\Z/p^2\Z)$. Its action on $\Z/p^2\Z$ allows us to write $\mathrm{AGL}(1,\Z/p^2\Z) \subset S_{p^2}$.
 \be
\item[(a)] Prove that $\mathrm{AGL}(1,\Z/p^2\Z)$ is solvable and transitive of order $p^3(p-1)$.
\item[(b)] Prove that $\mathrm{AGL}(1,\Z/p^2\Z) \subset S_{p^2}$ is imprimitive.
\ee
}
\begin{proof}
\item[(a)] Recall that the group of invertible elements of $\Z/p^2\Z$ is $(\Z/p^2\Z)^*$, where $$|(\Z/p^2\Z)^*| = \phi(p^2) = p^2 - p.$$
As in section 6.4, if $a,b \in \Z/p^2\Z$, we define $\gamma_{a,b} : \Z/p^2\Z \to \Z/p^2\Z$ by $\gamma_{a,b}(x) = ax +b$. Note that $\gamma_{a,b}$ is bijective if and only if $a \in (\Z/p^2\Z)^*$.
We define
$$G = \mathrm{AGL}(1,\Z/p^2\Z) = \{\gamma_{a,b} \, | \, a \in (\Z/p^2\Z)^*, b \in \Z/p^2\Z\}.$$
We note that $\gamma_{a,b} = \gamma_{c,d}$ implies $\gamma_{a,b} (0)= \gamma_{c,d}(0)$, thus $b = d$, and then $\gamma_{a,b} (1)= \gamma_{c,d}(1)$ gives $a =c$:
$$\gamma_{a,b} = \gamma_{c,d} \Rightarrow (a,b) = (c,d).$$
Therefore the map
$$
\left\{
\begin{array}{ccc}
(\Z/p^2\Z)^* \times  \Z/p^2\Z & \to & G\\
(a,b) &\mapsto & \gamma_{a,b}
\end{array}
\right.
$$
is bijective, which proves $|G| = \phi(p^2) p^2 = (p^2-p)p^2 = p^3(p-1)$.

As in section 6.4 (and Exercise 6.4.1), we obtain, if $a,c \in (\Z/p^2\Z)^*$,
$$\gamma_{a,b} \circ \gamma_{c,d} = \gamma_{ac,ad+c} \in G,$$
since $ac \in (\Z/p^2\Z)^*$. Moreover, $\gamma_{a,b}^{-1} = \gamma_{a^{-1}, -a^{-1}b} \in G$, so that $G$ is a subgroup of $S(\Z/p^2\Z) \simeq S_{p^2}$.

The map 
$$
\varphi \left\{
\begin{array}{ccc}
G & \to & (\Z/p^2\Z)^*\\
\gamma_{a,b} &\mapsto & a
\end{array}
\right.
$$
is well-defined, and is a surjective homomorphism by definition of $G$. Moreover, the kernel of $\varphi$ is the subgroup
$$T = \{\gamma_{1,b}\, | \, b \in \Z/p^2\Z\}.$$
Therefore $ \mathrm{AGL}(1,\Z/p^2\Z)/T \simeq (\Z/p^2\Z)^*$. Since $T$ and $ (\Z/p^2\Z)^*$ are Abelian, this proves that $G = \mathrm{AGL}(1,\Z/p^2\Z)$ is solvable.

To prove that $G$ acts transitively on $\Z/p^2\Z$, for all $\alpha,\beta \in \Z/p^2 \Z$, we must find $\gamma_{a,b} \in G$ such that $\gamma_{a,b}(\alpha) = \beta$.

Write $\alpha = [u]_{p^2},  \beta = [v]_{p^2}, a = [A]_{p^2}, b = [B]_{p^2}$, where $u,v,A,B \in \Z$, and $\gcd(p,A) = 1$. Then
$$\gamma_{a,b}(\alpha) = \beta \iff a\alpha + b =\beta \iff A u + B \equiv v \pmod {p^2}.$$
If $\alpha \in (\Z/p^2\Z)^*$, $b = 0$ and $a = \alpha^{-1}\beta$ give a solution $\gamma_{a,b} = \gamma_{\alpha^{-1}\beta , 0}$.

Otherwise, $p \mid u$. If $p^2 \nmid u$, then $u = kp, k\in \Z, p \nmid k$. We can take $B = v-p$, so that
 \begin{align*}
 A u + B \equiv v \pmod {p^2} &\iff Akp + v-p \equiv v \pmod {p^2}\\
 & \iff Ak \equiv 1 \pmod p,
 \end{align*}
which has a solution $A \in \Z$ since $p \nmid A$. 

Finally, if $p^2 \mid u$, then $\alpha = 0$, and $a=1, b = \beta$ gives a solution $\gamma_{1,\beta}$. 

Thus $G$, viewed as a subgroup of $S_{p^2}$, is transitive.

\item[(b)] Consider the subgroup $H$ of $G$ defined by
$$H = p\Z/p^2 Z = \{\overline{0}, \overline{p},\overline{2p},\ldots,\overline{(p-1)p}\},$$
where we write, for every integer $a$, $\overline{a} = [a]_{p^2}$ the class of $a$ modulo $p^2$.

Then the cosets $R_i = \overline{i} + H, 0 \leq i \leq p-1$, partition $G$:
\begin{align*}
&R_0 = H = \{\overline{0}, \overline{p},\overline{2p},\ldots,\overline{(p-1)p}\},\\
&R_1 = \overline{1}+H =  \{\overline{1}, \overline{1+p},\overline{1+2p},\ldots,\overline{1+(p-1)p}\},\\
&\ldots\\
&R_{p-1} = \overline{p-1}+H = \{\overline{p-1}, \overline{p-1 + p},\overline{p-1+ 2p},\ldots,\overline{p^2-1}\}.
\end{align*}
Note that $R_i = \gamma_{1,i}(R_0)$ for all $i$ (here $\gamma_{1,i}$ refers to $\gamma_{\overline1 ,\overline i}$, and $i+H$ to $\overline{i} + H = R_i$).

 We prove
$$\gamma_{a,b}(R_0) = R_b.$$
Indeed, if $\alpha = \overline{kp} \in R_0$, then $\gamma_{a,b}(\overline{kp}) = a\overline{kp} + b \in b+ H = R_b$, thus $\gamma_{a,b}(R_0) \subset R_b$. 
Moreover the cosets $R_0, R_b$ have same cardinality, so that $\gamma_{a,b}(R_0) = R_b$.

Using this result, for all index $i$, since $R_i = i + R_0 = \gamma_{1,i}(R_0)$,
\begin{align*}
\gamma_{a,b}(R_i) &= (\gamma_{a,b}\,  \gamma_{1,i})(R_0)\\
&=\gamma_{a,a+ib}(R_0)\\
&=R_{ai+b}.
\end{align*}
Therefore $G$ permutes the blocks $R_0,\ldots,R_{p-1}$. 

This proves that $G = \mathrm{AGL}(1,\Z/p^2\Z)$ is imprimitive.
\end{proof}

\subsection{PRIMITIVE PERMUTATION GROUPS}

 \paragraph{Ex. 14.3.1}{\it The goal of this exercise is to prove that primitive permutation groups are transitive. Assume that $G \subset S_n$ is primitive but not transitive, and derive a contradiction as follows.
 \be
 \item[(a)] Explain why $n>1$.
 \item[(b)] Let the orbits of $G$ acting on $\{1,\ldots,n\}$ be $R_1,\ldots,R_k$ (see Section A.4 if you have forgotten about orbits). Explain why $k>1$ and why elements of $G$ map every orbit to itself.
 \item[(c)] Conclude that $G$ is imprimitive. Be sure to take into account the case when every orbit consists of a single element.
 \ee
 }
 \begin{proof}

 \item[(a)] If $n = 1$, then $S_n=\{e\}$ and $G = \{e\}$. Then $G$ is primitive (we can't partition $\{1,\ldots,n\} = \{1\}$ with classes $R_i$ such that $|R_i|>1$ for some $i$), but $G$ is transitive, since $e(1) = 1$. So the assumption $G \subset S_n$ is primitive but not transitive implies $n>1$.
 \item[(b)] If $k=1$, there is only one orbit $R_1 = G\cdot 1$. Then $G$ is transitive: if $i,j \in \{1,n\}$, $i = \sigma(1), j = \tau(1),\ \sigma, \tau \in G$, thus $(\tau \sigma^{-1})(i) = j$, where $\tau \sigma^{-1} \in G$. This shows that $G$ is transitive. Since $G$ is not transitive, then $k>1$.
 

 We know that the orbits partition $\{1,\ldots,n\}$. 
  
 Now we prove that, if $\sigma \in G$ and $R_i$ is an orbit, then $\sigma(R_i) = R_i$ is the same orbit $R_i$.
 
 Fix $x \in R_i$, so that $R_i = {\cal O}_x = G\cdot x$ is the orbit of $x$.
 
 Let $u \in R_i$. Then  $u = \tau(x)$ for some $\tau \in G$. Then $\sigma(u) = (\sigma \tau)(x) \in R_i$. This proves $\sigma(R_i) \subset R_i$.
 
 Conversely, for every $u = \tau(x) \in R_i$, $ u = \sigma((\sigma^{-1} \tau)(x))$ ,where $(\sigma^{-1} \tau)(x) \in R_i$, thus $u \in \sigma(R_i)$. Therefore $R_i \subset \sigma(R_i)$. For all $\sigma \in G$,
 $$\sigma(R_i) = R_i.$$
 
 \item[(c)] If $|R_i|>1$ for some $i$, then $G$ is imprimitive by Definition 14.2.5. By assumption, $G$ is primitive, thus $|R_i| = 1$ for all $i$, so that every orbit consists of a single element. This means that for all $\sigma \in G$, and for all $i \in \{1,\ldots,n\}$, $\sigma(i) = i$. Therefore $\sigma = e$ for all $\sigma \in G$, $G = \{e\}$. Since $n>1$ by part (a), 
 $G$ is imprimitive, because there are partitions with at least two classes, and several elements in a class, for instance $R_1 = \{1,2\},R_2 =\{1,\ldots,n\} \setminus R_1$, is preserved by $G = \{e\}$. This proves that $G = \{e\}$ is imprimitive when $n>1$, in contradiction with the assumption.
 
This contradiction proves $|R_i|>i$ for some index $i$, thus $G$ is imprimitive.
 
 To conclude, all primitive permutation groups are transitive.
 \end{proof}
 
 \paragraph{Ex. 14.3.2}
 {\it Let $\gamma_{I_n,v} \in \mathrm{AGL}(n,\F_q)$ be translation by $v \in \F_{q}^n$, and let $\gamma_{A,w} \in \mathrm{AGL}(n,\F_q)$ be arbitrary.}
 \be
 \item[(a)] Prove that $\gamma_{A,w}^{-1} = \gamma_{A^{-1},-A^{-1}w}$.
 \item[(b)] Prove that $\gamma_{A,w} \circ \gamma_{I_n,v} \circ \gamma_{A,w}^{-1} = \gamma_{I_n,Av}$.
 \item[(c)] Part (b) shows that the translation subgroup $\F_q^n \subset \mathrm{AGL}(n,\F_q)$ is normal. Prove that the quotient group $\mathrm{AGL}(n,\F_q)/\F_q^n$ is isomorphic to $\mathrm{GL}(n,\F_q)$.
 \item[(d)] Prove that $\mathrm{AGL}(n,\F_q)$ is isomorphic to the semidirect product $\F_q^n \rtimes \mathrm{GL}(n,\F_q)$, where $\mathrm{GL}(n,\F_q)$ acts on $\F_q^n$ by matrix multiplication.
 \ee
 
\begin{proof}
We give first the product of two elements in $\mathrm{AGL}(n,\F_q)$, to generalize the results of Section 6.4A. Using the definition $\gamma_{A,v}(u) = Au+v$, we obtain, for all $u \in \F_q^n$,
\begin{align*}
(\gamma_{A,v} \circ \gamma_{B,w})(u) &= \gamma_{A,v}(Bu + w)\\
&= A(Bu + w) + v\\
&= ABu + Aw +v\\
&= \gamma_{AB,Aw+v}(u),
\end{align*}
thus
\begin{align}
\gamma_{A,v} \circ \gamma_{B,w} = \gamma_{AB,Aw+v}
\end{align}
\item[(a)] For all $u,v \in \F_q^n$,
\begin{align*}
v = \gamma_{A,w}(u) &\iff v = Au + w\\
&\iff u = A^{-1}(v - w) = A^{-1} v - A^{-1}w,
\end{align*}
thus
\begin{align}
\gamma_{A,w}^{-1} = \gamma_{A^{-1}, - A^{-1}w}.
\end{align}
\item[(b)]
Using the formulas (1) and (2), we obtain
\begin{align*}
\gamma_{A,w} \circ \gamma_{I_n,v} \circ \gamma_{A,w}^{-1} &= \gamma_{A,w} \circ \gamma_{I_n,v} \circ \gamma_{A^{-1}, - A^{-1}w}\\
&= \gamma_{A,w} \circ \gamma_{A^{-1}, -A^{-1}w + v}\\
&=\gamma_{I_n,A(-A^{-1}w + v) + w}\\
&=\gamma_{I_n,Av}.
\end{align*}
We have proved
\begin{align}
\gamma_{A,w} \circ \gamma_{I_n,v} \circ \gamma_{A,w}^{-1}  = \gamma_{I_n,Av}.
\end{align}


\item[(c)] Consider the map
$$
\varphi
\left\{
\begin{array}{ccc}
\mathrm{AGL}(n,\F_q) & \to &\mathrm{GL}(n,\F_q)\\
\gamma_{A,v} & \mapsto &A.
\end{array}
\right.
$$
The map $\varphi$ is well defined, since for all $A,B \in \mathrm{GL}(n,\F_q)$ and for all $v,w \in \F_q^n$,
\begin{align}
\gamma_{A,v} = \gamma_{B,w} \iff (A,v) = (B,w).
\end{align}
Then 
$$
\varphi(\gamma_{A,v} \circ \gamma_{B,w}) =\varphi( \gamma_{AB,Aw+v}) = AB = \varphi(\gamma_{A,v}) \varphi(\gamma_{B,w}),
$$
thus $\varphi$ is a group homomorphism.

Since every $A \in \mathrm{GL}(n,\F_q)$ is the image of $\gamma_{A,0}$, $\varphi$ is surjective, and
$$\ker \varphi = \{\gamma_{I,w} \, |\, w \in \F_q^*\} \simeq \F_q^n.$$
Therefore, the first Isomorphism Theorem gives
$$\mathrm{AGL}(n,\F_q)/ \F_q^n \simeq \mathrm{GL}(n,\F_q),$$
with the identification of vectors $w$ with the translations $\gamma_{I,w}$.

\item[(d)] Consider
$$
\psi
\left\{
\begin{array}{ccc}
\mathrm{AGL}(n,\F_q) & \to & \F_q^n \rtimes \mathrm{GL}(n,\F_q)\\
\gamma_{A,v} & \mapsto & (v,A).
\end{array}
\right.
$$
By formula (4), the map $\psi$ is well defined. It is a bijection, with inverse $(v,A) \mapsto \gamma_{A,v}$.

Following the formula (6.9), $(h,g)\, (h',g') = (h(g\cdot h'), gg')$, which defines the product in $H \rtimes G$, the product in $\F_q^n \rtimes \mathrm{GL}(n,\F_q)$ is given by
\begin{align}
(v,A)\, (w,B) = (v+ Aw,AB),\qquad v,w \in \F_q^n, \quad A,B \in \mathrm{GL}(n,\F_q).
\end{align}
Therefore, using (5) and (1),
\begin{align*}
\psi(\gamma_{A,v})  \psi(\gamma_{B,w}) &= (v,A)(w,B)\\
&= (v+ Aw,AB)\\
&=\psi(\gamma_{AB,Aw+v})\\
&=\psi(\gamma_{A,v} \circ \gamma_{B,w}), 
\end{align*}
thus $\psi$ is a group isomorphism, and
$$\mathrm{AGL}(n,\F_q) \simeq \F_q^n \rtimes \mathrm{GL}(n,\F_q).$$
\end{proof}

 \paragraph{Ex. 14.3.3}{\it
 Consider the affine semilinear group $\mathrm{A\Gamma L}(n,\F_q)$ for $q = p^m$.
 \be
 \item[(a)] Prove that $\mathrm{AGL}(n,\F_q)$ is a normal subgroup of $\mathrm{A\Gamma L}(n,\F_q)$ of index $m$.
 \item[(b)] Prove that $\F_q^n$ is a normal subgroup of $\mathrm{A\Gamma L}(n,\F_q)$.
 \item[(c)] Prove that elements of $\mathrm{A\Gamma L}(n,\F_q)$ give maps $\F_q^n \to \F_q^n$ that are affine linear over $\F_p$.
 \ee
 }
 
 \bigskip
 
 We give some preliminary formulas.
 
Note that $\gamma_{A,v} = \gamma_{A,e,v}, \ A \in \mathrm{GL}(n,\F_q), v \in \F_q$, where $e$ is the identity element of $\Gal(F_{q^m}/\F_p)$, thus $\mathrm{AGL}(n,\F_q) \subset \mathrm{A\Gamma L}(n,\F_q)$.

For all $u = (\alpha_1,\ldots,\alpha_n) \in \F_q^n$, we write $\sigma(u) = (\sigma(\alpha_1),\ldots,\sigma(\alpha_n))$. 

If $A =(a_{ij})_{1\leq i,j \leq n} \in \mathrm{GL}(n,\F_q)$, then 
\begin{align*}
\sigma(A\cdot u) &= \sigma\left(\sum_{j=1}^n a_{1j} \alpha_j,\ldots,\sum_{j=i}^n a_{ij} \alpha_j,\ldots,\sum_{j=1}^n a_{nj} \alpha_j\right)\\
&=\left(\sum_{j=1}^n \sigma(a_{1j}) \sigma(\alpha_j),\ldots,\sum_{j=i}^n \sigma(a_{ij})\sigma( \alpha_j),\ldots,\sum_{j=1}^n \sigma(a_{nj})\sigma( \alpha_j)\right)\\
&=\sigma(A)\cdot \sigma(u),
\end{align*}
where $\sigma(A) = (\sigma(a_{ij}))_{1\leq i,j \leq n}$, and similarly 
$$\sigma(A\cdot u + v) = \sigma(A)\cdot  \sigma(u) + \sigma(v).$$
Then, for all $u \in \F_q^n$,
\begin{align*}
(\gamma_{A,\sigma,v} \circ \gamma_{B,\tau,w})(u) &= A\sigma(B\tau(u)+w) + v\\
&=A\sigma(B) (\sigma\tau)(u) + A\sigma(w) + v\\
&=\gamma_{A\sigma(B),\sigma \tau,A \sigma(w) + v}(u),
\end{align*}
thus
\begin{align}
\gamma_{A,\sigma,v} \circ \gamma_{B,\tau,w} = \gamma_{A\sigma(B),\sigma \tau,A \sigma(w) + v}.
\end{align}

For all $u,s \in \F_q^n$,
\begin{align*}
s = \gamma_{A,\sigma,v}(u) &\iff s = A \sigma(u) + v\\
&\iff \sigma(u) = A^{-1}(s -v)\\
&\iff u = \sigma^{-1}(A^{-1} s - A^{-1}v)\\
&\iff u = \sigma^{-1}(A^{-1}) \sigma^{-1}(s) - \sigma^{-1}(A^{-1})\sigma^{-1}(v)\\
&\iff u = \gamma_{\sigma^{-1}(A^{-1}),\sigma^{-1},- \sigma^{-1}(A^{-1})\sigma^{-1}(v)}(s),
\end{align*}
thus
\begin{align}
\gamma_{A,\sigma,v}^{-1} = \gamma_{\sigma^{-1}(A^{-1}),\sigma^{-1},- \sigma^{-1}(A^{-1})\sigma^{-1}(v)}.
\end{align}
By (6) and (7), $\mathrm{A\Gamma L}(n,\F_q)$ is a subgroup of $S(\F_q^n)$.
Moreover, if $\gamma_{A,\sigma,v}, \gamma_{B,\tau,w} \in  \mathrm{A\Gamma L}(n,\F_q)$, then
\begin{align}
 \gamma_{A,\sigma,v} =  \gamma_{B,\tau,w} \Rightarrow (A,\sigma,v) = (B,\tau,w).
\end{align}
 Indeed, if $\gamma_{A,\sigma,v} =  \gamma_{B,\tau,w}$, then for all $u \in \F_q^n$, $A \sigma(u) + v = B \tau(u) + w$. With $u=0$, we obtain $v = w$. If we take $u =e_i =(0,\ldots,0,1,0,\ldots,0)$, where $(e_1,\ldots,e_n)$ is the standard base of $\F_q^n$, then $\sigma(e_i) = e_i = \tau(e_i)$, so that $Ae_i = Be_i,\ i=1,\ldots,n$. This implies $A = B$, and $\sigma(u) = \tau(u)$ for all $u\in \F_q^n$. Therefore, with $u = (\alpha,0,\ldots,0),\ \alpha \in \F_q$, we obtain $\sigma(\alpha) = \tau(\alpha)$, thus $\sigma = \tau$.

 \begin{proof}
 \item[(a)] 
 By (8),
 $$
 \varphi
 \left\{
 \begin{array}{ccc}
 \mathrm{A\Gamma L}(n,\F_q) &\to & \Gal(\F_{p^m},\F_p)\\
 \gamma_{A,\sigma,v} & \mapsto & \sigma
 \end{array}
 \right.
 $$
 is well defined. Moreover, by (6), if $\gamma_{A,\sigma,v}, \gamma_{B,\tau,w} \in \mathrm{A\Gamma L}(n,\F_q)$,
 $$\varphi(\gamma_{A,\sigma,v} \circ \gamma_{B,\tau,w} ) = \varphi( \gamma_{A\sigma(B),\sigma \tau,A \sigma(w) + v}) =\sigma \tau = \varphi(\gamma_{A,\sigma,v})\varphi( \gamma_{B,\tau,w}),
 $$
 thus $\varphi$ is a group homomorphism. If $\sigma$ is any element of $\Gal(\F_{p^n},\F_p)$, then $\sigma = \varphi(\gamma_{I_n, \sigma,0})$, thus $\varphi$ is surjective. The kernel of $\varphi$ is
 $$\ker \varphi = \{\gamma_{A,e,v}\, | \, A\in \mathrm{GL}(n,\F_q), v \in \F_q^n\} = \{\gamma_{A,v}\, | \, A\in \mathrm{GL}(n,\F_q), v \in \F_q^n\} = \mathrm{AGL}(n,\F_q).$$

This shows that $\mathrm{AGL}(n,\F_q)$ is a normal subgroup of $\mathrm{A\Gamma L}(n,\F_q)$, and the first Isomorphism Theorem gives
$$ \mathrm{A\Gamma L}(n,\F_q)/ \mathrm{AGL}(n,\F_q) \simeq \Gal(\F_{p^m},\F_p).$$
Since $\Gal(\F_{p^m},\F_p)$ is a cyclic group of order $m$, $\mathrm{AGL}(n,\F_q)$ is a normal subgroup of $\mathrm{A\Gamma L}(n,\F_q)$ of index $m$.

\item[(b)] Here, $\F_q^n$ is identified with the group of translations $\{\gamma_{I_n,e,w}\,|\, w \in \F_q^n\}$, and, using (6) and (7),
\begin{align*}
\gamma_{A,\sigma,v} \circ \gamma_{I_n,e,w} \circ \gamma_{A,\sigma,v}^{-1} &= \gamma_{A,\sigma,v} \circ \gamma_{I_n,e,w}  \circ \gamma_{\sigma^{-1}(A^{-1}),\sigma^{-1},- \sigma^{-1}(A^{-1})\sigma^{-1}(v)}\\
&= \gamma_{A,\sigma,v} \circ  \gamma_{\sigma^{-1}(A^{-1}),\sigma^{-1},-\sigma^{-1}(A^{-1}) \sigma^{-1}(v) + w}\\
&=\gamma_{A \sigma \sigma^{-1}(A^{-1}),\sigma \sigma^{-1}, A\sigma(-\sigma^{-1}(A^{-1}) \sigma^{-1}(v) + w) + v}\\
&= \gamma_{I_n,e,A\sigma(w)}.
\end{align*}
We have proved
\begin{align}
\gamma_{A,\sigma,v} \circ \gamma_{I_n,e,w} \circ \gamma_{A,\sigma,v}^{-1} = \gamma_{I_n,e,A\sigma(w)} \in \F_q^n,
\end{align}
therefore $\F_q^n$ is a normal subgroup of $\mathrm{A\Gamma L}(n,\F_q)$.

\item[(c)]  $\F_q^n$, which is a vector space over $\F_q$, is also a vector space over $\F_p$, by restriction of the extern operation 
$
\left\{
\begin{array}{ccc}
\F_q \times \F_q^n & \to & \F_q^n\\
(\lambda,u) & \mapsto &\lambda u
\end{array}
\right.
$
to $\F_p \times \F_q^n$.

Let ${\cal B} = (e_1,\ldots,e_m)$ be a base of $\F_q$ over $\F_p$.
As $\F_p$-vector spaces, $\F_q^n$ and $\F_p^{nm}$ are isomorphic, where an isomorphism $\varphi$ is given by
$$
\varphi \left\{
\begin{array}{ccc}
\F_q^n & \to & \F_p^{nm}\\
(\alpha_1,\ldots,\alpha_n)& \mapsto &(x_1^1,\ldots,x_m^1;\ldots;x_1^n,\ldots,x_m^n),
\end{array}
\right.
$$
where $\alpha_i = \sum_{j=1}^m x_j^i e_j,\ i=1,\ldots,n.$ (this isomorphism depends of the choice of the base $\cal B$). This proves $\dim_{\F_p} \F_q^n = nm$.
Consider the two maps
$$
\psi
\left\{
\begin{array}{ccc}
\F_q^n &\to &\F_q^n\\
u  = (\alpha_1,\ldots,\alpha_n)& \mapsto &\sigma(u) = (\sigma(\alpha_1),\ldots,\sigma(\alpha_n)),
\end{array}
\right.
\qquad 
\chi
\left\{
\begin{array}{ccc}
\F_q^n &\to &\F_q^n\\
v& \mapsto &A \cdot v.
\end{array}
\right.
$$
The automorphism $\sigma \in \Gal(\F_q/\F_p)$ is $\F_q$-linear: if $\lambda \in \F_p, \alpha \in \F_q$, $\sigma(\lambda \alpha) = \sigma(\lambda) \sigma(\alpha) = \lambda \sigma(\alpha)$. Thus $\psi$ is $\F_p$-linear. Moreover, $\chi$ is $\F_q$-linear, a fortiori $\F_p$-linear. Therefore, $\chi \circ \psi = u \mapsto A\cdot \sigma(u)$ is $\F_p$-linear, so that
$$
\gamma_{A,\sigma,v}
\left\{
\begin{array}{ccc}
\F_q^n & \to & \F_q^n\\
u &\mapsto & A\cdot \sigma(u) + v
\end{array}
\right.
$$
is affine linear over $\F_p$.
 \end{proof}
 
 
\paragraph{Ex. 14.3.4}{\it Let $F$ be any field. The definition of $\mathrm{AGL}(n,\F_q)$ given in the text extend to $\mathrm{AGL}(n,F)$.
The goal of this exercise is to prove that $\mathrm{AGL}(n,F)$ is doubly transitive when we regard elements of $\mathrm{AGL}(n,F)$ as permutations of the vector space $F^n$.
\be
\item[(a)] Use $F^n \subset \mathrm{AGL}(n,F)$ to show that $\mathrm{AGL}(n,F)$ acts transitively on $F_n$.
\item[(b)] Inside $\mathrm{AGL}(n,F)$, we have the isotropy subgroup of $\,0\in F_n$. Prove that this isotropy subgroup is $\mathrm{GL}(n,F)$.
\item[(c)] Prove that $\mathrm{GL}(n,F)$ acts transitively on $F^n\setminus \{0\}$.
\item[(d)]Use Exercise 19 below to conclude that $\mathrm{AGL}(n,F)$ is doubly transitive.
\ee
}
\begin{proof}
\item[(a)] Let $u,u'$ be any vectors in $F^n$. The equality $I_n\cdot u  + (u'- u) = u'$ shows that $\gamma_{I_n,u'-u}(u) = u'$, where $\gamma_{I_n,u'-u} \in \mathrm{AGL}(n,F)$.

Therefore $\mathrm{AGL}(n,F)$ acts transitively on $F^n$.

\item[(b)] Write $G_0$ the isotropy group of $0$. Then
$$\gamma_{A,v} \in G_0 \iff A\cdot 0 + v = 0 \iff v = 0.$$
Therefore $G_0 = \{\gamma_{A,0}\, | \,  A \in \mathrm{GL}(n,F)\} \simeq \mathrm{GL}(n,F)$.

In section 14.3.B, we identified $\{\gamma_{A,0}\mid A \in \mathrm{GL}(n,F)\} $ with $\mathrm{GL}(n,F)$, so that
$$G_0 = \mathrm{GL}(n,F).$$

\item[(c)]

 Let $u,v \in F^n \setminus \{0\}$. Since $u \ne 0$, we can complete $u$ in a base ${\cal B} _1= (u_1,\ldots,u_n)$ of $F_n$, where $u_1 = u$. Similarly, we can complete $v\ne 0$ in a base ${\cal B}_2 = (v_1,\ldots,v_n)$, where $v_1 = v$. 

Since ${\cal B}_1,{\cal B}_2$ are two bases, there exists some bijective linear map $f : F^n \to F^n$ such that $f(u_i) = v_i,\ i=1,\ldots,n$, so that $f(u) = v$.

Let ${\cal B} = (e_1,\ldots,e_n)$ be the standard base of $F^n$. Then the matrix $A = {\cal M}_{\cal B}(f)$ of $f$ satisfies $A \in \mathrm{GL}(n,F)$, and $A \cdot u = v$. This proves that $\mathrm{GL}(n,F)$ acts transitively on $F^n \setminus\{0\}$.

\item[(d)] Since  the isotropy group $G_0$ acts transitively on $F^n\setminus \{0\}$, Exercise 14.3.19 (a) shows that $\mathrm{AGL}(n,F)$ acts transitively on $F^n$, using that  $\mathrm{AGL}(n,F)$ can be viewed as a subgroup $G$ of $S_{|F^n|}$.
\end{proof} 
 
\paragraph{Ex. 14.3.5}{\it Let $A$ and $B$ be non-Abelian simple groups. You will show that $A \times \{e_B\}$ and $\{e_A\}\times B$ are the only nontrivial normal subgroups of $A \times B$. Let $N \subset A \times B$ be a normal subgroup different from $\{(e_A,e_B)\},A \times\{e_B\}$, and $\{e_A\} \times B$.
\be
\item[(a)] Prove that $A \times\{e_B\}$ and $\{e_A\}\times B$ are normal in $A \times B$. Hence, if we can show that $N = A\times B$, then we will be done.
\item[(b)] Prove that we can find $(a,b)\in N$ such that $e_A \ne a \in A$ and $e_B \ne b\in B$.
\item[(c)] Let $(a,b)\in N$ be as in part (b). Show that $(aa_1a^{-1}a_1^{-1}, e_B) \in N$ for any $a_1 \in A$.
\item[(d)] Given $e_A\ne a \in A$, prove that there is $a_1\in A$ such that $aa_1\ne a_1a$. Then combine this with parts (b) and (c) to show that $N\cap(A\times\{e_B\}) = A \times\{e_B\}$.
\item[(e)] Part (d) implies that $A \times \{e_B\} \subset N$, and the inclusion $\{e_A\} \times B \subset N$ is proved similarly. Use this to prove that $N = A\times B$.
\ee
Exercise 18 will explore various aspects of this argument.
}

\begin{proof}
\item[(a)] Take $(a_0,e_B)$ in $A\times\{e_B\}$, and $(a,b) \in A\times B$. Then
\begin{align*}
(a,b)(a_0,e_B)(a,b)^{-1} 
&=(a,b)(a_0,e_B)(a^{-1},b^{-1})\\
&=(aa_0a^{-1},be_Bb^{-1})\\
&=(aa_0a^{-1}, e_B) \in A \times\{e_B\},
\end{align*}
thus $A\times \{e_B\}$ is normal in $A \times B$, and similarly $\{e_A\} \times B$ is normal in $A \times B$.

\item[(b)] Reasoning by contradiction, suppose that we can't find $(a,b) \in N$ such that $a \ne e_A,b\ne e_B $ , then $N \subset (A \times \{e_B\}) \cup (\{e_A\} \times B)$. 

Note that $A\times \{e_B\} \simeq A$ is simple. If $N \subset A \times \{e_B\}$, then $N$ is normal in $A \times \{e_B\} \subset A \times B$, thus $N = A\times\{e_B\}$ or $N = \{(e_A,e_B)\}$, but this contradicts the assumptions on $N$. Therefore $N \not \subset A \times \{e_B\}$, and similarly $N \not \subset \{e_A\} \times B$, so that
\begin{align*}
&N \subset (A \times \{e_B\}) \cup (\{e_A\} \times B), \\
&N \not \subset A \times \{e_B\}, \quad N \not \subset \{e_A\} \times B.
\end{align*}
Since $N \subset (A \times \{e_B\}) \cup (\{e_A\} \times B)$ and $N \not \subset A \times \{e_B\}$, there exists some $n_1 = (a',b) \in N$ such that $n_1 = (a',b) \in \{e_A\} \times B$ and $n_1 = (a',b) \not \in A\times \{e_B\}$. Thus $n_1 = (e_A,b) \in N, b \ne e_B$.

Similarly, since  $N \subset (A \times \{e_B\}) \cup (\{e_A\} \times B)$ and $N \not \subset \{e_A\} \times B$, there exists some $n_2 = (a,b') \in N$ such that $n_2 = (a,b') \in A \times
 \{e_B\}$ and $n_2 = (a,b') \not \in \{e_A\} \times B$. Thus $n_2 = (a,e_B) \in N, a \ne e_A$.

Then $n = n_1n_2 =(e_A,b)(a,e_B) = (a,b) \in N$, and $a \ne e_A,b \ne e_B$.



\item[(c)] $N$ is normal, and  $(a^{-1},b^{-1}) = (a,b)^{-1} \in N$, thus
$$(a_1,e_B)(a,b)^{-1}(a_1,e_B)^{-1} = (a_1 a^{-1}a_1^{-1},b^{-1}) \in N.$$
Left multiplication by $(a,b) \in N$ gives
$$(a,b)[(a_1,e_B)(a,b)^{-1}(a_1,e_B)^{-1} ] = (aa_1a^{-1}a_1^{-1}, e_B) \in N.$$

\item[(d)] Let $a\ne e_A$ be a fixed element in $A$.

Consider the center of $A$:
$$Z(A) = \{x \in A\, | \, \forall y \in A, xy = yx\}.$$
$Z(A)$ is a normal subgroup of $A$ (it is the kernel of $f : A \to \mathrm{Aut(A)}$ defined by $f(y) =\phi_y$, where $\phi_y(x) = yxy^{-1}$ for all $x\in A$). Since $A$ is non-Abelian, $Z(A) \ne A$, and since $A$ is simple, $Z(A) = \{e_A\}$.

Therefore, $a\ne e_A$ implies $a \not \in Z(A)$, thus there exists $a_1 \in A$ such that $aa_1 \ne a_1 a$.

Since $N$ and $A\times \{e_B\}$ are normal subgroups of $A\times B$, the intersection $N \cap (A\times \{e_B\})$ is normal in $A\times B$, a fortiori in the simple group $A\times \{e_B\}$. Therefore  $N \cap (A\times \{e_B\}) =\{(e_A,e_B)\}$ or $N \cap (A\times \{e_B\}) = A \times \{e_B\}$. But $N \cap (A\times \{e_B\}) =\{(e_A,e_B)\}$ is impossible, because $(e_A,e_B) \ne (aa_1a^{-1}a_1^{-1}, e_B) \in N$. Hence $N \cap (A\times \{e_B\}) = A \times \{e_B\}$.

\item[(e)] Part (d) implies that $A \times \{e_B\} \subset N$, and the inclusion $\{e_A\} \times B \subset N$ is proved similarly. Let $(a,b)$ be any element in $A\times B$. Then 
$n_1 = (a,e_B) \in A\times\{e_B\} \subset N$, and $n_2 = (e_A,b) \in \{e_A\} \times B \subset N$, thus $n_1 \in N, n_2 \in N$, so that $n =n_1n_2 = (a,b) \in N$. This proves $A\times B \subset N$, where $N \subset A\times B$. Therefore $N = A \times B$.

To conclude, if $A,B$ are non-Abelian simple groups, the only non trivial normal subgroups of $A\times B$ are $A \times \{e_B\}$ and $\{e_A\} \times B$.
\end{proof}

\paragraph{Ex. 14.3.6}{\it Let $A \subset N$ be a minimal normal subgroup, where $N$ is normal in a larger group $G$. Given $g \in G$, we set $A_g = gAg^{-1}$.
\be
\item[(a)] Prove that $A_g$ is isomorphic to $A$ and is a minimal normal subgroup of $N$.
\item[(b)] Fix $g_1 \in G$ and consider $AA_{g_1}$. By Exercise 7, we know that $AA_{g_1}$ is a subgroup of $N$. Assume that $A_g \subset AA_{g_1}$ for all $g \in G$. Prove that $AA_{g_1}$ is normal in $G$.
\item[(c)]Use the following idea to complete the proof of Proposition 14.3.10. Let $\cal A$ be the set of all subgroups of $N$ of the form $A_{g_1}\cdots A_{g_n}$ such that the map $(a_1,\ldots,a_n) \mapsto a_1\ldots a_n$ defines an isomorphism
$$A_{g_1}\times\cdots\times A_{g_n} \simeq A_{g_1}\cdots A_{g_n}.$$
Note that $A = A_e \in {\cal A}$. Then pick an element of $\cal A$ of maximal order.
\ee
}
\begin{proof}
\item[(a)] Consider
$$
\varphi_g
\left\{
\begin{array}{ccc}
N & \mapsto & N\\
a & \mapsto &gag^{-1}.
\end{array}
\right.
$$
For all $a,b \in N$, $\varphi_g(ab) = gabg^{-1} =gag^{-1} g bg^{-1} = \varphi_g(a) \varphi_g(b)$, thus $\varphi_g$ is a group homomorphism.

Moreover, $\varphi_g \circ \varphi_{g^{-1}} =  1_N = \varphi_{g^{-1}} \circ \varphi_g$, so that $\varphi$  is bijective: $\varphi$ is a group automorphism.

Since $A$ is normal subgroup of  $N$, $A_g = \varphi_g(A)$ is also a normal subgroup of $N$.

Now take $H$ a non trivial subgroup of $A_g$. Reasoning by contradiction assume that $H$ is normal in $N$. Then  $K = \varphi_g^{-1}(H) = g^{-1}H g$ is a non trivial subgroup of $A$. Since $H$ is normal in $N$, and $\varphi_g$ is an automorphism of $N$, $K = \varphi_g^{-1}(H)$ is normal in $N$. This contradicts the fact that $A$ is a minimal normal subgroup of $N$. This  contradiction proves that $H$ is not normal in $N$. We can conclude that $A_g$ is a minimal normal subgroup of $N$.

\item[(b)]  Let $g$ be any element of $G$. The hypothesis gives the inclusions $A_g = gAg^{-1} \subset AA_{g_1}$ and $A_{gg_1} = (gg_1) A (gg_1)^{-1} \subset AA_{g_1}$, therefore, since $AA_{g_1}$ is a subgroup,
\begin{align*}
gAA_{g_1} g^{-1} &= gA g^{-1}\ gA_{g_1} g^{-1}\\
&= gAg^{-1}\ gg_1A g_1^{-1}g^{-1} \subset AA_{g_1}.
\end{align*}
This proves that $A A_{g_1}$ is a normal subgroup of $G$.

\item[(c)] Let $\cal A$ be the set of all subgroups of $N$ of the form $A_{g_1}\cdots A_{g_n}$ such that the map $(a_1,\ldots,a_n) \mapsto a_1\ldots a_n$ defines an isomorphism
$$A_{g_1}\times\cdots\times A_{g_n} \simeq A_{g_1}\cdots A_{g_n}.$$
Note that $A = A_e \in {\cal A}$. Then pick an element $H$ of $\cal A$ of maximal order. Then 
$$H = A_{g_1}\cdots A_{g_n},\qquad g_1,\ldots,g_n \in G  \subset N.$$
If $H = N$, then we are done. Assume now that $H \ne N$.

If $A_g \subset H$ for all $g\in G$, then $H$ is normal in $G$: 

for all $g \in G$, $A_{gg_i}= (gg_i)A(gg_i)^{-1} \subset H$, thus
$$gHg^{-1} = gA_{g_1}g^{-1} gA_{g_2}g^{-1} \cdots gA_{g_n}g^{-1} = A_{gg_1}\cdots A_{gg_n} \subset H.$$
Since $H \ne N$, and $H \ne \{e\}$ ($|H| \geq |A|>1$), this is impossible by the minimality of $N$. Hence there is $g_{n+1} \in G$ such that $A_{g_{n+1}} \not \subset H$.

We don't know if $H$ is normal in $G$, but $H = A_{g_1}\cdots A_{g_n}$ is normal in $N$, since all $A_{g_i}$ are normal in $N$ (see Exercise 7(a)). Hence $H \cap A_{g_{n+1}}$ is a normal subgroup in $N$, and lies in the minimal normal subgroup $A_{g_{n+1}}$ of $N$. Since $A_{g_{n+1}} \not \subset H$, $H \cap A_{g_{n+1}} \ne H$, therefore $H \cap A_{g_{n+1}} = \{e\}$. By Exercise 7, this proves that the map $(a_1,\ldots,a_{n+1}) \mapsto a_1\ldots a_n$ defines an isomorphism
$$A_{g_1}\times\cdots\times A_{g_n}\times A_{g_{n+1}} \simeq A_{g_1}\cdots A_{g_n} A_{g_{n+1}} \subset N.$$
But $A_{g_{n+1}} \not \subset H$, thus $H =  A_{g_1}\cdots A_{g_n} A_{g_{n+1}}   \subsetneq A_{g_1}\cdots A_{g_n} A_{g_{n+1}} A_{g_{n+1}}$, and this contradicts the maximality of $H$. This contradiction proves that $H = N$, so that
$$N = A_{g_1}\cdots A_{g_n} \simeq A^n.$$
\end{proof}


\paragraph{Ex. 14.3.7}{\it Let $H$ and $K$ be normal subgroups of a group $G$. 

Let $HK =\{hk \mid h \in H, k \in K\}$.
\be
\item[(a)] Prove that $HK$ is a normal subgroup of $G$.
\item[(b)] Assume that $H \cap K = \{e\}$. Prove that $hk = kh$ for all $h\in H, k \in K$.
\item[(c)] As in part (b), assume that $H\cap K = \{e\}$. Prove that the map $H \times K \to HK$ defined by $(h,k) \mapsto hk$ is a group isomorphism.
\ee
}
\begin{proof}
\item[(a)] 

  \item[$\bullet$] Since $e\in H, e \in K$, $e = e e \in HK$, thus $HK \ne \varnothing$.
  \item[$\bullet$] Let $x = hk, y = h'k'$ be any elements of $HK$, where $h,h' \in H,\ k,k'\in K$. Then
  \begin{align*}
  xy &= hkh'k'\\
  &= hh' h'^{-1}kh' k' = h''k'',
  \end{align*}
  where $h''= hh'\in H, k'' = (h'^{-1}kh' )k' $, and $h'^{-1}kh' \in K$, since $K$ is normal in $G$, so that $k'' =  (h'^{-1}kh' )k' \in K$. Thus $xy \in HK$
  \item[$\bullet$] If $x = hk \in HK$, where $h \in H, k \in K$, then
  $$x^{-1} = k^{-1}h^{-1} = h^{-1}(h k^{-1} h^{-1}),$$
  where $h^{-1} \in H$, and $h k^{-1} h^{-1} \in K$, since $K$ is normal in $G$. Thus $x^{-1} \in HK$.
  
  We have proved that $HK$ is a subgroup of $G$. Moreover, if $x = hk \in HK$, and $g \in G$ then
  $$gx g^{-1} = ghk g^{-1} = (ghg^{-1})(gkg^{-1}),$$
  where $ghg^{-1} \in H, gkg^{-1}$ in $K$, since $H,K$ are normal subgroups. Therefore $HK$ is a normal subgroup of $G$.

\item[(b)] Assume that $H \cap K = \{e\}$. If $h \in H,$ and $k \in K$, consider $z = hkh^{-1}k^{-1} \in G$. Then $z = h(kh^{-1}k^{-1})$, where $h \in H, k h^{-1} k^{-1} \in H$, so that $z \in H$. Similarly $z = (hkh^{-1}) k^{-1}$, where $hkh^{-1} \in K$ and $k^{-1} \in K$, so that $z \in H \cap K = \{e\}$. Thus $hkh^{-1}k^{-1} = e$, which proves $hk = kh$.

$$(H \lhd G, K \lhd G, H \cap K = \{e\} ) \Rightarrow \forall h\in H, \forall k \in K, \ hk = kh.$$

\item[(c)]  As in part (b), assume that $H\cap K = \{e\}$. Define
$$
\varphi
\left\{
\begin{array}{ccc}
H \times K & \mapsto & HK\\
(h,k) & \mapsto &hk.
\end{array}
\right.
$$
  \item[$\bullet$] If $u = (h,k) \in H\times K, v = (h',k' )\in H\times K$, then part (b) shows that $h'k = kh'$, thus
  \begin{align*}
  \varphi(uv) &=\varphi(hh',kk')\\
  &= hh'kk'\\
  &=hkh'k'\\
  &=\varphi(u) \varphi(v),
  \end{align*}
thus $\varphi$ is a group homomorphism.

 \item[$\bullet$] If $x$ is any element of $HK$, then by definition there are some $h,k,\ h\in H,k \in K$ such that $x = hk = \varphi(h,k)$, where $(h,k) \in H\times K$. Therefore $\varphi$ is surjective.
 
 \item[$\bullet$] if $u = (h,k) \in \ker \varphi$, then $hk = e$, thus $h = k^{-1} \in H\cap K = \{e\}$, and $(h,k) = (e,e)$. This proves $\ker \varphi = \{(e,e)\}$, and $\varphi$ injective.
 
 We have proved that $\varphi$ is a group isomorphism.
\end{proof}

 \paragraph{Ex. 14.3.8}{\it Suppose that $\gamma, \gamma' : T \to \{1,\ldots,l\}$ are one-to-one and onto. As explained in the text, these give isomorphisms $\hat{\gamma},\hat{\gamma}' : S(T)\simeq S_l$.
 \be
 \item[(a)] Explain why $\sigma = \gamma \circ (\gamma')^{-1}$ is an element of $S_l$.
 \item[(b)] Let $\sigma \in S_l$ be as in part (a), and let $\hat{\sigma} : S_l \to S_l$ be conjugation by $\sigma$. Thus $\hat{\sigma}(\tau) = \sigma\tau\sigma^{-1}$ for $\tau \in S_l$. Prove that $\hat{\gamma} = \hat{\sigma} \circ \hat{\gamma}'$.
 \ee
 This proves that $\hat{\gamma}$ and $\hat{\gamma}' $ differ by conjugation by an element of $S_l$.
 }
 \begin{proof}
 \item[(a)] Since $\gamma, \gamma' : T \to \{1,\ldots,l\}$ are bijective, then $\sigma =  \gamma \circ (\gamma')^{-1} : \{1,\ldots,l\} \to \{1,\ldots,l\}$ is bijective, so 
 $\sigma \in S_l$.
 
 \item[(b)] For all $\varphi \in S(T)$,
 \begin{align*}
 (\hat \sigma \circ \hat \gamma')(\varphi)&= \hat \sigma(\gamma' \circ \varphi \circ (\gamma')^{-1})\\
 &= \sigma \circ \gamma' \circ \varphi \circ (\gamma')^{-1} \circ \sigma^{-1}\\
 &= \gamma \circ (\gamma')^{-1} \circ \gamma' \circ \varphi \circ (\gamma')^{-1} \circ \gamma' \circ \gamma^{-1}\\
 &=\gamma \circ \varphi \circ \gamma^{-1}\\
 &= \hat \gamma(\varphi),
 \end{align*}
 thus $\hat \sigma \circ \hat \gamma' = \hat \gamma$.
 
 Note: Let $G$ be a subgroup of $S(T)$, and $G_1 = \hat \gamma(G), G_2 = \hat\gamma'(G)$ the corresponding subgroups in $S_l$, then $G_1 = \hat \sigma(G_2) = \sigma G_2 \sigma^{-1}$ are conjugate subgroups.
 \end{proof}       
 
  \paragraph{Ex. 14.3.9}{\it Let $G$ be a group of order $n$. In Section 7.4 we constructed a subgroup $H \subset S_n$ isomorphic to $G$. Prove that $H$ is regular in $S_n$.
  }
  \begin{proof}
  Recall the construction of $G \simeq H \subset S_n$ (see ex. 7.4.4, 7.4.6).
  
  Let $f : \{1,\ldots,n\} \to G$ be a bijection, and write $g_i = f(i)$, so that $G = \{g_1,\ldots,g_n\}$.
  
  For each $i \in \{1,\ldots,n\}$ there is some permutation $\sigma_i$ such that $$g_i g_j = g_{\sigma_i(j)}.$$
  
  Consider the maps
  $$
  \begin{array}{cclllll}
  \{1,\ldots,n\} & \overset{f}{\to}  & G                          &\overset{\phi}{\to}               & G'=\{\phi_g \mid g \in G \}       & \overset{\psi} {\to}     & H\\
  i                   & \mapsto            & g = g_i                  &\mapsto                             &\sigma = \phi_g 
  \left\{
  \begin{array}{ccc}
  G & \to &G\\
  h & \mapsto & gh
  \end{array}
  \right.
                        & \mapsto                       &  f^{-1}\circ \phi_g \circ f,
  \end{array}
  $$
where $G' = \{\phi_g \mid g\in G \} \subset S(G), H \subset S_n$, and $G \simeq H$. Here $f,\phi,\psi$ are bijective, and $\phi,\psi$ are group isomorphisms (Ex .7.4.4).

Note that, as already seen in ex. 7.4.4, for all $j\in \{1,\ldots,n\}$,
\begin{align*}
(f^{-1} \circ \phi_{g_i} \circ f)(j) &= (f^{-1} \circ \phi_{g_i})(g_j)\\
&= f^{-1}(g_i g_j)\\
&= f^{-1}(g_{\sigma_i(j)})\\
&= \sigma_i(j),
\end{align*}
thus
$$\sigma_i = f^{-1} \circ \phi_{g_i} \circ f,$$
and $H = \{\sigma_1,\ldots,\sigma_n\}$.

Write $\delta = \psi \circ \phi \circ f$, so that $\delta(i) = ( \psi \circ \phi \circ f)(i)  =f^{-1} \circ \phi_{g_i} \circ f,\ i=1,\ldots,n.$

We define $\gamma = \delta^{-1}$, thus for all $i \in \{1,\ldots,n\}$, 
$$\gamma(f^{-1} \circ \phi_{g_i} \circ f) = i.$$

Here $T = \{1,\ldots,n\}$, since $H \subset S(\{1,\ldots,n\})$. Then $\gamma : G \to T$ induces an isomorphism $\hat \gamma : S(G) \to S(T) = S_n$ defined by 
$$\hat \gamma(\sigma) =\gamma \circ \sigma \circ \gamma^{-1}.$$
Now we compute $\hat \gamma (\varphi_h)$, where $h \in H$ is a permutation, and $\varphi_h(k) = hk = h \circ k,\ k\in H$. 
Since $h \in H$, there is some $i \in \{1,\ldots,n\}$ such that $h = f^{-1} \circ \phi_{g_i} \circ f =\sigma_i$.

For all $j \in \{1,\ldots,n\}$,
\begin{align*}
\hat \gamma (\varphi_h)(j) &= (\gamma \circ \varphi_h \circ \gamma^{-1})(j)\\
&= (\gamma \circ \varphi_h)(f^{-1} \circ \phi_{g_j} \circ f)\\
&= \gamma(h \circ f^{-1} \circ \phi_{g_j} \circ f)\\
&= \gamma(f^{-1} \circ \phi_{g_i} \circ f \circ f^{-1} \circ \phi_{g_j} \circ f)\\
&=\gamma(f^{-1}  \circ \phi_{g_i}\circ \phi_{g_j} \circ f)\\
&= \gamma(f^{-1}  \circ \phi_{g_i g_j} \circ f)\\
&= \gamma(f^{-1}  \circ \phi_{g_{\sigma_i(j)}} \circ f)\\
&=\sigma_i(j).
\end{align*}
Therefore $\hat \gamma (\varphi_h) = \sigma_i = h\in H$.

We have proved that the isomorphism $\hat \gamma$ takes $\{\varphi_h \mid h \in H\}$ in $H$, and since these two subgroups have same order $n$,
$$\hat \gamma(\{\varphi_h \mid h \in H\}) = H.$$
The definition of a regular subgroup of $S(T)$ is satisfied, so $H$ is a regular subgroup of $S_n$.
  \end{proof}
 
  \paragraph{Ex. 14.3.10}{\it A permutation group $G \subset S_l$ is regular if there is a  one-to-one onto map $\gamma : G \to \{1,\ldots,l\}$ such that $\hat \gamma : S(G) \simeq S_l$ maps $\{\varphi_g \mid g \in G\} \subset S(G)$ to $G \subset S_l$. Recall that $\varphi_g \in S(G)$ is defined by $\varphi_g(h) = gh$ for $h\in G$. The goal of this exercise is to show that $G$ is regular if and only if it is transitive with trivial isotropy subgroups.
  \be
  \item[(a)] Let $G \subset S_l$ be regular. Prove that $G$ is transitive and that the isotropy subgroups of $G$ are trivial.
  \item[(b)] For the rest of the exercise, assume that $G$ is transitive with trivial isotropy subgroups. Define $\gamma : G \to \{1,\ldots,l\}$ by $\gamma(\tau) = \tau(1)$ for $\tau \in G$. Prove that this map is one-to-one and onto.
  \item[(c)] The map $\gamma$ of part (b) gives $\hat \gamma:S(G) \simeq S_l$. Show that $\hat \gamma(\varphi_g) = g$, and conclude that $G$ is regular.
  \ee
  }
  \begin{proof}
  \item[(a)] Recall that $\hat \gamma(\sigma) = \gamma \circ \sigma \circ \gamma^{-1}$ for all $\sigma \in S_l$.
  
  Let $i,j$ be any elements in $\{1,\ldots,l\}$. We must find $g' \in G$ such that $g'(i) = j$.
  
  Take $g = \gamma^{-1}(j) [\gamma^{-1}(i)]^{-1} \in G$, and $g' = \hat \gamma(\varphi_g)$. Then $g' \in G$ since $\hat \gamma$ maps ${\{\varphi_g \mid g \in G\}}$ to $G$.
  
  Then $g \gamma^{-1}(i) = \gamma^{-1}(j)$, thus $(\gamma \circ \varphi_g \circ \gamma^{-1})(i) = j$, that is $g' (i) = (\hat \gamma(\varphi_g))(i) = j$, where $g' \in G$. This proves that $G$ is transitive.
  
  Let $i$ be any element in $\{1,\ldots,l\}$, and take  $g' \in G_i$, where $G_i$ is the isotropy subgroup of $i$, so that $g'(i) = i$. Since $\hat \gamma$ maps $\{\varphi_g \mid g \in G\} \subset S(G)$ to $G$, there exists some $g \in G$ such that $g' = \hat \gamma(\varphi_g) =\gamma \circ \phi_g \circ \gamma^{-1}$.
  
  Then $g'(i) = i$ implies $g \gamma^{-1}(i) = \gamma^{-1}(i)$, thus $g = e , \phi_g = e$, and $g' = \gamma \circ \phi_g \circ \gamma^{-1} = e$, where $e = 1_G$ is the identity element of $G$. Therefore $G_i=\{e\}$ for all $i \in \{1,\ldots,l\}$.
  
 \item[(b)] We prove first that $\gamma$ is injective. Suppose that $\gamma(\tau) = \gamma(\tau')$, where $\tau,\tau' \in G$. Then $\tau(1) = \tau'(1)$, thus $(\tau^{-1} \circ \tau')(1) = 1$, and $\tau^{-1} \circ \tau' \in G_1$, where $G_1$ is the isotropy subgroup of $1$. But $G_1 = \{e\}$, thus $\tau^{-1} \circ \tau' = e$, and $\tau = \tau'$. We have proved that $\gamma$ is injective.
 
 Now take $i$ be any element in $\{1,\ldots,l\}$. Since $G$ is transitive, there is some $\tau \in G$ such that $\tau(1) = i$, that is $\gamma(\tau) = i, \tau \in G$. This proves that $\gamma$ is surjective.
 
 $\gamma : G \to \{1,\ldots,n\}$ is bijective.
 
 \item[(c)] let $i$ be any element in $\{1,\ldots,l\}$. Since $G$ is transitive, there is some $\tau \in G$ such that $\tau(1) = i$, thus $\gamma(\tau) = i$. Therefore
 \begin{align*}
 [\hat \gamma(\varphi_g)](i) &= (\gamma \circ \varphi_g \circ \gamma^{-1})(i)\\
 &= (\gamma \circ \varphi_g)(\tau)\\
 &= \gamma(g\tau)\\
 &= (g\tau)(1) = g(i).
 \end{align*}
 Since this is true for all $i$,
 $$\hat \gamma(\varphi_g) = g.$$


This proves that the isomorphism $\hat \gamma$ maps $\{\varphi \mid g \in G\}$ in $G$, and since these two subgroup have same order $l$, $\hat \gamma$ maps $\{\varphi_g\mid g \in G\}$ on $G$.

$G$ is a regular subgroup of $S_l$.

  \end{proof}

Note: This proves also that $G \subset S_l$ is regular iff there is some bijective $\gamma : G \to \{1,\ldots l\}$ such that $\hat \gamma$ maps $\varphi_g$ to $g$, for every $g\in G$.

 \paragraph{Ex. 14.3.11}{\it We can regard $\F_p^n$ as both a group (under addition) and a vector space over $\F_p$ (under addition and scalar multiplication). However, since we are over $\F_p$, scalar multiplication can be build out of addition. Use this observation to prove the following:
 \be
 \item[(a)] Any subgroup of $\F_p^n$ is a subspace.
 \item[(b)] Any group homomorphism $\gamma : \F_p^n \to \F_p^n$ is linear.
 \ee
 }
 
 \begin{proof}
 \item[(a)] Let $H$ be a subgroup of $\F_p^n$.
If $v = (\alpha_1,\ldots,\alpha_n) \in H$, we show by induction on $k \in \N$ that $[k]_p v \in H$.
 First $[0]_p v = (0,\ldots,0) \in H$. If $[k]_p v \in H$, then $[k+1]_p v = [k]_pv + v \in H$. Therefore 
 $$\forall k \in \N,\ [k]_p v \in H.$$
 Since for every $\lambda \in \F_p$, there is some $k\in \N$ such that $\lambda =[k]_p$, we can conclude that, for all $v \in H$, for all $\lambda \in \F_p$, $\lambda v \in H$. Thus $H$ is a subspace of $\F_p^n$.
 
 \item[(b)] Let $\gamma : \F_p^n \to \F_p^n$ a group homomorphism. if $v \in \F_p^n$, We show by induction that $\gamma([k]_p v) = [k]_p \gamma(v)$ for $k \in \N$.
 
For $k = 0$, $ \gamma([0]_p v) = 0 =  [0]_p \gamma(v)$. If $\gamma([k]_p v) = [k]_p \gamma(v)$ for some $k\in \N$, then, since $\gamma$ is a group homomorphism,
$$\gamma([k+1]_p v) = \gamma([k]_p v + v) = \gamma([k]_pv) + \gamma(v) = [k]_p \gamma(v) + \gamma(v) = ([k]_p + 1)\gamma(v) = [k+1]_p \gamma(v).$$
Therefore $\gamma([k]_p v) = [k]_p \gamma(v)$  for all $k \in \N$. This proves that $\gamma(\lambda v) = \lambda f(v)$ for all $v \in \F_p^n$, for all $\lambda \in \F_p$, thus $\gamma$ is linear.
 
 \end{proof}
 
  \paragraph{Ex. 14.3.12}{\it This exercise will use the notation of the proof of Proposition 14.3.20.
  \be
  \item[(a)] Suppose that $V \subset \F_p^n$ is a nontrivial subspace such that $g(V) \subset V$ for all $g\in G_0$. Use the cosets of $V$ in $\F_p^n$ to prove that $G$ is imprimitive.
  \item[(b)] Explain why $\F_p^n$ is normal in $G$, and prove that $G/\F_p^n \simeq G_0$. Use this to prove part (b) of Proposition 14.3.20.
  \ee
 }
 \begin{proof}
 Here
 $$\F_p^n \subset G \subset \mathrm{AGL}(n,\F_p) \subset S_{p^n}.$$
 \item[(a)] Consider a complete system $\{a_1,\ldots,a_k\}$ of representatives of cosets of $V$ in $\F_p^n$, so that the cosets of $V$ are the disjoint cosets $$R_1 = a_1+V, \ldots,R_k = a_k + V.$$
 Then $\F_p^n = \bigcup\limits_{i=1}^k (a_i + V)$ is the disjoint union of parallel affine subspaces.
 
 Let $g \in G \subset \mathrm{AGL}(n,\F_p)$. Recall (see the text following Corollary 14.3.18) that
 $$G = \{\gamma_{A,w} \mid w \in \F_p^n, A \in G_0\},$$
 thus $g = \gamma_{A,w}$, for some $w\in \F_p^n$ and $A\in G_0$. For all $v \in V$
 $$g\cdot (a_i + v) = A\cdot (a_i + v) + w =A\cdot a_i + w + A\cdot v.$$
Since $A \in G_0$, $A \cdot V \subset V$, thus $g\cdot (a_i + v) \in (A\cdot a_i + w)+ V.$ The cosets $R_i,\ 1\leq i \leq k$, partition $\F_p^n$, thus there is some index $j$ such that $A\cdot a_i + w \in R_j = a_j + V$, so that $A\cdot a_i + w + V = a_j + V$, and 
 $$g(a_i + V) \subset a_j + V.$$ 
Moreover, the cosets $R_i = a_i +V$ have same cardinality, and $g$ is bijective, therefore $|a_j + V| = |a_i + V| = |g(a_i + V)|$. This proves
  $$g(a_i + V) = a_j + V.$$ 
  To conclude, there is a partition of $\F_p^n$ such that for every $1\leq i \leq k$, we have $g(R_i) = R_j$ for some $1\leq j \leq k$. Since $1<|V|<p^n$, then $k>1$ and $|R_1| =\cdots=|R_k|>1$. This proves that $G\subset S_{p^n}$ is imprimitive.
  
  \item[(b)] We have seen in Exercise 2 that the group homomorphism
  $$
\varphi
\left\{
\begin{array}{ccc}
\mathrm{AGL}(n,\F_q) & \to &\mathrm{GL}(n,\F_q)\\
\gamma_{A,v} & \mapsto &A.
\end{array}
\right.
$$
induces an isomorphism 
$$\mathrm{AGL}(n,\F_q)/ \F_q^n \simeq \mathrm{GL}(n,\F_q).$$
Consider the restriction $\varphi_0$ of $\varphi$, defined by
 $$
\varphi_0
\left\{
\begin{array}{ccc}
G & \to &G_0\\
\gamma_{A,v} & \mapsto &A.
\end{array}
\right.
$$
Since $G = \{\gamma_{A,w} \mid w \in \F_p^n, A \in G_0\}$, $\varphi_0$ is surjective, and $\ker \varphi_0 = \{\gamma_{I_n,w} \mid w \in \F_{p^n}\} \simeq \F_p^n$, thus, with the usual identification $ \{\gamma_{I_n,w} \mid w \in \F_{p^n}\} = \F_p^n$, $\F_p^n$  is a normal subgroup of $G$, and
$$G/\F_p^n \simeq G_0.$$
Now we prove (b).

By Theorem 8.1.4, $G$ is solvable if and only if $\F_p^n$ and $G/\F_p^n$ are normal. Since we know that  $\F_p^n$  is a normal subgroup of $G$, we obtain the equivalence
$$G \text{ is solvable } \iff G_0 \text { is solvable.}$$
 \end{proof}
 
 
  \paragraph{Ex. 14.3.13}{\it Consider the definition of $k$-transitive given in the Mathematical Notes.
  \be
  \item[(a)] Prove that $S_n$ is $n$-transitive.
  \item[(b)] Prove that $A_n$ is $(n-2)$-transitive when $n\geq 3$.
  \ee
  }
  \begin{proof}
 \item[(a)] Take any ordered $n$-tuple $(a_1,\ldots,a_n)$ of distinct elements of $\{1,\ldots n\}$. Then consider the map 
  $$
  \sigma
  \left\{
  \begin{array}{ccc}
  \{1,\ldots,n\} & \to &\{1,\ldots,n\}\\
  i &\mapsto &a_i.
  \end{array}
  \right.
  $$
  Then $\{a_1,\ldots,a_n\} \subset \{1,\ldots,n\}$ and, since the $a_i$ are distinct, $|\{a_1,\ldots,a_n\}| = n$, so that $\{a_1,\ldots,a_n\} = \{1,\ldots,n\}$. This shows that $\sigma$ is surjective, and since the $a_i$ are distinct, $\sigma$ is injective, thus $\sigma$ is a permutation: $\sigma \in S_n$
  
   ($\sigma =
   \left(
   \begin{array}{cccc}
   1&2&\cdots&n\\
   a_1&a_2&\cdots&a_n
   \end{array}
   \right)
   $ is the only permutation such that $\sigma(i) = a_i,\ i=1,\ldots,n$).
 
 Since 
$$\sigma\cdot(1,\ldots,n) = (a_1,\ldots,a_n),$$
the orbit of $(1,\ldots,n)$ is the whole set $P_n$ of ordered $n$-tuples of distinct elements of $\{1,\ldots,n\}$. This proves that there is only one orbit, and that $G$ acts transtitively on $P_n$, i.e. $S_n$ is $n$-transitive.

\item[(b)] Take any ordered $(n-2)$-tuple $(a_1,\ldots,a_{n-2})$ of distinct elements of $\{1,\ldots,n\}$. Name $a,b$ the two remaining elements:
$$\{a,b\} = \{1,\ldots,n\} \setminus \{a_1,\ldots,a_{n-2}\}.$$
Consider the two permutations
$$
\sigma =
   \left(
   \begin{array}{cccccc}
   1&2&\cdots&n-2 & n-1&n\\
   a_1&a_2&\cdots&a_{n-2}& a & b
   \end{array}
   \right),
 $$
 $$
\tau =
   \left(
   \begin{array}{cccccc}
   1&2&\cdots&n-2 & n-1&n\\
   a_1&a_2&\cdots&a_{n-2}& b & a
   \end{array}
   \right).
 $$
 Then $\sigma\cdot (1,\ldots,n-2) = (a_1,\ldots,a_{n-2}) = \tau\cdot (1,\ldots,n-2)$.
 
But $\tau = (a\, b) \sigma$, thus one of the two permutations $\sigma, \tau$ is in $A_n$, therefore $(a_1,\ldots,a_{n-2}) $ is in the orbit $A_n\cdot (1,\ldots,n-2)$ of $(1,\ldots,n-2) \in P_{n-2}$. Therefore $A_n$ acts transitively on $P_{n-2}$. $A_n$ is $(n-2)$-transitive.
  \end{proof}
 
 
  \paragraph{Ex. 14.3.14}{\it Consider the groups $\mathrm{GL}(2,\F_q), \mathrm{SL}(2,\F_q), \mathrm{PGL}(2,\F_q)$, and $\mathrm{PSL}(2,\F_q)$ defined in the Mathematical Notes.
  \be
  \item[(a)] Prove that $|\mathrm{GL}(2,\F_q)| = q(q-1)(q^2-1)$.
  \item[(b)] Prove that $|\mathrm{SL}(2,\F_q)|= | \mathrm{PGL}(2,\F_q)| = q(q^2-1)$.
  \item[(c)] Prove that $\mathrm{PSL}(2,\F_q) \simeq \mathrm{SL}(2,\F_q)/\{\pm I_2\}$, and conclude that
  $$|\mathrm{PSL}(2,\F_q)| = 
  \left\{
  \begin{array}{ll}
  \frac{1}{2} q (q^2-1),&q\ne 2^n\\
  q(q^2-1), & q=2^n.
  \end{array}
  \right.
  $$
  \item[(d)] Compute $|\mathrm{PSL}(2,\F_q)|$ for $q=2,3,4,5,7$.
  \item[(e)] Show that $|\mathrm{GL}(3,\F_2)| = |\mathrm{PSL}(3,\F_2)| = 168$.
  \ee
  }
  \begin{proof}
  \item[(a)] To build an $2\times 2$ matrix $A \in \mathrm{GL}(2,\F_q)$, we must first choose a nonzero vector $u \in \F_q^2$ (the first column of the matrix), then a vector $v \not \in \F_q u$ (the second column). Since $|\F_q^2 \setminus \{0\}| = q^2-1$, and $|\F_q^2 \setminus \F_q u| = q^2 -q$, we obtain
  $$|\mathrm{GL}(2,\F_q)| = (q^2 - 1)(q^2 -q) =q(q-1)(q^2 - 1).$$
  
  \item[(b)] $\mathrm{SL}(2,\F_q)$ is the kernel of the surjective homomorphism $det : \mathrm{GL}(2,\F_q) \to \F_q^*$, so that
  $$\mathrm{GL}(2,\F_q)/ \mathrm{SL}(2,\F_q) \simeq \F_q^*.$$
  Therefore $|\mathrm{GL}(2,\F_q)|/| \mathrm{SL}(2,\F_q)| = |\F_q^*|$, thus
  $$| \mathrm{SL}(2,\F_q)| = q(q-1)(q^2 - 1)/(q-1) = q(q^2-1).$$
  The definition
  $$\mathrm{PGL}(2,\F_q) = \mathrm{GL}(2,\F_q)/\F_q^* I_n$$
  gives
  $$|\mathrm{PGL}(2,\F_q)| = q(q-1)(q^2 - 1)/ (q-1) = q(q^2-1).$$
  \item[(c)] Consider the restriction of the canonical projection $\pi : \mathrm{GL}(2,\F_q) \to \mathrm{GL}(2,\F_q)/\F_q^* I_n$ to $\mathrm{SL}(2,\F_q)$. We obtain
  $$
  \varphi
  \left\{
  \begin{array}{ccc}
  \mathrm{SL}(2,\F_q) & \to & \mathrm{PGL}(2,\F_q)\\
  A & \mapsto & [A].
  \end{array}
  \right.
  $$
  
  If $A \in   \mathrm{SL}(2,\F_q) $ is such that $[A] = [I_2]$, then $A = \lambda I_2, \lambda \in \F_q^*$, and $\det(A) = \lambda^2 = 1$, so that $\lambda = \pm 1$. We have proved $\ker(\pi_1) = \{I_2,-I_2\}$ (where $I_2 = -I_2$ if the characteristic is 2). Therefore
  $$\im(\varphi) \simeq \mathrm{SL}(2,\F_q)/\{I_2,-I_2\}.$$
  
  Note that $\varphi$ is not surjective.
  
  By the definition given in the mathematical notes, $\im \varphi = \mathrm{PSL}(2,\F_q)$, thus
  $$\mathrm{PSL}(2,\F_q) \simeq \mathrm{SL}(2,\F_q)/\{I_2,-I_2\}.$$
  
  If $q=2^n$, then the characteristic of $\F_q$ is 2, and $I_2 = - I_2$, so that $|\mathrm{PGL}(2,\F_q) | = | \mathrm{SL}(2,\F_q)|$. Otherwise $|\mathrm{PGL}(2,\F_q) | = \frac{1}{2}| \mathrm{SL}(2,\F_q)|$. This proves
   $$|\mathrm{PSL}(2,\F_q)| = 
  \left\{
  \begin{array}{ll}
  \frac{1}{2} q (q^2-1),&q\ne 2^n\\
  q(q^2-1), & q=2^n.
  \end{array}
  \right.
  $$
  \item[(d)]
  $$
  \begin{array}{c|ccccc}
   q & 2 & 3 & 4 & 5 & 7\\
   \hline
  |\mathrm{PSL}(2,\F_q)| & 6 & 12 & 60 & 60 & 168
  \end{array}
   $$
   \item[(e)]  The same reasoning as in part (a) shows that
   $$|\mathrm{GL}(3,\F_q)| = (q^3 - 1)(q^3 -q)(q^3 -q^2),$$
   thus, for $q=2$,
      $$|\mathrm{GL}(3,\F_2)| = 168.$$
     As in parts (b) and (c),
      $$\mathrm{GL}(3,\F_q)/ \mathrm{SL}(3,\F_q) \simeq \F_q^*,\qquad  \mathrm{PSL}(3,\F_q) \simeq \mathrm{SL}(3,\F_q)/\{\pm I_2\}.$$
    For $q=2$, $I_2 = -I_2$, and  $\F_q^* = \{1\}$, thus
    $$|\mathrm{PSL}(3,\F_2)| = |\mathrm{SL}(3,\F_2)|  = |\mathrm{GL}(3,\F_2)| = 168.$$
      
  \end{proof}

 
  \paragraph{Ex. 14.3.15}{\it Prove that $\mathrm{GL}(2,\F_2) = \mathrm{SL}(2,\F_2) \simeq \mathrm{PSL}(2,\F_2) \simeq S_3$, and ${\mathrm{PSL}(2,\F_3) \simeq A_4}$.
 }
 
 \begin{proof}
 Since $\mathrm{SL}(2,\F_2)$ is the kernel of the surjective homomorphism $det : \mathrm{GL}(2,\F_2) \to \F_2^*$, where $\F_2^* = \{1\}$ is trivial, we obtain $\mathrm{SL}(2,\F_2) = \mathrm{GL}(2,\F_2)$. 
 
 By part (c) of Exercise 14, $\mathrm{PGL}(2,\F_q) \simeq \mathrm{SL}(2,\F_q)/\{I_2,-I_2\}.$
   For $q = 2$, $I_2 = -I_2$, thus 
  $$\mathrm{PGL}(2,\F_2) \simeq \mathrm{SL}(2,\F_2) = \mathrm{GL}(2,\F_2).$$
  Moreover,
  $$\mathrm{GL}(2,\F_2) = \left\{
   \begin{pmatrix}
  1 & 0\\
  0 & 1
  \end{pmatrix},
  \begin{pmatrix}
  0 & 1\\
  1 & 1
  \end{pmatrix},
   \begin{pmatrix}
  0 & 1\\
  1 & 0
  \end{pmatrix},
   \begin{pmatrix}
  1 & 0\\
  1 & 1
  \end{pmatrix},
   \begin{pmatrix}
  1 & 1\\
  0 & 1
  \end{pmatrix},
    \begin{pmatrix}
  1 & 1\\
  1 & 0
  \end{pmatrix}
  \right\},
  $$
  thus $|\mathrm{GL}(2,\F_2)| = 6$, and $\mathrm{GL}(2,\F_2) $ is not Abelian:
  $$
   \begin{pmatrix}
  1 & 1\\
  1 & 0
  \end{pmatrix}
  \begin{pmatrix}
  1 & 1\\
  0 & 1
  \end{pmatrix}
  =
    \begin{pmatrix}
  1 & 0\\
  1 & 1
  \end{pmatrix}
  \ne
    \begin{pmatrix}
  0 & 1\\
  1 & 0
  \end{pmatrix}
=
 \begin{pmatrix}
  1 & 1\\
  0 & 1
  \end{pmatrix}
  \begin{pmatrix}
  1 & 1\\
  1 & 0
  \end{pmatrix}.
  $$
Since there is only one non Abelian group of order 6, up to isomorphism,
  $$\mathrm{GL}(2,\F_2) = \mathrm{SL}(2,\F_2) \simeq \mathrm{PSL}(2,\F_2) \simeq S_3.$$
  
  \bigskip
  
To prove the last inclusion, we use the action of $\mathrm{PGL}(2,F)$ on $F \cup \{\infty\}$ described in section 7.5. We resume this description in the following lemma:


{\bf Lemma. }{\it The operation defined for all  $M = 
\left(
\begin{array}{cc}
 a&   b  \\
 c&   d   
\end{array}
\right)$ and $z \in \dot F$ by 
$$
[M] \cdot z = \frac{az+b}{cz+d}, \qquad (z \in F \setminus \{-d/c\}),
$$
and also
$$ [M] \cdot (-d/c) = \infty, \qquad [M] \cdot \infty = a/c \qquad ([M] \cdot \infty = \infty \text { if } c = 0).$$
is a (left) action of $\mathrm{PGL}(2, F)$ on the projective line $\dot F$.
}

\bigskip

In this exercise,  $F = \F_3$. Consider the restriction of this action to the subgroup $\mathrm{PSL}(2,\F_3)$ on $\dot{\F}_3= \F_3 \cup \{\infty\} = \{0,1,2,\infty\}$. 


 This action defines a group homomorphism
 $$\varphi
 \left\{
 \begin{array}{ccc}
 \mathrm{PSL}(2,\F_3) & \to & S(\dot{\F}_3)\\
 {[}M{]} & \mapsto & 
     \varphi([M]) \left \{
     \begin{array}{ccc}
      \dot{\F}_3 & \to & \dot{\F}_3\\
      z & \mapsto &[M]\cdot z.
     \end{array}
     \right.
  \end{array}
 \right.
 $$
 We show that $\ker(\varphi)$ is trivial (the action is faithful): Let $[M] \in  \mathrm{PSL}(2,\F_3)$, with $M  =\left(
\begin{array}{cc}
 a&   b  \\
 c&   d   
\end{array}
\right) \in  \mathrm{SL}(2,\F_3)$, such that $\varphi([M])$ is the identity of $\dot \F_3$. Then, for all $z \in \dot \F_3, [M]\cdot z = z$. The equality $[M]\cdot \infty = \infty$ shows that $c=0$. Thus for every $z\in \F_3$,
$$\frac{az+b}{d} = z,$$
and then $az + b = dz, (a-d)z + b = 0$. With $z = 0$, we obtain $b=0$, and with $z=1$, $a-d = 0$. Therefore $M  = a I_2, \ a \in \F_3$, and $[M] = [I_2]$. We have proved that the kernel of the action is trivial, in other words $\varphi$ is injective, and $ \mathrm{PSL}(2,\F_3) \simeq \im(\varphi) \subset S(\dot \F_3)$ is isomorphic to a subgroup of  $S(\dot \F_3) \simeq S_4$, therefore $  \mathrm{PSL}(2,\F_3)$ is isomorphic to a subgroup of $S_4$. By Exercise 14, parts (c) and (d), $| \mathrm{PSL}(2,\F_3)| = 12$, and the only subgroup of $S_4$ of order $12$ is $A_4$. Therefore
$$ \mathrm{PSL}(2,\F_3) \simeq A_4.$$

  \end{proof}
  
 Note:  We explain why the permutations corresponding to elements of $ \mathrm{PSL}(2,\F_3) \simeq A_4$ are even, with another proof of this isomorphism.
 
 Let $[T],[S]$ the two elements of $\mathrm{PSL}(2,\F_3)$ defined by $[T ]\cdot z = z+1$ and $[S]\cdot z = -1/z$, corresponding to the matrices 
 $$T = 
 \begin{pmatrix}
  1 & 1\\
   0 & 1
  \end{pmatrix},
  \qquad 
  S = 
   \begin{pmatrix}
  0 & -1\\
   1 & 0
  \end{pmatrix},
  \qquad T,S \in \mathrm{SL}(2,\F_3).
$$
The permutation of $\dot \F_3$ corresponding to $[T],[S]$ by $\varphi$ are
$$\varphi([T]) = 
\left(
\begin{array}{cccc}
0 & 1 & 2 & \infty\\
1 & 2 & 0 & \infty
\end{array}
\right) =(0\,1\,2), \qquad 
\varphi([S]) = \left(
\begin{array}{cccc}
0 & 1 & 2 & \infty\\
\infty & 2 & 1 & 0
\end{array}
\right) = (0\,\infty)(1\, 2)
$$
If we take a numbering of $\dot \F_3$ to $\{1,2,3,4\}$, for instance $ f : 0 \mapsto 1, 1 \mapsto 2, 2 \mapsto 3, \infty \mapsto 4$, the permutations corresponding to $T,S$ are 
$(1\,2\,3) , (1\,4)(2\,3)$, both even.

A well known theorem shows that the matrices $[T],[S]$ generate $\mathrm{PSL}(2,\Z)$ : every matrix $M \in \mathrm{SL}(2,\Z)$ is of the form
$$M = \pm U_1\cdots U_k,\quad U_i \in \{T,S\}.$$

Reducing modulo 3, every matrix $M \in \mathrm{SL}(2,\F_3)$ is of the same form, thus $[T],[S]$ generate $\mathrm{PSL}(3,\F_3).$ This proves that the permutations corresponding to  $\mathrm{PSL}(3,\F_3)  = \langle [S], [T] \rangle$ are all even, and the argument of cardinality shows that $\mathrm{PSL}(3,\F_3) \simeq \im(\varphi) \simeq A_4$.

With Sage:
 \begin{verbatim}
 G = PermutationGroup(['(1,4)(2,3)','(1,2,3)'])
 G.order()
 		12
 G.structure_description()
	        'A4'		
 \end{verbatim}
 
 This confirms $A_4 = \langle (1\,4)(2\, 3), (1\,2\,3) \rangle$, and consequently $\mathrm{PSL}(3,\F_3)  = \langle [S], [T] \rangle$.
 
 \bigskip
 
 Note 2: The same method can be applied to prove anew the first isomorphism $\mathrm{PSL}(2,\F_2) \simeq S_3$. Indeed, the action of $\mathrm{PGL}(2,\F_2)$ on $\F_2 \cup \{\infty\} = \{0,1,\infty\}$, restricted to $\mathrm{PSL}(2,\F_2)$, gives an injective group homomorphism
 
 $$\varphi
 \left\{
 \begin{array}{ccc}
 \mathrm{PSL}(2,\F_2) & \to & S(\dot{\F}_2)\simeq S_3\\
 {[}M{]} & \mapsto & 
     \varphi([M]) \left \{
     \begin{array}{ccc}
      \dot{\F}_2 & \to & \dot{\F}_2\\
      z & \mapsto &[M]\cdot z.
     \end{array}
     \right.
  \end{array}
 \right.
 $$
 
 Moreover $| \mathrm{PSL}(2,\F_2)| = |S(\dot{\F}_2)| = 6$, thus $\varphi$ is bijective. Therefore $\mathrm{PSL}(2,\F_2) \simeq S_3$.
 
 
 
  \paragraph{Ex. 14.3.16}{\it Let $G$ be a finite group with socle $H$. Prove that $H$ is isomorphic to a product of finite simple groups.
 
 }
 
 \begin{proof}
 If $G$ has a unique minimal normal subgroup $N$, then the socle $H$ is $N$, and by Proposition 14.3.10, $H = N$ is a product of finite simple groups.
  
 
 Suppose now that $G$ contains two minimal normal subgroups $N_1\ne N_2$.  Then $N_1 \cap N_2$ is normal in $G$, and $N_1$ is a minimal normal subgroup, therefore $N_1 \cap N_2 = N_1$ or $N_1 \cap N_2 = \{e\}$, and using the minimality of $N_2$, $N_1 \cap N_2 = N_2$ or $N_1 \cap N_2 = \{e\}$. If $N_1 \cap N_2 \ne \{e\}$, then $N_1 = N_1 \cap N_2 = N_2$, in contradiction with the assumption, thus $N_1 \cap N_2 = \{e\}$.
 
 By Exercise 7, $N_1N_2$ is a normal subgroup of $G$, such that $(x_1,x_2) \mapsto x_1x_2$ is an isomorphism from $N_1 \times N_2$ to $N_1N_2$.
 
Consider $\cal A$ the set of all subgroups of $G$ of the form $N_1 \cdots N_k$ such that $N_1\cdots N_k$ is a normal subgroup of $G$, and such that the map $(x_1,\ldots,x_k) \mapsto x_1\cdots x_k$ is an isomorphism from $N_1 \times \cdots \times N_k$ to $N_1\cdots N_k$.
 
Then pick $A =N_1 \cdots N_k$ an element of $\cal A$ of maximal order. Reasoning by contradiction, suppose that $A \ne H$. Then there is some minimal normal subgroup $N$ such that $N\not \subset A$, otherwise the socle, which is the subgroup $H$ generated by the minimal normal subgroups of $G$ would be $A = N_1\cdots N_k$. Write $N_{k+1} = N$.

Then $N_1\cdots N_k \cap N_{k+1} \ne N_{k+1}$, thus the minimality of $N$ shows that $N_1\cdots N_k \cap N_{k+1} = \{e\}$. Then Exercise 7 shows that $(N_1\cdots N_k)N_{k+1}$ is normal, and the map $(x_1,\ldots,x_k) \mapsto x_1\cdots x_{k+1}$ is an isomorphism from $N_1 \times \cdots \times N_{k+1}$ to $N_1\cdots N_{k+1}$.

Therefore $N_1\cdots N_{k+1} \in {\cal A}$, and since $|N_1\cdots N_{k+1}|> |N_1\cdots N_{k}| = |A|$, this contradicts the maximality of $A$.

This proves $H = N_1\cdots N_k$, so that the socle $ H =N_1\cdots N_k \simeq N_1 \times \cdots N_k$ is a direct product of minimal normal subgroups, and by Proposition 14.3.10, each of them is isomorphic to a direct product of simple group. We have proved that $H$ is isomorphic to a product of finite simple groups.

 \end{proof}
 
 
\paragraph{Ex. 14.3.17}{\it  Prove Galois' formula (14.18) for $\mathrm{AGL}(n,\F_p)$:
$$|\mathrm{AGL}(n,\F_p)| = p^n(p^n-1)(p^n-p)\cdots(p^n - p^{n-1}).$$
 }
 \begin{proof}
We have seen in Exercise 2 that $\mathrm{AGL}(n,\F_p)/ \F_p^n \simeq \mathrm{GL}(n,\F_p)$, thus
 $$|\mathrm{AGL}(n,\F_p)| = p^n |\mathrm{GL}(n,\F_p)|.$$
 To construct a matrix $A \in \mathrm{GL}(n,\F_p)$, we must choose the successive columns $c_i$ of the matrix $A$, so that $(c_1,\ldots,c_n)$ is a base of $\F_p^n$. This is equivalent to take  $c_1 \ne 0$, then $c_2 \not \in \langle c_1 \rangle$, then $c_3 \not \in \langle c_1,c_2 \rangle$,$\ldots$, up to $c_n \not \in \langle c_1,\ldots,c_{n-1} \rangle $. Since $|\langle c_1,\ldots,c_{k} \rangle| = p^k$, when $c_1,\ldots,c_k$ are linearly independent, we obtain $|\mathrm{GL}(n,\F_p)| = (p^n-1)(p^n-p)\cdots(p^n - p^{n-1})$, thus
 $$|\mathrm{AGL}(n,\F_p)| = p^n(p^n-1)(p^n-p)\cdots(p^n - p^{n-1}).$$
 \end{proof}
 
 \paragraph{Ex. 14.3.18}{\it Here are some observations related to Exercise 5.
 \be
 \item[(a)] Give an example to show that Exercise 5 is false if we drop the assumption that $A$ and $B$ are non-Abelian.
 \item[(b)] Let $A_1,\ldots,A_r$ be non-Abelian simple groups. Determine all nontrivial normal subgroups of $A_1\times\cdots\times A_r$.
 \ee
}
\begin{proof}
\item[(a)] Consider the group $G = \Z/3\Z \times \Z/3\Z = A \times B$, where $A = B = \Z/3\Z$ is simple. This is the additive group of the vector space $\F_3^2$. If $v = (1,2)$ (for instance), the vectorial line 
$$N = \F_3\cdot v = \{(0,0), (1,2),(2,1)\}$$
 is a subspace of $\F_3^2$, therefore a subgroup  of $G$. Moreover $N$ is normal and not trivial, and $N \ne \{0\} \times B , N\ne A \times \{0\}$. This counterexample shows that Exercise 5 is false if we drop the assumption that $A$ and $B$ are non-Abelian.
\item[(b)] We will follow the steps of Exercise 5.

\item[$\bullet$]  First, note that the subsets $H_1\times \cdots H_r \subset G$, where $H_i = \{e_{A_i}\}$ or $H_i = A_i$ for all $i = 1,\ldots r$, are normal subgroups of $G = A_1\times\cdots \times A_{r}$:

Take $h = (h_1,\ldots,h_r) \in N = H_1\times \cdots H_r$, and $a= (a_1,\ldots,a_r) \in G$, then $a h a^{-1} = (a_1h_1a_1^{-1},\ldots,a_rh_ra_r^{-1})$. If $H_i = \{e_{A_i}\}$, then $a_i h_i a_i^{-1} = a_i a_i^{-1} = e_{A_i}\in H_i$, and if $H_i = A_i$, then $a_i h_i a_i^{-1} \in H_i$, thus $a h a^{-1} \in N$.

We will call these subgroups standard normal subgroups.

\item[$\bullet$]  To prove that these standard subgroups are the only normal subgroups of $G$, we use induction on $r$. The case $r=2$ is done is Exercise 5.

For simplicity, we write $e$ for the identity of each group $H_i$.

Now suppose that for any $k<r$, and for any $k$-tuple $(A_1,\ldots,A_k)$ of non-Abelian simple groups, the only normal subgroups of $A = A_1 \times \cdots \times A_k$ are $H_1\times \cdots H_k$, where $H_i = \{e\}$ or $H_i = A_i$ for all $i = 1,\ldots k$. 

Now let $A_1, \ldots,A_r$ be non-Abelian simple groups, and let $N$ be a normal subgroup of $A_1\times \cdots \times A_{r}$.

\item[$\bullet$]  Consider first the case where, for some $i \in \{1,\ldots,n\}$,

\begin{align*}
N &\subset A_1\times \cdots \times A_{i-1} \times \{e\} \times A_{i+1} \cdots \times A_r \\
    &\simeq   A_1\times \cdots \times A_{i-1}  \times A_{i+1}\cdots \times A_r,
\end{align*}
where the isomorphism is the map $(a_1,\ldots,a_{i-1},e,a_{i+1},\ldots,a_r) \mapsto (a_1,\ldots,a_{i-1},a_{i+1},\ldots,a_r)$. This isomorphism sends $N$ on a normal subgroup $N'$.

$ A_1\times A_{i-1}  \times A_{i+1}\cdots \times A_r$ is a direct product of $r-1$ simple non-Abelian subgroups, thus we can apply the induction hypothesis: its only normal subgroups are standard subgroups, thus $N'$ is standard. This implies, via the isomorphism, that 
$$N = H_1\times \cdots H_{i-1} \times \{e\} \times H_{i+1} \times \cdots \times H_r, \text{ where } H_j = \{e\} \text{ or } H_j = A_j \text{ for all } j.$$
Therefore $N$ is standard in this case.

\item[$\bullet$] In the remaining case,  $N \not \subset A_1\times \cdots \times A_{i-1} \times \{e\} \times A_{i+1}\times \cdots \times A_r $ for all $i=1,\ldots,r$. We want to prove that $N$ is the whole group $G$. 

\item[$\bullet$]  First, we prove that there is some $n=(n_1,\ldots,n_r) \in N$ such that $a_i \ne e$ for all $i,\ 1 \leq i \leq r$. 

Fix some $i \in \{1,\ldots,r\}$. Since $N \not \subset A_1\times \cdots \times A_{i-1} \times \{e\} \times A_{i+1}\times \cdots \times A_r $, there exists some
$n_i = (a_1,\ldots,a_i,\ldots,a_r) \in N$ such that $a_i \ne e$.

 Since  $A_i$ is non-Abelian and simple, the centers $Z_i$ of $A_i$ are different from $A_i$, and $Z_i$ is normal in the simple group $A_i$, thus $Z_i = \{e\}$. Using $a_i \ne e$, this shows that there is some $c \in A_i$ such that $a_i c \ne c a_i$. Now take $a = (e,\ldots,e,c,e,\ldots,e) \in G$. Since $N$ is normal, $n_i(an_i^{-1}a^{-1}) \in N$, and
 \begin{align*}
 n_ian_i^{-1}a^{-1} &= (a_1,\ldots,a_i, \ldots, a_r) (e,\ldots,c,\ldots,e)(a_1^{-1},\ldots,a_i^{-1},\ldots,a_r^{-1})(e,\ldots,c^{-1},\ldots,e)\\
 &=(e,\ldots,a_ica_i^{-1}c^{-1},\ldots,e),
 \end{align*}
thus $ n_ian_i^{-1}a^{-1} =(e,\ldots,n_i,\ldots,e)\in N$, where $n_i = a_ica_i^{-1}c^{-1} \ne e$. Since we can find such $n_i\ne e$ for each index $i$,
$$n= (n_1,\ldots,n_r) = (n_1,e,\ldots,e)(e,n_2,\ldots,e)\cdots (e,\ldots,n_r) \in N, \text{ where } n_1 \ne e,\ldots,n_r \ne e.$$
$\bullet$ Write $b = n_r$. $N$ is normal in $G$, thus, for any $a= (a_1,\ldots,a_{r-1},e) \in A_1\times\cdots\times A_r$,
\begin{align*}
n (an^{-1}a^{-1}) &= (n_1,\ldots,n_{r-1},b)(a_1,\ldots,a_{r-1},e) (n_1^{-1},\ldots,n_{r-1}^{-1},b^{-1})(a_1^{-1},\ldots,a_{r-1}^{-1},e)\\
&=(n_1a_1n_1^{-1}a_1^{-1},\ldots,n_{r-1}a_{r-1}n_{r-1}^{-1}a_{r-1}^{-1}, e) \in N
\end{align*}
Since $Z_i = \{e\}$, we can find $(a_1,\ldots,a_{r-1},e) \in A$ such that $n_ia_in_i^{-1}a_i^{-1} \ne e$ for every $i, 1\leq i \leq r-1$.

Write for simplicity $A = A_1\times \cdots \times A_{r-1},B = A_r$, and $e_A = (e,\ldots,e)$ the identity of $A$.

By the induction hypothesis, $N\cap (A\times \{e\})$ is a direct product of trivial subgroups $H_i \subset A_i$:
$$\begin{array}{ll}
N\cap (A\times \{e\}) = H_1\times \cdots \times H_{r-1}\times\{e\},  &H_i =\{e\} \text{ or } H_i=A_i,\  i=1,\ldots,r-1,\\
\end{array} 
$$

Since $$nan^{-1}a^{-1} = (n_1a_1n_1^{-1}a_1^{-1},\ldots,n_{r-1}a_{r-1}n_{r-1}^{-1}a_{r-1}^{-1}, e) \in N \cap (A\times\{e\}) = H_1\times\cdots \times H_{r-1} \times \{e\},$$
every $H_i$ contains an element $n_ia_in_i^{-1}a_i^{-1} \ne e$, thus $H_i = A_i$ for all $i,\ 1\leq i \leq r-1$.   This proves that
$$N\cap (A\times \{e\})  = A\times \{e\},$$
thus $A_1 \times \cdots \times A_{r-1} \times \{e\} \subset N$.

With the same reasoning, $\{e\}\times A_2 \times \cdots \times A_r \subset N.$

\item[$\bullet$]  If $a =(a_1,\ldots,a_r)$ is any element of $A_1\times \cdots \times A_r$, then
$$a = (a_1,\ldots,a_{r-1},e)(e,\ldots,e,a_r),$$
where $(a_1,\ldots,a_{r-1},e) \in A_1\times \cdots A_{r-1} \times \{e\} \subset N$, and $(e,\ldots,e,a_r) \in \{e\}\times A_2 \times\cdots \times A_r \subset N$. Therefore $a \in N$.

This proves $N = A \times B = A_1\times \cdots \times A_r \times A_{r+1}$: for every normal subgroup $N$ of $G$ such that there is some $n=(n_1,\ldots,n_r) \in N$ where $n_i \ne e$ for all $i$, $N$ is the whole group $G$. In this case, $N$ is also standard, and the induction is done.

To conclude, if $A_1,\ldots,A_r$ are non-Abelian simple groups, the only normal subgroups of $A_1\times\cdots \times A_r$ are the subgroups
$$N = H_1\times \cdots \times H_r, \text{ where } H_i = \{e_{A_i}\} \text{ or } H_i = A_i \text { for all } i \in \{1,\ldots,r\}.$$

\end{proof}


 \paragraph{Ex. 14.3.19}{\it Let $G \subset S_n$ be transitive, and let $G_i$ be the isotropy subgroup of $i\in \{1,\ldots,n\}$. Thus $G_i = \{\sigma \in G \mid  \sigma(i) = i\}$.
 \be
 \item[(a)] Prove that $G$ is doubly transitive if and only if $G_i$ acts transitively on $\{1,\ldots,n\} \setminus \{i\}$.
 \item[(b)] More generally, let $k\geq 2$. Prove that $G$ is $k$-transitive if and only if $G_i$ acts $(k-1)$-transitively on $\{1,\ldots,n\} \setminus \{i\}$.
 \ee
 }
 \begin{proof}
 \item[(a)] Suppose that $G$ is doubly transitive. If  $j,k\in \{1,\ldots,n\} \setminus \{i\}$, as $i \ne j, i\ne k$, there exists $\sigma \in G$ such that $\sigma(i) = i, \sigma(j) = k$, so that $\sigma \in G_i$  and $\sigma(j) = k$. This proves that $G_i$ acts transitively on $\{1,\ldots,n\} \setminus \{i\}$.
 
 Conversely, suppose that $G_i$ acts transitively on $\{1,\ldots,n\} \setminus \{i\}$ for some $i \in \{1,\ldots,n\}$.
We first show that $G_{i'}$ acts also transitively on $\{1,\ldots,n\} \setminus \{i'\}$ for all $i' \in \{1,\ldots,n\}$. 

First, since $G$ is transitive, there is some $\tau \in G$ such that $\tau(i) = i'$.

Let $j' , k' \in \{1,\ldots,n\} \setminus \{i'\}$, and define $j = \tau^{-1}(j'), k = \tau^{-1}(k')$. Then $j \ne i, k \ne i$, otherwise $j'= i'$ or $k' =i'$.
Then the hypothesis gives $\sigma \in G_i$ such that $\sigma(j) =k$, that is $\sigma(\tau^{-1}(j')) =\tau^{-1}(k')$.

Therefore
\begin{align*}
(\tau \sigma \tau^{-1})(i') &= i'\\
(\tau \sigma \tau^{-1})(j') &= k',
\end{align*}
 which shows that $\sigma'  = \tau \sigma \tau^{-1}\in G_{i'}$ satisfies $\sigma'(j') = k'$, thus $G_{i'}$ acts transitively on $\{1,\ldots,n\} \setminus \{i'\}$, and this is true for all $i'$.

Now we prove that $G$ is doubly transitive. Let $(r,s),(r',s'),r\ne s, r'\ne s'$ be pairs of distinct elements of $\{1,\ldots,n\}$. 	
Since $G_{r}$ acts transitively on $\{1,\ldots,n\} \setminus \{r\}$, and $r\ne s$, there exists $\sigma_1 \in G_r$ such that $\sigma_1(r) = r, \sigma_1(s) = s'$. Note that $r \ne s'$, otherwise $\sigma_1(r) = \sigma_1(s) ,r\ne s$, which is impossible since $\sigma_1$ is a permutation. Thus there exists $\sigma_2 \in G_{s'}$ such that $\sigma_2(r) = r', \sigma_2(s') = s'$, where $r'\ne s'$. Then $\sigma = \sigma_2\sigma_1$ satisfies $\sigma(r) = r', \sigma(s) = s'$. This proves that $G$ is doubly transitive.

\item[(b)] Suppose first that $G$ is $k$ transitive, and let $(i_1,\ldots,i_{k-1}), (j_1,\ldots,j_{k-1})$ be any ${(k-1)}$-tuples of distinct elements of $\{1,\ldots,n\}\setminus \{i\}$. Then $(i, i_1,\ldots,i_{k-1}), (i, j_1,\ldots,j_{k-1})$ are $k$-tuples of distinct elements of $\{1,\ldots,n\}$, thus there is some $\sigma \in G$ such that $\sigma\cdot(i, i_1,\ldots,i_{k-1}) = (i, j_1,\ldots,j_{k-1})$, so that $\sigma \in G_i$ and $\sigma(i_1) = j_1,\ldots,\sigma(i_{k-1}) = j_{k-1}$. This proves that $G_i$ acts $(k-1)$-transitively on $\{1,\ldots,n\} \setminus \{i\}$.

Conversely, suppose that $G_i$ acts $(k-1)$-transitively on $\{1,\ldots,n\} \setminus \{i\}$ for some $i \in \{1,\ldots,n\}$. As in part (a), we first show that $G_{i'}$ acts $(k-1)$-transitively on $\{1,\ldots,n\} \setminus \{i'\}$ for all $i' \in \{1,\ldots,n\}$, using a permutation $\tau \in G$ satisfying $\tau(i) = i'$. Take $(i'_1,\ldots,i'_{k-1}), (j'_1,\ldots,j'_{k-1})$ two $(k-1)$-tuples of distinct elements in $\{1,\ldots,n\} \setminus \{i'\}$. We define $i_l= \tau^{-1}(i'_l), j_l = \tau^{-1}(j'_l),\ l=1,\ldots,k-1$. Since $\tau^{-1}$ is a permutation, $(i,i_1,\ldots,i_{k-1})$ and $(i,j_1,\ldots,j_{k-1})$ are $k$-tuples of distinct elements. The hypothesis gives $\sigma \in G_i$ such that $\sigma(i_l) = j_l,\ l=1,\ldots,k-1$. Then $\sigma'= \tau \sigma \tau^{-1}$ satisfies
$$(\tau \sigma \tau^{-1})(i'_l) = j'_l,\ l=1,\ldots,k-1,\text { and } \sigma'(i') = i'.$$
Therefore $G_{i'}$ acts $(k-1)$-transitively on $\{1,\ldots,n\} \setminus \{i'\}$, for all $i' \in \{1,\ldots,n\}$.

Now let $(r_1,\ldots,r_k), (s_1,\ldots,s_k)$ be any $k$-tuples of distinct elements. There exists $\sigma_1 \in G_{r_k}$ such that
$$\sigma_1\cdot (r_1,\ldots,r_{k-1},r_k) = (s_1,\ldots,s_{k-1},r_k),$$
where $s_1,\ldots,s_{k-1},r_k$ are distinct, since $r_k =s_i$ for some $i\ne k$ implies $\sigma_1^{-1}(r_k) = \sigma_1^{-1}(s_i)$, that is $r_k = r_i$, which is false. Thus there exists $\sigma_2  \in G_{s_1}$ such that
$$\sigma_2\cdot (s_1,s_2,\ldots,s_{k-1}, r_k) = (s_1,s_2,\ldots,s_{k-1},s_k).$$
Then $\sigma = \sigma_2\sigma_1$ satisfies $\sigma \cdot (r_1,\ldots,r_k) = (s_1,\ldots,s_k)$. This proves that $G$ is $k$-transitive.
 \end{proof}
 
  \paragraph{Ex. 14.3.20}{\it Let $G \subset S_n$ be doubly transitive. Proposition 14.3.3 implies that $G$ is transitive. Prove that $G$ is transitive directly from the definition of doubly transitive.
  }
  
  \begin{proof}
  If $n= 1$,then $G = \{e\}$ is transitive. Suppose now that $n\geq 2$, and that $G \subset S_n$ is doubly transitive.
  
   Let $i,j$ be any elements in $\{1,\ldots,n\}$. Since $n\geq 2$, there is some $i' \in \{1,\ldots,n\}$ such that $i\ne i'$, and some $j'$ such that $j \ne j'$. By definition of doubly transitive, there is $\sigma \in G$ such that $\sigma(i) =j$ and $\sigma(i') = j'$. We have proved
   $$\forall i \in\{1,\ldots,n\},\ \forall j \in \{1,\ldots,n\},\ \exists \sigma \in G,\ \sigma(i) = j,$$
   so that $G$ is transitive.
  \end{proof}
  
   
  \paragraph{Ex. 14.3.21}{\it Generalize (14.15) by showing that we have inclusions
  $$\F_p^{mn} = \F_{p^m}^n \subset \mathrm{AGL}(n,\F_{p^m}) \subset \mathrm{A\Gamma L}(n,\F_{p^m}) \subset \mathrm{AGL}(nm,\F_p)\subset S_{p^{nm}}.$$
  }
  \begin{proof}
  We write $q = p^m$.
  \item[$\bullet$] 
  Let ${\cal B} = (e_1,\ldots,e_m)$ be a base of $\F_q$ over $\F_p$.
As $\F_p$-vector spaces, $\F_q^n$ and $\F_p^{nm}$ are isomorphic, where an isomorphism $\varphi$ is given by
$$
\varphi \left\{
\begin{array}{ccc}
\F_q^n & \to & \F_p^{nm}\\
(\alpha_1,\ldots,\alpha_n)& \mapsto &(x_1^1,\ldots,x_m^1;\ldots;x_1^n,\ldots,x_m^n),
\end{array}
\right.
$$
where $\alpha_i = \sum_{j=1}^m x_j^i e_j,\ i=1,\ldots,n.$ (this isomorphism depends of the choice of the base $\cal B$).
 So we can write $$\F_p^{mn} = \F_{p^m}^n.$$
   \item[$\bullet$]  $ \F_{p^m}^n = \F_q^n$ is identified with $\{\gamma_{I_n,v} \mid v \in  \F_q^n\} \subset \mathrm{AGL}(n,\F_q)$, via the isomorphism
     $$
  \left\{
  \begin{array}{ccc}
  \F_{q}^n & \to  &\{\gamma_{I_n,v} \mid v \in  \F_q^n\}\\
  v&\mapsto & \gamma_{I_n,v}.
  \end{array}
  \right.
  $$
  This shows that $$\F_{p^m}^n  \subset \mathrm{AGL}(n,\F_{p^m}).$$
  
  \item[$\bullet$] If $f = \gamma_{A,v}$ is any element of $\mathrm{AGL}(n,\F_{q})$, then $f = \gamma_{A,e,v} \in  \mathrm{A\Gamma L}(n,\F_{q})$ (where $e$ is the identity is $S_{q}$). Therefore
  $$\mathrm{AGL}(n,\F_{p^m}) \subset \mathrm{A\Gamma L}(n,\F_{p^m}).$$
  
    \item[$\bullet$] By Exercise 3(c), we know that elements of $\mathrm{A\Gamma L}(n,\F_{q})$ give maps $\F_q^n = \F_p^{nm} \to \F_q^n = \F_p^{nm}$ that are affine linear over $\F_p$, so that these elements are in $\mathrm{AGL}(nm,\F_p)$.
    $$\mathrm{A\Gamma L}(n,\F_{p^m}) \subset \mathrm{AGL}(nm,\F_p).$$
  
    \item[$\bullet$] The elements of $ \mathrm{AGL}(nm,\F_p)$ are bijective maps from $\F_p^{nm}$ to $\F_p^{nm}$, thus $ \mathrm{AGL}(nm,\F_p) \subset S(\F_p^{nm}).$ 
    
  Via an arbitrary numbering of $\F_p^{nm}$, say $\gamma : \{1,\ldots,p^{nm}\} \to \F_p^{nm}$, we obtain the isomorphism $\psi : S(\F_p^{nm}) \to S_{nm}$ defined by $\psi(\sigma) = \gamma^{-1} \circ \sigma \circ \gamma$,  which allows us to identify  $S(\F_p^{nm}) \simeq S_{nm}$. 
  $$\mathrm{AGL}(nm,\F_p)\subset S_{p^{nm}}.$$
  
  To conclude, using several improper non-canonical identifications, we write
    $$\F_p^{mn} = \F_{p^m}^n \subset \mathrm{AGL}(n,\F_{p^m}) \subset \mathrm{A\Gamma L}(n,\F_{p^m}) \subset \mathrm{AGL}(nm,\F_p)\subset S_{p^{nm}}.$$
  \end{proof}
  
  \paragraph{Ex. 14.3.22}{\it  Show that $\mathrm{AGL}(n,\F_q)$ is isomorphic to the subgroup
  $$
  \left\{
  \begin{pmatrix}
  A & v\\
  0 & 1
  \end{pmatrix}
  \mid A \in \mathrm{GL}(n,\F_q), v \in \F_q^n
  \right\}
  \subset \mathrm{GL}(n+1,\F_q),
  $$
  where $\begin{pmatrix}
  A & v\\
  0 & 1
  \end{pmatrix}
$ is the $(n+1) \times (n+1)$ matrix such that the upper left $n\times n$ corner is $A$, the first $n$ entries of the last column are $v$, and the first $n$ entries of the last row are all zero.

  \begin{proof} Consider the map
  $$
  \varphi
  \left\{
  \begin{array}{ccc}
  \mathrm{AGL}(n,\F_q)& \to & \mathrm{GL}(n+1,\F_q)\\
  \gamma_{A,v} & \mapsto & 
  \begin{pmatrix}
  A & v\\
  0 & 1
  \end{pmatrix}
  \end{array}
  \right.
  $$
  This map is well defined, since $\gamma_{A,v} = \gamma_{B,w} \Rightarrow (A ,v) = (B,w)$. We verify first that $\varphi$ is an injective group homomorphism.
  Since (see (1) in Exercise 2)
 $$ \gamma_{A,v} \circ \gamma_{B,w} = \gamma_{AB,Aw+v}$$
 (see (1) in Exercise 2), we obtain
 \begin{align*}
 \varphi(\gamma_{A,v} \circ \gamma_{B,w}) &= \varphi(\gamma_{AB,Aw+v})\\
 &=
 \begin{pmatrix}
 AB & Aw+v\\
 0 & 1
 \end{pmatrix}.
 \end{align*}
 Moreover,
 \begin{align*}
  \varphi(\gamma_{A,v} ) \varphi( \gamma_{B,w})&= 
  \begin{pmatrix}
  A & v\\
  0 & 1
  \end{pmatrix}
\begin{pmatrix}
  B & w\\
  0 & 1
  \end{pmatrix}\\
&=
\begin{pmatrix}
  AB & Aw+v\\
  0 & 1
  \end{pmatrix}\\
  &=  \varphi(\gamma_{A,v} \circ \gamma_{B,w}).
 \end{align*}
 $\varphi$ is a group homomorphism, and if $\gamma_{A,v} \in \ker(\varphi)$ then 
 $$
   \begin{pmatrix}
  A & v\\
  0 & 1
  \end{pmatrix}=
    \begin{pmatrix}
  I_n & 0\\
  0 & 1
  \end{pmatrix},
 $$
 Therefore $A = I_n, v = 0$, so that $\gamma_{A,v} = \gamma_{I_n,0} = 1_{F_q}$. We have proved that $\varphi$ is an injective group homomorphism, thus
 $$
  \im(\varphi) = \left\{
  \begin{pmatrix}
  A & v\\
  0 & 1
  \end{pmatrix}
\mid A \in \mathrm{GL}(n,\F_q), v \in \F_q^n
  \right\},
  $$
  is a subgroup of $\mathrm{GL}(n+1,\F_q)$ isomorphic to $ \mathrm{AGL}(n,\F_q)$.
  \end{proof}  
  
\paragraph{Ex. 14.3.23}{\it Use Theorem 14.3.21 to show that $\mathrm{AGL}(2,\F_p)$ is not solvable for $p>3$.
 }
    
\begin{proof}
Let $p$ be a prime such that $p>3$. By Theorem 14.3.21, we know that $\mathrm{PSL}(2,\F_p)$ is simple, and non-Abelian, therefore $\mathrm{PSL}(2,\F_p)$ is not solvable.

Since  $$\mathrm{PSL}(2,\F_p) \subset \mathrm{PGL}(2,\F_p), $$
$\mathrm{PGL}(2,\F_p)$ is not solvable.

Moreover, $$\mathrm{PGL}(2,\F_p) = \mathrm{GL}(2,\F_p)/\F_p^* I_n.$$
Since $\F_p^* I_n$ is cyclic, therefore solvable, $\mathrm{GL}(2,\F_p)$ is not solvable.

But $\mathrm{AGL}(2,\F_p)$ contains the subgroup $\{\gamma_{A,0} \mid A \in \mathrm{GL}(2,\F_p)\} \simeq \mathrm{GL}(2,\F_p)$.

This proves that $\mathrm{AGL}(2,\F_p)$ is not solvable if $p>3$.
\end{proof}

\paragraph{Ex. 14.3.24}{\it
The action of $\mathrm{PGL}(2,F)$ on $\hat F = F \cup \{\infty\}$ was introduced in Section 7.5 In particular, Exercise 11 of that section implies that the isotropy subgroup of $\mathrm{PGL}(2,F)$ at the point $\infty$ can be identified with $\mathrm{AGL}(1,F)$. Use part (c) of Exercise 4 and Exercise 19 to prove that the action of $\mathrm{PGL}(2,F)$ on $\hat F$ is 3-transitive (also called triply transitive).
}
\begin{proof}
By Exercise 4 (c), $\mathrm{GL}(2,F)$ acts transitively on $F^2 \setminus \{0\}$. This proves that $\mathrm{PGL}(2,F)$ acts transitively on the projective line $\mathbb{P}_1(F)$, so that the action of $\mathrm{PGL}(2,F)$ on $\hat F$ is transitive.

This allows us to apply Exercise 19 (b). To prove that the action of $\mathrm{PGL}(2,F)$ on $\hat F$ is triply transitive, it is sufficient to prove that the isotropy group $G_{\infty} = \mathrm{AGL}(1,F)$ acts 2-transitively on $ \hat F \setminus \{\infty\} = F$. Example 14.3.2 shows that this is true. Therefore the action of $\mathrm{PGL}(2,F)$ on $\hat F$ is 3-transitive.
\end{proof}

\paragraph{Ex. 14.3.25}{\it Prove that $\mathrm{AGL}(1,\F_4) \simeq A_4$ and $\mathrm{A\Gamma L}(1,\F_4) \simeq S_4$.
}
\begin{proof}
$\mathrm{AGL}(1,\F_4)$ acts on $\F_4$, via the action defined by $$\gamma_{a,b} \cdot i = \gamma_{a,b}(i) = ai +b,$$ where $a \in \F_4^*, b \in \F_4$.

Since $\gamma_{a,b}$ is bijective, $\mathrm{AGL}(1,\F_4) \subset S(\F_4) \simeq S_4$.

Moreover, $|\mathrm{AGL}(1,\F_4)| = 3 \times 4 = 12$, and the only subgroup with $12$ elements of $S_4$ is $A_4$. Therefore 
$$\mathrm{AGL}(1,\F_4) \simeq A_4.$$

Similarly, $\mathrm{A\Gamma L}(1,\F_4)$ acts on $\F_4$, via the action defined by 
$$\gamma_{a,\sigma,b} \cdot i = \gamma_{a,\sigma,b}(i) = a \sigma(i) + b,$$
where $a \in \F_4^*, b \in F_4, \sigma \in \Gal(F_4/F_2) = \{e,F\}$, $F$ being the Frobenius isomorphism $i \mapsto i^2$.

Since $\gamma_{a,\sigma,b}$ is bijective, $\mathrm{AGL}(1,\F_4) \subset S(\F_4) \simeq S_4$.

By Exercise 3(a), $\mathrm{AGL}(2,\F_4)$ is a subgroup of $\mathrm{A\Gamma L}(2,\F_4)$ of index 2, therefore $|\mathrm{A\Gamma L}(2,\F_4) | = 24$. This proves that $\mathrm{A\Gamma L}(2,\F_4) = S(\F_4)$, thus
$$\mathrm{AGL}(2,\F_4) \simeq S_4.$$
\end{proof}

\paragraph{Ex. 14.3.26}{\it Compute the orders of the groups in (14.15).
}
$$(14.15)\qquad \F_p^2 = \F_{p^2}  \subset \mathrm{AGL}(1,\F_{p^2}) \subset \mathrm{A\Gamma L}(1,\F_{p^2}) \subset \mathrm{AGL}(2,\F_p) \subset S_{p^2}.$$
\begin{proof}
 \item[$\bullet$] $|\F_p^2| = |\F_{p^2}| = p^2.$
 \item[$\bullet$] Every element of $\mathrm{AGL}(1,\F_{p^2})$ is of the unique form $\gamma_{a,b},\  a \in \F_{p^2}^*, b \in \F_{p^2}$, thus
  $$|\mathrm{AGL}(1,\F_{p^2})| = p^2(p^2-1).$$
   \item[$\bullet$] By Exercise 3(a), $\mathrm{AGL}(1,\F_{p^2})$ has index 2 in  $\mathrm{A\Gamma L}(1,\F_{p^2})$, therefore
   $$| \mathrm{A\Gamma L}(1,\F_{p^2})| = 2p^2(p^2-1).$$
   \item[$\bullet$] Every element of $\mathrm{AGL}(2,\F_{p})$ is of the unique form $\gamma_{A,v},\  A \in \mathrm{GL}(2,\F_p), v \in \F_{p}^2$.
   Moreover, by Exercise 14(a), $|\mathrm{GL}(2,\F_p)| = p(p-1)(p^2-1)$, therefore
   $$|\mathrm{AGL}(2,\F_{p})| = p^3(p-1)(p^2-1).$$
   \item[$\bullet$]  To be complete, $|S_{p^2}| = (p^2)!$.
\end{proof}

\subsection{PRIMITIVE POLYNOMIALS OF PRIME-SQUARED DEGREE}
\paragraph{Ex. 14.4.1}{\it Prove that $M_1 = \mathrm{A\Gamma L}(1,\F_{p^2})$ is solvable, and compute its order.
}
\begin{proof}
Recall that $\mathrm{AGL}(1,\F_{p^2})/\F_p \simeq \F_p^*$, where $(\F_p,+)$ and $(\F_p^*,\times)$ are cyclic, thus solvable, hence $\mathrm{AGL}(1,\F_{p^2})$ is solvable by Theorem 8.1.4.

By Exercise 14.3.3(a), $\mathrm{AGL}(1,\F_{p^2})$ has index 2 in  $\mathrm{A\Gamma L}(1,\F_{p^2})$, therefore $\mathrm{AGL}(1,\F_{p^2})$ is a normal subgroup of $\mathrm{AGL}(1,\F_{p^2})$, and
$$\mathrm{A\Gamma L}(1,\F_{p^2})/  \mathrm{AGL}(1,\F_{p^2}) \simeq \{-1,1\}.$$
The group $ \{-1,1\}$ is cyclic, of prime order $2$, therefore is solvable, and $\mathrm{AGL}(1,\F_{p^2})$ is solvable. The same Theorem 8.1.4 shows that $M_1 = \mathrm{A\Gamma L}(1,\F_{p^2})$ is solvable.

By Exercise 14.3.26,
$$| M_1| = 2p^2(p^2-1).$$
\end{proof}

\paragraph{Ex. 14.4.2}{\it This exercise will study the subgroup $M_2 \subset \mathrm{AGL}(2,\F_p)$ defined in (14.21).
\be
\item[(a)] Prove that the map $\delta$ defined in (14.2) gives an element of $\mathrm{AGL}(2,\F_p)$.
\item[(b)] Prove that
$\begin{pmatrix}
0 & 1\\
1 & 0
\end{pmatrix}
$
has order $\,2$ and normalizes $\mathrm{AGL}(1,\F_p) \times \mathrm{AGL}(1,\F_p) \subset \mathrm{AGL}(2,\F_p)$.
\item[(c)] Prove that $M_2$ is solvable, and compute its order.
\item[(d)] Prove that $(M_2)_0$ is generated by the matrices in (14.22).
\item[(e)] Prove that $\mathrm{AGL}(1,\F_p) \times \mathrm{AGL}(1,\F_p) \subset \mathrm{AGL}(2,\F_p)$ is imprimitive in $S_{p^2}$.
\ee
}
\begin{proof}
\item[(a)] Write $\gamma = \gamma_{a,b}, \gamma' = \gamma_{a',b'}$, where $a,a'\in \F_p^*,\ b,b \in \F_p$.

Then $$\delta(\alpha,\beta) = (a\alpha+b,a' \beta + b'), \qquad \text{for all } (\alpha,\beta) \in \F_p^2.$$
Since
$$
\begin{pmatrix}
a\alpha + b\\
a' \beta + b' 
\end{pmatrix}
=
\begin{pmatrix}
a & 0\\
0&a'
\end{pmatrix}
\begin{pmatrix}
\alpha\\
\beta
\end{pmatrix}
+
\begin{pmatrix}
b\\
b'
\end{pmatrix},
$$
$\delta = \gamma_{A,v} \in \mathrm{AGL}(2,\F_p)$, where 
$A = 
\begin{pmatrix}
a & 0\\
0&a'
\end{pmatrix} \in \mathrm{GL}(2,\F_p)$ (since $\det(A) = aa' \ne 0$), and 
$v = (b,b') \in \F_p^2$.

\item[(b)] 
$\begin{pmatrix}
0&1\\
1&0
\end{pmatrix}
\begin{pmatrix}
0&1\\
1&0
\end{pmatrix}
=I_2
$, thus 
$B = \begin{pmatrix}
0&1\\
1&0
\end{pmatrix}$ has order $2$, and is identified with the element $\gamma_{B,0} \in \mathrm{AGL}(2,\F_p)$.

Let $\delta = \gamma_{A,v}$ be any element of $\mathrm{AGL}(1,\F_p) \times \mathrm{AGL}(1,\F_p)$. With the same notations as in part (a), using (1),
\begin{align*}
\gamma_{B,0} \circ \delta \circ \gamma_{B,0}^{-1} &= \gamma_{B,0} \circ \gamma_{A,v} \circ \gamma_{B,0}\\
&=\gamma_{B,0} \circ \gamma_{AB,v}\\
&=\gamma_{BAB,Bv}
\end{align*}
where 
$$
BAB = 
\begin{pmatrix}
0&1\\
1&0
\end{pmatrix}
\begin{pmatrix}
a & 0\\
0&a'
\end{pmatrix} 
\begin{pmatrix}
0&1\\
1&0
\end{pmatrix}
=
\begin{pmatrix}
a' & 0\\
0&a
\end{pmatrix} 
$$ 
is diagonal, therefore $\gamma_{B,0} \circ \delta \circ \gamma_{B,0}^{-1} \in \mathrm{AGL}(1,\F_p) \times \mathrm{AGL}(1,\F_p)$. This proves that $B$ normalizes $\mathrm{AGL}(1,\F_p) \times \mathrm{AGL}(1,\F_p)$: 
$$B \left(\mathrm{AGL}(1,\F_p) \times \mathrm{AGL}(1,\F_p)\right)B^{-1} = \mathrm{AGL}(1,\F_p) \times \mathrm{AGL}(1,\F_p).$$

\item[(c)] Since $M_2$ is the subgroup of $ \mathrm{AGL}(2,\F_p)$ generated by $\mathrm{AGL}(1,\F_p) \times \mathrm{AGL}(1,\F_p)$ and $B$, part (b) shows that $H = \mathrm{AGL}(1,\F_p) \times \mathrm{AGL}(1,\F_p)$ is a normal subgroup of $ M_2$.

Note that, if $A \in H$, $AB= B(BAB) = BA'$, where $A' = BAB = BAB^{-1} \in H$. Therefore every element of $M_2$ is of the form $A$ or $BA$, where $A\in H$. This proves that $M_2 = H \cup (B\cdot H)$, so that the group homomorphism
$$
\pi
\left\{
\begin{array}{ccc}
\{I_2,B\} & \to & M_2/H\\
I_2 & \mapsto& H\\
B & \mapsto & B\cdot  H
\end{array}
\right.
$$
is surjective. Since $B \not \in H, B \cdot H \ne H$, thus $\ker(\pi) = \{I_2\}$. This proves that $\pi$ is an isomorphism, and
$$M_2/ (\mathrm{AGL}(1,\F_p) \times \mathrm{AGL}(1,\F_p)) \simeq \{I_2,B\}.$$

This gives $|M_2| = 2 |\mathrm{AGL}(1,\F_p)|^2$, so
$$|M_2| = 2p^2(p-1)^2.$$
Since $(\mathrm{AGL}(1,\F_p) \times \mathrm{AGL}(1,\F_p))/\mathrm{AGL}(1,\F_p) \simeq \mathrm{AGL}(1,\F_p)$, where $\mathrm{AGL}(1,\F_p)$ is solvable, the group $\mathrm{AGL}(1,\F_p) \times \mathrm{AGL}(1,\F_p)$ is solvable.

The group $\{I_2,B\}$ is cyclic, therefore solvable. By Theorem 8.1.4, the isomoprphism $M_2/ (\mathrm{AGL}(1,\F_p) \times \mathrm{AGL}(1,\F_p)) \simeq \{I_2,B\}$ shows that $M_2$ is solvable.

\item[(d)] Note that $\delta \in H$ defined by $\delta(\alpha,\beta) = (\lambda \alpha+b, \mu  \beta + b')$ fixes $0$ if and only if $b = b' = 0$. Therefore $H \cap (M_2)_0$ is generated by the matrices 
$
\begin{pmatrix}
\lambda & 0\\
0&\mu
\end{pmatrix} 
,\ \lambda,\mu \in \F_p^*$.

By part (c), $M_2 = H \cup (B\cdot H)$, and $B \in (M_2)_0$, therefore $(M_2)_0 \subset \mathrm{GL}(2,\F_p)$ is generated by the matrices
$$
\begin{pmatrix}
0&1\\
1&0
\end{pmatrix},
\begin{pmatrix}
\lambda & 0\\
0&\mu
\end{pmatrix} 
,\ \lambda,\mu \in \F_p^*.
$$

\item[(e)] If $H$ is the smaller subgroup $\mathrm{AGL}(1,\F_p) \times \mathrm{AGL}(1,\F_p)$, then the isotropy group $H_0$ is generated by the matrices 
$
\begin{pmatrix}
\lambda & 0\\
0&\mu
\end{pmatrix} 
,\ \lambda,\mu \in \F_p^*$, therefore is the subgroup of diagonal matrices in $\mathrm{GL}(2,\F_p)$.

We prove that $H_0$ is not irreducible. The nontrivial subspace $V = \F_p \times \{0\} \subset \F_p^2$ is such that $h(V) \subset V$ for all $h = \begin{pmatrix}
\lambda & 0\\
0&\mu
\end{pmatrix} \in H_0$: For all $(\gamma,0) \in V$,
$$
\begin{pmatrix}
\lambda & 0\\
0&\mu
\end{pmatrix}
\begin{pmatrix}
\gamma\\
0
\end{pmatrix}
=
\begin{pmatrix}
\lambda \gamma\\
0
\end{pmatrix}
\in V.
$$
\end{proof}

\paragraph{Ex. 14.4.3}{\it Let $M_1$ and $M_2$ be the subgroups defined in the text, and assume that $p>3$. Prove that $M_2$ is not doubly transitive and not isomorphic to a subgroup of $M_1$.
}
\begin{proof} By Theorem 14.3.4, if $M_2 \subset S_{p^2}$ was doubly transitive, then $$p^2(p^2-1) \mid |M_2| = 2p^2(p-1)^2.$$
Then $p+1 \mid 2(p-1)$, thus $p+1 \mid  2(p+1) - 2(p-1) = 4$, where $p$ is prime. The only solution is $p=3$. We can conclude:
\begin{center}
If $p>3$, then $M_2$ is not doubly transitive.
\end{center}
If $M_2$ was isomorphic to a subgroup of $M_1$, then, by Lagrange's Theorem,
$$
\frac{|M_1|}{|M_2|} = \frac{p+1}{p-1} \in \Z.
$$
In this case, $p-1 \mid p+1$, therefore $p-1 \mid (p+1)- (p-1) = 2$, which is only possible if $p = 2$ or $p=3$. Here $p>3$, so we can conclude:
\begin{center}
If $p>3$, then $M_2$ is not isomorphic to a subgroup of $M_1$.
\end{center}
\end{proof}

\paragraph{Ex. 14.4.4}{\it Let $V$ be a vector space of dimension $2$ over a field $F$, and let $T : V \to V$ be a linear map that is not a multiple of the identity. Also assume that $T$ is an isomorphism. Prove that there is $v \in V$ such that $v$ and $T(v)$ form a basis of $V$ over $F$.
}

\begin{proof} Consider a basis ${\cal B} = (e,f)$ of $V$.

Reasoning by contradiction, suppose that for all $v \in V$, the vectors $v,T(v)$ are linearly dependent. Then, since $e\ne 0,f \ne 0, e+f \ne 0$, there are some $\lambda, \mu,\nu \in F$ such that
$$T(e) = \lambda e,\qquad T(f) = \mu f, \qquad T(e+f) = \nu (e+f).$$
Since $T$ is linear,
$$\nu e + \nu f = T(e+f) = T(e) + T(f) = \lambda e + \mu f;$$
But $(e,f)$ is a basis, therefore $\lambda = \mu = \nu$, so that the matrix of $T$ in the basis $\cal B$ is
$
\begin{pmatrix}
\lambda & 0 \\
0 &\lambda
\end{pmatrix}
= \lambda I_2
$, and $T$ would be a multiple of the identity, which is in contradiction with the hypothesis.

This proves that there is some $v \in V$ such that $v,T(v)$ are linearly independent. Since the dimension of $V$ is $2$, $v$ and $T(v)$ form a basis of $V$ over $F$.
\end{proof}

Note: The hypothesis ``$\,T$ is an isomorphism'' was useless.

\paragraph{Ex. 14.4.5}{\it Fix $a \in \F_p, p>2$. The goal of this exercise is to find $s,t \in \F_p$ with $s^2+t^2 =a$.
\be
\item[(a)] Let $S = \{s^2 \mid s \in \F_p\}$. Prove that $|S| = (p+1)/2$.
\item[(b)] Let $S' = \{a-s^2 \mid s \in \F_p\}$. Show that $S \cap S' \ne \varnothing$, and use this to prove the existence of $s,t \in \F_p$ such that $s^2+t^2 = a$.
\ee
}
\begin{proof}

\item[(a)]  Consider the map
$$
\varphi
\left\{
\begin{array}{ccc}
\F_p^* & \to &\F_p^*\\
s & \mapsto & s^2.
\end{array}
\right.
$$
For all $s,t \in \F_p^*$, $\varphi(st) = (st)^2 = s^2t^2 = \varphi(s) \varphi(t)$, thus $\varphi$ is a group homomorphism.

Not that, $\im(\varphi) =\{s^2 \mid s \in \F_p^*\} = S \setminus \{0\}$. Moreover $s \in \ker(\varphi)$ iff $s^2 = 1$, that is $(s-1)(s+1) = 0$. Since $\F_p$ is a field, this is equivalent to $s = 1$ or $s=-1$, thus $\ker(\varphi) = \{-1,1\}$, where $-1 \ne 1$ since $p>2$.
By the first Isomorphism Theorem,
$$\F_p^*/\{-1,1\} \simeq \{s^2 \mid s \in \F_p^*\} = S \setminus\{0\}.$$
This shows that there are $(p-1)/2$ squares in $\F_p^*$. If we add the square $0 = 0^2$, we obtain
$$|S| = \frac{p+1}{2}.$$

\item[(b)] The two maps $u, t_a : \F_p \to \F_p$ defined for all $s \in \F_p$ by $u(s) = -s$, and $t_a(s)= a+s$ are bijective (since $u \circ u = 1_{\F_p}$ and $t_a \circ t_{-a} = t_{-a} \circ t_a = 1_{\F_p})$. Therefore
$$
f = t_a \circ u
\left\{
\begin{array}{ccc}
\F_p & \to & \F_p\\
s & \mapsto & a-s^2
\end{array}
\right.
$$
is bijective.

Since $S' = \{f(s) \mid s \in \F_p\} = f(S)$,
$$|S'| = |S| = \frac{p+1}{2}.$$

Reasoning by contradiction, if $S \cap S' =\varnothing$, then the inclusion $S \cup S' \subset \F_p$ shows that
$$p + 1 = |S| + |S'| = |S \cup S'| \leq |\F_p| = p,$$
and $p+1 \leq p$ gives a contradiction. Therefore $S \cap S' \ne \varnothing$, so that we can find some $c \in \F_p$ such that $c \in S \cap S'$. Such an element $c$ verifies
$c = s^2$ and $c = a-t^2$ for some $s,t \in \F_p$. This proves that the equation $s^2 + t^2 = a$ has a solution $(s,t) \in \F_p \times \F_p$.
\end{proof}

\paragraph{Ex. 14.4.6}{\it Let $A = \begin{pmatrix} a & b \\ c & d \end{pmatrix}$ be a $2 \times 2$ matrix with entries in a field $F$.
\be
\item[(a)] Prove that the characteristic polynomial of $A$ is $P(x) = x^2 - \mathrm{tr}(A) x + \det(A)$, where $\mathrm{tr}(A) = a+d$ and $\det(A) =ad-bc$ are the trace and determinant of $A$.
\item[(b)] Prove that $P(A) = A^2 - \mathrm{tr}(A) A + \det(A) I_2$ is the zero matrix.
\ee
The Cayley-Hamilton Theorem generalizes part (b) by showing that $P(A)$ is the zero matrix when $P(x)$ is the characteristic polynomial of an $n\times n$ matrix $A$.
}
\begin{proof}
\item[(a)] By definition of the characteristic polynomial,
\begin{align*}
P(x) &= 
\begin{vmatrix}
a - x & b\\
c &d-x
\end{vmatrix}\\
&= (a-x)(d-x) - bc\\
&= x^2 -(a+d) x + ad-bc\\
&=x^2 - \mathrm{tr}(A) x + \det(A).
\end{align*}
\item[(b)] Moreover,
\begin{align*}
P(A) &=  A^2 - \mathrm{tr}(A) A + \det(A) I_2\\
&=  \begin{pmatrix} a & b \\ c & d \end{pmatrix}^2 - (a+d)  \begin{pmatrix} a & b \\ c & d \end{pmatrix} + (ad- bc)I_2\\
&=\begin{pmatrix} a^2 + bc & ab + bd\\ ca + dc & cb + d^2 \end{pmatrix} +\begin{pmatrix} -a^2 -da & -ab - db\\ -ac -dc& -ad - d^2\end{pmatrix} + \begin{pmatrix} ad-bc&0\\ 0 & ad-bc\end{pmatrix} \\
&=0.
\end{align*}

\end{proof}

\paragraph{Ex. 14.4.7}{\it Complete the proof of $C(H) = H$ from Proposition 14.4.4 begun in the text.
}
\begin{proof}
Recall the context:

Here $g,h \in \mathrm{GL}(2,\F_p)$ are such that
\begin{align}
gh = -hg, g^2 = h^2 = -I_2, \det(g) = \det(h) = 1.
\end{align}
$$H = \langle [g],[h]\rangle = \{[I_2],[g],[h],[g][h]\rangle \subset \mathrm{PGL}(2,\F_p).$$
By (14.23),
$$
[m] \in C(H)  \iff 
\left\{
\begin{array}{ll}
{[}m{]}{[}g{]} &= {[}g]{[}m{]}\\
{[}m{]}{[}h{]} &= {[}h{]}{[}m{]}
\end{array}
\right.
\iff
\left\{
\begin{array}{ll}
mg &= \pm gm\\
mh &= \pm hm.
\end{array}
\right.
$$
Let $[m] \in C(H)$. There are four cases.

\item[$\bullet$] If $mg = gm, mh = hm$, then by Lemma 14.4.3,
$$m = aI_2+bg =cI_2 + dh,\qquad a,b,c,d \in \F_p.$$
If $b\ne 0$, then $g = b^{-1}(c-a)I_2 + b^{-1}dh \in \mathrm{Vect}(I_2,h)$, thus $gh = hg$. This is impossible since $gh = -hg \ne 0$ and $p>2$. Thus $b=0$, and  $m = aI_2$, so that $[m] = [I_2] \in H$.

\item[$\bullet$] If $mg = -gm, mh = hm$, then
\begin{align*}
&(mh) g = m(hg) = m(-gh) = (-mg)h = (gm)h = g(mh)\\
&(mh) h = (hm)h = h(mh),
\end{align*}
thus $mh$ is in the centralizer of $g$ and $h$. By the first bullet, $[mh] = [I_2]$, and using $[h]^2 = [I_2]$, we obtain $[m] = [h] \in H$.

\item[$\bullet$] If $mg = gm, mh = -hm$, exchanging $g$ and $h$ in the previous case, we obtain similarly $[m] = [g] \in H$.

\item[$\bullet$] If $mg = -gm, mh = -hm$, then, using (10),
\begin{align*}
&(mgh) g = mg(hg) = mg(-gh) = m(-g^2) h = mh,\\
&g(mgh) = (gm)(gh) = (-mg)gh = m(-g^2)h = mh,\\
\\
&(mgh) h = mg(h^2) = -mg,\\
&h(mgh) = (hm)(gh) = (-mh)(-hg) = mh^2g = -mg.
\end{align*}
Therefore $mgh$ commutes with $g$ and $h$. By the first bullet, $[mgh] = [I_2]$, thus $[mg] = [-(mgh)h] = [-h]$, and $[m] = [-mg^2]= [mg][-g] = [-h][-g] = [h][g] = [g][h] \in H$. 

We have proved $C(H) \subset H$. Since $H$ is Abelian $H \subset C(H)$. Thus $C(H) = H$.
\end{proof}

\paragraph{Ex. 14.4.8}{\it Let $G$ be a group with a normal subgroup $H \simeq (\Z/2\Z)^2$ such that $C_G(H) = H$ and the map $G \to \mathrm{Aut}(H)$ given by conjugation is onto. The goal of this exercise is to prove that $G \simeq S_4$. Note that $|G| = 24$ by the proof of Proposition 14.4.4. 
\be
\item[(a)] Use the Sylow Theorems to show that $G$ has one or four $3$-Sylow subgroups. Then use $C_G(H) = H$ to show that the number is four.
\item[(b)] Let $H_1$ be a $3$-Sylow subgroup of $G$. Use part (a) and the Sylow Theorem to show that its normalizer has order $6$.
\item[(c)] Now consider the homomorphism $\phi : G \to S_4$ given by the action of $G$ by conjugation on the $3$-Sylow subgroups. Use part (b) to prove that $\ker(\phi)$ cannot contain an element of order $3$.
\item[(d)] Conclude that the image of $\phi$ contains $A_4$. It follows that if $\phi$ is not an isomorphism, then $G$ contains a normal subgroup of order $2$.
\item[(e)] Prove that $G$ cannot contain a normal subgroup of order $2$. Thus $\phi : G \simeq S_4$.
\ee
This exercise is closely related to Exercise 3 of Section 14.2.
}

\begin{proof}
First, we try to understand the first sentence, and why this is related to the proof of Proposition 14.4.4.

Note that every permutation of $\Z/2\Z \times \Z/2\Z$ fixing $(0,0)$ is an outer automorphism (the Cayley tables are identical if we exchange the two elements of any pair of nonzero elements). Since $\Z/2\Z \times \Z/2\Z$ is Abelian, there is no non trivial inner automorphism. Thus $\mathrm{Aut}(H) \simeq S_3$.

In the proof of Proposition 14.4.4, if we write $G = N(H) = N_{\mathrm{PGL}(2,\F_p)}(H)$, then we proved that
$$
\varphi
\left\{
\begin{array}{ccc}
G &\to &\mathrm{Aut}(H) \simeq S_3\\
g & \mapsto & \varphi_g 
  \left\{
  \begin{array}{ccc}
  H &\to & H\\
  g & \mapsto &  ghg^{-1}
  \end{array}
  \right.
\end{array}
\right.
$$
is surjective (onto). Note that $\varphi$ is well defined because $H$ is a normal subgroup of $G$, so $ghg^{-1} \subset H$ for all $g \in G$ and for all $h \in H$.

By hypothesis, $\varphi$ is surjective. As in the text, $\ker(\varphi) = C_G(H) =  H$, thus
$$S_3 \simeq \mathrm{Aut}(H) \simeq G/\ker(\varphi) = G/H.$$
This proves $|G| = |H| \times |S_3| = 4 \times 6 = 24$.

\item[(a)] Let $N$ be the number of $3$-Sylow subgroups of $G$. By the third Sylow Theorem,
$$N \mid 24 = |G|, \qquad N \equiv 1 \pmod 3.$$
Therefore $N = 1$ or $N = 4$.

If $N = 1$, then there is only one $3$-Sylow subgroup, say $K$. Since  $gKg^{-1}$ is also a $3$-Sylow subgroup, $gKg^{-1} = K$, so $K$ is normal in $G$.

Since $|K| = 3$ and $|H| =4$, then $|H\cap K|$ divides $4$ and $3$, thus $H\cap K = \{e\}$.

By Exercise 14.3.7, knowing that $H,K$ are normal in $G$, and that $H\cap K = \{e\}$, then $HK$ is a normal subgroup of $G$, and $HK \simeq H \times K$. Moreover $H$ and $K$ are Abelian subgroups, thus $HK \simeq H \times K$ is abelian. This shows that $K \subset C_G(H)$, thus $H = C_G(H) \supset H K \supsetneq H$. This is a contradiction.

We have proved that $G$ has four $3$-Sylow subgroups.

\item[(b)] Let $H_1,H_2,H_3,H_4$ be the $3$-Sylow subgroup of $G$. For all $g \in G$, $gH_ig^{-1}$ is a $3$-Sylow of $G$, thus $G$ acts on the set $\{H_1,H_2,H_3,H_4\}$.

By the second Sylow Theorem, all $3$-Sylow subgroups are conjugate, thus $G$ acts transitively on $\{H_1,H_2,H_3,H_4\}$, thus the orbit of $H_1$ is 
$${\cal O}_{H_1} = \{H_1,H_2,H_3,H_4\}.$$
By definition of the normalizer, the isotropy group of $H_1$ is $G_{H_1} = N_G(H_1)$. The Fundamental Theorem of Group Actions gives
$$4 = |{\cal O}_{H_1} | = (G : N_G(H_1)),$$
thus $|N_G(H_1)| = |G|/4 = 6$.

\item[(c)] Write $E = \{H_1,H_2,H_3,H_4\}$. Consider the homomorphism
$$
\psi
\left\{
\begin{array}{ccc}
G & \to & S(E) \simeq S_4\\
g & \mapsto &\psi_g
   \left\{
    \begin{array}{ccc}
 E & \to &E\\
   M& \mapsto & gMg^{-1}.
   \end{array}
   \right.
\end{array}
\right.
$$
$(M = H_i,\ i=1,2,3,4).$

Reasoning by contradiction, suppose that $\ker(\psi)$ contains an element $h$ of order $3$.

 For all $g\in G$, $g \in \ker(\psi)$ if and only if $\psi_g$ is the identity of $S(E)$, thus $$\ker(\psi) =  N_G(H_1) \cap N_G(H_2)\cap N_G(H_3) \cap N_G(H_3).$$
 Therefore  $\{e,h,h^2\}$ is a subgroup of $N_G(H_1)$. But $H_1 \subset N_G(H_1)$ is another subgroup of $N_G(H_1)$ of order $3$.
 
 Since $N_G(H_1) = 6$, $N_G(H_1)$ has a unique subgroup of order $3$: the number $n$ of $3$-Sylow subgroups of $N_G(H_1)$ satisfies $n \mid 6, n \equiv 1 \pmod 3$, thus $n=1$. Therefore $H_1 = \{e,h,h^2\}$. Reasoning with $H_2\subset N_G(H_2)$, we obtain similarly $H_2 = \{e,h,h^2\}$. Thus $H_1 = H_2$. This is a contradiction since the $3$-Sylow subgroups $H_1,H_2,H_3,H_4$ are distinct. Therefore $\ker(\psi)$ cannot contain an element of order $3$.
 
 \item[(d)] By part (b), $\ker(\psi)$ is a subgroup of the group $N_G(H_1)$, whose  order is $6$. By part (c), $\ker(\psi)$ has no element of order 3, hence the order of $\ker(\psi)$ is not $3$ or $6$, otherwise, by Cauchy Theorem, $\ker(\psi)$ would contain an element of order $3$. Thus the order of $\ker(\phi)$ is $1$ or $2$. This implies that
 $$|\im(\psi)| = |G|/|\ker(\psi)| = 12 \text{ or } 24.$$
 
 If we choose some numbering $\gamma$ of $E$, where $\gamma : \{1,2,3,4\} \to E$ is given by $\gamma(i) = H_i$, then $\hat \gamma : S(E) \to S_4$ defined by $\hat \gamma(\sigma) = \gamma^{-1} \circ \sigma \circ \gamma$ is an isomorphism. Thus $\phi = \hat \gamma \circ \psi$, $\phi :  G \to S_4$, is a group homomorphism, such that $|\im(\phi)| = |\im(\psi)| = 12$ or $24$.  Since $S_4$ has only one subgroup of order $12$, we obtain that
 $$\im(\phi) = A_4 \text{ or } \im(\phi) = S_4.$$
We have proved $\im(\phi) \supset A_4$.

If $\im(\phi) = S_4$, then $\phi : G \to S_4$ is surjective, and since $|G| = |S_4|$, $\phi$ is bijective, thus $\phi$ is a group isomorphism and $G \simeq S_4$.

It follows that if $\phi$ is not an isomorphism, then $\im(\phi) = A_4 $, and $|\ker(\varphi)| = 2$. Then $K= \ker(\varphi)$ is a normal subgroup of $G$ of order $2$.

\item[(e)] It remains to show that $G$ cannot contain a normal subgroup $K$ of order $2$. Reasoning by contradiction, suppose that there exists such a subgroup $K$. Write $K = \{e,a\}, a \ne e$. 

Since $K$ is normal in $G$, for every $g \in G$, $gag^{-1} \in K = \{e,a\}$, and $gag^{-1} \ne e$, otherwise $a = e$. Thus $gag^{-1} =a$.
$$\forall g \in G, \ ga = ag.$$
This means that $a$ is in the center of $G$, a fortiori, $a \in C_G(H)$. Since $C_G(H) = H$, it follows that
$$K \subset H.$$
Write $H = \{e,a,b,c\}$.

Consider anew the homomorphism $\varphi : G \to \mathrm{Aut}(H) \simeq S_3$ described in the introduction. By hypothesis, $\varphi$ is surjective. But for all $g \in G$, $\varphi_g(a) = gag^{-1} =a$, thus any automorphism $\chi$ of $G$ such that $\chi(a) \ne a$ , for instance the automorphism defined by $\chi(e) = e, \chi(a) = b, \chi(b) = c, \chi(c) = a$, is not in $\im(\phi)$, in contradiction with the surjectivity of $\varphi$.

This proves that $G$ has no normal subgroup of order 2. By part (d), it follows that $S(E) \simeq S_4$.
\end{proof}

\paragraph{Ex. 14.4.9}{\it Let $g$ and $C(g)$ be as in the proof of part (c) of Proposition 14.4.4.
\be
\item[(a)] Show that $C(g)$ is Abelian and contains $\F_p^* I_2$.
\item[(b)] If $m \in C(g)$, then it is easy to see that $\det(m) m^{-2} \in C(g)$. By part (a), it follows that $\phi(m) = \det(m) m^{-2}$ defines a group homomorphism $\phi : C(g) \to C(g)$. Prove that $\ker(\phi) = \F_p^* I_2$ and $|\im(\phi)| = |C(g)|/(p-1)$.
\item[(c)] Prove that $\im(\phi) \subset \{w \in C(g) \mid \det(w) = 1\}$.
\item[(d)] Explain why we may assume that $g = \begin{pmatrix} 0 & -1\\1 & 0 \end{pmatrix}$. Then use Lemma 14.4.3 and Exercise 5 to show that $\det:C(g) \to \F_p^*$ is onto. Conclude that
$$\im(\phi) = \{w \in C(g) \mid \det(w) = 1\}.$$
\ee
The equality proved in part (d) shows that every element of $C(g)$ of determinant $1$ is of the form $\det(m)m^{-2}$ for some $m\in C(g)$. This will be used in the proof of part (c) of Proposition 14.4.4.
}

\begin{proof}
\item[(a)] Since $g^2 = -I_2$, $g \in \mathrm{GL}(2,\F_p)$, and since $gh = -hg \ne hg (p>2)$, $g \not \in \F_p^* I_2$. Then Lemma 14.4.3 shows that
$$C(g) = \{m \in \mathrm{GL}(2,\F_p) \mid \exists a \in \F_p, \ \exists b \in \F_p, \ m = aI_2 + bg\} = \mathrm{GL}(2,\F_p) \cap \mathrm{Vect}_{\F_p}(I_2,g).$$
if $m, m' \in C(g)$, then $m = aI_2+bg, m' =a'I_2 + b'g,\ a,b,a,b,' \in F_p$, thus
$$mm' =  (aI_2+bg)( a'I_2+b'g) = aa'I_2 + ba'g + a b' g + bb'g^2 = ( a'I_2+b'g)( aI_2+bg),$$
so $C(g)$ is Abelian. 

Moreover, $C(g)$ contains the matrices $aI_2 \in \mathrm{GL}(2,\F_p)$, which is equivalent to $a \in \F_p^*$.
$$C(g)\supset \F_p^* I_2.$$

\item[(b)] If $m\in C(g)$, then $m^2 g = gm^2$, thus $gm^{-2} = m^{-2} g$ and $(\det(m) m^{-2}) g = g (\det(m) m^{-2})$, so that $\det(m) m^{-2} \in C(g)$. This permits us to define
$$
\phi
\left\{
\begin{array}{ccc}
C(g) & \to &C(g)\\
m & \mapsto &\det(m) m^{-2}.
\end{array}
\right.
$$
For $m,m' \in C(g)$, $\phi(mm') = (\det(mm') (mm')^{-2} = (\det(m)m^{-2})(\det(m') m'^{-2})$, thus $\phi$ is a group homomorphism.

If $m = \lambda I_2, \lambda \in \F_p^*$, then $\phi(m) = \lambda^2 (\lambda I_2)^{-2} = I_2$. Therefore $\F_p^* I_2 \subset \ker(\phi)$.

Conversely, if $m\in \ker(\phi)$, $\det(m) m^{-2} = I_2$, thus $m^2 = \mu I_2$, where $\mu = \det(m) \in \F_p^*$.

Reasoning by contradiction, suppose that $m \not \in \F_p^* I_2$. Since $m\in C(g)$, Lemma 14.4.3 shows that $m = aI_2 + bg$, where $b\ne 0$.

Then $(aI_2 + bg)^2 = \mu I_2$, which gives, using $g^2 = -I_2$,
$$(a^2 - b^2 - \mu) I_2 + 2ab g = 0.$$
If $a \ne 0$, then $g = -(2ab)^{-1} (a^2 -b^2 - \mu) I_2 \in \F_p^* I_2$, which is false. Therefore $a= 0$, and $m= bg$. Then 
\begin{align*}
I_2 = \phi(m) &= \det(bg) (bg)^{-2}\\
&= \det(g) g^{-2},
\end{align*}
thus $-I_2 = g^2 = \det(g) I^2$, and $\det(g) = -1$, which contradicts $\det(g) = 1\ (p>2)$. This contradiction shows that $m \in \F_p^* I_2$. To conclude,
$$\ker(\phi) = \F_p^* I_2.$$

Since $\im(\phi) \simeq C(g)/\ker(\phi) = C(g)/\F_p^* I_2$,
$$|\im(\phi)| = |C(g)|/(p-1).$$

\item[(c)] If $w \in \im(\phi)$, then $w = \det(m) m^{-2}$ for some $m \in C(g)$. Then
\begin{align*}
\det(w) &= \det(m)^2 \det(m)^{-2} = 1,
\end{align*}
thus
$$\im(\phi) \subset \{w \in C(g) \mid \det(w) = 1\}.$$

\item[(d)] In the proof of part (c) of Proposition 14.4.4, we saw that there is some $Q \in \mathrm{GL}(2,\F_p)$ such that
$$g' = Q^{-1} g Q = \begin{pmatrix}0 & -1\\1 & 0 \end{pmatrix}.$$
Note that $$C(g') = Q^{-1} C(g) Q.$$
Indeed, if $m \in C(g)$, then $mg = gm$, thus
\begin{align*}
(Q^{-1} m Q) g' &= (Q^{-1} m Q )(Q^{-1} g Q)\\
&= Q^{-1} mg Q = Q^{-1} g m Q\\
&= (Q^{-1} g Q)(Q^{-1} m Q)\\
&= g' (Q^{-1} m Q).
\end{align*}
This proves $Q^{-1} C(g) Q \subset C(g')$. Symmetrically,  $QC(g')Q^{-1} \subset  C(g)$, thus $C(g') = Q^{-1} C(g) Q.$

Let $\phi' : C(g') \to C(g')$ the group homomorphism defined by $\phi'(m') = \det(m') m'^{-2}$. Suppose that we can prove that $\im(\phi') = \{w' \in C(g') \mid \det(w') = 1\}.$
Now, take $w \in C(g)$ such that $\det(w) = 1$. Since $C(g) = Q C(g') Q^{-1}$,  there is some $w' \in C(g')$ such that
$$w = Q w' Q^{-1},\qquad w' \in C(g'), \det(w') = 1.$$
By our hypothesis, $w' \in \im(\phi')$, so that $w' = \det(m') m'^2, m' \in C(g')$. 

Put $m = Qm' Q^{-1}$. Then $m \in C(g)$, and
\begin{align*}                                                                                                                                                                                                                                                                                                                              
w &= Qw' Q^{-1}\\
&=\det(Qm'Q^{-1}) (Q m'Q^{-1})^2\\
&=\det(m) m^2\\
&= \phi(m)
\end{align*}

This proves that $\im(\phi) = \{w \in C(g) \mid \det(w) = 1\},$ if we can show first that $\im(\phi') = \{w' \in C(g') \mid \det(w') = 1\}.$
This explain why we may assume now that $g = \begin{pmatrix} 0 & -1\\1 & 0 \end{pmatrix}$.

For all $a,b \in \F_p$,
$$\det(aI_2 + b g) = \begin{vmatrix} a & -b\\ b & a \end{vmatrix} = a^2 + b^2.$$
By Exercise 5, for all $\lambda \in \F_p$, there is some pair $(a,b) \in \F_p^2$ such that $a^2  + b^2 = \lambda$. If $\lambda \in \F_p^*$, then $\det(aI + bg) = \lambda \ne 0$, thus 
$aI_2 + bg \in \mathrm{GL}(2,\F_p)$. Moreover $aI_2 + bg  \in C(g)$ (this is the easy part of Lemma 14.4.3), thus $\det : C(g) \to \F_p^*$ is a surjective homomorphism.

For any $w \in \mathrm{GL}(2,\F_p)$,
\begin{align*}
w \in C(g) &\iff \begin{pmatrix} r & t\\s & u \end{pmatrix}\begin{pmatrix} 0 & -1\\1 & 0 \end{pmatrix}  = \begin{pmatrix} 0 & -1\\1 & 0 \end{pmatrix} \begin{pmatrix} r & t\\s & u \end{pmatrix}\\
&\iff \begin{pmatrix} t & -r\\ u & -s\end{pmatrix} =\begin{pmatrix}  -s & -u\\ r & t\end{pmatrix}\\
&\iff
\left\{
\begin{array}{ll}
t &= -s,\\
u &=r.
\end{array}
\right.
\end{align*}
Thus
$$C(g) = \left\{ w \in \mathrm{GL}(2,\F_p) \mid \exists r \in \F_p,\ \exists s \in \F_p,\  w = \begin{pmatrix} r & -s\\s & r \end{pmatrix} \right\}.$$

Now take any $w = \begin{pmatrix} r & t\\s & u \end{pmatrix} \in C(g)$ such that $\det(w) = 1$. 

Then  $w = \begin{pmatrix} r & -s\\s & r \end{pmatrix} , \ r^2 + s^2 = 1$, and $w^{-1} =  \begin{pmatrix} r & s\\-s & r \end{pmatrix}$.

Put $m = \begin{pmatrix} 1+r & s\\- s & 1+r \end{pmatrix}$ (we found this matrix by trigonometrical analogy). By the previous remark, $m\in C(g)$, and $\det(m) = (1+r)^2 +s^2 = 2(1+r)$.

Then, using $r^2+s^2 = 1$,
\begin{align*}
m^2 &= \begin{pmatrix} 1+r & s\\- s & 1+r \end{pmatrix}\begin{pmatrix} 1+r & s\\- s & 1+r \end{pmatrix}\\
&=\begin{pmatrix} (1+r)^2 -s^2 & 2s(1+r)\\- 2s(1+r) & (1+r)^2 -s^2 \end{pmatrix}\\
&=\begin{pmatrix} 2r^2 + 2r & 2s(1+r)\\- 2s(1+r) & 2r^2+2r \end{pmatrix}\\
&=2(1+r) \begin{pmatrix} r & s\\- s & r \end{pmatrix}\\
&= \det(m)w^{-1}
\end{align*}
Thus $w = \det(m) m^{-2} = \phi(m)$, where $m = \begin{pmatrix} 1+r & s\\- s & 1+r \end{pmatrix} \in C(g)$.

(We have not used with this method the surjectivity of $\det : C(g) \to \F_p^*$.)

We can conclude, using also part (c),
$$\im(\phi) = \{w \in C(g) \mid \det(w) = 1\}.$$
(for $g= \begin{pmatrix} 0 & -1\\1 & 0 \end{pmatrix}$, and by the above remark, for any $g \in \mathrm{GL}(2,\F_p)$ such that $g^2 = -I_2$.)

This equality shows that every element of $C(g)$ of determinant $1$ is of the form $\det(m) m^{-2}$ for some $m \in C(g)$.
\end{proof}

\paragraph{Ex. 14.4.10}{\it Consider the subgroup $N(H) \subset \mathrm{PGL}(2,\F_p)$ defined in Proposition 14.4.4.
\be
\item[(a)] Prove that the images of the matrices (14.25) generate $N(H)$ when $p\equiv 1 \pmod 4$.
\item[(b)] Prove that generators of $H$ and the images of the matrices
$$
\begin{pmatrix} 1 & -1\\1 & 1 \end{pmatrix} \text { and } \begin{pmatrix} s &  t+ 1\\ t- 1& -s \end{pmatrix}
$$
from [17, p.163] generate $N(H)$ when $p\equiv 3 \pmod 4$.
\ee
}

\begin{proof}
\item[(a)] In this part, $p \equiv 1 \pmod 4$. Then $\legendre{-1}{p} = (-1)^{\frac{p-1}{2}} = 1$, therefore there exists some element $i \in \F_p$ such that $i^2 = 1$.

By Proposition 14.4.4, $H \simeq \Z/2\Z \times \Z/2\Z$ is given, up to conjugacy, by
$$H = \langle [g],[h] \rangle = \{[I_2], [g],[h],[k]\},$$
where
$$g = \begin{pmatrix} 0 & -1\\1 & 0 \end{pmatrix}, \qquad h = \begin{pmatrix} i & 0 \\0 & -i \end{pmatrix}, \qquad k = gh = \begin{pmatrix} 0 & i\\i & 0 \end{pmatrix}, $$
since $s = i, t = 0$ is a solution of the equation $s^2+t^2 = -1$.

Then $N(H)$ is the normalizer of $H$ in $\mathrm{PGL}(2,\F_p)$. Write
$$a = \begin{pmatrix} 0 & 1\\1 & 0\end{pmatrix}, \qquad b= \begin{pmatrix} i & 0\\0 & 1 \end{pmatrix}, \qquad c = \begin{pmatrix} 1 & -1\\1 & 1 \end{pmatrix}.$$
We verify first that $\langle[a],[b],[c]\rangle \subset N(H)$.
\begin{align*}
aga^{-1}& = \begin{pmatrix} 0 & 1\\1 & 0 \end{pmatrix}\begin{pmatrix} 0 & -1\\1 & 0 \end{pmatrix}\begin{pmatrix} 0 & 1\\1 & 0 \end{pmatrix}
= \begin{pmatrix} 0 & 1\\-1 & 0 \end{pmatrix} = -g,\\
aha^{-1} &=\begin{pmatrix} 0 & 1\\1 & 0 \end{pmatrix}\begin{pmatrix} i & 0 \\0 & -i \end{pmatrix}\begin{pmatrix} 0 & 1\\1 & 0 \end{pmatrix}
= \begin{pmatrix} -i & 0 \\0 & i \end{pmatrix} = -h,\\
bgb^{-1} &=\begin{pmatrix} i & 0\\0 & 1 \end{pmatrix}\begin{pmatrix} 0 & -1\\1 & 0 \end{pmatrix}   \begin{pmatrix} i & 0\\0 & 1 \end{pmatrix}
= \begin{pmatrix} 0 & -i\\-i & 0 \end{pmatrix} =  -k,\\
bhb^{-1} &= \begin{pmatrix} i & 0\\0 & 1 \end{pmatrix} \begin{pmatrix} i & 0 \\0 & -i \end{pmatrix}   \begin{pmatrix} i & 0\\0 & 1 \end{pmatrix}
=  \begin{pmatrix} i & 0 \\0 & -i \end{pmatrix} = h,\\
cgc^{-1} &=\frac{1}{2} \begin{pmatrix} 1 & -1\\1 & 1 \end{pmatrix}  \begin{pmatrix} 0 & -1\\1 & 0 \end{pmatrix}\begin{pmatrix} 1 & 1\\-1 & 1 \end{pmatrix}
=\begin{pmatrix} 0 & -1\\1 & 0 \end{pmatrix} = g,\\
chc^{-1} &=\frac{1}{2} \begin{pmatrix} 1 & -1\\1 & 1 \end{pmatrix}  \begin{pmatrix} i & 0 \\0 & -i \end{pmatrix}\begin{pmatrix} 1 & 1\\-1 & 1 \end{pmatrix}
=\begin{pmatrix} 0& i \\i & 0 \end{pmatrix}  = k.
\end{align*}
Therefore
\begin{align*}
[a][g][a]^{-1} = [g], \qquad &[a][h][a]^{-1} = [h],\qquad  ( \Rightarrow [a][k][a]^{-1} = [k]),\\
[b][g][b]^{-1} = [k], \qquad &[b][h][b]^{-1} = [h],\qquad  ( \Rightarrow [b][k][b]^{-1} = [g]),\\  
[c][g][c]^{-1} = [g], \qquad &[c][h][c]^{-1} = [k],\qquad  ( \Rightarrow [c][k][c]^{-1} = [h]).\\
\end{align*}
Since $g,h,k \in H$, we have proved $\langle[a],[b],[c]\rangle \subset N(H)$.

Note that $H \subset K = \langle {[}a{]},{[}b{]},{[}c{]} \rangle$:

$[k] = [ia] = [a] \in H$, thus $[b][g][b]^{-1} = [k] = [a]$, so that
\begin{align*}
[g] &= [b]^{-1} [a][b] \in K,\\
[h] &= [g]^{-1} [gh] =[b]^{-1}[a]^{-1} [b] [a] \in K.
\end{align*}
This proves $$H = \langle g,h \rangle \subset K = \langle a,b,c \rangle.$$

Consider the action of $K =\langle {[}a{]},{[}b{]},{[}c{]} \rangle  \subset N(H)$ on $H\setminus\{[I_2]\}$ by conjugation. This gives a group homomorphism
$$
\varphi
\left\{
\begin{array}{ccc}
\langle {[}a{]},{[}b{]},{[}c{]} \rangle & \to & S(\{{[}g{]},{[}h{]},{[}k{]}\} \simeq S_3\\
m & \mapsto &\varphi_m
    \left\{
    \begin{array}{ccc}
    \{{[}g{]},{[}h{]},{[}k{]}\} & \to & \{{[}g{]},{[}h{]},{[}k{]}\} \\
    x &\mapsto & mxm^{-1}.
    \end{array}
    \right.
\end{array}
\right.
$$

\item[$\bullet$] $\varphi$ is surjective: $\varphi_{[b]}$ is the transposition $(g\, k)$, and $\varphi_{[c]}$ is the transposition $(h\,k)$. Since $S_3$ is generated by any pair of distinct transpositions, $S(\{{[}g{]},{[}h{]},{[}k{]}\} = \langle (g\, k), (h\,k) \rangle$. This proves that $\varphi$ is surjective.

\item[$\bullet$] Since $H$ is Abelian, the elements of $H$ fix $[g],[h],[k]$ by conjugation, therefore $H \subset \ker(\varphi)$. To prove the converse, we note that $\ker(\varphi) \subset C(H) = H$, because every element of $\ker(\varphi)$ fixes $[I_2], [g],[h],[k]$, thus is in $C(H)$, which is $H$ (Proposition 14.4.4).
$$\ker(\varphi) = H.$$
Therefore $S_3 \simeq \im(\varphi) \simeq K/\ker(\varphi) = K/H$. This gives
$$|\langle {[}a{]},{[}b{]},{[}c{]} \rangle | = |K| = |S_3| |H| = 6 \times 4 = 24. $$
Since $\langle {[}a{]},{[}b{]},{[}c{]} \rangle \subset N(H)$, and $|\langle {[}a{]},{[}b{]},{[}c{]} \rangle | = 24 = |N(H)|$, we obtain 
$$N(H) = \left \langle
\begin{bmatrix} 0 & 1\\1 & 0\end{bmatrix}, \begin{bmatrix} i & 0\\0 & 1 \end{bmatrix},  \begin{bmatrix} 1 & -1\\1 & 1 \end{bmatrix}
 \right \rangle.
$$
If we replace our $g$ and $h$ by any pair $g,h$ satisfying the conditions of Proposition 14.4.4, then $N(H)$ is a conjugate subgroup of $K = \langle {[}a{]},{[}b{]},{[}c{]} \rangle$ by part (c) of this Proposition.

\item[(b)] In this part, $p \equiv 3 \pmod 4$. There is no element $i$ of order 4, but by Proposition 14.1.4, there exist $s,t \in \F_p$ such that
$$H = \langle  [g],[h] \rangle,$$ where
$$g = \begin{pmatrix} 0 & -1\\1 & 0 \end{pmatrix}, \qquad h = \begin{pmatrix} s& t \\t & -s \end{pmatrix}, \qquad s^2+t^2 = -1,$$
and
$$k = gh = \begin{pmatrix} 0 & -1\\1 & 0 \end{pmatrix}\begin{pmatrix} s& t \\t & -s \end{pmatrix}= \begin{pmatrix} -t& s \\s & t \end{pmatrix}.$$
Write
$$c = \begin{pmatrix} 1 & -1\\1 & 1 \end{pmatrix},\qquad  d = \begin{pmatrix} s & t-1\\t+1 & -s \end{pmatrix}.$$
Put $K = \langle [g],[h],[c],[d]\rangle$. We want to show that $N(H) = K = \langle [g],[h],[c],[d]\rangle.$

We know that $H \subset N(H)$, so $[g],[h] \in N(H)$. It remains to verify that $[c],[d] \in N(H)$.
\begin{align*}
cgc^{-1} &=\frac{1}{2} \begin{pmatrix} 1 & -1\\1 & 1 \end{pmatrix}  \begin{pmatrix} 0 & -1\\1 & 0 \end{pmatrix}\begin{pmatrix} 1 & 1\\-1 & 1 \end{pmatrix}
=\begin{pmatrix} 0 & -1\\1 & 0 \end{pmatrix} = g,\\
chc^{-1} &=\frac{1}{2} \begin{pmatrix} 1 & -1\\1 & 1 \end{pmatrix}  \begin{pmatrix} s& t \\t & -s \end{pmatrix}\begin{pmatrix} 1 & 1\\-1 & 1 \end{pmatrix}
=\begin{pmatrix} -t& s\\s& t \end{pmatrix}  = k,\\
dgd^{-1} &=\frac{1}{2} \begin{pmatrix} s & t-1\\t+1 & -s \end{pmatrix}  \begin{pmatrix} 0 & -1\\1 & 0 \end{pmatrix}  \begin{pmatrix} -s & -t+1\\-t-1 & s \end{pmatrix}
= \begin{pmatrix} s& t \\t & -s \end{pmatrix} = h,\\
dhd^{-1} &= \frac{1}{2} \begin{pmatrix} s & t-1\\t+1 & -s \end{pmatrix}  \begin{pmatrix} s& t \\t & -s \end{pmatrix}  \begin{pmatrix} -s & -t+1\\-t-1 & s \end{pmatrix}
=\begin{pmatrix} 0 & -1\\1 & 0 \end{pmatrix} = g.
\end{align*}
Since $g,h,k \in H$, it follows that 
\begin{align*}
[c][g][c]^{-1} = [g], \qquad &[c][h][c]^{-1} = [k]\qquad (\Rightarrow [c][k][c]^{-1} = [h]),\\
[d][g][d]^{-1} = [h],\qquad &[d][h][d]^{-1} = [g]\qquad (\Rightarrow [d][k][d]^{-1} = [k]),
\end{align*}

$K =\langle [g],[h],[c],[d]\rangle \subset N(H)$.

As in part (a), consider the homomorphism
$$
\phi
\left\{
\begin{array}{ccc}
\langle {[}g{]},{[}h{]},{[}c{]},{[}d{]} \rangle & \to & S(\{{[}g{]},{[}h{]},{[}k{]}\} \simeq S_3\\
m & \mapsto &\phi_m
    \left\{
    \begin{array}{ccc}
    \{{[}g{]},{[}h{]},{[}k{]}\} & \to & \{{[}g{]},{[}h{]},{[}k{]}\} \\
    x &\mapsto & mxm^{-1}.
    \end{array}
    \right.
\end{array}
\right.
$$
Then $\phi_{[c]}$ is the transposition $(h\,k)$, and $\phi_{[d]}$ is the transposition $(g \,h)$. Since $S_3$ is generated by any pair of two distinct transpositions, $S([g],[h],[k]) = \langle (h\,k), (g\,h)$, thus $\phi$ is surjective. Since $H \subset K$, as in part (a), $\ker(\varphi) = H$.

Therefore $S_3 \simeq \im(\varphi) \simeq K/\ker(\varphi) = K/H$. This gives
$$|\langle {[}g{]},{[}h{]},{[}c{]},{[}d{]}  \rangle | = |K| = |S_3| |H| = 6 \times 4 = 24. $$
Since $\langle {[}g{]},{[}h{]},{[}c{]},{[}d{]}  \rangle \subset N(H)$, and $|\langle {[}g{]},{[}h{]},{[}c{]},{[}d{]}  \rangle | = 24 = |N(H)|$, we obtain 
$$N(H) = \left \langle
 \begin{bmatrix} 0 & -1\\1 & 0 \end{bmatrix},\begin{bmatrix} s& t \\t & -s \end{bmatrix}, \begin{bmatrix} 1 & -1\\1 & 1 \end{bmatrix},\begin{bmatrix} s & t-1\\t+1 & -s \end{bmatrix}
 \right \rangle.
$$
\end{proof}

\paragraph{Ex. 14.4.11}{\it Let $M_3$ be as in Proposition 14.4.5. Show that $M_3$ is solvable of order $24p^2(p-1)$.
}
\begin{proof}
By definition, $M_3 = \pi^{-1}(N(H))$, where
$$
\pi
\left\{
\begin{array}{ccc}
\mathrm{AGL}(2,\F_p) & \to & \mathrm{PGL}(2,\F_p)\\
\gamma_{A,v} & \mapsto &[A].
\end{array}
\right.
$$

Consider the restriction (and co-restriction) of $\pi$, given by
$$
\tilde{\pi}
\left\{
\begin{array}{ccc}
M_3& \to &N(H)\\
\gamma_{A,v} & \mapsto &[A].
\end{array}
\right.
$$
Since $M_3 = \pi^{-1}(N(H))$, $\tilde{\pi}$ is surjective.
Moreover, $\gamma_{A,v} \in \ker(\tilde{\pi})$ if and only if $[A] = [I_2]$, that is $A \in \F_p^* I_2$, thus
$$\ker(\tilde{\pi}) = \{ \gamma_{\lambda I_2, v} \mid \lambda \in \F_p^*, v \in \F_p^2\}.$$
We know that $\gamma_{A,v} = \gamma_{B,w} \Rightarrow A = B, v = w$, therefore the map
$$
\psi
\left\{
\begin{array}{ccc}
 \F_p^2\times \F_p^* & \to &\ker(\tilde \pi)\\
(v, \lambda) & \mapsto & \gamma_{\lambda I_2,v}
\end{array} 
\right.
$$
is bijective. This gives
$$
|\ker(\tilde{\pi})| = |\F_p^*| \times |\F_p^2| = (p-1)p^2.
$$
The First Isomorphism Theorem gives
$$S_4 \simeq N(H) \simeq M_3/ \ker(\tilde{\pi}),$$
therefore
$$|M_3|= 24 p^2(p-1).$$

Moreover, if we give the structure of semidirect product to $\F_p^2\times \F_p^*$, we see that
$$
\psi
\left\{
\begin{array}{ccc}
 \F_p^2\rtimes \F_p^* & \to &\ker(\tilde \pi)\\
(v, \lambda) & \mapsto & \gamma_{\lambda I_2,v}
\end{array} 
\right.
$$
is a group isomorphism:
By formula (1) in Ex. 14.3.2, we obtain
$$\gamma_{\lambda I_2,v} \circ \gamma_{\mu I_2,w} = \gamma_{\lambda \mu I_2, Aw+v}.$$
We have seen that $\psi$ is bijective, and, following the formula (6.9) for the semidirect product, 
$$(v,\lambda)(w, \mu) = (v + \lambda w, \lambda \mu).$$
Therefore
\begin{align*}
\psi^{-1}(\gamma_{\lambda I_2,v}) \psi^{-1} (\gamma_{\mu I_2,w}) &= (v,\lambda)(w, \mu)\\
&=(v + \lambda w, \lambda \mu)\\
&= \psi^{-1}(\gamma_{\lambda \mu I_2, Aw+v})\\
&=\psi^{-1}(\gamma_{\lambda I_2,v} \circ \gamma_{\mu I_2,w}).
\end{align*}
This proves that $\psi$ is a group isomorphism, and $\ker(\tilde \pi) \simeq \F_p^2 \rtimes \F_p^*$.

By exercise 6.4.8, we know that $\F_p^2 \rtimes \F_p^*/ \F_p^2\times \{1\} \simeq \F_p^*$, where $\F_p^2 \times \{1\} \simeq \F_p^2$. Since $\F_p^*$ is cyclic, $\F_p^*$ is solvable, and $\F_p^2$ is also cyclic and solvable, Theorem 8.1.4 shows that the semidirect product $\F_p^2 \rtimes \F_p^*$ is solvable, thus $\ker(\tilde \pi)$ is solvable.

We know that $S_4$ is solvable (Exercise 8.1.1), thus $N(H) \simeq S_4$ is solvable.

Using again Theorem 8.1.4 with $N(H) \simeq M_3/ \ker(\tilde{\pi})$, we obtain that $M_3$ is solvable.

\end{proof}


\paragraph{Ex. 14.4.12}{\it Consider the subgroups $M_1,M_2$, and $M_3$ defined in the text.
\be
\item[(a)] Show that $(M_1)_0$ and $(M_2)_0$ have Abelian subgroups of index $2$, and use this to prove that neither can contain $(M_3)_0$. This proves that $M_3 \not \subset M_1$ and $M_3 \not \subset M_2$.
\item[(b)] Explain why $M_3 = \mathrm{AGL}(2,\F_3)$ when $p = 3$.
\item[(c)] Show that $(M_1)_0/\F_p^* I_2$ has an element of order $p+1$, and use this to prove that $M_1 \not \subset M_3$ when $p>3$.
\item[(d)] Show that $M_2 \not \subset M_3$ when $p>5$.
\item[(e)] Show that $M_2 \subset M_3$ when $p=5$.
\ee
It follows that the only exceptions to (14.27) are $M_1 \subset M_3$ and $M_2 \subset M_3$, when $p=3$ and $M_2 \subset M_3$ when $p=5$. This result is due to Jordan.
}

\begin{proof} We recall the Second and Third Isomorphism Theorems (see for instance [14] Rose, A course on Group Theory, p.  77-79).

\bigskip

{\bf Second Isomorphism Theorem.} If $H$ is a subgroup of $G$, and $N$ is a normal subgroup of $G$, then $N$ is a normal subgroup of $HN \subset G$, and
$$H/H\cap N \simeq HN/N.$$
\begin{center}
\begin{tikzpicture}
    \node (HcapN) at (1,4) {$H \cap N$};
    \node (N) at (1,2) {$N$};
    \node (H) at (4,4) {$H$};
    \node (G) at (4,2) {$HN \subset G$};
    \draw[->] (HcapN) edge (H);
    \draw[->] (N) edge (G);
    \draw[->] (HcapN) edge (N);
    \draw[->] (H) edge (G);
\end{tikzpicture}
\end{center}

{\bf Third Isomorphism Theorem.} If $K,H$ are normal subgroups of $G$ such that $K \subset H$, then $H/K$ is a normal subgroup of $G/K$, and
$$G/H \simeq (G/K)/(H/K).$$

\item[(a)] 
 By (8) in Exercise 14.3.3, the map
$$
\left\{
\begin{array}{ccc}
\F_{p^2}^*  \times \Gal(\F_{p^2}/\F_p) & \to & (M_1)_0 = \{\gamma_{a,\sigma,v}\in M_1 \mid v=0 \}\\
(a,\sigma) & \mapsto &\gamma_{a,\sigma,0}
\end{array}
\right.
$$
is bijective, thus
$$|(M_1)_0| = 2(p^2-1).$$
Note that $(M_1)_0 \cap \mathrm{AGL}(1,\F_{p^2}) = \{\gamma_{a,0} \in \mathrm{AGL}(1,\F_{p^2}) \mid a \in \F_{p^2}^*\}$. Indeed, if $\gamma_{a,\sigma,0} \in \mathrm{AGL}(1,\F_{p^2})$, where $\sigma \in \Gal(\F_{p^2}/\F_p) = \{e,F\}$. If $\sigma = F$ is the Frobenius isomorphism, there are some $c,d \in \F_{p^2}, c \ne 0$ such that $\gamma_{a,\sigma,0} = \gamma_{c,d}$, thus, for all $u \in \F_{p^2}$,
$$ a u^p = c u +d.$$
For $u = 0$, we obtain $d = 0$, and for $u = 1$, we obtain $a= c$, thus $u^p = u$, which implies $u \in \F_p$. Since $\F_p \subsetneq \F_{p^2}$, this is a contradiction, therefore $\sigma = e$, and $\gamma_{a,\sigma,0} = \gamma_{a,0} \in \mathrm{GL}(1,\F_{p^2}) \subset \mathrm{AGL}(1,\F_{p^2})$.

We have proved
$$K = (M_1)_0 \cap \mathrm{AGL}(1,\F_{p^2}) = \{\gamma_{a,0} \in \mathrm{AGL}(1,\F_{p^2}) \mid a \in \F_{p^2}^*\} \simeq \F_{p^2}^*,$$
therefore the subgroup $K$ of $(M_1)_0$ is Abelian, with order $|\F_{p^2}^*|= p^2-1$, thus $K$ is an Abelian subgroup of index 2 in $(M_1)_0$.

\bigskip

Similarly, $(M_2)_0$ is the subgroup of $M_2$ generated by $\mathrm{GL}(1,\F_p) \times \mathrm{GL}(1,\F_p)$ and $B = \begin{pmatrix} 0 & 1 \\ 1 & 0 \end{pmatrix}$, thus
\begin{align*}
(M_2)_0 &= \left \{  \begin{pmatrix} a & 0 \\ 0 & b \end{pmatrix} \mid a,b \in \F_p^* \right\} \cup \left \{  \begin{pmatrix} 0 & a \\ b & 0 \end{pmatrix} \mid a,b \in \F_p^* \right\} \\
&= L \cup (B\cdot L),
\end{align*}
where $L$ is the group of diagonal matrices in $\mathrm{GL}(2,\F_p)$, with order $|L| = (p-1)^2$, and $L \simeq \F_p^* \times \F_p^*$ is Abelian, of index $2$ in $(M_2)_0$, where $|(M_2)_0| = 2(p-1)^2$.
\bigskip

Reasoning by contradiction, suppose that $(M_3)_0 \subset (M_1)_0$. The idea is to prove that this implies the existence of an Abelian subgroup of index $2$ in $S_4$, which is false.

Consider the Abelian subgroup $K$ of index $2$ in $(M_1)_0$, whose existence is proved above, and the following diagram, where all arrows are inclusions. 

\begin{center}
\begin{tikzpicture}
    \node (HcapN) at (1,4) {$K \cap (M_3)_0$};
    \node (N) at (1,2) {$K$};
    \node (H) at (4,4) {$(M_3)_0$};
    \node (G) at (4,2) {$(M_1)_0$};
    \draw[->] (HcapN) edge (H);
    \draw[->] (N) edge (G);
    \draw[->] (HcapN) edge (N);
    \draw[->] (H) edge (G);
\end{tikzpicture}
\end{center}


The subgroup $K$ has index $2$ in $(M_1)_0$, thus is a normal subgroup of $(M_1)_0$. Hence $K (M_3)_0$ is a subgroup of $(M_1)_0$, and
$$K \subset K(M_3)_0 \subset (M_1)_0.$$
Since 
$$2 = ((M_1)_0 : K) = ((M_1)_0 : K(M_3)_0)(K(M_3)_0 : K),$$
we obtain $(K(M_3)_0 : K) = 1$ or $2$. Moreover, if $(K(M_3)_0 : K) = 1$, then $K(M_3)_0 = K$ and $(M_3)_0 \subset K$. But this is impossible since $K$ is Abelian, but $(M_3)_0$ is not ($\gamma_{A,0} \mapsto [A]$ maps $(M_3)_0$ on $N(H) \simeq S_4$, which is not Abelian). Hence $(K(M_3)_0 : K) = 2$, thus $K(M_3)_0 = (M_1)_0$. Then the Second Isomorphism Theorem shows that
$$(M_3)_0/ K \cap (M_3)_0 = (M_1)_0/K = 2.$$
If we write $ M= K\cap (M_3)_0$, then $M$ is an Abelian subgroup of $(M_3)_0$ of index $2$ in $(M_3)_0$. 

Note that $K = (M_1)_0 \cap \mathrm{AGL}(1,\F_{p^2})$, thus
$$M = K \cap (M_3)_0 = ((M_1)_0 \cap \mathrm{AGL}(1,\F_{p^2})) \cap (M_3)_0 = (M_3)_0 \cap \mathrm{AGL}(1,\F_{p^2}).$$
Moreover, if we identify $\gamma_{A,0}$ with $A \in \mathrm{GL}(2,\F_p)$, then $\{\gamma_{\lambda I_2,0} \mid \lambda \in \F_{p}^* \} \simeq \F_p^* I_2$, and for all $\lambda \in \F_{p^2}^*$, $\gamma_{\lambda I_2,0} \in (M_3)_0 \cap \mathrm{AGL}(1,\F_{p^2})$, therefore $\F_p^* I_2 \subset M$.


$M$ is an Abelian subgroup of $(M_3)_0$ of index $2$ in $(M_3)_0$ containing $\F_p^* I_2$. It remains to prove that this is impossible.

Consider the group homomorphism (where $A$ is identified with $\gamma_{A,0}$).
$$
\varphi
\left\{
\begin{array}{ccl}
(M_3)_0 & \mapsto & N(H) \subset \mathrm{PGL}(2,\F_p)\\
A & \mapsto &[A]
\end{array}
\right.
$$

Since $(M_3)_0 = \{A \in \mathrm{GL}(2,\F_p) \mid [A] \in N(H)\}$, $\varphi$  is surjective, and $\ker(\varphi) = \F_p^* I_2 \subset M$.


In the following diagram, where $P = \varphi(M)$,
\begin{center}
\begin{tikzpicture}
    \node(ker) at (2,4){$\ker(\varphi)$};
     \node(e) at (2,2){$\{e\}$};
    \node (HcapN) at (5,4) {$M$};
    \node (N) at (5,2) {$P $};
    \node (H) at (8,4) {$(M_3)_0$};
    \node (G) at (8,2) {$N(H)\simeq S_4$};
    \draw[->] (ker) edge (HcapN);
    \draw[->] (e) edge (N);
     \draw[->] (ker) edge (e);
    \draw[->] (HcapN) edge (H);
    \draw[->] (N) edge (G);
    \draw[->] (HcapN) edge (N);
    \draw[->] (H) edge (G);
\end{tikzpicture}
\end{center}
the horizontal arrows are inclusions, and the vertical arrows are restrictions of $\varphi$.

The  restriction
$
\varphi_1
\left\{
\begin{array}{ccl}
M & \mapsto & P = \varphi(M)\\
g & \mapsto &\varphi(g)
\end{array}
\right.
$ is surjective, and $\ker(\varphi_1) = \ker(\varphi) \cap M = \ker(\varphi)$.

By the First Isomorphism Theorem,
$$ M/\ker(\varphi) \simeq P = \varphi(M),\qquad (M_3)_0/\ker(\varphi) \simeq N(H) = \varphi((M_3)_0).$$
Then the Third Isomorphism Theorem gives
$$N(H) / P \simeq ((M_3)_0/\ker(\varphi))/(M/\ker(\varphi) ) \simeq (M_3)_0/M \simeq \{-1,1\}.$$
Since $\varphi(M) = P$, $P$ is Abelian, and is a subgroup of index $2$ in  $N(H) \simeq S_4$. Hence $S_4$ would contain an Abelian subgroup of index 2, but this is impossible, since the only subgroup of index 2 in $S_4$ is $A_4$, which is not Abelian.

We have proved $(M_3)_0 \not \subset (M_1)_0$. Now, if $M_3 \subset M_1$, then $$\{g \in M_3 \mid g\cdot 0 = 0\} \subset \{g \in M_1 \mid g \cdot 0 = 0\},$$ so $(M_3)_0 \subset (M_1)_0$, which is false. Therefore
$$M_3 \not \subset M_1.$$

Since $M_2$ has also an Abelian subgroup of index 2 which contains $\F_p^* I_2$, we prove similarly that
$$M_2 \not \subset M_1.$$

\item[(b)] By Exercise 14.3.26, for $p=3$, we obtain
\begin{align*}
|\mathrm{AGL}(2,\F_3)| &= p^3(p-1)(p^2-1)\\
&=27 \times 2 \times 8 =432,
\end{align*}
and by Exercise 14.4.11,
\begin{align*}
|M_3| &= 24 p^2(p-1)\\
&= 24\times 9 \times 2 = 432.
\end{align*}
Since $M_3 \subset \mathrm{AGL}(2,\F_3)$ by definition, and $|M_3| = |\mathrm{AGL}(2,\F_3)|$, we obtain
$$M_3 = \mathrm{AGL}(2,\F_3) \qquad (p=3).$$

\item[(c)] We recall (see Cox, p. 453, and Ex. 16) that $(M_1)_0$ is the group $\mathrm{\Gamma L}(1,\F_{p^2})$ consisting of semi-linear maps $\gamma_{\alpha,\sigma} : \F_{p^2} \to \F_{p^2}, \alpha \in \F_{p^2}, \sigma \in \Gal(\F_{p^2}/\F_p) = \{e, F\}$, linear over $\F_p$, defined by
$$\gamma_{\alpha,\sigma}(u) = \alpha \sigma(u),\qquad u\in\F_{p^2},$$
so that $(M_1)_0 \subset \mathrm{GL}(2,\F_p),$ and we can consider the group $(M_1)_0/\F_p^* I_2$.

Now, let $c$ be a generator of $\F_{p^2}^*$, and consider $f = \gamma_{c,e} \in (M_1)_0$ defined by $f(u) = c u$ for $u\in \F_{p^2}$, and $M \in \mathrm{GL}(2,\F_p)$ the matrix representing the $\F_p$-linear map $f$.

Then for all $u \in \F_{p^2}$, $f^{p+1}(u) = c^{p+1} u = h u$, where $h^{p-1} = (c^{p+1})^{p-1} = 1$, which proves that $h = c^{p+1} \in \F_p^*$. Therefore the matrix representing $f^{p+1}$ is $\begin{pmatrix} h & 0 \\ 0 & h \end{pmatrix}$, so that $[M]^{p+1} = [I_2].$ The order of $[f]$ in $(M_1)_0/ \F_p^* I_2$ divides $p+1$.

Conversely, if $[M]^k = [I_2]$ for some integer $k >0$, then $M^k = \lambda I_2$ for some $\lambda \in \F_p^*$, thus
$$\forall u \in \F_{p^2},\ f^k(u) = c^k u = \lambda u.$$
Therefore $c^k = \lambda \in \F_p^*$, which implies $c^{k(p-1)} = 1$. Since the order of $c$ is $o(c) = p^2-1$, we obtain $p^2 - 1 \mid k(p-1)$, thus $p+1 \mid k$. This proves that the order of $[f]$ (or [M]) in $(M_1)_0/ \F_p^* I_2$ is $p+1$.

To conclude, $(M_1)_0/ \F_p^* I_2$ contains elements of order $p+1$.

\bigskip

It remains to prove that $M_1 \not \subset M_3$ when $p>3$.

If $M_1 \subset M_3$, then $\F_p^* I_2 \subset (M_1)_0 \subset (M_3)_0$, therefore 
$$(M_1)_0 / \F_p^* I_2 \subset (M_3)_0/ \F_p^*I_2.$$
We proved in part (a) that $(M_3)_0/ \F_p^* I_2 \simeq N(H) \simeq S_4$.

The first group $(M_1)_0 / \F_p^*I_2$ contains an element of order $p+1$, which is also in $(M_3)_0/ \F_p^* I_2\simeq S_4$, and this implies that
$$p+1 \mid 24 =  |(M_3)_0/ \F_p^* I_2|,$$
which gives $p = 5,7,11,23$ (if $p>3$). But in these cases we obtain in $S_4$ some elements of order $6,8,12$ or $24$. But all elements in $S_4$ have orders $1,2,3$ or $4$. This is a contradiction.

To conclude, 
$$M_1 \not \subset M_3 \text{ if } p>3.$$
\bigskip

{\bf Example.} Take $p=11$. Since $x^2 - 4d - 9$ is irreducible over $\F_{11}$, we use the basis ${\cal B} = (1,c)$ of $\F_{p^2}$ over $\F_p$, where $c$ is a generator of $\F_{11}^*$ such that $c^2 = 4c + 9$. Then the matrix of $f$ defined by $f(u) = c u$ in the basis $\cal B$ is
$$A = \begin{pmatrix} 0 & 9\\1 & 4 \end{pmatrix}.$$
The multiplicative order of $A$ is 120, but with Sage, we obtain
$$A^{12} =  \begin{pmatrix} 2 & 0\\0 & 2 \end{pmatrix},$$
(where $2 = c^{12}$), so that $[A]^{12} = [I_2]$.

The matrix $S$ corresponding to the Frobenius isomorphism $\sigma = F$ is, using $\sigma(c) = c^{11} = 10c + 4$, 
$$S = \begin{pmatrix} 1 & 4\\0 & 10 \end{pmatrix}.$$
Then $S^2 = I_2$, and we can build the whole group $ (M_1)_0$ by
$$(M_1)_0 = \{A^k \mid 0 \leq k < 120\} \cup \{A^k S \mid 0 \leq k < 120\}.$$
The only elements of order $p+1$ in $(M_1)_0/ \F_p^* I_2$ are in the $[A^k]$, and not in the $[A^k S]$.

\item[(d)] By Propositions 14.4.2 and 14.4.5, we know that
\begin{align*}
|M_2| &= 2p^2(p-1)^2,\\
|M_3| &= 24p^2(p-1).
\end{align*}
If $M_2 \subset M_3$, by Lagrange Theorem, $|M_2| = 2 p^2(p-1)^2$ divides  $|M_3| = 24 p^2(p-1)$, therefore
$$p-1 \mid 12.$$
The only prime solution $p$ with $p>5$ are $p=7, p=13$. 

If $p=13$, then $|M_2| = |M_3| = 48672 = 2^5 3^2 13^2$, and $M_2 \subset M_3$, thus $M_2 = M_3$. But this is impossible, since, by part (a), $(M_2)_0$ has a subgroup of index 2 containing $\F_p^* I_2$, but not $(M_3)_0$.

It remains to prove that $M_2 \not \subset M_3$ if $p=7$. Then $|M_2| = 3528 = 2^3 3^2 7^2 = 3528, |M_3| = 2^4 3^2 7^2 =7056$, thus $(M_3 : M_2) = 2$, and $M_2$ would be a normal subgroup of index $2$ in $M_3$.

Consider the surjective homomorphism
$$\varphi
\left\{
\begin{array}{ccl}
M_3 & \mapsto & N(H) \subset \mathrm{PGL}(2,\F_p)\\
\gamma_{A,v} & \mapsto &[A].
\end{array}
\right.
$$
Note that $\ker(\varphi) = \{\gamma_{\lambda I_2, v} \mid \lambda \in \F_p^*, v \in \F_p^2\} \subset M_2$, since for all $u = (\alpha,\beta) \in \F_p^2$, and $v = (a,b)$
$$\gamma_{\lambda I_2,v} = \begin{pmatrix}\lambda & 0\\ 0 & \lambda \end{pmatrix} \begin{pmatrix} \alpha \\ \beta \end{pmatrix} + \begin{pmatrix} a \\ b \end{pmatrix} = \begin{pmatrix} \lambda \alpha + a \\ \lambda \beta + b \end{pmatrix},$$
so that $\ker(\varphi) \subset  \mathrm{AGL}(1,\F_p) \times \mathrm{AGL}(1,\F_p)  \subset M_2$.

Write $N = \mathrm{AGL}(1,\F_p) \times \mathrm{AGL}(1,\F_p)$, and $M = \varphi(N)$. In the following diagram, 
\begin{center}
\begin{tikzpicture}
    \node(ker) at (-1,4){$\ker(\varphi)$};
    \node(e) at (-1,2){$\{e\}$};
    \node(N) at (2,4){$N$};
    \node(M) at (2,2){$M$};
    \node (M2) at (5,4) {$M_2$};
    \node (L) at (5,2) {$L $};
    \node (M3) at (8,4) {$M_3$};
    \node (G) at (8,2) {$N(H)\simeq S_4$};
    \draw[->] (ker) edge (N);
    \draw[->] (N) edge (M2);
    \draw[->] (M2) edge (M3);
    \draw[->] (e) edge (M);
    \draw[->] (M) edge (L);
    \draw[->] (L) edge (G);
    \draw[->] (ker) edge (e);
    \draw[->] (N) edge (M);
    \draw[->] (M2) edge (L);
     \draw[->] (M3) edge (G);
\end{tikzpicture}
\end{center}
the horizontal arrows are inclusions, and the vertical arrows are restrictions of $\varphi$, whose kernels are $\ker(\varphi)$, because $\ker(\varphi) \subset N$.  Therefore
$$N(H)/L \simeq (M_3/\ker(\varphi))/ (M_2/\ker(\varphi)) \simeq M_3/M_2 \simeq \{-1,1\}.$$

This proves that $L$ is a subgroup of index 2 in $N(H) \simeq S_4$. But $S_4$ has a unique subgroup of order 2, which is $A_4$. Thus $L \simeq A_4$.

Similarly, $N$ has index 2 in $M_2$, thus $M$ has index 2 in $L \simeq A_4$. This is impossible, since $A_4$ has no subgroup of index $2$.

To conclude: If $p>5$,
$$M_2 \not \subset M_3.$$

\item[(e)] If $p=5$, $|M_2| = 2p^2(p-1)^2 = 800, |M_3| = 24p^2(p-1) = 2400$. 

Here $p \equiv 1 \pmod 4$, and $i=2$ is such that $i^2 = -1$. By definition of $H$, $H =\langle [g],[h] \rangle$, where
$$ g = \begin{pmatrix} 0 & -1 \\ 1 & 0 \end{pmatrix}, h =  \begin{pmatrix} 2 & 0  \\0 & -2 \end{pmatrix}$$
(by Proposition 14.4.4, other choices of $g,h$ as in part (a) give conjugate subgroups). Thus
$$H = \left \{ [I_2],  \begin{bmatrix} 0 & -1 \\ 1 & 0 \end{bmatrix},  \begin{bmatrix} 0 & 2  \\2 & 0 \end{bmatrix}, \begin{bmatrix} -2 & 0 \\ 0 & 2 \end{bmatrix}\right\} \simeq C_2\times C_2.$$
We have seen in part (a) that
$$
\varphi
\left\{
\begin{array}{ccl}
(M_3)_0 & \mapsto & N(H) \subset \mathrm{PGL}(2,\F_p)\\
A & \mapsto &[A]
\end{array}
\right.
$$
satisfies $\varphi((M_3)_0) = N(H)$, and that
\begin{align*}
(M_2)_0 &= \left \{  \begin{pmatrix} a & 0 \\ 0 & b \end{pmatrix} \mid a,b \in \F_5^* \right\} \cup \left \{  \begin{pmatrix} 0 & a \\ b & 0 \end{pmatrix} \mid a,b \in \F_5^* \right\} \\
&= L \cup (B\cdot L).
\end{align*}
Therefore
\begin{align*}
\varphi((M_2)_0 )&= \left \{  \begin{bmatrix} a & 0 \\ 0 & b \end{bmatrix} \mid a,b \in \F_5^* \right\} \cup \left \{  \begin{bmatrix} 0 & a \\ b & 0 \end{bmatrix} \mid a,b \in \F_5^* \right\} \\
&= \varphi(L) \cup ([B]\cdot \varphi(L)).
\end{align*}
Since $[I_2], \begin{bmatrix} -2 & 0 \\ 0 & 2 \end{bmatrix}  \in \varphi(L)$, and $\begin{bmatrix} 0 & -1 \\ 1 & 0 \end{bmatrix}, \begin{bmatrix} 0 & 2  \\2 & 0 \end{bmatrix} \in [B]\cdot \varphi(L)$, we obtain
$$H \subset \varphi((M_2)_0 ),$$
where $\varphi((M_2)_0 ) \simeq (M_2)_0 / \F_p^*I_2$ has order $|(M_2)_0|/|\F_p^*| = 32/4 = 8$.

Since $H$ is a subgroup of index $2$ in  $\varphi((M_2)_0 )$, $H$ is a normal subgroup of $\varphi((M_2)_0 )$, so that $\varphi((M_2)_0 )$ normalizes $H$:
$$\varphi((M_2)_0 ) \subset N(H) = \varphi((M_3)_0).$$
If $A \in (M_2)_0$, then $[A] = \varphi(A) \in \varphi((M_2)_0 ) \subset \varphi((M_3)_0) = N(H)$, so that $A \in (M_3)_0$.
$$(M_2)_0 \subset (M_3)_0.$$
Then we can prove $M_2 \subset M_3$. 
Let $\gamma = \gamma_{A,v}$ be any element in $M_2 \subset \mathrm{AGL}(2,\F_p)$. Then $\gamma = \gamma_{I_2,v} \gamma_{A,0}$, where $\gamma_{A,0} \in (M_2)_0 \subset (M_3)_0$, and $\gamma_{I_2,v} \in M_3$. Therefore $\gamma \in M_3$.

If $p=5$, $$ M_2 \subset M_3.$$
\end{proof}

\paragraph{Ex. 14.4.13}{\it Let $G_0 \subset \mathrm{GL}(2,\F_p)$ be solvable. Prove that the subgroup generated by $G_0$ and $\F_p^* I_2$ is also solvable.
}

\begin{proof}
The subgroup $N = \F_p^* I_2$  is a normal subgroup of $G = \mathrm{GL}(2,\F_p)$: 

If $A \in G$, and $H = \lambda I_2$, then $A H A^{-1} = A \lambda I_p A^{-1} = \lambda A A^{-1} = \lambda I_2 \in H$.

Therefore $G_0N$ is a subgroup of $G$. This implies that $G_0 N$ is the smallest subgroup containing $G_0$ and $N$, so that $G_0N$ is the subgroup generated by $G_0$ and $\F_p^* I_2$.

By the Second Isomorphism Theorem (see the beginning of Exercise 12),
$$G_0N/N \simeq G_0/N\cap G_0.$$
By hypothesis, $G_0$ is solvable, thus its subgroup $N \cap G_0$ is solvable, and the quotient group $G_0/N\cap G_0$ is solvable (Proposition 8.2.4).

$N$ is cyclic, thus $N$ is solvable. Using Proposition 8.2.4 anew, $G_0N$ is solvable.

We have proved that if $G_0 \subset \mathrm{GL}(2,\F_p)$ is solvable, the subgroup generated by $G_0$ and $\F_p^* I_2$ is also solvable.
\end{proof}

\paragraph{Ex. 14.4.14}{\it Let $g = \begin{pmatrix} \alpha & 0 \\ 0 & \beta \end{pmatrix}$, where $\alpha, \beta \in \F_p^*$ and $\alpha \ne \beta$.
\be
\item[(a)] Prove (14.29):
$$C(g) = \left\{ \begin{pmatrix} \mu & 0 \\ 0 & \nu \end{pmatrix} \mid \mu,\nu \in \F_p^*\right\}.$$
\item[(b)] Let $m =  \begin{pmatrix} a & b \\ c & d \end{pmatrix} \in N(C(g))$. In the argument following (14.29), we claimed that $b=c=0$ or $a=d=0$. Supply the missing details.
\item[(c)] Prove that $(M_2)_0 \subset N(C(g))$.
\ee
}

\begin{proof}
\item[(a)] First $\left\{ \begin{pmatrix} \mu & 0 \\ 0 & \nu \end{pmatrix} \mid \mu,\nu \in \F_p^*\right\} \subset C(g)$, since for all $\mu,\nu \in \F_p^*$,
$$ \begin{pmatrix} \mu & 0 \\ 0 & \nu \end{pmatrix}\begin{pmatrix} \alpha & 0 \\ 0 & \beta \end{pmatrix}
= \begin{pmatrix} \mu\alpha & 0 \\ 0 & \nu\beta \end{pmatrix}
=\begin{pmatrix} \alpha & 0 \\ 0 & \beta \end{pmatrix}\begin{pmatrix} \mu & 0 \\ 0 & \nu \end{pmatrix}.$$
Conversely, if $h =  \begin{pmatrix} \mu & \xi \\ \eta & \nu \end{pmatrix} \in C(g)$, since $g\not \in \F_p^*I_2$, Lemma 14.4.3 shows that $h = aI_2+bg$ for some $a,b \in \F_p$, thus 
$$h =  \begin{pmatrix} \mu & \xi \\ \eta & \nu \end{pmatrix} =  \begin{pmatrix} a + b\alpha & 0 \\ 0 & a + b \beta \end{pmatrix},$$
so that $\xi = \eta = 0$, and $h =  \begin{pmatrix} \mu & 0 \\ 0 & \nu \end{pmatrix}$.

We have proved 
$$C(g) = \left\{ \begin{pmatrix} \mu & 0 \\ 0 & \nu \end{pmatrix} \mid \mu,\nu \in \F_p^*\right\}.$$
\item[(b)] If $m =  \begin{pmatrix} a & b \\ c & d \end{pmatrix} \in N(C(g))$, then the text p. 452 shows that
$$\begin{pmatrix} a & b \\ c & d \end{pmatrix} \begin{pmatrix} \alpha & 0 \\ 0 & \beta \end{pmatrix} = \begin{pmatrix} \mu & 0 \\ 0 & \nu \end{pmatrix} \begin{pmatrix} a & b \\ c & d \end{pmatrix},\qquad \mu, \nu  \in \F_p^*,\  \mu \ne \nu,$$
thus
$\begin{pmatrix} a\alpha & b\beta \\ c\alpha & d\beta \end{pmatrix}  = \begin{pmatrix} \mu a & \mu b \\ \nu c & \nu d \end{pmatrix},$ that is
$\begin{pmatrix} a(\alpha - \mu) & b(\beta - \mu) \\ c(\alpha - \nu) & d(\beta - \nu) \end{pmatrix}  = 0$.

If $\alpha = \mu$, then $\alpha \ne \nu$ (since $\mu \ne \nu$), and $\beta \ne \mu$ (since $\alpha \ne \beta$), then $b = c = 0$. 

Similarly, if $\beta = \mu$, then $\beta \ne \nu$ and $\alpha \ne \mu$, thus $a= d = 0$.

It remains the case where $\alpha \ne \mu$ and $\beta \ne \mu$. Then $a = b = 0$, but then $m \not \in \mathrm{GL}(2,\F_p)$: this is in contradiction with $m \in N(C(g)) \subset \mathrm{GL}(2,\F_p)$.

To conclude,
$$m =  \begin{pmatrix} a & b \\ c & d \end{pmatrix} \in N(C(g)) \Rightarrow b=c=0 \text{ or } a=d=0.$$

\item[(c)] Let $m \in(M_2)_0$. Then 
$$m = \begin{pmatrix} a& 0 \\ 0 & d \end{pmatrix} \text{ or }m = \begin{pmatrix} 0& b\\  c & 0 \end{pmatrix}, \qquad a,b,c,d \in \F_p^*.$$

In the first case, for all $c = \begin{pmatrix} \mu & 0 \\ 0 & \nu \end{pmatrix} \in C(g)$,  since $m,c$ are diagonal, $mc= cm$, and $mcm^{-1} = c \in C(g)$, and $m \in N(C(g))$.

In the other case, $m = \begin{pmatrix} 0& b\\  c & 0 \end{pmatrix}$, and $m^{-1} =   \begin{pmatrix} 0& c^{-1}\\  b^{-1} & 0 \end{pmatrix}$, then
\begin{align*}
mcm^{-1} &= \begin{pmatrix} 0& b\\  c & 0 \end{pmatrix} \begin{pmatrix} \mu & 0 \\ 0 & \nu \end{pmatrix} \begin{pmatrix} 0& c^{-1}\\  b^{-1} & 0 \end{pmatrix}\\
&=\begin{pmatrix} 0& b\nu\\  c\mu & 0 \end{pmatrix}\begin{pmatrix} 0& c^{-1}\\  b^{-1} & 0 \end{pmatrix}\\
&=\begin{pmatrix} \nu & 0\\  0 & \mu \end{pmatrix} \in C(g)\\
\end{align*}
Thus, for all $c \in C(g)$, $mcm^{-1} \in C(g)$, therefore $m \in N(C(g))$.
$$(M_2)_0 \subset N(C(g)).$$
\end{proof}

\paragraph{Ex. 14.4.15}{\it Prove (14.30) and (14.31).
\begin{align*}
C(g) &= \left\{\begin{pmatrix} \mu & \nu\\ 0 & \mu \end{pmatrix} \mid \mu,\nu \in \F_p, \ \mu \ne 0\right\}\\
N(C(g)) &=  \left\{\begin{pmatrix} \mu & \nu\\ 0 & \lambda \end{pmatrix} \mid \mu,\nu,\lambda \in \F_p,\  \mu \lambda \ne 0\right\}
\end{align*}
}

\begin{proof}
\item[(a)]
 Here $g = \begin{pmatrix} \alpha & \beta\\ 0 & \alpha \end{pmatrix} \in \mathrm{GL}(2,\F_p)$, where $\beta \ne 0$ since $g \not \in \F_p^* I_2$.

Let $m=\begin{pmatrix} \mu & \nu\\ \xi  & \eta \end{pmatrix} \in \mathrm{GL}(2,\F_p)$.
Since 
\begin{align*}
gm = \begin{pmatrix} \alpha & \beta\\ 0 & \alpha \end{pmatrix} \begin{pmatrix} \mu & \nu\\ \xi  & \eta \end{pmatrix}&= 
\begin{pmatrix} \alpha \mu + \beta \xi & \alpha \nu+\beta \eta\\ \alpha \xi & \alpha \eta \end{pmatrix},\\
mg = \begin{pmatrix} \mu & \nu\\ \xi  & \eta \end{pmatrix}\begin{pmatrix} \alpha & \beta\\ 0 & \alpha \end{pmatrix} &= 
\begin{pmatrix} \alpha \mu& \alpha \nu + \beta \mu\\ \alpha \xi & \alpha \eta + \beta \xi, \end{pmatrix}
\end{align*}
we obtain, using $\beta \ne 0$,
$$ m \in C(g) \iff \xi = 0, \mu = \nu.$$
Therefore
$$C(g) = \left\{\begin{pmatrix} \mu & \nu\\ 0 & \mu \end{pmatrix} \mid \mu,\nu \in \F_p, \mu \ne 0\right\}.$$
\item[(b)] Let  $h = \begin{pmatrix} a & b\\ c & d \end{pmatrix} \in N(C(g))$. Since $g \in C(g)$, we obtain $hgh^{-1} \in C(g)$, thus there are some $\mu,\nu \in \F_p, \mu \ne 0,$ such that 
$$hgh^{-1}  = h \begin{pmatrix} \alpha & \beta\\ 0 & \alpha \end{pmatrix} h^{-1} = \begin{pmatrix} \mu & \nu\\ 0 & \mu \end{pmatrix}.$$
This gives
$$ \begin{pmatrix} a & b\\ c & d \end{pmatrix}\begin{pmatrix} \alpha & \beta\\ 0 & \alpha \end{pmatrix}
= \begin{pmatrix} \mu & \nu\\ 0 & \mu \end{pmatrix} \begin{pmatrix} a & b\\ c & d \end{pmatrix},
$$
that is
$$
\begin{pmatrix} a\alpha & a \beta + b \alpha\\ c\alpha & c\beta + d \alpha \end{pmatrix} = \begin{pmatrix} a \mu + c \nu& b\mu + d \nu \\ c \mu& d \mu \end{pmatrix}.
$$
If $c \ne 0$, then $\alpha = \mu$. Thus $c \beta + d\alpha = d \mu = d\alpha$, so $c\beta = 0$, which is impossible since $\beta \ne 0$. Therefore $c=0$, and every $h \in N(C(g))$ is of the form
$$h = \begin{pmatrix} a & b\\ 0 & d \end{pmatrix}, \qquad ad \ne 0.$$
Conversely, if $h = \begin{pmatrix} a & b\\ 0 & d \end{pmatrix},  ad \ne 0$, and $n = \begin{pmatrix} \mu & \nu\\ 0 & \mu \end{pmatrix} $ is any element in $C(g)$, then
\begin{align*}
h n h^{-1}&= (ad)^{-1} \begin{pmatrix} a & b\\ 0 & d \end{pmatrix} \begin{pmatrix} \mu & \nu\\ 0 & \mu \end{pmatrix}  \begin{pmatrix} d & -b\\ 0 & a \end{pmatrix}\\
&=(ad)^{-1}\begin{pmatrix} a & b\\ 0 & d \end{pmatrix}  \begin{pmatrix} \mu d  & a \nu - b \mu\\ 0 & a \mu \end{pmatrix} \\
&=(ad)^{-1}\begin{pmatrix} a d \mu  & a(a\nu - b \mu) + ab \mu\\ 0 & ad \mu \end{pmatrix}\\
&=\begin{pmatrix} \mu  & ad^{-1}\nu\\ 0 & \mu  \end{pmatrix} \in C(g).
\end{align*}
We have proved
$$N(C(g)) =  \left\{\begin{pmatrix} a & b \\0 & d \end{pmatrix} \mid a,b,d \in \F_p,\ ad\ne 0\right\},$$
which is the same as (14.31):
$$N(C(g)) =  \left\{\begin{pmatrix} \mu & \nu\\ 0 & \lambda \end{pmatrix} \mid \mu,\nu,\lambda \in \F_p,\  \mu \lambda \ne 0\right\}.$$
\end{proof}

\paragraph{Ex. 14.4.16}{\it Let $V,W$ be vector spaces over a field $F$, and let $\mathrm{Aut}_F(V)$ be the group of vector space isomorphisms $V \simeq V$. Also let $T : V \to W$ be a vector space isomorphism.
\be
\item[(a)] Prove that $\phi \mapsto T \circ \phi \circ T^{-1}$ induces a group isomorphism $\gamma_T : \mathrm{Aut}_F(V) \simeq \mathrm{Aut}_F(W)$.
\item[(b)] Let $T' : V\to W$ be another isomorphism. Prove that there is $\Phi \in \mathrm{Aut}_F(W)$ such that $T'= \Phi \circ T$. In the notation of part (a), $\gamma_\Phi : \mathrm{Aut}_F(W) \simeq \mathrm{Aut}_F(W)$ is conjugation by $\Phi$.
\item[(c)] In the situation of part (b), prove that $\gamma_{T'} = \gamma_\Phi \circ \gamma_T$.
\ee
}

\begin{proof}
\item[(a)] Let $ T : V \to W$ be a vector space isomorphism, and consider the map
$$
\gamma_T
\left\{
\begin{array}{ccc}
\mathrm{Aut}_F(V) & \to &\mathrm{Aut}_F(W)\\
\phi & \mapsto & T \circ \phi \circ T^{-1}.
\end{array}
\right.
$$
$\gamma_T$ is a group homomorphism: if $\phi,\psi \in \mathrm{Aut}_F(V)$,
$$ \gamma_T(\phi)\circ \gamma_T(\psi) = T \circ \phi \circ T^{-1} \circ T \circ \psi \circ T^{-1} = T \circ \phi \circ \psi \circ T^{-1} = \gamma_T(\phi \circ \psi).$$
Moreover $\gamma_{T^{-1} } : \mathrm{Aut}_F(W) \to \mathrm{Aut}_F(V)$ is such that, for all $\phi \in \mathrm{Aut}_F(V)$,
$$(\gamma_{T_{-1}} \circ \gamma_T)(\phi) =  T^{-1} \circ (T \circ \phi \circ T^{-1}) \circ T = \phi,$$
thus $\gamma_{T^{-1}} \circ \gamma_T = 1_{\mathrm{Aut}_F(V)}$, and similarly $\gamma_{T} \circ \gamma_{T^{-1}}= 1_{\mathrm{Aut}_F(W)}$. This proves that $\gamma_T$ is bijective, and that $\gamma_T^{-1} = \gamma_{T^{-1}}$. Thus $\gamma_T$ is a group isomorphism.

\item[(b)] Let $T' : V \to W$ be another isomorphism, and put $\Phi =T' \circ T^{-1}$. Then $\Phi : W \to W$ is bijective and linear, so $\Phi \in \mathrm{Aut}_F(W)$, and $T' = \Phi \circ T$. 

By definition,
$$
\gamma_\Phi
\left\{
\begin{array}{ccc}
\mathrm{Aut}_F(W) & \to &\mathrm{Aut}_F(W)\\
\psi  & \mapsto & \Phi \circ \psi \circ \Phi^{-1},
\end{array}
\right.
$$
and by part (a), $\gamma_\Phi$ is the conjugation by $\Phi$, so is an inner group automorphism: $$\gamma_\Phi \in \mathrm{Int}(\mathrm{Aut}_F(W)) \subset \mathrm{Aut}(\mathrm{Aut}_F(W)).$$
\item[(c)] If $\phi \in \mathrm{Aut}_F(V)$, then
\begin{align*}
(\gamma_\Phi \circ \gamma_T)(\phi) &= \Phi \circ (T \circ \phi \circ T^{-1}) \circ \Phi^{-1}\\
&= (\Phi \circ T) \circ \phi \circ (\Phi \circ T)^{-1}\\
&= T' \circ \phi \circ (T')^{-1}\\
&= \gamma_{T'}(\phi),
\end{align*}
so that
$$\gamma_{T'} = \gamma_{\Phi} \circ \gamma_T.$$
\end{proof}

\paragraph{Ex. 14.4.17}{\it Fix $\alpha \in \F_{p^2} \setminus \F_p$, and let $\gamma_\alpha$ be as defined just before (14.35).
\be
\item[(a)] Prove (14.37) and (14.39). For (14.37), you should use the argument from the proof of Lemma 14.4.3.
\item[(b)] Prove that $\mathrm{\Gamma L}(1,\F_{p^2}) \subset N(C(\gamma_\alpha))$.
\ee
}
\begin{proof}
\item[(a)] Write $e =1_{\F_{p^2}} \in \mathrm{Aut}_{\F_p}(\F_{p^2})$ the identity of $\F_{p^2}$, which corresponds to $I_2 \in \mathrm{GL}(2,\F_p)$. We want to show that
$$C(\gamma_\alpha) = \{ ae + b \gamma_\alpha \in \mathrm{Aut}_{\F_p}\F_{p^2} \mid a,b \in \F_p\}.$$
First $ae + b \gamma_\alpha \in C(\gamma_\alpha)$, since
$$ (ae + b \gamma_\alpha)\gamma_\alpha =a\gamma_\alpha + b \gamma_\alpha^2  = \gamma_\alpha(ae + b \gamma_\alpha).$$
As in the proof of Lemma 14.4.3, we take $m \in C(\gamma_\alpha)  \subset \mathrm{Aut}_{\F_p}(\F_{p^2})$.

Since $\alpha \not \in \F_p$, $(1,\alpha)$ is a basis of $\F_{p^2}$ over $\F_p$, thus there are $r,s$ such that
$$m(1) = a+ b\alpha,\qquad a,b \in \F_p.$$
Then, using $m \circ \gamma_\alpha = \gamma_\alpha \circ m$, we obtain
\begin{align*}
m(\alpha) &= (m\circ \gamma_\alpha)(1)\\
&=(\gamma_{\alpha} \circ m)(1)\\
&= \gamma_{\alpha}(a + b \alpha)\\
&=\alpha a+ b \alpha^2,
\end{align*}
therefore
\begin{align*}
m(1) &= \phantom{\alpha} a+ b\alpha\ = (ae + b \gamma_\alpha)(1),\\
m(\alpha) &= \alpha a+ b \alpha^2 = (ae + b \gamma_\alpha)(\alpha).
\end{align*}
Since $m$ is $\F_p$-linear, and $(1,\alpha)$ is a basis, we obtain that $m = ae + b\gamma_\alpha$.
We have proved that
$$C(\gamma_\alpha) = \{ ae + b \gamma_\alpha \in \mathrm{Aut}_{\F_p}\F_{p^2} \mid a,b \in \F_p\}.$$
Moreover, for all $u\in \F_{p^2}$,
$$(ae + b \gamma_\alpha)(u) = a u + b \alpha u = (a+b\alpha) u = \beta u \text{ where } \beta = a+b \alpha \in \F_{p^2},$$
and $\beta \ne 0$, since $ae + b \gamma_\alpha$ is a linear automorphism.
Conversely, if $\beta \in \F_{p^2}^*$, $\gamma_\beta$ is a $\F_p$ linear automorphism. The decomposition of $\beta$ on the basis $(1,\alpha)$ gives $a,b\in \F_p$ such that $\beta = a + b \alpha$, and 
$$\gamma_\beta(u) = (a+b\alpha)u = (ae + b\gamma_{\alpha})(u).$$
To conclude,
$$C(\gamma_\alpha) = \{ ae + b \gamma_\alpha \in \mathrm{Aut}_{\F_p}\F_{p^2} \mid a,b \in \F_p\} = \{\gamma_\beta \mid \beta \in \F_{p^2}^*\}.$$

\bigskip

Now we prove (14.39):
\begin{align*}
\beta = \alpha &\Rightarrow m = \gamma_\delta = \gamma_{\delta,e} \in \mathrm{\Gamma L}(1,\F_{p^2}),\\
\beta = \sigma(\alpha) &\Rightarrow m = \gamma_\delta \circ \sigma = \gamma_{\delta,\sigma} \in \mathrm{\Gamma L}(1,\F_{p^2}).
\end{align*}
By the proof of Theorem 14.4.6 (p. 455), if $m \in N(C(\gamma_\alpha)$, we know that 
$$m\circ \gamma_\alpha = \gamma_\beta \circ \gamma,\qquad \beta \in \{\alpha,\sigma(\alpha)\} = \mathrm{Gal}(\F_{p^2}/\F_p).$$
  \be
  \item[$\bullet$] If $\beta = \alpha$, then $m \circ \gamma_\alpha = \gamma_\alpha \circ \gamma$, thus $m \in C(\gamma_\alpha)$. We have proved above that $m = \gamma_\delta,$ where $\delta \in      \F_{p^2}$, and  $m(1) = \gamma_\delta(1) = \delta$. Therefore $m = \gamma_\delta = \gamma_{\delta,e} \in \mathrm{\Gamma L}(1,\F_{p^2})$.

  \item[$\bullet$] If $\beta = \sigma(\alpha)$, then $m\circ \gamma_\alpha = \gamma_{\sigma(\alpha)} \circ m$. For all $u \in \F_{p^2}$, $m(\alpha u ) = \sigma(\alpha) m(u)$, thus, using $\sigma(1) = 1$, and $\delta = m(1)$,
$$
\begin{array}{lll}
  m(1) &= \delta &=(\gamma_\delta \circ \sigma)(1),\\
  m(\alpha) &= \sigma(\alpha) \delta &= (\gamma_\delta \circ \sigma)(\alpha).
  \end{array}
$$
  This proves that $m = \gamma_\delta \circ \sigma$. Moreover, for all $u \in \F_{p^2}$, $m(u) = \delta \sigma(u) = \gamma_{\delta,\sigma}(u)$, therefore
  $$m = \gamma_{\delta,\sigma} \in \mathrm{\Gamma L}(1,\F_{p^2}).$$
  \ee

\item[(b)] Let $m \in \mathrm{\Gamma L}(1,\F_{p^2})$, and $n \in C(\gamma_\alpha)$. By part (a), $n  =ae + b \gamma_\alpha,\ a,b \in \F_p$.

Since $\Gal(\F_{p^2}/\F_p) = \{e,\sigma\}$, $m = \gamma_\delta$, or $m = \gamma_{\delta, \sigma} = \gamma_\delta \circ \sigma$ for some $\delta \in \F_{p^2}$.
\be
\item[$\bullet$] If $m = \gamma_\delta$, since $n \in C(\gamma_\alpha)$,  $m\circ n \circ m^{-1} = \gamma_\alpha \circ n \circ \gamma_\alpha^{-1} = n \in C(\gamma_\alpha)$.

\item[$\bullet$] If $m = \gamma_\delta \circ \sigma$, then for all $u \in \F_{p^2}$, $m(u) = \delta \sigma(u)$, thus
\begin{align*}
(m \circ \gamma_\alpha)(u) &= \delta \sigma(\alpha u) = \delta \sigma(\alpha) \sigma(u),\\
(\gamma_{\sigma(\alpha)} \circ m)(u) &= \sigma(\alpha) \delta \sigma(u),
\end{align*}
therefore $ m \circ \gamma_\alpha =\gamma_{\sigma(\alpha)} \circ m$, and $ m \circ \gamma_\alpha \circ m^{-1} = \gamma_{\sigma(\alpha)}.$ Then, using the linearity of $m,n$,
\begin{align*}
 m \circ n \circ m^{-1} &= m \circ (a\, e + b\,  \gamma_\alpha)\circ m^{-1} \\
   &= a\,e + b\, m \circ \gamma_\alpha \circ m^{-1}\\
   &= a\, e + b \, \gamma_{\sigma(\alpha)}  \\
   &= \gamma_\beta,    
\end{align*}
where $\beta = a+b \sigma(\alpha) \in \F_{p^2}$. 

By part (a), $ m \circ n \circ m^{-1} =\gamma_\beta \in C(\gamma_\alpha)$. We have proved
$$\mathrm{\Gamma L}(1,\F_{p^2}) \subset N(C(\gamma_\alpha)).$$

\ee
\end{proof}

\paragraph{Ex. 14.4.18}{\it Let $M$ be a finite group.
\be
\item[(a)] Let $A \subset M$ be a minimal normal subgroup, and let $g\ne e$ be in $A$. Prove that $A$ is generated by the elements $hgh^{-1}$ as $h$ varies over all elements of $M$.
\item[(b)] Let $A \subset M$ be a normal subgroup. Prove that the center $Z(A)$ of $A$ is normal in $M$.
\ee
}
\begin{proof} 
\item[(a)] Let $g\ne e$ be in $A$, and consider
$$B = \left \langle h g h^{-1} \mid h \in M \right \rangle,$$
the subgroup of $M$ generated by the elements $hgh^{-1}$ as $h$ varies over all elements of $M$.
Since $g \in A$, where $A$ is normal in $M$, then $hgh^{-1} \in A$ and $hg^{-1}h^{-1} \in A$ for all $h \in M$, thus $B \subset A$.

Moreover $B$ is normal in $M$: if $m \in M$, for all generators $hgh^{-1}$ of $B$, $m(hgh^{-1})m^{-1} = (mh) g (mh)^{-1} \in B$, and similarly $m(hgh^{-1})^{-1}m^{-1}  = (mh) g^{-1}(mh)^{-1}\in B$. 

If $g_1 = h_1 g h_1^{-1},\ldots,g_l = h_l g h_l^{-1}$ are the generators of $B$, every element $b \in B$ is of the form
$$b = g_{i_1}^{\varepsilon_1} \cdots g_{i_k}^{\varepsilon_k}, \text{ where } \varepsilon_i = \pm 1, i=1,\ldots,k.$$
Therefore
$$m b m^{-1} = mg_{i_1}^{\varepsilon_1}m^{-1} \cdots m g_{i_k}^{\varepsilon_k}m^{-1} \in B.$$

Here $B \subset A \subset M$, where $B$ is normal in $M$, and $A$ is a minimal normal subgroup of $M$, therefore $ B = \{e\}$ or $B = A$. But $B \ne \{e\}$, since $e \ne g \in B$. Thus $B = A$:
$$A = \left \langle h g h^{-1} \mid h \in M \right \rangle.$$

\item[(b)] Let $g \in M, b \in Z(A)$. $A$ is normal in $G$, thus  for all $a \in A$, $a' = g^{-1} a g \in A$. Since $b \in Z(A)$, $a' b  = b a'$, that is $(g^{-1}a g) b = b (g^{-1} a g)$, which implies, by left-multiplication by $g$ and right-multiplication by $g^{-1}$,
$$a(gbg^{-1}) = (g b g^{-1}) a.$$
This relation is true for every $a \in A$, thus $gbg^{-1} \in Z(A)$, for all $g \in M$ and for all $b \in Z(A)$. This proves that $Z(A)$ is a normal subgroup of $M$.

\bigskip

Note: More generally, if $K \subset A \subset M$, where $K$ is a characteristic subgroup of $A$, and $A$ a normal subgroup of $G$, then $K$ is normal in $G$ (see [Issacs] Finite Group Theory, Lemma 1.10 p. 11): ``conjugation by $g$ maps $A$ onto itself, and it follows that the restriction of this conjugation map to $A$ is an automorphism of $A$ (but not necessarily an inner automorphism of $A$). Since $K$ is characteristic in $A$, it is mapped onto itself by this automorphism of $N$''. Here $Z(A)$ is a characteristic subgroup of $A$, so this Lemma gives a solution to part (b).
\end{proof}

\paragraph{Ex. 14.4.19}{\it In the Mathematical Notes, we mentioned that all subgroups of $\mathrm{PSL}(2,\F_q)$ are known up to conjugacy. We will do a small part of this classification by proving that $\mathrm{PSL}(2,\F_q)$ contains a subgroup isomorphic to $S_4$ when $p\equiv 1 \pmod 8$. To begin, note that by Exercise 10, the images of the matrices (14.25) generate a subgroup of $\mathrm{PGL}(2,\F_q)$ isomorphic to $S_4$.
\be
\item[(a)] Explain why $\F_p^*$ has an element $\zeta$ of order $8$. Then $i = \zeta^2$ has order $4$.
\item[(b)] Compute $(1+i)^2$ and use this to prove that there is $\alpha \in \F_p$ such that $\alpha^2 = 2$.
\item[(c)] Show that the matrices (14.25) lie in $\mathrm{SL}(2,\F_p)$ after multiplication by suitable elements of $\F_p^*$. Hence their images generate a subgroup of $\mathrm{PSL}(2,\F_p)$ isomorphic to $S_4$.
\item[(d)] Over $\C$, $\zeta_8 = \cos(2\pi/8) + i \sin(2\pi/8) = (1+i)/\sqrt{2}$. How does this relate to part (b)?
\ee

}
\begin{proof}
\item[(a)] We know that $\F_p^*$ is cyclic. Let $g$ a generator of $\F_p^*$. Then the order of $g$ is $p-1$, and $8 \mid p-1$, thus $\zeta = g^{(p-1)/8}$ has order $8$ (and $i = \zeta^2$ has order $4$, therefore $i^2 = -1$).
\item[(b)] Since $i^2 = -1$, $(1+i)^2 = 2i = 2 \zeta^2$, thus $\left(\frac{1+i}{\zeta}\right)^2 = 2$. If 
$$\alpha = \frac{1+i}{\zeta} = \frac{1+\zeta^2}{\zeta} = \zeta + \zeta^{-1},$$ then $\alpha \in \F_p$, and $\alpha^2 = 2$.
\item[(c)] Consider the matrices (14.25):
$$g = \begin{pmatrix} 0 & 1\\1 & 0\end{pmatrix},\qquad  h = \begin{pmatrix} i & 0\\0 & 1 \end{pmatrix},\qquad   k= \begin{pmatrix} 1 & -1\\1 & 1 \end{pmatrix}.$$
In Exercise 10, we have proved that, if $p\equiv 1 \pmod 8$,
$$\langle [g],[h],[k]\rangle = \left \langle
\begin{bmatrix} 0 & 1\\1 & 0\end{bmatrix}, \begin{bmatrix} i & 0\\0 & 1 \end{bmatrix},  \begin{bmatrix} 1 & -1\\1 & 1 \end{bmatrix}
 \right \rangle = N(H) \simeq S_4,
$$
where $N(H) \subset \mathrm{PGL}(2,\F_p)$.
Note that 
$$\det(f) = -1 = i^2,\qquad \det(h) = i = \zeta^2,\qquad \det(k) = 2 = \alpha^2, $$
where $i,\zeta, \alpha \in \F_p^*$, so that
$$\frac{1}{i} g \in \mathrm{SL}(2,\F_p), \qquad \frac{1}{\zeta} h  \in \mathrm{SL}(2,\F_p), \qquad \frac{1}{\alpha} k  \in \mathrm{SL}(2,\F_p).$$
Thus $[g] = [\frac{1}{i} g], [h] = [ \frac{1}{\zeta} h], [k] = [ \frac{1}{\alpha} k]$ are in $\mathrm{PSL}(2,\F_p)$. Therefore
$$S_4 \simeq N(H) = \langle [g],[h],[k]\rangle \subset \mathrm{PSL}(2,\F_p).$$
$\mathrm{PSL}(2,\F_p)$ contains a copy of $S_4$ if $p\equiv 1 \pmod 8$.
\item[(d)] We know that $\zeta_8$ is such that 
$$\zeta_8 + \zeta_8^{-1} = \zeta_8 + \overline{\zeta_8} = 2 \cos(\pi/4) = \sqrt{2},$$
so that $(\zeta_8 + \zeta_8^{-1})^2 = 2$ (draw a regular octagon).

If $\zeta$ has order $8$ in $\F_p^*$, as $\zeta_8$ has order $8$ in $\C^*$, it is not surprising that $(\zeta + \zeta^{-1})^2 = 2$,

To give a direct proof in these two groups, note that, since the order of $\zeta$ is $8$,  
$(\zeta^4)^2 = 1$, and $\zeta^4 \ne 1$, thus $\zeta^4 = -1 = \zeta^{-4}$. Therefore $(\zeta^2 + \zeta^{-2})^2 = \zeta^4 + \zeta^{-4} + 2 = 0$, so that $\zeta^2 + \zeta^{-2} = 0$. Then 
$$(\zeta + \zeta^{-1})^2 = \zeta^2 + \zeta^{-2} + 2 = 2.$$
This equality is used to compute the quadratic character of $2$ modulo $p$, $\legendre{2}{p} = (-1)^{(p^2-1)/8}$.
\end{proof}

\paragraph{Ex. 14.4.20}{\it Assume that $g,h \in \mathrm{GL}(2,\F_p)$ satisfy $gh = -hg$ and $\det(g) = \det(h) = 1$, as in part (a) of Proposition 14.4.4. Also assume that $p>2$.
\be
\item[(a)] Prove that the subgroup $\langle g,h\rangle \subset \mathrm{GL}(2,\F_p)$ is isomorphic to the quaternion group $Q = \{\pm 1, \pm i, \pm j, \pm k\}$, where $i^2 = j^2 = k^2 = -1, ij = -ji = k$, and $-1 \in Z(Q)$.
\item[(b)] Prove that $(M_3)_0$ is the normalizer of $\langle g,h \rangle$ in $\mathrm{GL}(2,\F_p)$.
\ee

}
\begin{proof}
\item[(a)] The given conditions on $Q$ give the following Cayley's table of the quaternion group:
\begin{center}
$
\begin{array}{c|cccccccc|}
  \times &1  &-1 &i &-i &j &-j &k &-k   \\
  \hline
         1 &1  &-1 &i &-i &j &-j &k &-k    \\
   	-1 &-1  &1 &-i &i &-j &j &-k &k    \\
	i &i  &-i  &-1 &1 &k &-k &-j & j   \\
	-i &-i  &i &1 &-1 &-k &k &j &-j    \\
	j &j  &-j &-k &k &-1 &1 &i &-i    \\
	-j &-j  &-j &k&-k &1 &-1&-i &i    \\
	k &k  &-k &j &-j &-i &i &-1 &1    \\
	-k &-k  &k &-j &j &i &-i &1 &-1    \\
	\hline
\end{array}
$
\end{center}
Assume that $g,h \in \mathrm{GL}(2,\F_p)$ satisfy $gh = -hg$ and $\det(g) = \det(h) = 1$, as in part (a) of Proposition 14.4.4. ($p>2$). In the proof of this proposition, we have seen that
$$g^2 = -I_2,\qquad h^2 = -I_2.$$ 
Write $l = gh$. Then $l^2 = ghgh = -g^2h^2= -I_2$. Moreover $-I_2 \in \Z(\mathrm{GL}(2,\F_p))$. These properties give the table of $G = \{\pm I_2,\pm g, \pm h, \pm k\}$: 
\begin{center}
$
\begin{array}{c|cccccccc|}
  \times &I_2  &-I_2 & g &-g &h &-h &l &-l   \\
  \hline
         I_2 &I_2  &-I_2 &g &-g &h &-h &l &-l    \\
   	-I_2 &-I_2  &I_2 &-g &g&-h &h &-l &l    \\
	g&g  &-g  &-I_2 &I_2 &k &-k &-h & h   \\
	-g &-g  &g &I_2 &-I_2 &-k &k &h&-h    \\
	h &h  &-h &-l &l &-I_2 &I_2&g &-g    \\
	-h &-h  &-h &l&-l &I_2 &-I_2&-g &g    \\
	l &l  &-l &h &-h &-g &g &-I_2 &I_2    \\
	-l &-l  &l &-h &h&g &-g &I_2 &-I_2    \\
	\hline
\end{array}
$
\end{center}
Thus $G\ne \varnothing$ is stable for multiplication, and each element has an inverse. Therefore $G$ is a subgroup of $\mathrm{GL}(2,\F_p)$. Moreover the two preceding tables show that the map $\phi : Q \to G$ defined by $$\phi(1) = I_2, \phi(-1) = -I_2, \phi(i) = g, \phi(-i) = -g, \phi(j) = h, \phi(-j) = h, \phi(k) = l, \phi(-k) = -l,$$ is a group isomorphism.

Finally $\{g , h\}  \subset G$, and $G$ is a group, therefore $\langle g,h \rangle \subset G$. Conversely, since $-I_2 =g^2, l = gh$, every element of $G$ is generated by $g,h$. Therefore
$$\langle g,h \rangle = G = \{\pm I_2,\pm g, \pm h, \pm k\},$$
and 
$$\langle g,h \rangle \simeq Q.$$

Example with Sage:
\begin{verbatim}
F = GF(5)
A = matrix(F,2,[0,-1,1,0])
B = matrix(F,2,[2,0,0,-2])
gens = [A,B]
gens
\end{verbatim}
$$
\left[\left(\begin{array}{rr}
0 & 4 \\
1 & 0
\end{array}\right), \left(\begin{array}{rr}
2 & 0 \\
0 & 3
\end{array}\right)\right]
$$
\begin{verbatim}
G = MatrixGroup(gens)
G.list()
\end{verbatim}
$$
\left(\left(\begin{array}{rr}
0 & 1 \\
4 & 0
\end{array}\right), \left(\begin{array}{rr}
0 & 2 \\
2 & 0
\end{array}\right), \left(\begin{array}{rr}
0 & 4 \\
1 & 0
\end{array}\right), \left(\begin{array}{rr}
0 & 3 \\
3 & 0
\end{array}\right), \left(\begin{array}{rr}
1 & 0 \\
0 & 1
\end{array}\right), \left(\begin{array}{rr}
2 & 0 \\
0 & 3
\end{array}\right), \left(\begin{array}{rr}
4 & 0 \\
0 & 4
\end{array}\right), \left(\begin{array}{rr}
3 & 0 \\
0 & 2
\end{array}\right)\right)
$$
\begin{verbatim}
G.structure_description()
\end{verbatim}
\begin{center}
Q8
\end{center}

\item[(b)] Let 
$$\pi : 
\left \{
\begin{array}{ccc}
\mathrm{GL}(2,\F_p) & \to & \mathrm{PGL}(2,\F_p)\\
1 & \mapsto & [A].
\end{array}
\right.
$$
By definition, $(M_3)_0 = \pi^{-1}(N(H))$, where we identified $A \in \mathrm{GL}(2,\F_p)$ with $\gamma_{A,0} \in \mathrm{AGL}(2,\F_p)$.
Therefore, for all $A \in \mathrm{GL}(2,\F_p)$,
\begin{align*}
A \in (M_3)_0 &\iff [A] \in N(H) = N(\langle [g],[h] \rangle)\\
&\iff 
\left\{
\begin{array}{lll}
{[}A{]} {[}g{]} {[}A{]}^{-1} &\in& \langle {[}g{]},{[}h{]}\rangle,\\
{[}A{]}{[}h{]} {[}A{]}^{-1} &\in& \langle {[}g{]},{[}h{]} \rangle.
\end{array}
\right.
\end{align*}
Note that, if $[f] \in \langle {[}g{]},{[}h{]}\rangle = \{[I_2], [g], [h], [gh]\}$, where we can take $f \in \mathrm{SL}(2,\F_p)$ (since $I_2,g,h,gh \in  \mathrm{SL}(2,\F_p)$),
$$[A] [g] [A]^{-1} = [f] = \iff [AgA^{-1}] = [f] \iff AgA^{-1} = \lambda f,\ \lambda \in \F_p^*.$$
In this case, since $\det(g) = \det(h) = \det(f) = 1$,
$$1 = \det(AgA^{-1}) = \det(\lambda f) = \lambda^2 ,$$
thus $\lambda = \pm 1$, and $g = \pm f$. Therefore
\begin{align*}
[A][g][A]^{-1} \in \langle {[}g{]},{[}h{]}\rangle & \iff [A][g][A^{-1}] = [f], [f] \in \langle {[}g{]},{[}h{]}\rangle\\
&\iff AgA^{-1} = \pm f, \ f \in \langle g,h \rangle\\
&\iff AgA^{-1} \in  \langle g,h \rangle,
\end{align*}
We have the same result with $[A]h[A]^{-1}$, therefore, for all $A \in \mathrm{GL}(2,\F_p)$,
$$A \in (M_3)_0  \iff 
\left\{
\begin{array}{lll}
AgA^{-1} \in  \langle g,h \rangle\\ 
AhA^{-1} \in  \langle g,h \rangle\\ 
\end{array}
\right.
\iff A \in N_{\mathrm{GL}(2,\F_p)}(\langle g,h \rangle)
$$
This equivalence proves that 
$$(M_3)_0 = N_{\mathrm{GL}(2,\F_p)}(\langle g,h \rangle).$$
\end{proof}

\end{document}
