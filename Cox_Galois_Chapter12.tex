%&LaTeX
\documentclass[11pt,a4paper]{article}
\usepackage[frenchb,english]{babel}
\usepackage[applemac]{inputenc}
\usepackage[OT1]{fontenc}
\usepackage[]{graphicx}
\usepackage{amsmath}
\usepackage{amsfonts}
\usepackage{amsthm}
\usepackage{amssymb}
\usepackage{tikz}
%\input{8bitdefs}

% marges
\topmargin 10pt
\headsep 10pt
\headheight 10pt
\marginparwidth 30pt
\oddsidemargin 40pt
\evensidemargin 40pt
\footskip 30pt
\textheight 670pt
\textwidth 420pt

\def\imp{\Rightarrow}
\def\gcro{\mbox{[\hspace{-.15em}[}}% intervalles d'entiers 
\def\dcro{\mbox{]\hspace{-.15em}]}}

\newcommand{\be} {\begin{enumerate}}
\newcommand{\ee} {\end{enumerate}}
\newcommand{\deb}{\begin{eqnarray*}}
\newcommand{\fin}{\end{eqnarray*}}
\newcommand{\ssi} {si et seulement si }
\newcommand{\D}{\mathrm{d}}
\newcommand{\Q}{\mathbb{Q}}
\newcommand{\Z}{\mathbb{Z}}
\newcommand{\N}{\mathbb{N}}
\newcommand{\R}{\mathbb{R}}
\newcommand{\C}{\mathbb{C}}
\newcommand{\F}{\mathbb{F}}
\newcommand{\U}{\mathbb{U}}
\newcommand{\re}{\,\mathrm{Re}\,}
\newcommand{\im}{\,\mathrm{Im}\,}
\newcommand{\ord}{\mathrm{ord}}
\newcommand{\Gal}{\mathrm{Gal}}
\newcommand{\legendre}[2]{\genfrac{(}{)}{}{}{#1}{#2}}

\title{Solutions to David A.Cox  "Galois Theory''}
\author{Richard Ganaye}
\refstepcounter{section} \refstepcounter{section} \refstepcounter{section} \refstepcounter{section}
\refstepcounter{section}\refstepcounter{section}\refstepcounter{section}\refstepcounter{section}
\refstepcounter{section}\refstepcounter{section}\refstepcounter{section}

\begin{document}
\maketitle

\section{Chapter 12 : LAGRANGE, GALOIS, AND KRONECKER}

\subsection{LAGRANGE}
\paragraph{Ex. 12.1.1}

{\it Let $\theta(x)$ be the resolvent polynomial defined in (12.3). Use the second bullet following (12.1) to show that $\theta(x) \in K[x]$.
}

\begin{proof} Let $\sigma$ be any permutation of $S_n$. Since 
$$\theta(x) = \prod_{i=1}^r (x-\varphi_i),$$
then
\begin{align*}
\sigma \cdot \theta(x) &= \sigma \cdot  \prod_{i=1}^r (x-\varphi_i)\\
&= \prod_{i=1}^r \sigma \cdot (x-\varphi_i)\\
&= \prod_{i=1}^r   (x-\varphi_{\sigma(i)})\\
&= \prod_{j=1}^r   (x-\varphi_j) \qquad (j = \sigma(i))\\
&=\theta(x).
\end{align*}
By  Exercise 2.2.8, $\sigma \cdot \theta(x)  = \theta(x)$ implies that $\theta(x) \in K(x)$.
\end{proof}

\paragraph{Ex. 12.1.2}

{\it Work out the details of Example 12.1.2.
}

\begin{proof}
Let $F = \Q(\omega)$, $z_1 = \frac{1}{3}(x_1+\omega^2 x_2+\omega x_3) \in K=\Q(\omega)(x_1,x_2,x_3)$,  and $\theta(z) \in \Q(\omega)[z]$ be the resolvent polynomial of $z_1$.
The orbit of $z_1$ under the action of $S_n$ is composed of 
\begin{align*}
z_1& =\frac{1}{3}(x_1+\omega^2 x_2+\omega x_3),\\
  (2,3)\cdot z_1 &= \frac{1}{3}(x_1+\omega^2 x_3+\omega x_2) = \frac{1}{3}(x_1+\omega x_2+\omega^2 x_3)=z_2\\
  (1,3) \cdot z_1 &=\frac{1}{3}(x_3+\omega^2 x_2+\omega x_1) =  \frac{1}{3}(\omega x_1+\omega^2 x_2+ x_3) = \omega z_2\\
  (1,2) \cdot z_1 &=\frac{1}{3}(x_2+\omega^2 x_1+\omega x_3) =  \frac{1}{3}(\omega^2 x_1+ x_2+ \omega x_3) = \omega^2 z_2\\
 (1,2,3)\cdot z_1 &= \frac{1}{3}(x_2+\omega^2 x_3+\omega x_1) =  \frac{1}{3}(\omega x_1+ x_2+\omega^2 x_3) = \omega z_1\\
 (1,3,2)\cdot z_1 &= \frac{1}{3}(x_3+\omega^2 x_1+\omega x_2) =  \frac{1}{3}(\omega^2 x_1+ \omega x_2+ x_3) = \omega^2 z_1.\\
\end{align*}
So the orbit of $z_1$ is
$${\cal O}_{z_1} = \{z_1,z_2,\omega z_1,\omega z_2,\omega^2 z_1,\omega^2 z_2\},$$
and these six elements are distinct in $F(x_1,x_2,x_3)$.

Moreover,
\begin{align*}
\theta(z) &= (z-z_1)(z-z_2)(z-\omega z_1)(z-\omega z_2) (z- \omega^2 z_1)(z-\omega^2 z_2)\\
&=(z^3-z_1^3)(z^3 - z_2^3)\\
&=z^6 - (z_1^3+z_2^3)z^3 +(z_1z_2)^3
\end{align*}
and 
\begin{align*}
z_1z_2 &= \frac{1}{9}(x_1+\omega^2 x_2+\omega x_3)(x_1+\omega x_2+\omega^2 x_3)\\
&= \frac{1}{9}(x_1^2+x_2^2+x_3^2-x_1x_2-x_2x_3-x_1x_3)\\
&= \frac{1}{9}[(x_1+x_2+x_3)^2 - 3(x_1x_2+x_2x_3+x_1x_3)]\\
&=\frac{1}{9}(\sigma_1^2 - 3 \sigma_2),
\end{align*}
so 
$$z_1^3z_2^3 = \frac{1}{3^6}(\sigma_1^2-3\sigma_1)^3 = -\frac{1}{27}\left(-\frac{\sigma_1^2}{3} + \sigma_2 \right)^3 = -\frac{p^3}{27},\text{ where } p = -\frac{\sigma_1^2}{3} + \sigma_2.$$

\begin{align*}
z_1^3 + z_2^3 &= \frac{1}{27}\left[2 (x_{1}^{3} +  x_{2}^{3} +  x_{3}^{3}) - 3 (x_{1}^{2} x_{2} + x_{1} x_{2}^{2} +x_{1}^{2} x_{3} + x_{2}^{2} x_{3} + x_{1}x_{3}^{2} +x_{2} x_{3}^{2}) + 12 x_{1} x_{2} x_{3}\right]
\end{align*}
\begin{align*}
s &= x_{1}^{2} x_{2} + x_{1} x_{2}^{2} +x_{1}^{2} x_{3} + x_{2}^{2} x_{3} + x_{1}x_{3}^{2} +x_{2} x_{3}^{2}\\
&= (x_1x_2+x_2x_3+x_1x_3)(x_1+x_2+x_3)-3x_1x_2x_3\\
&= \sigma_2 \sigma_1 - 3 \sigma_3
\end{align*}

\begin{align*}
x_1^3+x_2^3+x_3^3 &= (x_1^2+x_2^2+x_3)^2(x_1+x_2+x_3) - (x_{1}^{2} x_{2} + x_{1} x_{2}^{2} +x_{1}^{2} x_{3} + x_{2}^{2} x_{3} + x_{1}x_{3}^{2} +x_{2} x_{3}^{2})\\
&=(\sigma_1^2-2\sigma_2)\sigma_1 - (\sigma_2 \sigma_1 - 3 \sigma_3)\\
&=\sigma_1^3 -3\sigma_1\sigma_2+3\sigma_3.
\end{align*}
Thus
\begin{align*}
z_1^3 + z_2^3 &= \frac{1}{27}\left[2(\sigma_1^3 -3\sigma_1\sigma_2+3\sigma_3) -3(\sigma_1 \sigma_2 - 3 \sigma_3) + 12 \sigma_3\right]\\
&= \frac{1}{27} (2 \sigma_1^3-9\sigma_1 \sigma_2 + 27 \sigma_3)\\
&=  \frac{2\sigma_1^3}{27} - \frac{\sigma_1\sigma_2}{3} + \sigma_3
\end{align*}
Finally,
$$\theta(z) = z^6 +q z^3 -\frac{p^3}{27},$$
where
$$p = -\frac{\sigma_1^2}{3} + \sigma_2,\quad q = -\frac{2\sigma_1^3}{27} + \frac{\sigma_1\sigma_2}{3}- \sigma_3.$$
\end{proof}

\paragraph{Ex. 12.1.3}

{\it This exercise concerns Examples 12.1.3 and 12.1.5.
\be
\item[(a)] Compute the resolvent $\theta(y)$ of Example 12.1.3. This can be done using the methods of Section 2.3.
\item[(b)] Let $y_1 = x_1x_2+x_3x_4$. Show that $H(y_1) = \langle (1\, 2),(1 \, 3\, 2\,4) \rangle \subset S_4$.
\item[(c)] Show that $H(y_1)$ is not normal in $S_4$.
\item[(d)] Show that $H(y_1)$ is isomorphic to $D_8$, the dihedral group of order 8.

\ee
}

\begin{proof}
\be
\item[(a)] $y_1 = x_1x_2+x_3x_4, y_2 = (2\,3)\cdot y_1= x_1 x_3 +  x_2 x_4, y_3 = (2\,4) \cdot y_1 = x_1x_4+x_2x_3$ are distinct elements of the orbit of $y_1$.

Since $|H(y_1)| = |\mathrm{Stab}_{S_4}(y_1)| = 8$ (see Part (b)), $|{\cal O}_{y_1} |= 3$, so $y_1,y_2,y_3$ are all the elements of ${\cal O}_{y_1}$.
$${\cal O}_{y_1} = \{y_1,y_2,y_3\} = \{x_1 x_3 +  x_2 x_4, x_1 x_3 +  x_2 x_4, x_1x_4+x_2x_3\}.$$
Therefore
\begin{align*}
\theta(y) &=\left((y - (x_1x_2+x_3x_4)\right)\left(y - (x_1 x_3 +  x_2 x_4)\right)\left(y - (x_1x_4+x_2x_3)\right)
\end{align*}
Using the methods of section 2.3, we obtain with the following Sage instructions
\begin{verbatim}
e = SymmetricFunctions(QQ).e()
e1, e2, e3 , e4 =
	 e([1]).expand(4),e([2]).expand(4),e([3]).expand(4), e([4]).expand(4)
R.<y,x0,x1,x2,x3,y1,y2,y3,y4> = PolynomialRing(QQ, order = 'degrevlex')
J = R.ideal(e1-y1, e2-y2, e3-y3,e4-y4)
G = J.groebner_basis()

z1 = x0*x1 + x2*x3
z2 = x0*x2 + x1*x3
z3 = x0*x3 + x1*x2
f = (y-(x0*x1 + x2*x3))*(y-(x0*x2 + x1*x3))*(y-(x0*x3 + x1*x2))

var('sigma_1,sigma_2,sigma_3,sigma_4')
g=f.reduce(G).subs(y1=sigma_1,y2=sigma_2,y3=sigma_3,y4=sigma_4)
g.collect(y)
\end{verbatim}
$$-\sigma_{1}^{2} \sigma_{4} - \sigma_{2} y^{2} + y^{3} - \sigma_{3}^{2} + 4 \, \sigma_{2} \sigma_{4} + {\left(\sigma_{1} \sigma_{3} - 4 \,\sigma_{4}\right)} y.
$$
So  
$$ \theta(y) = y^{3}  - \sigma_{2} y^{2} + {\left(\sigma_{1} \sigma_{3} - 4 \,\sigma_{4}\right)} y- \sigma_{3}^{2}  -\sigma_{1}^{2} \sigma_{4}+ 4 \, \sigma_{2} \sigma_{4}  .
$$
\item[(b)]
$$(1\, 2)\cdot  y_1 = x_2x_1+x_3x_4 = y_1,\qquad (1\,3\,2\,4)(y_1) = x_3x_4+x_2x_1 = y_1,$$
therefore
$$\langle (1\, 2), (1\,3\,2\,4)  \rangle \subset H(y_1).$$
Moreover
$$\langle (1\, 2), (1\,3\,2\,4) \rangle= \{(), (1\,2), (1\,3\,2\,4), (1\,3)(2\,4), (1\,2)(3\,4), (1\,4)(2\,3), (3\,4), (1\,4\,2\,3)\}.$$
We obtain this by hand, or with the Dimino's algorithm, or with the Sage instructions:
\begin{verbatim}
G = PermutationGroup([(1,2),(1,3,2,4)])
G.list()
\end{verbatim}
The orbit of $y_1$ contains three distinct elements $y_1,y_2,y_3$, so $|{\cal O}_{y_1} |\geq 3$. Since $|{\cal O}_{y_1} |\ = (S_n:H(y_1))$, $|H(y_1)|\leq 8$. But $H(y_1)$ contains the 8 elements of $\langle (1\, 2), (1\,3\,2\,4)  \rangle$, thus
$$H(y_1) = \langle (1\, 2), (1\,3\,2\,4) \rangle.$$

\item[(c)] $(2\,3) (1\,3\,2\,4)(2\,3)^{-1} = (1\,2\,3\,4) \not \in H(y_1)$, so $H(y_1)$ is not normal in $S_4$.

\item[(d)] If we number the 4 consecutive summits of the square in the order $(1,3,2,4)$, then $H(y_1)$ is isomorphic to the group generated by the rotation of angle $\pi/2$ corresponding to  $(1\,3\,2\,4)$ and the reflection relative to the diagonal $(3,4)$ corresponding to $(1\,2)$, and this is the dihedral group $D_8$. 
$$H(y_1) \simeq D_8.$$
\ee
\end{proof}

\paragraph{Ex. 12.1.4}

{\it Verify (12.9) and (12.10).
}

\begin{proof}
Starting from
$$x^4 - \sigma_1x^3 = -\sigma_2x^2+\sigma_3x-\sigma_4,$$
wee add the quantity
$$yx^2+\frac{1}{4}(-\sigma_1x+y)^2 = \left(y+\frac{\sigma_1^2}{4}\right) x^2 -\frac{\sigma_1}{2} yx + \frac{y^2}{4},$$
so
$$x^4 - \sigma_1x^3 + yx^2+\frac{1}{4}(-\sigma_1x+y)^2 = -\sigma_2x^2+\sigma_3x-\sigma_4 + \left(y+\frac{\sigma_1^2}{4}\right) x^2 -\frac{\sigma_1}{2} yx + \frac{y^2}{4},$$
Since 
\begin{align*}
x^4 - \sigma_1x^3 + yx^2+\frac{1}{4}(-\sigma_1x+y)^2  &= x^4 + (-\sigma_1x+y) x^2 +\frac{1}{4}(-\sigma_1x+y)^2 \\
&= \left(x^2 + \frac{1}{2}(-\sigma_1x+y)\right)^2\\
&=\left(x^2 - \frac{\sigma_1}{2} x + \frac{y}{2}\right)^2,
\end{align*}
we obtain 
$$\left(x^2 - \frac{\sigma_1}{2} x + \frac{y}{2}\right)^2 = \left(y+\frac{\sigma_1^2}{4}-\sigma_2\right) x^2+\left(-\frac{\sigma_1}{2} y + \sigma_3 \right) x +\frac{y^2}{4} - \sigma_4.$$
The discriminant of the right member $Ax^2+Bx+C$ is
$$\Delta = B^2-4AC= \left(-\frac{\sigma_1}{2} y + \sigma_3 \right)^2 - 4 \left(y+\frac{\sigma_1^2}{4}-\sigma_2\right)\left(\frac{y^2}{4} - \sigma_4\right).$$
\begin{align*}
4\Delta &= (-\sigma_1 y + 2\sigma_3)^2 - (4y +\sigma_1^2-4\sigma_2)(y^2 -4 \sigma_4)\\
&=(\sigma_1^2y^2-4\sigma_1\sigma_3 y + 4 \sigma_3^2) - (4y^3 - 16 \sigma_4 y + (\sigma_1^2 - 4 \sigma_2)y^2 - 4 \sigma_1^2 \sigma_4 + 16 \sigma_2 \sigma_4\\
&= -4 y^3 + 4 \sigma_2y^2+(-4\sigma_1\sigma_3 + 16 \sigma_4)y+ (4\sigma_3^2+4\sigma_4\sigma_1^2-16\sigma2 \sigma_4)\\
&=-4(y^3 - \sigma_2 y^2 + (\sigma_1\sigma_3 - 4 \sigma_4)y - \sigma_3^2-\sigma_1^2\sigma_4 + 4 \sigma_2\sigma_4).
\end{align*}

So the second member is a perfect square if and only if the Ferrari resolvent
$$R(y) = y^3 - \sigma_2 y^2 + (\sigma_1\sigma_3 - 4 \sigma_4)y - \sigma_3^2-\sigma_1^2\sigma_4 + 4 \sigma_2\sigma_4 $$
is zero for the chosen $y$.
\end{proof}

\paragraph{Ex. 12.1.5}

{\it This exercise will study the quadratic equations (12.11). Each quadratic has two roots, which together make up the four roots $x_1,x_2,x_2,x_4$ of our quadric.
\be
\item[(a)] For the moment, forget all the theory developed so far, and let $y$ be some root of the Ferrari resolvent (12.10). Given only this, can we determine how $y$ relates to the $x_i$? This is surprisingly easy to do. Suppose $x_i,x_j$. are the roots of (12.11) for one choice of sign, and $x_k,x_l$ are the roots for the other. Thus $i,j,k,l$ are the number 1,2,3,4 in some order. Prove that $y$ is given by $y=x_ix_j + x_kx_l$.

\item[(b)] Now let $y = x_1x_2 +x_3x_4$, and define the square root in (12.11) using (12.12). Show that the roots of (12.11) are $x_1,x_2$ for the plus sign and $x_3,x_4$ for the minus sign.
\ee
}

\begin{proof}
\be
If $y$ is some root of the Ferrari resolvent, then
$x_i,x _j$ are the roots of
$$x^2 - \frac{\sigma_1}{2} x + \frac{y}{2} = + \sqrt{y + \frac{\sigma_1^2}{4} - \sigma_2} \, \left(x + \frac{\frac{-\sigma_1}{2} y + \sigma_3 }{2(y+\frac{\sigma_1^2}{4} - \sigma_2)} \right).$$
The product $x_i x_j$ is given by 
$$x_ix_j = \frac{y}{2} - \sqrt{y + \frac{\sigma_1^2}{4} - \sigma_2} \left(\frac{\frac{-\sigma_1}{2} y + \sigma_3 }{2(y+\frac{\sigma_1^2}{4} - \sigma_2)} \right).$$
Similarly $x_k,x _l$ are the roots of
$$x^2 - \frac{\sigma_1}{2} x + \frac{y}{2} = - \sqrt{y + \frac{\sigma_1^2}{4} - \sigma_2} \, \left(x + \frac{\frac{-\sigma_1}{2} y + \sigma_3 }{2(y+\frac{\sigma_1^2}{4} - \sigma_2)} \right).$$
and the product $x_kx_l$ is given by 
$$x_kx_l = \frac{y}{2} + \sqrt{y + \frac{\sigma_1^2}{4} - \sigma_2} \left(\frac{\frac{-\sigma_1}{2} y + \sigma_3 }{2(y+\frac{\sigma_1^2}{4} - \sigma_2)} \right).$$
Adding these two formulas, we obtain
$$x_ix_j + x_kx_l = y.$$

\item[(b)]
Using $y_1 = x_1x_2+x_3x_4$, and setting 
$$t_1 = x_1+x_2-x_3-x_4,$$
then
\begin{align*}
&y_1 + \frac{\sigma_1^2}{4} - \sigma_2 \\
&= x_1x_2 + x_3 x_4 +\frac{1}{4}(x_1+x_2+x_3+x_4)^2 - (x_1x_2+x_1x_3+x_1x_4+x_2x_3+x_2x_4+x_3x_4)\\
&=\frac{1}{4}\left[x_1^2+x_2^2+x_3^2+x_4^2 -2(x_1x_2+x_1x_3+x_1x_4+x_2x_3+x_2x_4+x_3x_4) + 4(x_1x_2+x_3x_4)\right]\\
&=\frac{1}{4}\left[x_1^2+x_2^2+x_3^2+x_4^2 + 2x_1x_2+2x_3x_4 - 2(x_1x_3+x_1x_4+x_2x_3+x_2x_4)\right]\\
&=\frac{1}{4}\left[(x_1+x_2)^2+(x_3+x_4)^2  - 2(x_1+x_2)(x_3+x_4)\right]\\
&=\frac{1}{4} (x_1+x_2-x_3-x_4)^2\\
&= \frac{t_1^2}{4}
\end{align*}
We choose the square root such that
$$ \sqrt{y_1 + \frac{\sigma_1^2}{4} - \sigma_2 } = \frac{t_1}{2}.$$
Then the quadratic equation with $y=y_1$ and the plus sign is
$$x^2 - \frac{\sigma_1}{2} x + \frac{y_1}{2} = + \sqrt{y_1 + \frac{\sigma_1^2}{4} - \sigma_2} \, \left(x + \frac{\frac{-\sigma_1}{2} y_1 + \sigma_3 }{2(y_1+\frac{\sigma_1^2}{4} - \sigma_2)} \right),$$
or otherwise
$$x^2 - \left(\frac{\sigma_1}{2} + \frac{t_1}{2}\right) x + \frac{y_1}{2} +\frac{1}{2t_1} (\sigma_1 y_1 - 2 \sigma_3).$$

Let $u,v$ be the roots of this equation, and $S = u+v, P= uv$ be the sum and product of these roots. Then
\begin{align*}
S &= \frac{\sigma_1}{2}  + \frac{t_1}{2}\\
&= \frac{1}{2}(x_1+x_2+x_3+x_4 +x_1+x_2-x_3-x_4) \\
&= x_1+x_2
\end{align*}
\begin{align*}
P &= \frac{y_1}{2} +\frac{1}{2t_1} (\sigma_1 y_1 - 2 \sigma_3)\\
 &= \frac{y_1}{2} +\frac{1}{2t_1} \left[ (x_1+x_2+x_3+x_4)(x_1x_2+x_3x_4) - 2(x_1x_2x_3+x_1x_2x_4+x_1x_3x_4+x_2x_3x_4)\right]\\
 &=\frac{y_1}{2} + \frac{1}{2t_1} \left[ x_1^2x_2 + x_1x_2^2+x_3^2x_4 +x_3x_4^2 - x_1x_3x_4-x_2x_3x_4-x_1x_2x_3-x_1x_2x_4\right]\\
 &= \frac{y_1}{2} + \frac{1}{2t_1} (x_1+x_2-x_3-x_4)(x_1x_2-x_3x_4)\\
 &=\frac{1}{2} (x_1x_2+x_3x_4 +x_1x_2-x_3x_4)\\
 &= x_1x_2
\end{align*}
Thus $u,v$ are the roots of $x^2-Sx+P = (x-x_1)(x-x_2)$, so $\{u,v\} = \{ x_1,x_2\}$.

$x_1,x_2$ are the roots of (12.11) with the plus sign, so $x_3,x_4$ are the roots of (12.11) with the minus sign.
\ee
\end{proof}

\paragraph{Ex. 12.1.6}

{\it Explain why the polynomial $\theta(t)$ (12.13) has coefficients in $K = F(\sigma_1,\sigma_2,\sigma_3,\sigma_4)$.
}

\begin{proof}
$$\theta(t) = (t^2 - 4y_1-\sigma_1^2+4\sigma_2)(t^2 - 4y_2-\sigma_1^2+4\sigma_2)(t^2 - 4y_3-\sigma_1^2+4\sigma_2).$$
Recall that
\begin{align*}
y_1 &= x_1x_2+x_3x_4\\
y_2 &= x_1x_3+x_2x_4\\
y_3 &= x_1x_4 + x_2x_3
\end{align*}
Let $\tau = (1\, 2), \sigma = (1\,2\,3\,4)$. Then
$$\tau \cdot y_1 = x_2 x_1 + x_3x_4 =  y_1,\quad \tau \cdot y_2 = x_2x_3 + x_1x_4= y_3,\quad \tau \cdot y_3 = x_2x_4+x_1x_3= y_2,$$
and of course $\tau \cdot \sigma_1 = \sigma_1,\tau \cdot \sigma_2 = \sigma_2$.

Therefore $\tau\cdot \theta(t) = \theta(t)$.

Similarly,
$$\sigma\cdot y_1 = x_2x_3+x_4x_1 = y_3,\quad \sigma\cdot y_2=x_2x_4+x_3x_1 = y_2,\quad \sigma \cdot y_3 = x_2x_1+x_3x_4 = y_1.$$
Therefore $\sigma \cdot \theta(t) = \theta(t)$.

Since $S_n = \langle \sigma, \tau \rangle$, every permutation in $S_n$ lets the coefficients of $\theta(t)$ unchanged, therefore $\theta(t)$ has coefficients in $K = F(\sigma_1,\sigma_2,\sigma_3,\sigma_4)$ and $\theta(t) \in K[t]$.
\end{proof}

\paragraph{Ex. 12.1.7}

{\it Show that (12.15) implies the equations for $x_1,x_2,x_3,x_4$ given in the text.
}

\begin{proof}
We know that
\begin{align*}
\sigma_1 &= x_1+x_2+x_3+x_4,\\
t_1 &= x_1+x_2-x_3-x_4,\\
t_2 &= x_1-x_2+x_3-x_4,\\
t_3 &= x_1-x_2-x_3+x_4.
\end{align*}
The sum of these equations gives
$$\sigma_1 + t_1 + t_2 + t_3 = 4x_1,$$
so
$$x_1=\frac{1}{4} \left(\sigma_1 + t_1 + t_2 + t_3 \right).$$
We can compute similarly $\sigma_1+t_1-t_2-t_3$, ...

More conceptually, let $\sigma = (1\,2)(3\,4)$. Then 
$$\sigma \cdot x_1 = x_2,\quad  \sigma \cdot t_1 = t_1,\quad  \sigma \cdot t_2 = -t_2,\quad \sigma \cdot t_3 = -t_3.$$
Therefore 
$$x_2 = \frac{1}{4} \left(\sigma_1 + t_1 - t_2 - t_3 \right).$$
Similarly, if $\tau = (1\,3)(2\,4)$,
$$\sigma \cdot x_1 = x_3,\quad  \tau \cdot t_1 = -t_1,\quad  \tau \cdot t_2 = t_2,\quad \tau \cdot t_3 = -t_3.$$
Therefore 
$$x_3 = \frac{1}{4} \left(\sigma_1 - t_1 + t_2 - t_3 \right).$$
Finally, if $\zeta = (1\,4)(2\,3)$,
$$\zeta \cdot x_1 = x_4,\quad  \zeta \cdot t_1 = -t_1,\quad  \zeta \cdot t_2 = -t_2,\quad \zeta \cdot t_3 = t_3.$$
Therefore 
$$x_4 = \frac{1}{4} \left(\sigma_1 - t_1 - t_2 + t_3 \right).$$
In conclusion
\begin{align*}
x_1 &=\frac{1}{4} \left(\sigma_1 + t_1 + t_2 + t_3 \right),\\
x_2 &= \frac{1}{4} \left(\sigma_1 + t_1 - t_2 - t_3 \right),\\
x_3 &= \frac{1}{4} \left(\sigma_1 - t_1 + t_2 - t_3 \right),\\
x_4 &= \frac{1}{4} \left(\sigma_1 - t_1 - t_2 + t_3 \right).
\end{align*}
\end{proof}

\paragraph{Ex. 12.1.8}

{\it Let $t_1,t_2,t_3$ defined as in (12.15).
\be
\item[(a)]
Lagrange noted that any transposition fixes exactly one of $t_1,t_2,t_3$ and interchanges the other two, possibly changing the sign of both. Prove this and use it to show that $t_1t_2t_3$ is fixed by all elements of $S_4$.
\item[(b)]
Use the methods of Chapter 2 to express $t_1t_2t_3$ in terms of the $\sigma_i$. The result should be the identity (12.16).
\ee
}

\begin{proof}
\be
\item[(a)] By (12.15),
\begin{align*}
t_1 &= x_1+x_2-x_3-x_4,\\
t_2 &= x_1-x_2+x_3-x_4,\\
t_3 &= x_1-x_2-x_3+x_4.
\end{align*}

Since $H(t_1) = \langle (1\,2),(3\,4) \rangle$ has order 4, the orbit ${\cal O}_{t_1}$ of $t_1$ under $S_n$ has $4!/4 = 6$ elements, so $${\cal O}_{t_1} =  \{t_1,t_2,t_3,-t_1,-t_2,-t_3\}.$$

$$(1 \,2) \cdot t_1 = t_1, \quad (1 \,2) \cdot t_2 = -t_3, \quad (1 \,2) \cdot t_3 = -t_2,$$
therefore 
$$(1 \,2) \cdot (t_1t_2t_3) = t_1(-t_3)(-t_2) = t_1t_2t_3.$$
\begin{align*}
(1 \,2 \, 3 \, 4) \cdot t_1 &= x_2+x_3-x_4-x_1\\
&=-x_1+x_2+x_3-x_4\\
&=-(x_1-x_2-x_3+x_4)\\
&=-t_3
\end{align*}
With similar computations, we obtain
$$(1 \,2 \, 3 \, 4) \cdot t_1 = -t_3, \quad (1 \,2 \, 3 \, 4)\cdot t_2 = -t_2, \quad (1 \,2 \, 3 \, 4)\cdot t_3 = t_1,$$
thus
$$ (1 \,2 \, 3 \, 4) \cdot (t_1t_2t_3) = (-t_3)(-t_2)t_1 = t_1t_2t_3.$$
Since $(1 \,2) \cdot (t_1t_2t_3)  = t_1t_2t_3, (1 \,2 \, 3 \, 4) \cdot (t_1t_2t_3) = t_1t_2t_3$, and $S_4 = \langle (1 \,2), (1 \,2 \, 3 \, 4) \rangle$, then $t_1t_2t_3$ is fixed by all elements of $S_4$, and so is in $F(\sigma_1,\sigma_2,\sigma_3,\sigma_4)$.

\item[(b)] With the methods of Chapter 2, the following Sage instructions
\begin{verbatim}
e = SymmetricFunctions(QQ).e()
e1,e2,e3,e4 = e([1]).expand(4),e([2]).expand(4),
        e([3]).expand(4),e([4]).expand(4)
R.<x0,x1,x2,x3,y1,y2,y3,y4> = PolynomialRing(QQ, order = 'lex')
J = R.ideal(e1-y1,e2-y2,e3-y3,e4-y4)
G = J.groebner_basis()
t1= x0+x1-x2-x3; t2 = x0-x1+x2-x3; t3 = x0-x1-x2+x3
u = t1*t2*t3
var('sigma_1,sigma_2,sigma_3,sigma_4')
v = u.reduce(G).subs(y1=sigma_1, y2 = sigma_2,y3=sigma_3,y4=sigma_4)
\end{verbatim}
give
$$\sigma_{1}^{3} - 4 \, \sigma_{1} \sigma_{2} + 8 \, \sigma_{3}.
$$
So
\begin{align*}
t_1t_2t_3 &= (x_1+x_2-x_3-x_4)(x_1-x_2+x_3-x_4)( x_1-x_2-x_3+x_4)\\
& = \sigma_{1}^{3} - 4 \, \sigma_{1} \sigma_{2} + 8 \, \sigma_{3}.
\end{align*}
\ee
\end{proof}

\paragraph{Ex. 12.1.9}

{\it Let $H$ be a subgroup of $S_n$. In this exercise you will give two proofs that there is $\varphi \in L$ such that $H = H(\varphi)$.
\be
\item[(a)](First Proof.) The fixed field $L_H$ gives an extension $K \subset L_H$. Explain why the Theorem of the Primitive Element applies to give $\varphi \in L_H$ such that $L_H = K(\varphi)$. Show that this $\varphi$ has the desired property.

\item[(b)] (Second Proof.) Let $m = x_1^{a_1}\cdots x_n^{a_n}$ be a monomial in $x_1,\ldots,x_n$ with distinct exponents $a_1,\ldots,a_n$. Then define
$$\varphi = \sum_{\sigma \in H} \sigma \cdot m = \sum_{\sigma \in H} x_{\sigma(1)}^{a_1} \cdots x_{\sigma(n)}^{a_n}.$$
Prove that $H(\varphi) = H$.
\ee
}

\begin{proof}
\be
\item[(a)] Here $K = F(\sigma_1,\ldots,\sigma_n), L = F(x_1,\ldots,x_n)$, where $F$ has characteristic $0$.

We know (Theorem 6.4.1) that $K \subset L$ is a Galois extension, and that
$$
\psi : 
\left\{
\begin{array}{ccc}
  S_n& \to   &\Gal(L/K)   \\
  \tau & \mapsto    &  \tilde{\tau}
  	\left\{
		\begin{array}{ccc}
  		L& \to   &  L \\
  		f&\mapsto   &  \tau \cdot f 
		\end{array}
	\right.
\end{array}
\right.
$$
(where $\tau \cdot f(x_1,\ldots,x_n) = f(x_{\tau(1)},x_{\tau(2)},\ldots,x_{\tau(n)})$)

is an isomorphism from $S_n$ to $\Gal(L/K)$.

Write $\tilde{H} = \psi(H)$ the subgroup of $\Gal(L/K)$ corresponding to $H \subset S_n$, and $L_{\tilde{H}}$ its fixed field (we can write $L_H = L_{\tilde{H}}$).

 $K \subset L$ is a finite extension, and $K \subset L_H \subset L$, so $K \subset L_H$ is a finite extension. Since the characteristic of $F$ is $0$, the Theorem of the Primitive Element (Corollary 5.4.2 (b)) applies to give $\varphi \in L_H$ such that $L_H = K(\varphi)$.

Since $K \subset L$ is a Galois extension, the Galois correspondence (Theorem 7.3.1) gives
$$\tilde{H} = \Gal(L/L_{\tilde{H}}) = \Gal(L/K(\varphi)).$$
We show that $H = H(\varphi)$:
	\be
	\item[$\bullet$] If $\tau \in H$, then $\tilde{\tau} = \psi(\tau) \in \tilde{H} = \Gal(L/K(\varphi))$. Since $\varphi \in K(\varphi)$, $\tau \cdot \varphi = \tilde{\tau}(\varphi) =  \varphi$, so $ \tau \in H(\varphi)$.
	\item[$\bullet$] If $\tau \in H(\varphi)$, then $\tau \cdot \varphi = \varphi$. If $u(x_1,\ldots,x_n) \in K(\varphi)$, then $u(x_1,\ldots,x_n) = f(\varphi(x_1,\ldots,x_n))$, where $f \in K(x)$.  Therefore $$\tau \cdot u(x_1,\ldots,x_n) = f(\varphi(x_{\tau(1)},\ldots, x_{\tau(n)}) = f(\varphi(x_1,\ldots,x_n)) = u(x_1,\ldots,x_n),$$
	 so $\tilde{\tau}(u) = \tau \cdot u = u$ for all $u \in K(\varphi)$, thus $\tilde{\tau} \in  \Gal(L/K(\varphi)) = \tilde{H}$, and so $\tau \in H$.
	\ee
	
 Conclusion: if $H$ is a subgroup of $S_n$, there is $\varphi \in L$ such that $H = H(\varphi)$.
 
 \item[(b)] 
 Let $\varphi = \sum_{\sigma \in H} \sigma \cdot m $, where $m = x_1^{a_1}\cdots x_n^{a_n}$ with distinct exponents $a_1,\ldots,a_n$. 
  \be
 \item[$\bullet$] If $\tau \in H$, by (6.7),
 $$\tau \cdot \varphi = \sum_{\sigma \in H} (\tau \sigma)\cdot m =  \sum_{\sigma' \in H}  \sigma' \cdot m = \varphi \qquad (\sigma' = \tau \sigma).$$
 
 Therefore $\tau \in H(\varphi)$.
 
 \item[$\bullet$] If $\tau \in H(\varphi)$,  $\tau \cdot \varphi = \varphi$, where $\varphi = \sum_{\sigma \in H} \sigma \cdot m $, so
 $$\sum_{\sigma \in H} (\tau \sigma)\cdot m = \sum_{\chi \in H} \chi \cdot m,
 $$
 $$\sum_{\sigma \in H} x_{(\tau \sigma)(1)}^{a_1} \cdots x_{(\tau \sigma)(n)}^{a_n} =  \sum_{\chi \in H} x_{\chi(1)}^{a_1}\cdots x_{\chi(n)}^{a_n}.$$
 Moreover,
 $$\prod_{i=1}^n x_{\chi(i)}^{a_i} = \prod_{j=1}^n x_j ^{a_{\chi^{-1}(j)}}, \qquad (j = \chi(i)),$$
 so
 $$ \sum_{\sigma \in H} x_1 ^{a_{(\tau \sigma)^{-1}(1)}} \cdots x_n ^{a_{(\tau \sigma)^{-1}(n)}} = \sum_{\chi \in H} x_1 ^{a_{\chi^{-1}(1)}} \cdots x_n ^{a_{\chi^{-1}(n)}}$$
 Since the exponents $a_1,\ldots,a_n$ are distinct, the $k$ terms of $\sum_{\chi \in H} \chi \cdot m$, where $k = |H|$,  are distinct, so there exists exactly one term in the right member which is the same as the term $ x_1 ^{a_{\tau^{-1}(1)}} \cdots x_n ^{a_{\tau^{-1}(n)}}$ of the left member corresponding to $\sigma = e$, so there exists $\chi \in H$ such that
 $$x_1 ^{a_{\tau^{-1}(1)}} \cdots x_n ^{a_{\tau^{-1}(n)}} = x_1 ^{a_{\chi^{-1}(1)}} \cdots x_n ^{a_{\chi^{-1}(n)}}.$$
 This implies $a_{\tau^{-1}(i)} = a_{\chi^{-1}(i)}, \ 1\leq i \leq n$. Since the exponents are distinct, $a_k = a_l$ implies $k=l$, so we obtain $\tau^{-1}(i) = \chi^{-1}(i)$ for all $i$, therefore $\tau^{-1} = \chi^{-1}$ and  $\tau = \chi \in H$.
 
 We have proved $H = H(\varphi)$.
 \ee
\ee
\end{proof}

\paragraph{Ex. 12.1.10}

{\it Prove that the subset $N \subset S_n$ defined in the proof of Theorem 12.1.10 is a subgroup of $S_n$.
}

\begin{proof}
Let 
$$N = \{\sigma \in S_n \ | \ \sigma\cdot \varphi_i = \varphi_i \text{ for all } i=1,\ldots, r\}.$$
Then
$$N = \bigcap_{1\leq i \leq r} {\mathrm{Stab}_{S_n}(\varphi_i}) =  \bigcap_{1\leq i \leq r} H(\varphi_i)$$
is the intersection of $r$ subgroups of $S_n$, so is a subgroup of $S_n$.
\end{proof}

\paragraph{Ex. 12.1.11}

{\it Let $H$ be a proper subgroup of $A_n$ with $n \geq 5$. Prove that $[A_n:H] \geq n$.
}

\begin{proof}
As $H$ is a subgroup of $A_n$, by Exercise 9, there exists $\varphi \in A_n$ such that $H = H(\varphi)$. Let ${\cal O}_{\varphi}$ the orbit of $\varphi$ under the action of $A_n$:
$${\cal O}_{\varphi} = \{\sigma \cdot \varphi\ | \ \sigma \in H\} = \{\varphi_1=\varphi, \varphi_2,\ldots,\varphi_s\},$$
and let $G$ the subgroup of $A_n$ defined by
$$G = \{\sigma \in A_n \ | \ \forall i \in \gcro 1, s \dcro,\, \sigma \varphi_i = \varphi_i\} = \bigcap_{1\leq i \leq s}\mathrm{Stab}_{A_n}(\varphi_i).$$
Then $G \subset H(\varphi_1) = H$. We show that $G$ is normal in $A_n$. 

Let $\tau \in A_n$ and $\sigma \in G$. Fix $i$ between 1 and $s$. Then $\tau \cdot \varphi_i \in {\cal O}_{\varphi}$, so $\tau \cdot \varphi_i = \varphi_j$ for some $j \in \gcro 1, s \dcro$. Then
$$(\tau^{-1} \sigma \tau)\cdot \varphi_i = (\tau^{-1} \sigma)\cdot \varphi_j = \tau^{-1} \cdot (\sigma \cdot \varphi_j) = \tau^{-1} \cdot \varphi_j = \varphi_i,$$
so $\tau^{-1} \sigma \tau \in G$.
Since $A_n$ is a simple group for $n\geq 5$, $G =\{e\}$ or $G = A_n$. Since $G \subset H$ and $H \subset A_n,H\ne A_n$, then $G \ne A_n$, therefore $G = \{e\}$.

$H = H(\varphi) = \mathrm{Stab}_{A_n}(\varphi)$, therefore $s = |{\cal O}_{\varphi}| = (A_n : H)$.

If we suppose that $(A_n:H) < n$, then $s<n$. Then $s \leq n-1$, therefore $s! \leq (n-1)! < n!/2$. Since there are $n!/2$ permutations in $A_n$, and only $s$ permutations of $ \{\varphi_1, \varphi_2,\ldots,\varphi_s\}$ there exist two distinct permutations $\tau_1,\tau_2 \in A_n$ such that
$$\tau_1\cdot \varphi_i = \tau_2 \cdot \varphi _i \qquad \text{for all } i=1,\ldots,r.$$
So $e \ne \tau_2^{-1} \tau_1 \in N$, $N \ne \{e\}$: this is a contradiction. This proves $(A_n:H) \geq n$.
\end{proof}

\paragraph{Ex. 12.1.12}

{\it The discussion following Theorem 12.1.10 shows that if we are going to use Lagrange's strategy when $n\geq 5$, then we need to begin with $\varphi = \sqrt{\Delta}$, which has isotropy subgroup $A_n$. Suppose that $\psi \in L$ is our next choice, and let $\theta(x)$ be the resolvent of $\psi$. Since we regard $K(\sqrt{\Delta})$ as known, we may assume that $\psi \not \in K(\sqrt{\Delta})$. The idea is to factor $\theta(x)$ over $K(\sqrt{\Delta})$, say $\theta = R_1\cdots R_s$, where $R_i \in K(\sqrt{\Delta})[x]$ is irreducible. This is similar to how (12.13) factors the resolvent of $t_1$ over $K(y_1)$. Suppose that $\psi$ enables us to continue Lagrange's inductive strategy. This means that some factor of $\theta$, say $R_j$, has degree $<n$. Your goal is to prove that this implies the existence of a proper subgroup of $A_n$ of index $<n$.
\be
\item[(a)] Prove that $\deg(R_j) \geq 2$.
\item[(b)] Since $\theta$ splits completely over $L$, the same is true for $R_j$. Let $\psi_j \in L$ be a root of $R_j$ and consider the fields
$$K \subset K(\sqrt{\Delta}) \subset M = K(\sqrt{\Delta},\psi_j) \subset L.$$
Let $H_j \subset S_n$ be the subgroup corresponding to $\Gal(L/M) \subset \Gal(L/K)$ under (12.1). Prove that $H_j \subset A_n$ and that $[A_n:H_j]$ is the degree of $R_j$.
\item[(c)] Conclude that $\deg(R_j)<n$ implies that $H_j$ is a proper subgroup of $A_n$ of index $<n$. With more work, one can show that $\deg(R_i) = [A_n:A_n\cap H(\psi)]$ for all $i$ and that
$$s = \frac{2}{[H(\psi):A_n \cap H(\psi)]}.$$
It follows that $s = 1$ or $2$.
\ee
}

\begin{proof}
\be
\item[(a)]
Here $K = F(\sigma_1,\ldots,\sigma_n)$ and $L = F(x_1,\ldots,x_n)$.

The roots of the resolvent $\theta$ are all the distinct $\sigma \cdot \psi$, where $\sigma \in S_n$. If $\deg(R_j)=1$, then $R_j(x) = x - \sigma \cdot \psi$ for some $\sigma \in S_n$. Since $R_j \in K(\sqrt{\Delta})[x]$, then $\sigma \cdot \psi \in K(\sqrt{\Delta})$. If $\sigma \in A_n$ then $\sigma^{-1} \in A_n$ fixes $\sqrt{\Delta}$,  and so $\psi = \sigma^{-1} \cdot (\sigma \cdot \psi) \in K(\sqrt{\Delta})$, which contradicts our assumption, therefore $\sigma \in S_n \setminus A_n$ and $\sigma \cdot \sqrt{\Delta} = - \sqrt{\Delta}$.

As $\sigma \cdot \psi \in K(\sqrt{\Delta})$, $\sigma \cdot \psi = A + B \sqrt{\Delta}, \ A,B \in K = F(\sigma_1,\ldots,\sigma_n)$. Therefore $\psi = \sigma^{-1}\cdot (A+B \sqrt{\Delta}) = A - B \sqrt{\Delta} \in K(\sqrt{\Delta})$: this is a contradiction.

Thus $\deg(R_j) \geq 2$.

\item[(b)] Since $K \subset K(\sqrt{\Delta}) \subset M$, the Galois correspondence being order reversing,
$$\Gal(L/M) \subset \Gal(L/K(\sqrt{\Delta})) \subset \Gal(L/K).$$
The same inclusions are true for the corresponding subgroups of $S_n$:
$$H_j \subset A_n \subset S_n.$$
By the fundamental Theorem (Theorem 7.3.1), since $K \subset L$, a fortiori $K(\sqrt{\Delta}) \subset L$ are Galois extensions,  the index  $(A_n:H_j) = (\Gal(L/K(\sqrt{\Delta}) : \Gal(L/M))$ is equal to $[M : K(\sqrt{\Delta})] = [K(\sqrt{\Delta},\psi_j ): K(\sqrt{\Delta})]$. The minimal polynomial of $\psi_j$ over $K(\sqrt{\Delta})$ being $R_j$, $[K(\sqrt{\Delta},\psi_j):K(\sqrt{\Delta})] = \deg(R_j)$, so
$$(A_n:H_j) = \deg(R_j).$$

\item[(c)] If $H_j = A_n$, then by the Galois correspondence $K(\sqrt{\Delta},\psi_j) = K(\sqrt{\Delta})$, and then $\psi_j \in K(\sqrt{\Delta})$. But this implies that $R_j = x - \psi_j$ has degree 1, which is impossible by part (a). So $H_j$ is a proper subgroup of $A_n$.  If $\deg(R_j) < n$, then $A_j$ is a proper subgroup of $A_n$ such that $(A_n:H_j) <n$. By Theorem 12.1.10(b), this is impossible for all $n\geq 5$.
\ee
\end{proof}

\paragraph{Ex. 12.1.13}

{\it Let $\zeta$ be a primitive $n$th root of unity, and let ${\alpha = x_1+\zeta x_2+\cdots + \zeta^{n-1}x_n}$. Prove that $H(\alpha^n) = \langle (1\,2\ldots n)\rangle \subset S_n$.
}

\begin{proof}
$(1\,2\ldots n)\cdot \alpha =  x_2+\zeta x_3+\cdots + \zeta^{n-1}x_1 = \zeta^{-1} \alpha$, therefore $(1\,2\ldots n)\cdot \alpha^n = (\zeta^{-1} \alpha)^n = \alpha^n$, so 
$$ \langle (1\,2\ldots n)\rangle \subset H(\alpha^n).$$

Conversely, suppose that $\sigma \in H(\alpha^n)$.  Then $\sigma \cdot \alpha^n = \alpha^n$, so
$$(x_{\sigma(1)}+ \zeta x_{\sigma(2)} + \cdots +\zeta^{n-1} x_{\sigma(n)})^n = (x_1+\zeta x_2+\cdots + \zeta^{n-1}x_n)^n.$$
Therefore, there exists a $n$th root of unity $\xi$ such that
$$x_{\sigma(1)}+ \zeta x_{\sigma(2)} + \cdots + \zeta^{n-1} x_{\sigma(n)} = \xi (x_1+\zeta x_2+\cdots + \zeta^{n-1}x_n).$$

Then
\begin{align*}
\xi \sum_{i=1}^n \zeta^{i-1} x_{i} &= \sum_{j=1}^n \zeta^{j-1} x_{\sigma(j)}\\
&=\sum_{i=1}^n \zeta^{\sigma^{-1}(i)-1} x_i, \qquad (i = \sigma(j))
\end{align*}
Therefore, for all $i=1,\ldots,n$,
$$\xi\, \zeta^{i-1} = \zeta^{\sigma^{-1}(i)-1}$$
For $i=1$, we obtain $\xi =\zeta^{\sigma^{-1}(1)-1}$, so $\zeta^{\sigma^{-1}(1)-1+ i-1} = \zeta^{\sigma^{-1}(i)-1}$.

Since $\zeta$ is a primitive $n$th root of unity,
$$\sigma^{-1}(1)+ i-1 \equiv \sigma^{-1}(i) \pmod n \qquad (1\leq i \leq n).$$
If $k = \sigma^{-1}(1) -1$, then 
$$\sigma^{-1}(i) \equiv i+k \pmod n,$$
therefore $\sigma^{-1} = (1\,2\ldots n)^k, \sigma = (1\,2\ldots n)^{n-k}$ are in the subgroup  $\langle (1\,2\ldots n)\rangle$.

$$H(\alpha^n) = \langle (1\,2\ldots n)\rangle.$$
\end{proof}

\paragraph{Ex. 12.1.14}

{\it Let $\alpha_i$ be as in (12.18), with $\sigma = (1\, 2 \ldots n) \in S_n \simeq \Gal(L/K)$:
\begin{align*}
\alpha_i &= x_1+\zeta^{-i} \sigma \cdot x_1+\zeta^{-2i} \sigma_2 \cdot x_1 + \cdots + \zeta^{-i(n-1)} \sigma^{n-1} \cdot x_1\\
&=x_1+\zeta^{-i} x_2 +\zeta^{-2i} x_3 + \cdots + \zeta^{-i(n-1)}\cdot x_n\\
\end{align*}
The quotation given in the discussion following (12.18) can be paraphrased as saying that the roots of the resolvent of $\theta_i =\alpha_i^n$ come from the permutations of the $n-1$ roots $x_2,\ldots,x_n$ that ignore the root $x_1$. What does this mean?
\be
\item[(a)] Show that each left coset of $\langle (1\, 2 \ldots n) \rangle$ in $S_n$ can be written uniquely as $\sigma \langle (1\, 2 \ldots n) \rangle$, where $\sigma$ fixes 1.
\item[(b)] Explain how Lagrange's statement follows from part (a).
\ee
}

\begin{proof}
\be
\item[(a)]
Write $\rho =  (1\, 2 \ldots n)  \in S_n$ and $H = \langle \rho \rangle$. Let $\tau H$ any coset relative to $H$, with $\tau \in S_n$. We must prove that there exists a unique $\sigma \in \tau H$ such that $\sigma(1) = 1$
	\be
	
	\item[$\bullet$] Existence.
	Let $k = \tau^{-1}(1)$ and $\sigma = \tau \rho^{k-1}$. Then $\sigma  \in \tau H$, and 
	$$\sigma(1) = (\tau \rho^{k-1})(1) = \tau(k) = 1.$$
	\item[$\bullet$] Unicity. If $\sigma H = \sigma' H$, with $\sigma(1) = \sigma'(1)=1$, then $\sigma' \in \sigma H$, so $$\sigma' = \sigma \rho^l, \quad l\in \Z.$$
	Since $\sigma'(1) = 1$, we have $\sigma(\rho^l(1)) = 1 = \sigma(1)$ and $\sigma$ is one-to-one, so $\rho^l(1) = 1$, therefore $l\equiv 0 \pmod n$, so $\rho^l =  e$ and $\sigma = \sigma'$.
	\ee
\item[(b)] As $H = \langle \rho \rangle$ is the stabilizer of $\theta_i = \alpha_i^n$, the value of $\tau \cdot \theta_i$ are the all the same when $\tau$ is in $\sigma H$, where $\sigma$ is the unique representative of the coset $\tau H$ such that $\sigma(1)=1$. We obtain the elements of the orbit ${\cal O}_{\theta_i}$  under the action of $S_n$, by taking the value of $\sigma \cdot \theta_i$ with $\sigma(1) = 1$. 
$${\cal O}_{\theta_i} = \{ \sigma \cdot \theta_i\ | \ \sigma \in S_n,\  \sigma(1)=1\}.$$
Moreover these values are distinct. Indeed, if $\sigma \cdot \theta_i = \sigma' \cdot \theta_i$, where $\sigma(1) = \sigma'(1) = 1$, then $\sigma'^{-1} \sigma \in H$, so $\sigma H = \sigma' H$. By part (a) (unicity), we obtain $\sigma = \sigma'$. (Thus $|{\cal O}_{\theta_i}| = (n-1)!$ is the degree of the Lagrange resolvent.)

So the resolvent is the product $$R(x) = \prod_{\sigma\in S_n,\ \sigma(1) = 1} (x - \sigma\cdot \alpha_i^n).$$

As Lagrange says,  the roots of the resolvent of $\theta_i =\alpha_i^n$ come from the permutations of the $n-1$ roots $x_2,\ldots,x_n$ that ignore the root $x_1$.
\ee
\end{proof}

\paragraph{Ex. 12.1.15}

{\it Given the Lagrange resolvent $\alpha_1,\ldots,\alpha_{p-1}$ defined in (12.19),
$$ \alpha_i = x_1+\zeta_p^ix_2+\zeta_p^{2i} x_3 + \cdots+\zeta_p^{(p-1) i} x_p,$$ 
the goal of this exercise is to prove that
$$x_i = \frac{1}{p}\left( \sigma_1 + \sum_{j=1}^{p-1} \zeta_p^{-j(i-1)} \alpha_j \right).$$
\be
\item[(a)] Write $\alpha_j = \sum_{l=1}^p \zeta_p^{j(l-1)}x_l$ for $1\leq j \leq p$, so that $
\alpha_p = \sigma_1$. Then show that
$$\sum_{j=1}^p \zeta_p^{-j(i-1)}\alpha_j = \sum_{j,l = 1}^p (\zeta_p^{l-i})^j x_l.$$
\item[(b)] Given an integer $m$, use Exercise 9 of section A.2 to prove that
$$
\sum_{j=1}^p (\zeta_p^m)^j =
\left\{
\begin{array}{ll}
 p, &  \text{if }m\equiv 0 \mod p,    \\
 0, &  \text{otherwise}.   
\end{array}
\right.
$$
\ee
}

\begin{proof}
\be
\item[(a)] By definition,
$$\alpha_j = \sum_{l=1}^p \zeta_p^{j(l-1)}x_l, \qquad 1\leq j \leq p.$$
Therefore
\begin{align*}
\sum_{j=1}^p \zeta_p^{-j(i-1)}\alpha_j &= \sum_{j=1}^p \zeta_p^{-j(i-1)} \sum_{l=1}^p \zeta_p^{j(l-1)}x_l\\
&= \sum_{l=1}^p \left [\sum_{j=1}^p( \zeta_p^{l-i})^j  \right ] x_l
\end{align*}

\item[(b)]
  \be
  \item[$\bullet$] If $m\equiv 0 \mod p$, then $\zeta_p^m = 1$, so $\sum\limits_{j=1}^p (\zeta_p^m)^j = p$.
  \item[$\bullet$] If $m\not \equiv 0 \mod p$, then $\zeta_p^m  \ne 1$, so
  $$\sum_{j=1}^p (\zeta_p^m)^j  =\zeta_p^m(1+\zeta_p^m+ \zeta_p^{2m}+ \cdots+ \zeta_p^{(p-1)m}) = \zeta_p^m \, \frac{1- (\zeta_p^{m})^p}{1 - \zeta_p^m} = 0.
  $$
  Thus,
  $$
\sum_{j=1}^p (\zeta_p^m)^j =
\left\{
\begin{array}{ll}
 p, &  \text{if }m\equiv 0 \mod p,    \\
 0, &  \text{otherwise}.   
\end{array}
\right.
$$
  \ee
  
\item[(c)] With $m=l-i$, part (b) gives
$$ \sum_{j=1}^p( \zeta_p^{l-i})^j =
\left\{
\begin{array}{ll}
 p, &  \text{if } l\equiv i \mod p,    \\
 0, &  \text{otherwise}.   
\end{array}
\right.
$$
Therefore, by part (a),
\begin{align*}
\sum_{j=1}^p \zeta_p^{-j(i-1)}\alpha_j &= \sum_{l=1}^p \left [\sum_{j=1}^p( \zeta_p^{l-i})^j  \right ] x_l\\
&= p x_i.
\end{align*}
For all $i=1,2,\ldots,p$,
\begin{align*}
x_i &= \frac{1}{p} \sum_{j=1}^p \zeta_p^{-j(i-1)}\alpha_j \\
&=\frac{1}{p} \left (\alpha_p +  \sum_{j=1}^p \zeta_p^{-j(i-1)}\alpha_j \right)\\
\end{align*}
Since $ \alpha_p =  \sum_{l=1}^p \zeta_p^{p(l-1)}x_l = x_1+\cdots+x_p = \sigma_1$, we obtain
$$x_i = \frac{1}{p}\left( \sigma_1 + \sum_{j=1}^{p-1} \zeta_p^{-j(i-1)} \alpha_j \right).$$
\ee
\end{proof}

\paragraph{Ex. 12.1.16}

{\it Prove that Theorem 7.4.4 follows from Theorem 12.1.6 and Proposition 2.4.1.
}

\begin{proof}
\be
\item[$\bullet$] Suppose that $\psi  \in F(x_1,\ldots,x_n)$ is invariant under $S_n$.

Let $\varphi = 1$. Then $\varphi$ is invariant under $S_n$, so $\psi$ is fixed by every permutation fixing $\varphi$. By Theorem 12.1.6. $\psi$ is a rational function of $\varphi$ with coefficients in $K = F(\sigma_1,\ldots,\sigma_n)$, i.e., $\psi \in K(\varphi) = K(1) = K$. So $\psi \in F(\sigma_1,\ldots,\sigma_n)$.

\item[$\bullet$] Suppose that $\psi  \in F(x_1,\ldots,x_n)$ is invariant under $A_n$.
Let $\varphi = \sqrt{\Delta}$. As the characteristic is not 2, by Proposition 2.4.1, $\sigma \cdot \sqrt{\Delta} = \sqrt{\Delta}$ if and only if $\sigma \in A_n$, so $H(\varphi) = H(\sqrt{\Delta})= A_n$.
Thus  $\psi$ is fixed by every permutation fixing $\varphi$.

By Theorem 12.1.6. $\psi$ is a rational function of $\varphi =\sqrt{\Delta}$ with coefficients in $K = F(\sigma_1,\ldots,\sigma_n)$, so $\psi \in K(\sqrt{\Delta})$.

$\sqrt{\Delta} \not \in K$, because $\tau \cdot \sqrt{\Delta} = - \sqrt{\Delta}\ne  \sqrt{\Delta}$ for every transposition $\tau$. Therefore $K \subset K(\sqrt{\Delta})$ is a quadratic extension, and $(1,\sqrt{\Delta} )$ is a basis of $K(\sqrt{\Delta})$ over $K$. Therefore
$$\psi = A+B \sqrt{\Delta},\qquad A,B \in K = F(\sigma_1,\ldots,\sigma_n).$$
So Theorem 7.4.4 follows from Theorem 12.1.6.
\ee
\end{proof}

\paragraph{Ex. 12.1.17}

{\it In Theorem 12.1.9, we used Galois correspondence to show that rational functions $\varphi$ and $\psi$ are similar if and only if $K(\varphi) = K(\psi)$. Give another proof of this result that uses only Theorem 12.1.6.
}

\begin{proof}
If $\varphi,\psi \in F(x_1,\ldots,x_n)$ are similar, then $H(\varphi) = H(\psi)$. So $\sigma \cdot \psi = \psi$ for every $\sigma \in H(\varphi)$. By Theorem 12.1.6, $\psi \in K(\varphi)$. Exchanging $\varphi$ and $\psi$, we obtain similarly $\varphi \in K(\psi)$. Therefore
$$K(\varphi) = K(\varphi,\psi) = K(\psi,\varphi) = K(\psi).$$

Conversely, if $K(\varphi) = K(\psi)$, then $\psi \in K(\varphi)$, so $\psi(x_1,\ldots,x_n)  = f(\varphi(x_1,\ldots,x_n))$, where $f \in K(x)$. Therefore, for all $\sigma \in H(\varphi)$,
$$\sigma \cdot \psi = f(\varphi(x_{\sigma(1)},\ldots,x_{\sigma(n)})) = f(\varphi(x_1,\ldots,x_n)) = \psi.$$
So $H(\varphi) \subset H(\psi)$, and similarly $H(\psi) \subset H(\varphi)$, thus $H(\varphi) = H(\psi)$.
\end{proof}

\paragraph{Ex. 12.1.18}

{\it Consider the quartic polynomial $f = x^4+2x^2-4x+2\in \Q[x]$.
\be
\item[(a)] Show that the Ferrari resolvent of (12.10) is $y^3-2y^2-8y$.

\item[(b)] Using the root $y_1 = 0$ of the cubic of part (a), show that (12.11) becomes
$$x^2 = \pm \sqrt{-2}(x-1)$$
and conclude that the four roots of $f$ are
$$\frac{\sqrt{2}}{2} i \pm \frac{1}{2} \sqrt{-2-4i\sqrt{2}} \text{ and }\frac{\sqrt{2}}{2} i \pm \frac{1}{2} \sqrt{-2+4i\sqrt{2}} .$$

\item[(c)] Use Euler's solution (12.17) to find the roots of $f$. The formulas are surprisingly different. We will see in Chapter 13 that this quartic is especially simple. For most quartics, the formulas for the roots are much more complicated.
\ee
}

\begin{proof}
\be
\item[(a)] The Ferrari resolvent $\theta(y)$ is given by Exercise 4:
$$\theta(y) = y^{3} - \sigma_{2} y^{2}+  {\left(\sigma_{1} \sigma_{3} - 4 \, \sigma_{4}\right)} y-\sigma_{1}^{2} \sigma_{4}   - \sigma_{3}^{2} + 4 \, \sigma_{2} \sigma_{4} .$$
As $f = x^4+2x^2-4x+2\in \Q[x]$, $\sigma_1 = 0, \sigma_2=2,\sigma_3 = 4, \sigma_4 = 2$, so
$$\theta(y) = y^3 - 2y^2-8y.$$

\item[(b)] We use the root $y_1=0$ of the Ferrari resolvent in (12.11)
$$x^2 - \frac{\sigma_1}{2} x + \frac{y_1}{2} = \pm \sqrt{y_1 + \frac{\sigma_1^2}{4} - \sigma_2} \, \left(x + \frac{\frac{-\sigma_1}{2} y_1 + \sigma_3 }{2(y_1+\frac{\sigma_1^2}{4} - \sigma_2)} \right),$$
Here $\sigma_1 = 0, \sigma_2=2,\sigma_3 = 4, \sigma_4 = 2$, therefore $y_1 + \frac{\sigma_1^2}{4} - \sigma_2 = -2$, so the roots of $f$ are the solutions of
$$x^2 = \pm \sqrt{-2} ( x -1), $$

(More directly, the equation is $$x^4 = -2x^2+4x-2 = -2(x^2-2x+1) = -2(x-1)^2 = [\sqrt{-2}(x-1)]^2,$$
so
$$x^2 = \pm \sqrt{-2} ( x -1).) $$
The roots of $f$ are the roots of
$$x^2 -i\sqrt{2} \, x+ i\sqrt{2} \qquad\text{or} \qquad  x^2 + i\sqrt{2} \,x - i\sqrt{2}.$$
\begin{align*}
x^2 -i\sqrt{2} \, x+ i\sqrt{2}&= \left(x - i\frac{\sqrt{2}}{2} \right)^2 + \frac{1}{2} + i \sqrt{2}\\
&=\left(x - i\frac{\sqrt{2}}{2} \right)^2 -\frac{1}{4}\left(-2 -4i\sqrt{2}\right)\\
&=\left(x - i\frac{\sqrt{2}}{2} \right)^2 -\left( \frac{1}{2} \sqrt{-2 - 4i\sqrt{2}} \right)^2\\
&= \left(x - i\frac{\sqrt{2}}{2}  - \frac{1}{2} \sqrt{-2 - 4i\sqrt{2}} \right)  \left(x - i\frac{\sqrt{2}}{2}  + \frac{1}{2} \sqrt{-2 - 4i\sqrt{2}} \right),
\end{align*}
and similarly
\begin{align*}
x^2 + i\sqrt{2} \,x - i\sqrt{2} &= \left(x + i\frac{\sqrt{2}}{2}  - \frac{1}{2} \sqrt{-2 +4i\sqrt{2}} \right)  \left(x + i\frac{\sqrt{2}}{2}  + \frac{1}{2} \sqrt{-2 + 4i\sqrt{2}} \right).
\end{align*}
so the roots of $f$ are
$$
 i\frac{\sqrt{2}}{2}  + \frac{1}{2} \sqrt{-2 - 4i\sqrt{2}},
  i\frac{\sqrt{2}}{2}  -  \frac{1}{2} \sqrt{-2 - 4i\sqrt{2}}, 
   -i\frac{\sqrt{2}}{2}  + \frac{1}{2} \sqrt{-2 + 4i\sqrt{2}},
 -i\frac{\sqrt{2}}{2}  - \frac{1}{2} \sqrt{-2 + 4i\sqrt{2}}
$$
Moreover 
\begin{align*}
(a+ib)^2 = -2 - 4i \sqrt{2} &\iff a^2+b^2 = |-2-4 i \sqrt{2} | = 6, \ a^2 - b^2 = -2, \ ab<0\\
&\iff a+ib = \pm(\sqrt{2} - 2i)
\end{align*}
so $$ \sqrt{-2 - 4i\sqrt{2}} = \pm (\sqrt{2} - 2i), \qquad  \sqrt{-2 + 4i\sqrt{2}} = \pm(\sqrt{2} +  2i).$$
The roots of $f$ are $x_1, x_2,x_3 = \overline{x_1}, x_4 = \overline{x_2}$, where
\begin{align*}
x_1 &= \frac{\sqrt{2}}{2} + i \left(\frac{\sqrt{2}}{2} - 1 \right),\\
x_2 &= -\frac{\sqrt{2}}{2} +i \left(-\frac{\sqrt{2}}{2} - 1 \right).
\end{align*}
Note: $x_1,x_2,x_3,x_4 \in \mathbb{Q}(i,\sqrt{2})$, so $\Q(x_1,x_2,x_3,x_4) \subset \mathbb{Q}(i,\sqrt{2})$. 

$\sqrt{2} = x_1 + \overline{x_1} = x_1 + x_3 \in \Q(x_1,x_2,x_3,x_4)$ and $i = -\frac{1}{2}(x_1+x_2) \in \mathbb{Q}(x_1,x_2,x_3,x_4)$. Therefore the splitting field of $f$ over $\Q$ is $L = \Q(i,\sqrt{2})$.

The Galois group is $\Gal(L/\mathbb{Q}) = \langle \sigma, \tau \rangle$, where $\sigma(\sqrt{2}) = -\sqrt{2}), \sigma(i) = i$, and $\tau$ is the complex conjugation. As permutation group,
$\Gal_{\Q}(f) = \langle (1 \, 2)(3 \,4), (1\, 3)(2\,4) \rangle \simeq \Z/2\Z \times \Z/2\Z$ has order 4.

\item[(c)] The Euler's solution gives the roots
$$\alpha = \frac{1}{4} \left(\sigma_1 + \varepsilon_1\sqrt{4y_1+\sigma_1^2-4\sigma_2}+ \varepsilon_2\sqrt{4y_2+\sigma_1^2-4\sigma_2}+ \varepsilon_3\sqrt{4y_3+\sigma_1^2-4\sigma_2}\right),$$
where $\sigma_1 = 0, \sigma_2=2$ and $y_1=0,y_2,y_3$ are the roots of 
$$y^3 - 2y^2-8y = y(y^2-2y-8) = y(y-4)(y+2), $$
so $y_1 = 0,y_2=4,y_3 = -2$.



Therefore
\begin{align*}
\alpha &= \frac{1}{4} (\varepsilon_1 \sqrt{-8} + \varepsilon_2 \sqrt{8} + \varepsilon_3\sqrt{-16})\\
&=\varepsilon_1 i \frac{\sqrt{2}}{2}  +\varepsilon_2  \frac{\sqrt{2}}{2} + \varepsilon_3 i
\end{align*}
Morever $\varepsilon_i = \pm 1$ satisfy 
$$t_1t_2t_3 = \varepsilon_1\varepsilon_2\varepsilon_3 (i \sqrt{8}) (\sqrt{8})4 i= \sigma_1^3-4\sigma_1\sigma_2+8\sigma_3 = 8\sigma_3= 32,$$
 so $\varepsilon_3 = -\varepsilon_1\varepsilon_2$.
 We obtain the four roots
 $$
\begin{array}{ll}
x_1 = \frac{\sqrt{2}}{2} + i \left(\frac{\sqrt{2}}{2} - 1 \right), &    x_3 = \overline{x_1} = \frac{\sqrt{2}}{2} - i \left(\frac{\sqrt{2}}{2} - 1 \right),  \\
  x_2 = -\frac{\sqrt{2}}{2} +i \left(-\frac{\sqrt{2}}{2} - 1 \right),&      x_4 =\overline{x_2}= -\frac{\sqrt{2}}{2} -i \left(-\frac{\sqrt{2}}{2} - 1 \right)  \\ 
\end{array}
$$
The formulas are NOT surprisingly different.
\ee
\end{proof}

\paragraph{Ex. 12.1.19}

{\it This exercise will prove a version of Theorem 12.1.10 for a subgroup $H$ of an arbitrary finite group $G$. When $G=S_n$, Theorem 12.1.10 used the action of $S_n$ on $L$ and wrote $H = H(\varphi)$ for some $\varphi \in L$. In general , we us the action of $G$ on the left cosets of $H$ defined by $g\cdot hH = ghH$ for $g,h \in G$.
\be
\item[(a)] Prove that $g\cdot hH = gh H$ is well defined, i.e., $hH = h'H$ implies that $ghH = gh'H$.
\item[(b)] Prove that $H$ is the isotropy subgroup of the identity coset $eH$.
\item[(c)] Let $m=[G:H]$, so that left cosets of $H$ can be labeled $g_1H,\ldots,g_mH$. Then, for $g\in G$, let $\sigma \in S_m$ be the permutation such that $g\cdot g_iH = g_{\sigma(i)}H$. Prove that the map $g\mapsto \sigma$ defines a group homomorphism $G \to S_m$.
\item[(d)] Let $N$ the kernel of the map of part (c). Thus $N$ is a normal subgroup of $G$. Prove that $N \subset H$.
\item[(e)] Prove that $[G:N]$ divides $m!$.
\item[(f)] Explain why you have proved the following result: If $H$ is a subgroup of a finite group $G$, then $H$ contains a normal subgroup of $G$ whose index divides $[G:H]!$.
\item[(g)] Use part (f) and Proosition 8.4.6 to give a quick proof of Theorem 12.1.10.
\ee
}

\begin{proof}
\be
\item[(a)] 
If $hH = h'H$, then $ghH = gh'H$. Indeed, if $u \in ghH$, then $u=ghx$, where $x \in H$. Since $hH = h'H$, then $hx \in hH$ implies $hx \in h'H$, so $hx = h'x'$ for some $x' \in H$. So $u = ghx = gh'x', x' \in H$, therefore $u \in gh'x'$, so $ghH \subset gh'H$, and similarly $gh'H \subset ghH$, so $ghH = gh'H$, and $g\cdot hH = gh H$ is well defined.

Moreover $e\cdot hH = ehH = hH$ and $g\cdot (g' \cdot H) = g \cdot g'H = gg'H = (gg')\cdot H$, so $g\cdot hH = gh H$ defines a left action of $G$ on the set of left cosets.

\item[(b)] Let $u$ any element of $G$. 
$$u \in \mathrm{Stab}_G(eH) \iff u \cdot eH = eH \iff ueH =e H \iff uH = H \iff u \in H.$$
The last equivalence is true, because $uH = H$ implies $u = ue\in H$, and conversely, if $u \in H$, $uH \subset H$ and every element $x \in H$ satisfies 
$x = u(u^{-1}x)$, where $u^{-1} x \in H$, so $x \in uH$.
$$\mathrm{Stab}_G(eH) = H.$$

\item[(c)] 
Let 
$$
\psi 
\left\{
\begin{array}{ccl}
  G& \to  &  S_m \\
  g& \mapsto  &   \sigma : \quad \forall i \in \gcro 1,m\dcro, \ g \cdot g_i H = g_{\sigma(i)} H  
\end{array}
\right.
$$
Let $g, g' \in G$, $\sigma = \psi(g), \sigma' = \psi(g')$. For all $i, 1\leq i \leq m$,
$$(gg')\cdot g_iH = g\cdot (g'\cdot g_iH) = g\cdot g_{\sigma'(i)}H = g_{\sigma(\sigma'(i))}H =g_ {(\sigma \circ \sigma')(i)} H.$$ 
Therefore $\psi(gg') = \sigma \circ \sigma'$, so $\psi:G \to S_m$ is a group homomorphism.

\item[(d)] Let $N$ be the kernel of $\psi$. For every $g \in G$,
\begin{align*}
g \in N  &\iff \forall i \in \gcro 1, m \dcro,\ g \cdot g_i H = g_iH \\
&\iff \forall h \in G, \ gh H = h H \\
&\iff \forall h \in G, \ h^{-1}gh H =  H\\
&\iff  \forall h \in G, \ h^{-1}gh \in H\\
&\iff \forall h \in G,\ g \in hHh^{-1}\\
& \iff g \in \bigcap_{h\in G} hHh^{-1}
\end{align*}
so
$$N = \bigcap_{h\in G} hHh^{-1}.$$
\ee
($N$ is the {\it core} of $H$ in $G$. We write $N = \mathrm{Core}_G(H)$.)

Since $H = eHe^{-1} \supset \bigcap_{h\in G} hHh^{-1}$, $H\supset N$.

\item[(e)] The first isomorphism theorem for groups gives the isomorphism $$G/N = G/\ker(\psi) \simeq \mathrm{Im}(\psi),$$ so $[G:N] = |\mathrm{Im}(\psi)|$ divides $|S_m| = m!$ by Lagrange's theorem.
$$[G:N] \mid m!.$$

\item[(f)] We can conclude that for any subgroup $H$ of a finite group $G$, then $H$ contains the core $N$ of $H$ in $G$, which is a normal subgroup of $G$ whose index divides $[G:H]!$.

\item[(g)]
   \be \item[$\bullet$] Let $H \subset S_n$ be a subgroup of index $[S_n:H]>1$, where $n\geq 5$.
   
   Let $N = \mathrm{Core}_{S_n}(H)$. Then $N\subset H \subset S_n$, and $N$ is normal in $S_n$, and $N\ne S_n$ (since $[S_n:H]>1$). By Proposition 8.4.6, $N = A_n$ or $N=\{e\}$.
   
   If $N = A_n$, then $N=A_n \subset H \subset S_n$, thus $1< [S_n:H] \leq [S_n:A_n] = 2$, therefore$ [S_n:H] = 2 = [S_n:A_n]$, where $A_n \subset H$, so $H = A_n$.
   
   In the other case, $N = \{e\}$. By part (e), $[S_n : N] \mid [S_n:H]!$, thus $n! \mid m!$, where $m = [S_n:H]$. So $n\leq m =[S_n:H]$. This proves part (a) of Theorem 12.1.10.
   
   \item[$\bullet$] Let $H \subset A_n$ be a subgroup of index $[A_n:H]>1$.
   
   Let $N = \mathrm{Core}_{A_n}(H)$. Then $N \subset H \subset A_n$ and $N$ is normal in $A_n$. Since $A_n$ is simple for $n\geq 5$, and $N \subset H \ne A_n$, $N =\{e\}$.
   
   By part (e), $[A_n:N] \mid [A_n:H] !$, so $n!/2 \mid m!$, where $m = [A_n:H]$.
   
   If $m<n$ then $m \leq n-1, m! \leq (n-1) < n!/2$ (since $n>2$), in contradiction with $n!/2 \mid m!$. Therefore 
   $$n \leq m = [A_n:H].$$
   This proves part (b) of Theorem 12.1.10.
   \ee
   \end{proof}
   
  
\paragraph{Ex. 12.1.20}

{\it Let $G$ be a finite group and let $p$ be the smallest prime dividing $|G|$. Prove that every subgroup of index $p$ in $G$ is normal.
}

\begin{proof}
Let $N = \mathrm{Core}_G(H)$. Then $N \subset H \subset G$, and $N$ is normal in $G$.

By Exercise 19 part (f),
$$[G:N] \mid [G:H]! = p!.$$
Moreover, 
$$[G:N] = [G:H] [H:N] = p [H:N],$$
so $$[H:N] \mid (p-1)!.$$
If $[H:N] \ne 1$, there exists a prime $q$ such that $q\mid [H:N]$. Since $[H:N] \mid (p-1)!$, $q<p$. But $q$ divides $[H:N]$ , so $q$ divides $|H|$, which divides $|G|$. But $p$ is the smallest prime divisor of $|G|$: this is a contradiction.

So $[H:N] = 1$, $N = H$. Therefore $H = N$ is normal in $G$.
\end{proof}

\paragraph{Ex. 12.1.21}

{\it Part (a) of Theorem 12.1.10 implies that when $n\geq 5$, the index of a proper subgroup of $S_n$ is either $2$ or $\geq n$.
\be
\item[(a)] Prove that $S_n$ always has a subgroup $H$ of index $n$. This means that equality can occur in the bound $[S_n:H] \geq n$.

\item[(b)] Give an example to prove that Theorem 12.1.10 is false when $n=4$.
\ee
}

\begin{proof}
\be
\item[(a)] The subgroup $H$ of $S_n$ of the permutations $\sigma$ that fix $n$ is a subgroup isomorphic to $S_{n-1}$, and $[S_n : H ] = n!/(n-1)! = n$.
\item[(b)] In the Exercise 3, we saw that $H = H(y_1)$, where $y_1 = x_1x_2+x_3x_4$ is a group isomorphic to $D_8$:
$$\langle (1\, 2), (1\,3\,2\,4) \rangle= \{(), (1\,2), (1\,3\,2\,4), (1\,3)(2\,4), (1\,2)(3\,4), (1\,4)(2\,3), (3\,4), (1\,4\,2\,3)\},$$
so $[S_4 : H] = 3 <n=4$. This proves that the Theorem 12.1.10 is false if we forget the hypothesis $n\geq 5$.
\ee
\end{proof}

\end{document}