%&LaTeX
\documentclass[11pt,a4paper]{article}
\usepackage[frenchb,english]{babel}
\usepackage[applemac]{inputenc}
\usepackage[OT1]{fontenc}
\usepackage[]{graphicx}
\usepackage{amsmath}
\usepackage{amsfonts}
\usepackage{amsthm}
\usepackage{amssymb}
%\input{8bitdefs}

% marges
\topmargin 10pt
\headsep 10pt
\headheight 10pt
\marginparwidth 30pt
\oddsidemargin 40pt
\evensidemargin 40pt
\footskip 30pt
\textheight 670pt
\textwidth 420pt

\def\imp{\Rightarrow}
\def\gcro{\mbox{[\hspace{-.15em}[}}% intervalles d'entiers 
\def\dcro{\mbox{]\hspace{-.15em}]}}

\newcommand{\be} {\begin{enumerate}}
\newcommand{\ee} {\end{enumerate}}
\newcommand{\deb}{\begin{eqnarray*}}
\newcommand{\fin}{\end{eqnarray*}}
\newcommand{\ssi} {si et seulement si }
\newcommand{\D}{\mathrm{d}}
\newcommand{\Q}{\mathbb{Q}}
\newcommand{\Z}{\mathbb{Z}}
\newcommand{\N}{\mathbb{N}}
\newcommand{\R}{\mathbb{R}}
\newcommand{\C}{\mathbb{C}}
\newcommand{\F}{\mathbb{F}}
\newcommand{\re}{\,\mathrm{Re}\,}
\newcommand{\ord}{\mathrm{ord}}
\newcommand{\legendre}[2]{\genfrac{(}{)}{}{}{#1}{#2}}

\title{Solutions to David A.Cox  "Galois Theory''}
\author{Richard Ganaye}

\begin{document}

\maketitle

\section{Chapter 1}

\subsection{CARDAN'S FORMULAS}

\paragraph{Ex. 1.1.1}

{\it Complete the demonstration (begun in the text) that the substitution $x = y-b/3$ transforms $x^3+bx^2+cx+d$ into $y^3+py+q$, where $p$ and $q$ are given by (1.2).
}

\begin{proof}
The equation $Ax^3+Bx^2+Cx+D=0$ ($A,B,C,D \in \mathbb{C},A\neq 0$) becomes,  after division by $A$,
$$x^3+bx^2+cx+d = 0.$$
with $b=B/A,c=C/A,d=D/A.$
The substitution $x = y-\frac{b}{3}$ gives
 \begin{align*}
 x\ & = y-\frac{b}{3},\\
 x^2 &= y^2 -\frac{2b}{3} y +\frac{b^2}{9},\\
 x^3 &= y^3 -by^2+\frac{b^2}{3} y - \frac{b^3}{27},
 \end{align*}
 
so
 
 \begin{align*}
 &x^3+bx^2+cx+d\\
 &=\left(y^3 -by^2+\frac{b^2}{3} y - \frac{b^3}{27}\right) +b\left(y^2 -\frac{2b}{3} y +\frac{b^2}{9}\right)+c\left(y-\frac{b}{3}\right) +d\\
 &=y^3 +\left(-\frac{b^2}{3}+c \right) y + \left(\frac{2b^3}{27}-\frac{bc}{3}+d\right).
 \end{align*}

In conclusion, if $x = y-\frac{b}{3}$, the equation $x^3+bx^2+cx+d = 0$ is equivalent to the equation
$$y^3+py+q=0,$$
where
\begin{align*}
p &= -\frac{b^2}{3}+c, \\
 q &= \frac{2b^3}{27}-\frac{bc}{3}+d, 
 \end{align*}
which is the equation (1.2).

Relatively to the first coefficients $A,B,C,D$ we obtain
\begin{align*}
p &= -\frac{B^2}{3A^2} + \frac{C}{A} = \frac{-B^2+3AC}{3A^2},\\  
q &= \frac{2B^3}{27A^3} - \frac{BC}{3A^2} + \frac{D}{A} = \frac{2B^3-9ABC+27 A^2D}{27A^3}.
 \end{align*}
\end{proof}

\paragraph{Ex. 1.1.2}

{\it In Example 1.1.1, show that $y_2$ and $y_3$ are complex conjugates of each other.
}

\begin{proof}
$z_1 = \sqrt[3]{\frac{-1+\sqrt{5}}{2}}$ and $z_2 = \sqrt[3]{\frac{-1-\sqrt{5}}{2}}$ are the real cube roots of $\frac{-1+\sqrt{5}}{2}$ and $\frac{-1-\sqrt{5}}{2}$.

Then
\begin{align*}
y_2 &= \omega z_1 +\omega^2 z_2,\\
y_3 &= \omega^2 z_1 +\omega z_2,\\
\end{align*}
are complex conjugates.

Indeed, $ \omega^3 =1$, so $\omega^2 = \frac{1}{\omega} = \overline{\omega}$, since $\omega \overline{\omega} = \vert \omega \vert ^2= 1$.

Moreover $\overline{\omega^2} = \overline{\omega}^2 = \omega^4 = \omega$.

Consequently 
\begin{align*}
\overline{y_2} &= \overline{\omega}\, \overline{z_1} + \overline{\omega}^2\,  \overline{z_2}\\
&= \omega^2 z_1 + \omega z_2\\
&=y_3.
\end{align*}
\end{proof}

\paragraph{Ex. 1.1.3}

{\it Show that Cardan's formulas give the roots of $y^3+py+q$ when $p=0$.
}

\begin{proof}
We choose a root of $q^2$ by writing $\sqrt{q^2} = q$ (the other choice $-q$ only exchange the two terms of the sum giving $y_1$, and exchange  $y_2,y_3$).

If $p=0$, 
\begin{align*}
\sqrt[3]{-\frac{q}{2}+\sqrt{\left(\frac{q}{2}\right)^2 + \left(\frac{p}{3}\right)^3}}=\sqrt[3]{-\frac{q}{2}+\frac{q}{2}} &= 0,\\
\sqrt[3]{-\frac{q}{2}-\sqrt{\left(\frac{q}{2}\right)^2 + \left(\frac{p}{3}\right)^3}}=\sqrt[3]{-\frac{q}{2}-\frac{q}{2}} &= \sqrt[3]{-q}.\\
\end{align*}

Thus $y_1 = \sqrt[3]{-q}, y_2 = \omega^2 \sqrt[3]{-q}, y_3 = \omega \sqrt[3]{-q}$.

These are the roots of $y^3 + q=0$ : the Cardan's formulas, established for $p\neq 0$, remain true if $p=0$.
\end{proof}

\paragraph{Ex. 1.1.4}

{\it Verify the formulas for $y_2$ and $y_3$ in Example 1.1.2.
}

\begin{proof}
The solutions of the equation $y^3 - 3y=0$ are $0,\pm \sqrt{3}$.

Here $p/3=-1$ and $q/2=0$, so the Cardan's formulas give, with the choice $\sqrt{-1} = i$, 

$$z_1 = \sqrt[3]{-\frac{q}{2}+\sqrt{\left(\frac{q}{2}\right)^2 + \left(\frac{p}{3}\right)^3}}= \sqrt[3]{i}.$$

As $(-i)^3 = i$, we can choose  $z_1 = \sqrt[3]{i} = -i$, which implies the value $z_2 = \sqrt[3]{-i} = i$, since $z_1z_2 = -p/3=1$.
The Cardan's formulas give
\begin{align*}
y_1 &= -i+i =0,\\
y_2 &= -i \omega + i \omega^2,\\
y_3 &= -i \omega^2 + i \omega.
\end{align*}
So $y_2 = -i \omega + i \omega^2 = -i\omega+i(-1-\omega) = -i(2\omega+1)$.

As  $\omega = -\frac{1}{2}+ i \frac{\sqrt{3}}{2}$, so $2 \omega +1 = i\sqrt{3}$, $y_2 = \sqrt{3}$, and $y_3 = -y_2 = -\sqrt{3}$.

$$y_1=0 , y_2 = \sqrt{3}, y_3 = -\sqrt{3}.$$

The Cardan's formulas give the three real roots of $y^3-3y=0$, with intermediate steps in the complex field.
\end{proof}

\paragraph{Ex. 1.1.5}

{\it The substitution $x= y - b/3$ can be adapted to other equations as follows.
\begin{enumerate}
\item[(a)]
Show that $x=y - b/2$ gets rid of the coefficient of $x$ in the quadratic equation $x^2+bx+c=0$. Then use this to derive the quadratic formula.

\item[(b)]For the quartic equation $x^4+bx^3+cx^2+dx+e=0$,what substitution should you use to
get rid of the coefficient of $x^3$? 

\item[(c)] Explain how part (b) generalizes to a monic equation of degree $n$.
\end{enumerate}
}

\begin{proof}
\begin{enumerate}
\item[(a)] The substitution $x = y - b/2$ in the equation $x^2+bx+c=0$ gives
\begin{align*}
0 &= x^2 + bx +c \\
&=\left ( y - \frac{b}{2}\right )^2 + b \left ( y - \frac{b}{2}\right ) + c\\
&=y^2 - \frac{b^2}{4} + c.
\end{align*}

This last equation has two solutions
$$y = \pm \frac{1}{2} \sqrt{b^2-4c}.$$
The two solutions of the equation $x^2+bx+c=0$ are so
$$x = \frac{-b  \pm \sqrt{b^2-4c}}{2}.$$


\item[(b)] The substitution $x = y-b/4$  in $x^4+bx^3+cx^2+dx+e=0$ gets rid of the coefficient of $y^3$.


\item[(c)] More generally, the substitution $x= y - a_{n-1}/n$ in the equation
$$x^n +  a_{n-1} x^{n-1}+ \cdots + a_0.$$
gets rid of the coefficient of $y^{n-1}$ in the transformed equation.

\end{enumerate}
\end{proof}

\paragraph{Ex. 1.1.6}

{\it Consider the equation $x^3+x-2 = 0$. Note that $x=1$ is a root.
\begin{enumerate}
\item[(a)] Use Cardan's formulas (carefully) to derive the surprising formula
$$1 = \sqrt[3]{1 + \frac{2}{3} \sqrt{\frac{7}{3}}} + \sqrt[3]{1 - \frac{2}{3} \sqrt{\frac{7}{3}}}$$
\item[(b)] Show that $1 + \frac{2}{3} \sqrt{\frac{7}{3}} = \left(\frac{1}{2}+ \frac{1}{2}\sqrt{\frac{7}{3}} \right)^3$, and use this to explain the result of part (a).
\end{enumerate}
}

\begin{proof}
\begin{enumerate}
\item[(a)] The polynomial $x^3+x-2 = (x-1)(x^2+x+2)$ has a unique real root $1$, since the discriminant of $x^2+x+2$ is negative.

As $p/3=1/3,q/2=-1$,
\begin{align*}
z_1 &= \sqrt[3]{-\frac{q}{2}+\sqrt{\left(\frac{q}{2}\right)^2 + \left(\frac{p}{3}\right)^3}}
= \sqrt[3]{1+\sqrt{1 +\frac{1}{27}}},\\
z_1&= \sqrt[3]{1 + \frac{2}{3} \sqrt{\frac{7}{3}}},\\
z_2 &= \sqrt[3]{1 - \frac{2}{3} \sqrt{\frac{7}{3}}}.
\end{align*}

The roots of $x^3+x-2$ are
\begin{align*}
x_1 &= z_1 + z_2,\\
x_2 &= \omega z_1 + \omega^2 z_2,\\
x_3 &= \omega^2 z_1 + \omega z_2.
\end{align*}

As $z_1 + z_2$ is real, it is the unique real root of $x^3+x-2$, so

$$1 = \sqrt[3]{1 + \frac{2}{3} \sqrt{\frac{7}{3}}} + \sqrt[3]{1 - \frac{2}{3} \sqrt{\frac{7}{3}}}.$$


\item[(b)] 
\begin{align*}
\left(\frac{1}{2}+ \frac{1}{2}\sqrt{\frac{7}{3}} \right)^3 &= \frac{1}{8} \left(1 + \sqrt{\frac{7}{3}}\right)^3\\
&=\frac{1}{8} \left( 1 + 3\sqrt{\frac{7}{3}} + 3 \frac{7}{3} + \frac{7}{3}\sqrt{\frac{7}{3}}\right)\\
&=1 + \frac{1}{8}\left(3 + \frac{7}{3} \right)\sqrt{\frac{7}{3}}\\
&=1 + \frac{2}{3} \sqrt{\frac{7}{3}}.
\end{align*}

So $\sqrt[3]{1 + \frac{2}{3} \sqrt{\frac{7}{3}}} = \frac{1}{2}+ \frac{1}{2}\sqrt{\frac{7}{3}} $, and similarly $\sqrt[3]{1 - \frac{2}{3} \sqrt{\frac{7}{3}}} = \frac{1}{2}- \frac{1}{2}\sqrt{\frac{7}{3}} $.

Consequently,
$$\sqrt[3]{1 + \frac{2}{3} \sqrt{\frac{7}{3}}} + \sqrt[3]{1 - \frac{2}{3} \sqrt{\frac{7}{3}}}  =  \frac{1}{2}+ \frac{1}{2}\sqrt{\frac{7}{3}}+  \frac{1}{2}- \frac{1}{2}\sqrt{\frac{7}{3}} = 1.$$

\end{enumerate}
\end{proof}

\paragraph{Ex. 1.1.7}

{\it Cardan's formulas, as stated in the text, express the roots as sums of two cube roots. Each cube root has three values, so there are nine different possible values for the sum of the cube roots. Show hat these nine values are the roots of the equations $y^3+py+q=0, y^3+\omega py+q$, and $y^3+\omega^2py+q=0$,where as usual $\omega = \frac{1}{2}(-1 + i \sqrt{3})$.
}

\begin{proof}
The nine possible sums of two cube roots are

\begin{align*}
&y_1 = z_1 + z_2, &y_4 = \omega z_1+\omega z_2 = \omega y_1,\hspace{1cm} &y_7 = \omega^2z_1+\omega^2 z_2 = \omega^2 y_1,\\
&y_2=\omega z_1+\omega^2 z_2, &y_5 = \omega^2z_1+z_2=\omega y_2, \hspace{1cm}  &y_8 = z_1+\omega z_2=\omega^2 y_2,\\
&y_3= \omega^2z_1+\omega z_2, &y_6 = z_1+\omega^2z_2 = \omega y_3 \hspace{1cm}  &y_9 = \omega z_1+z_2 = \omega^2 y_3.
\end{align*}

Let $t=\omega y$. Then
$y^3+py+q=0$ is equivalent to $\left ( \frac{t}{\omega} \right )^3+ p \left ( \frac{t}{\omega} \right )+q = 0$, so is equivalent to $t^3+\omega^2 t +q=0$.

Consequently $y_4,y_5,y_6$ are the roots of the polynomial $y^3 + \omega^2 p y + q$.

For the same reason, $y_7,y_8,y_9$ are the roots of the polynomial $y^3 + \omega p y +q$.
\end{proof}

\paragraph{Ex. 1.1.8} 

{\it Use Cardan's formulas to solve $y^3+3\omega y+1 = 0$.
}

\begin{proof}
The equation $y^3+3\omega y +1 =0$,  with the substitution $$t = \omega y, y =\omega^2 t,$$ becomes

$$t^3+3t+1 = 0.$$

 With $p/3 = 1, q/2 = 1/2$, the Cardan's formulas give
\begin{align*}
z_1 &= \sqrt[3]{-\frac{q}{2}+\sqrt{\left(\frac{q}{2}\right)^2 + \left(\frac{p}{3}\right)^3}}\\
z_1 &=\sqrt[3]{\frac{-1+\sqrt{5}}{2}}\\
z_2 &=\sqrt[3]{\frac{-1-\sqrt{5}}{2}}\\
\end{align*}

The roots of $t^3+3t+1$ are  $z_1+z_2,\omega z_1+\omega^2 z_2,\omega^2 z_1 + \omega z_2$.

Thus the roots of  $y^3+3\omega y + 1$ are $\omega^2 z_1 + \omega^2 z_2,z_1 + \omega z_2, \omega z_1 + z_2$.

The solutions of  $y^3+3\omega y +1 =0$ are
\begin{align*}
&y_1 = \omega^2  \sqrt[3]{\frac{-1+\sqrt{5}}{2}} + \omega ^2 \sqrt[3]{\frac{-1-\sqrt{5}}{2}},\\
&y_2 =  \sqrt[3]{\frac{-1+\sqrt{5}}{2}} + \omega \sqrt[3]{\frac{-1-\sqrt{5}}{2}},\\
&y_3 = \omega \sqrt[3]{\frac{-1+\sqrt{5}}{2}} + \sqrt[3]{\frac{-1-\sqrt{5}}{2}}.\\
\end{align*}
\end{proof}

\subsection{PERMUTATIONS OF THE ROOTS}

\paragraph{Ex. 1.2.1}

{\it Let $z_1,z_2$ be the roots of (1.9) chosen at the beginning of the section
\begin{enumerate}
\item[(a)] Show that $z_1,z_2,\omega z_1,\omega z_2, \omega^2 z_1, \omega^2 z_2$ are the six roots of the cubic resolvent.
\item[(b)] Prove (1.10)
\end{enumerate}
}

\begin{proof}
\begin{enumerate}
\item[(a)] If $p=0$,  by (1.7) and Ex. 1.1.3, $\{z_1,z_2\} = \{0,\sqrt[3]{-q}\}$, so $z_1,z_2$ are the solutions of  the equation $ 0 = z^6+qz^3-\frac{p^3}{27} = z^6+qz^3$.

We suppose now that $p\neq 0$.

By definition $z_1, z_2$ are such that
\begin{align}
&z_1 z_2 = -\frac{p}{3}, \label{eq:1}\\
&z_1^6+qz_1^3-\frac{p^3}{27}=0.\label{eq:2}
\end{align}
Let $Z_1=z_1^3,Z_2=z_2^3$. Then
\begin{align}
&Z_1 Z_2 = -\frac{p^3}{27}\label{eq:3}\\
&Z_1^2+qZ_1-\frac{p^3}{27}=0\label{eq:4}
\end{align}
As $p\neq 0$, then $Z_1\neq 0$ by \eqref{eq:4}, and using \eqref{eq:3} and \eqref{eq:4},
$$Z_1+Z_2 = Z_1 -\frac{p^3}{27Z_1} = \frac{1}{Z_1}\left(Z_1^2 -\frac{p^3}{27}\right) = \frac{1}{Z_1}(-q Z_1) = -q.$$

Thus, for all $Z \in \mathbb{C}$,
\begin{align}
(Z-Z_1)(Z-Z_2)=Z^2-(Z_1+Z_2)Z+Z_1Z_2 = Z^2+qZ-\frac{p^3}{27}, \label{eq:5}
\end{align}
and so 
$$Z_2^2+qZ_2-\frac{p^3}{27}=0,$$
$$z_2^6+qz_2^3-\frac{p^3}{27} = 0.$$

$z_2$ is a root of the resolvent equation.

By (5) the resolvent equation is
\begin{align}
0 = z^6+qz^3-\frac{p^3}{27} = (z^3-z_1^3)(z^3-z_2^3).
\end{align}
The solutions of this equation are so $z_1,z_2,\omega z_1,\omega z_2,\omega^2 z_1,\omega^2 z_2$.


\item[(b)] By 1.2  A :
\begin{align*}
&z_1 = \frac{1}{3}(x_1+\omega^2 x_2+\omega x_3)\\
&z_2 = \frac{1}{3}(x_1+\omega^2  x_3+\omega x_2)
\end{align*}
Multiplying by $\omega$, and by $\omega^2$ : 
\begin{align*}
&\omega z_1 = \frac{1}{3}(x_2+\omega^2 x_3+\omega x_1)\\
&\omega z_2 = \frac{1}{3}(x_3+\omega^2 x_2+\omega x_1)\\
&\omega^2 z_1 = \frac{1}{3}(x_3+\omega^2 x_1+\omega x_2)\\
&\omega^2 z_2 = \frac{1}{3}(x_2+\omega^2 x_1+\omega x_2)
\end{align*}
\end{enumerate}
\end{proof}

\paragraph{Ex. 1.2.2}

{\it Prove (1.14) and (1.15)
}

\begin{proof}
We know by 1.2(B) that
$$z_1-z_2 = \frac{-i}{\sqrt{3}}(x_2-x_3).$$
By (1.10) :
\begin{align*}
z_1 - \omega z_2 &= \frac{1}{3}\left [ (x_1 + \omega^2 x_2+\omega x_3) - (x_3+\omega^2 x_2 + \omega x_1) \right]\\
&=\frac{1}{3}(1-\omega)(x_1-x_3).
\end{align*}
and
\begin{align*}
\frac{1}{3}(1-\omega) &= \frac{1}{3} \left(1 - e^{2i\pi/3} \right)\\
&= \frac{1}{3} e^{i\pi/3} \left( e^{-i\pi/3} - e^{i\pi/3}\right)\\
&= \frac{1}{3} \omega^2 \left( e^{i\pi/3} - e^{-i\pi/3}\right)\\
&=\frac{2}{3} i \sin (\pi/3) \, \omega^2\\
&= \frac{i}{\sqrt{3}}\,  \omega^2.
\end{align*}
So
$$z_1 - \omega z_2 = \frac{i}{\sqrt{3}}\,  \omega^2 (x_1-x_3).$$
Similarly 
\begin{align*}
z_1 - \omega^2 z_2 &= \frac{1}{3}\left [ (x_1 + \omega^2 x_2+\omega x_3) - (x_2+\omega^2 x_1 + \omega x_3) \right]\\
&=\frac{1}{3}(1-\omega^2)(x_1-x_2),
\end{align*}
and  $\frac{1}{3}(1-\omega^2) = \overline{\frac{1}{3}(1-\omega) } = \overline{\frac{i}{\sqrt{3}}\,  \omega^2} =-\frac{i}{\sqrt{3}}\,  \omega.$

Thus
$$z_1 - \omega^2 z_2 = -\frac{i}{\sqrt{3}}\,  \omega (x_1-x_2).$$

Using these three results,
\begin{align*}
\sqrt{D} &= z_1^3 - z_2^3 = (z_1-z_2)(z_1-\omega z_2)(z_1-\omega^2 z_2)\\
&=\frac{-i}{\sqrt{3}}  \frac{i}{\sqrt{3}}\,  \omega^2 \left( -\frac{i}{\sqrt{3}}\,  \omega\right) (x_1-x_2)(x_1-x_3)(x_2-x_3)\\
&=-\frac{i}{3\sqrt{3}} (x_1-x_2)(x_1-x_3)(x_2-x_3),\\
\end{align*}
which is the formula (1.15).
\end{proof}

\paragraph{Ex. 1.2.3}

{\it Prove (1.20)
}

\begin{proof}
By (1.17), $\Delta = -4p^3-27q^2$.

Using (1.19), we obtain

\begin{align*}
p^3 &= \left( - \frac{b^2}{3}+c\right)^3\\
&=-\frac{b^6}{27} +  \frac{1}{3} b^4c - b^2c^2 + c^3,\\
-4p^3 &= \frac{4}{27}b^6 - \frac{4}{3} b^4c +4 b^2c^2 -4 c^3,\\
\end{align*}
\begin{align*}
q^2 &= \left(\frac{2b^3}{27} - \frac{bc}{3} + d \right)^2\\
&=\frac{4}{27^2} b^6 +\frac{b^2c^2}{9} + d^2 -\frac{4b^4c}{3\times 27} + \frac{4b^3d}{27} - \frac{2bcd}{3}\\
-27q^2&= -\frac{4}{27} b^6 - 3 b^2c^2 - 27 d^2+ \frac{4}{3} b^4c - 4 b^3d + 18bcd.\\
\end{align*}
So
\begin{align*}
\Delta&= -4p^3-27q^2\\
&= b^2c^2 + 18bcd-4c^3-4b^3d-27d^2.
\end{align*}
\end{proof}

\paragraph{Ex. 1.2.4}

{\it We say that a cubic $x^3+bx^2+cx+d$ has a multiple root if it can be written as $(x-r_1)^2(x-r_2)$. Prove that $x^3+bx^2+cx+d$ has a multiple root if and only if its discriminant is zero.
}

\begin{proof}
Let $f(x) = x^3+bx^2+cx+d$.

If  $r_1,r_2$ are such that $f(x) = (x-r_1)^2(x-r_2)$, naming the roots $x_1=r_1, x_2=r_1,x_3 = r_2$, we obtaint $x_1=x_2$, so
$$\Delta= (x_1-x_2)^2(x_1-x_3)^2(x_2-x_3)^2 = 0.$$

Reciprocally, if $\Delta = 0$, then $x_1=x_2$, or $x_1=x_3$, or $x_2=x_3$.

In the first case let $r_1=x_1, r_2=x_3$. Then
$$f(x) = (x-x_1)(x-x_2)(x-x_3) = (x-r_1)^2(x-x_2),$$
and similarly in the two other cases.
\end{proof}

\paragraph{Ex. 1.2.5}

{\it Since $\Delta = (x_1-x_2)^2(x_1-x_3)^2(x_2-x_3)^2$, we can define the square root of $\Delta$ to be $\sqrt{\Delta} = (x_1-x_2)(x_1-x_3)(x_2-x_3)$. Prove that an even permutation of the roots takes $\sqrt{\Delta}$ to $\sqrt{\Delta}$ while an odd permutation takes $\sqrt{\Delta}$ to $-\sqrt{\Delta}$. In section 2.4 we will see that this generalizes nicely to the case of degree $n$.
}

\begin{proof}
By definition,
$$\sqrt{\Delta} = (x_1-x_2)(x_1-x_3)(x_2-x_3).$$

If $\sigma \in S_3$, we define $$\sigma \cdot \sqrt{\Delta}  = (x_{\sigma(1)}-x_{\sigma(2)})(x_{\sigma(1)}-x_{\sigma(3)})(x_{\sigma(2)}-x_{\sigma(3)}).$$

If $\tau = (1\ 2)$, 
 $$\tau \cdot \sqrt{\Delta}  = (x_2-x_1)(x_2-x_3)(x_1-x_3) = -\sqrt{\Delta}.$$
 
Same result if $\tau = (1\ 3)$, or $\tau = (2\ 3)$.

If $\sigma = (1\  2\  3)$,
$$\sigma \cdot \sqrt{\Delta}  = (x_2-x_3)(x_2-x_1)(x_3-x_1) = \sqrt{\Delta},$$
with the  same result if $\sigma = (1 3 2)$ (and also if $\sigma$ is identity).

In conclusion, 

$\sigma \cdot \sqrt{\Delta}  = \sqrt{\Delta} $ if $\sigma$ is even,

$\sigma \cdot \sqrt{\Delta}  =- \sqrt{\Delta} $ if $\sigma$ is odd.
\end{proof}

\subsection{CUBIC EQUATIONS OVER THE REAL NUMBERS}

\paragraph{Ex. 1.3.1}

{\it Let $f(x) =y^3+py+q = (y-y_1)(y-y_2)(y-y_3)$, and set
$$\Delta = (y_1-y_2)^2(y_1-y_3)^2(y_2-y_3)^2.$$
The goal of this exercise is to give a different proof of (1.22).
\begin{enumerate}
\item[(a)] Use the product rule to show that $f'(y_1) = (y_1-y_2)(y_1-y_3)$, where $f'$ denotes the derivative of $f$. Also derive similar formulas for $f'(y_2)$ and $f'(y_3)$.
\item[(b)] Conclude that $\Delta = -f'(y_1)f'(y_2)f'(y_3)$. Be sure to explain where the minus sign comes from.
\item[(c)] The quadratic $f'(y) = 3y^2+p$ factors as $f'(y) = 3(y-\alpha)(y-\beta)$, where $\alpha = \sqrt{-p/3}$ and $\beta = - \sqrt{-p/3}$ (when $p>0$, we let $\sqrt{-p/3} = i\sqrt{p/3}$). Prove that $\Delta = -27 f(\alpha)f(\beta)$.
\item[(d)] Use $f(y) = y^3 + py +q$ and $\alpha = \sqrt{-p/3}$ to show that
$$f(\alpha) = (\sqrt{-p/3})^3 + p\sqrt{-p/3} + q = (2/3)p\sqrt{-p/3} + q.$$
Similarly, show that $f(\beta) = -(2/3) p \sqrt{-p/3}+q$.
\item[(e)] By combining parts (c) and (d), conclude that $\Delta = -4p^3-27q^2$.
\end{enumerate}
}

\begin{proof}
\begin{enumerate}
\item[(a)]
If $y_1, y_2,y_3$ are the complex roots of $f(y)= y^3+py+q \in \R[y]$, then
\begin{align*}
 f(y)&= y^3+py+q=(y-y_1)(y-y_2)(y-y_3),\\
f'(y)&=(y-y_2)(y-y_3)+(y-y_1)(y-y_3)+(y-y_1)(y-y_2),
\end{align*}
thus
\begin{align*}
f'(y_i) = \prod_{j\neq i}(y_i-y_j),\  i=1,2,3,
\end{align*}
explicitly
\begin{align*}
f'(y_1) = (y_1-y_2)(y_1-y_3),\\
f'(y_2) = (y_2-y_1)(y_2-y_3),\\
f'(y_3) = (y_3-y_1)(y_3-y_2).\\
\end{align*}
\item[(b)] Therefore
\begin{align*}
f'(y_1)f'(y_2)f'(y_3) &= [-(y_1-y_2)^2][-(y_1-y_3)^2][-(y_2-y_3)^2]\\
&=-\Delta,
\end{align*}
so
$$\Delta = -f'(y_1)f'(y_2)f'(y_3).$$
\item[(c)] Since $f'(y) = 3y^2 + p =3(y-\alpha)(y-\beta)$, where $\alpha =\sqrt{-p/3}, \beta = - \sqrt{-p/3}$,
\begin{align*}
\Delta&=-(3y_1^2+p)(3y_2^2+p)(3y_3^2+p)\\
&=-27(y_1-\alpha)(y_1-\beta)(y_2-\alpha)(y_2-\beta)(y_3-\alpha)(y_3-\beta)\\
&=-27f(\alpha)f(\beta).
\end{align*}
\item[(d)] Since $f(y) = y^3 + py +q$, 
\begin{align*}
f(\alpha) &=\left(\sqrt{-p/3}\right)^3 + p(\sqrt{-p/3})+q\\
&=(-p/3+p)\sqrt{-p/3}+q\\
&=(2/3)p\sqrt{-p/3}+q,
\end{align*}
and similarly $f(\beta) = f(-\alpha) = -(2/3)p\sqrt{-p/3}+q$.
\item[(e)] By combining parts (c) and (d), 
\begin{align*}
\Delta&=-27f(\alpha)f(\beta)\\
&=-27\left\{q^2-\left[(2/3)p\sqrt{-p/3}\right]^2\right\}\\
&=-27\left[q^2 -\frac{4}{9}p^2\left(-\frac{p}{3}\right)\right]\\
&=-4p^3-27q^2.
\end{align*}
Conclusion : the discriminant of  $y^3+py+q$ is $$\Delta = -4p^3-27q^2.$$
\end{enumerate}
\end{proof}

\paragraph{Ex. 1.3.2}

{\it Let $f(y) = y^3+py+q$. The purpose of Exercises 2 to 5 is to prove Theorem 1.3.1 geometrically using graphing techniques. The proof breaks up into three cases corresponding to $p>0,p=0$, and $p<0$. This exercise will consider the case $p>0$.
\begin{enumerate}
\item[(a)] Explain why $\Delta <0$.
\item[(b)] Analyse the sign of $f'(y)$, and show that $f(y)$ is always increasing.
\item[(c)] Explain why $f(y)$ has only one real root.
\end{enumerate}
}

\begin{proof}
\begin{enumerate}
 \item[(a)]
As $p>0$, $-4p^3<0$, and $-27q^2\leq 0$, thus $\Delta = -4p^3-27q^2<0$.

\item[(b)]
For all $y\in \mathbb{R}$, $f'(y) = 3y^2+p>0$,  thus $f$ is strictly increasing on $\mathbb{R}$.

\item[(c)]
As $f$ is strictly increasing on $\mathbb{R}$, $f$ is injective (one to one).

Moreover $\lim\limits_{y\to +\infty} f(y) = +\infty$, and $\lim\limits_{y\to -\infty} f(y) = -\infty$, so there exists $y_1$ such that $f(y_1)<0$ and there exists $y_2$ such that $f(y_2)>0$. As $f$ is a continuous function, the intermediate values theorem gives the existence of a real root of $f$. As $f$ is injective, this is the only real root.

\end{enumerate}
\end{proof}

\paragraph{Ex. 1.3.3}

{\it Next, consider the case $p=0$.
\begin{enumerate}
\item[(a)] Explain why $\Delta<0$.
\item[(b)] Explain why $f(y)$ has only one real root.
\end{enumerate}
}

\begin{proof}
\begin{enumerate}
 \item[(a)] $\Delta = -4q^2 \leq 0$
 
 The hypothesis of theorem 1.3.1 is $\Delta \neq 0$, so $\Delta <0$.
 
 \item[(b)]
For all $y \in \mathbb{R}$, $f'(y) = 3y^2 \geq 0$ and $f'(y) = 0$ only if $y=0$, so $f$ is strictly increasing. With the same argument already given in Ex. 1.3.2, $f$ has a unique real root, written $\sqrt[3]{-q}$.
\end{enumerate}
\end{proof}

\paragraph{Ex. 1.3.4} 

{\it Finally, consider the case $p<0$. In this case $f'(y) = 3y^2+p$ has roots $\alpha = \sqrt{-p/3}$ and $\beta = - \sqrt{-p/3}$, which are real and distinct.
\begin{enumerate}
\item[(a)] Show that the graph of $f(y)$ has a local minimum at $\alpha$ and a local maximum at $\beta$. Thus $f(\alpha)$ is a local minimum value and $f(\beta)$ is a local maximum value. Also show that $f(\alpha)<f(\beta)$.
\item[(b)] Explain why $f(y)$ has three real roots if $f(\alpha)$ and $f(\beta)$ have opposite signs and has one real root if they have the same sign. Illustrate your answer with a drawing of the three cases that can occur.
\item[(c)] Conclude that $f(y)$ has three real roots if and only if $f(\alpha)f(\beta)<0$.
\item[(d)] Finally, use part (c) of Exercise 1 to show that the roots are all real if and only if $\Delta >0$.

\end{enumerate}
}

\begin{proof}
Case $p<0$.
\begin{enumerate}

\item[(a)]
$f'(y) = 3y^2+p < 0 \iff y^2<-p/3 \iff \beta < f'(y) < \alpha\  (\alpha = \sqrt{-p/3} = -\beta)$.

$f$ is strictly increasing on $]-\infty,\beta]$ and on $[\alpha, +\infty[$, strictly decreasing on $[\beta,\alpha]$. $\beta$ is a local maximum, and $\alpha$ a local minimum. As $f$ is decreasing on  $[\beta,\alpha]$, $f(\beta) > f(\alpha)$.

\item[(b)] As $f$ is continuous, and  $\lim\limits_{y\to +\infty} f(y) = +\infty$,  $\lim\limits_{y\to -\infty} f(y) = -\infty$ and $f$ strictly  monotonous on each interval $]-\infty,\beta],[\beta,\alpha],[\alpha, +\infty[$, the intermediate value theorem gives three roots if $f(\alpha)f(\beta)<0$, and a unique root if $f(\alpha) f(\beta)>0$.

If  $f(\alpha)f(\beta) = 0$, then  $\Delta = -27 f(\alpha)f(\beta) = 0$ (Ex.1 c), which we can exclude by hypothesis.

\item[(c)] As these cases are mutually exclusive, (b) proves that $f$ has 3 real roots iff $f(\alpha)f(\beta)<0$. 

\item[(d)]
By Ex. 1.3.2 and 1.3.3, we know that $p\geq 0$ imply that $f$ has a unique real root. 

If $f$ has 3 distinct real roots, then $\Delta \neq 0$ (Ex.1.2.4), $p<0$, and (c) show that $f(\alpha)f(\beta) <0$, thus $\Delta = -27 f(\alpha)f(\beta) >0$.

Conversely, if $\Delta>0$, then $p<0$ and $f(\alpha)f(\beta)<0$. By (c), we know that $f$ has 3 real roots.

Conclusion :  $f$ has three distinct real roots iff  $\Delta >0$.

\end{enumerate}

\end{proof}

\paragraph{Ex. 1.3.5}

{\it Explain how Theorem 1.3.1 follows from Exercises 2,3, and 4.
}

\begin{proof}
With the hypothesis $\Delta \neq 0$, $f(y) = y^3+py+q$ has three distinct roots (Ex 1.2.4). We have proved in Ex. 1.3.4 that $\Delta >0$ iff the three roots of $f$ are real: This is the part (a) of Theorem 1.3.1.

If there exists a non real complex root, Ex.4(d) show that $\Delta <0$.

Reciprocally, if $\Delta <0$, there exists a non real root $y_1$, and $y_2 = \overline{y_1}$ is a root of $f$, with $y_2 \ne y_1$. As $f$ has always a real root by Ex. 2,3,4, $f$ has so exactly one real root, and two non real conjugates roots. This is the part (b) of the theorem.
\end{proof}

\paragraph{Ex. 1.3.6}

{\it Prove (1.24) : $z_1z_2 = -\frac{p}{3} \Rightarrow z_2 = \overline{z_1}$.
}

\begin{proof}
In the context of paragraph 1.3.A, 
$$z_1 = \sqrt[3]{\frac{1}{2}\left(-q+i\sqrt{\frac{\Delta}{27}} \right)}, z_2 = \sqrt[3]{\frac{1}{2}\left(-q-i\sqrt{\frac{\Delta}{27}} \right)}, \qquad p,q \in \R,\  \Delta>0.$$

We know $z_1z_2 = -\frac{p}{3}$. 

Then $z_1^3 = \overline{z_2}^3$, so $z_1 = \omega^k \, \overline{z_2},\ k=0,1,2$. Therefore

$$-\frac{p}{3} = z_1 z_2 = \omega^k \, \overline{z_2} z_2 =\omega^k \vert z_2 \vert ^2.$$

As $z_2 \neq 0$, $\omega^k = -\frac{p}{3} \frac{1}{\vert z_2 \vert ^2} \in \mathbb{R}$, which implies $k=0$, therefore $z_2 = \overline{z_1}$.
\end{proof}

\paragraph{Ex. 1.3.7}

{\it Example 1.3.2 expressed the root $y=4$ of $y^3-15y-4$ in terms of Cardan's formulas. Find the other two roots, and eplain how Cardan's formulas give these roots.
}

\begin{proof} Since
\begin{align*}
y^3-15y-4 &= (y-4)(y^2+4y+1)\\
&=(y-4)[(y+2)^2-3]\\
&=(y-4)(y+2-\sqrt{3})(y+2+\sqrt{3}),
\end{align*}
the three roots of $y^3-15y-4$ are $4,-2+\sqrt{3},-2-\sqrt{3}$ and are all real.

The Cardan's formulas, with $q/2 = -2,p/3=-5 $, give 
$$-\frac{q}{2}+ \sqrt{\left (\frac{q}{2}\right)^2+\left (\frac{p}{3}\right)^3} = 2 + \sqrt{4 - 125} = 2 + 11i.$$


Using the Bombelli's note :
$$(2+i)^3 = 2+11i,$$
so we can take for $z_1$ a cube root of $2+11i$ given by 
$$z_1 = 2+i.$$
$z_2$ is the cube root of $-\frac{q}{2}- \sqrt{\left (\frac{q}{2}\right)^2+\left (\frac{p}{3}\right)^3}= 2 - i \sqrt{11}$ verifying $z_1 z_2 = -p/3 = 5$. 

By Ex. 1.3.6,  $$z_2 = \overline{z_1} = 2 -i.$$
The roots of $y^3-15y-4$, using Cardan's formulas, are
\begin{align*}
y_1 &= z_1+ z_2,\\
y_2 &= \omega z_1 + \omega^2 z_2 = 2\mathrm{Re}(\omega z_1),\\
y_3 &= \omega^2 z_1 + \omega z_2 = 2\mathrm{Re}(\omega^2 z_1),\\
\end{align*}
so
\begin{align*}
y_1 &= 2+i + 2 -i = 4,\\
y_2 &=\mathrm{Re}[(-1+i\sqrt{3})(2+i)] = -2 - \sqrt{3},\\
y_3 &= \mathrm{Re}[(-1-i\sqrt{3})(2+i)] = -2 + \sqrt{3}.\\
\end{align*}
\end{proof}

\paragraph{Ex. 1.3.8}

{\it Derive the trigonometric identity $\cos(3\theta) = 4 \cos^3 \theta - 3 \cos \theta$ using ${\cos(x+y)} = \cos x \cos y - \sin x \sin y$ and $\cos^2 \theta + \sin^2 \theta = 1$.
}

\begin{proof}
\begin{align*}
e^{i3\theta} &= \left(e^{i\theta}\right)^3\\
&=(\cos \theta + i \sin \theta)^3\\
&= \cos^3 \theta - 3\cos \theta \sin^2 \theta + i (...),
\end{align*}
so $\cos 3 \theta = \mathrm{Re}(e^{i3\theta}) = \cos^3 \theta - 3\cos \theta \sin^2 \theta = \cos^3 \theta - 3\cos\theta (1-\cos^2 \theta) $.
So
$$\cos 3 \theta = 4 \cos^3 \theta - 3 \cos \theta.$$
\end{proof}

\paragraph{Ex. 1.3.9}

{\it When divided by 4, $4t^3 - 3t - \cos(3\theta)$ gives $t^3 -\frac{3}{4} t - \frac{1}{4}\cos(3\theta)$, which is monic. Show that the discriminant of this polynomial is $\frac{27}{16} \sin^2(3\theta)$.
}

\begin{proof}
Let $g(t) = 4 t^3-3t -\cos(3\theta) = 4 f(t)$, where 
$$f(t) = t^3 -\frac{3}{4}t - \frac{1}{4} \cos(3\theta).$$
The discriminant of $f$ is given by
\begin{align*}
\Delta &= -4 p^3 -27q^2\\
& = -4\left(-\frac{3}{4}\right)^3 - \frac{27}{16} \cos^2(3\theta)\\
&= \frac{27}{16}(1-\cos^2(3\theta))\\
&= \frac{27}{16} \sin^2(3 \theta).
\end{align*}
\end{proof}

\paragraph{Ex. 1.3.10}

{\it 
The goal of this exercise is to prove Theorem 1.3.3. Let $y^3+py+q=0$ be a cubic equation with positive discriminant. Consider the substitution $y = \lambda t$, which transforms the given equation into $\lambda^3 t^3 +\lambda pt+q = 0$.
\begin{enumerate}
\item[(a)] Show that Exercises 2 and 3 imply that $p<0$.
\item[(b)] The equation $\lambda^3 t^3 + \lambda pt + q = 0$ can be written as
$$4t^3 - \left(\frac{-4p}{\lambda^2}\right) t - \left(\frac{-4q}{\lambda^3}\right) = 0.$$
Show that this coincides with $4t^3 - 3t -\cos(3\theta) = 0$ if and only if
$$\lambda = 2 \sqrt{\frac{-p}{3}}\quad \mathrm{and}\quad  \cos(3\theta) = \frac{3\sqrt{3}q}{2p\sqrt{-p}}.$$
Note that $\sqrt{-p}$ is real and nonzero by part (a).
\item[(c)] Use $\Delta = -(4p^3+27q^3) >0$ to prove that
$$\left | \frac{3\sqrt{3}q}{2p\sqrt{-p}} \right | < 1.$$
\item[(d)] Explain how part (c) implies that the second equation of part (b) can be solved for $\theta$. Also show that $\theta>0$ implies that $\cos(3\theta) \neq \pm 1$.
\item[(e)] By (1.25), $t_1 = \cos \theta, t_2 = \cos\left(\theta + \frac{2\pi}{3}\right)$, and $t_3 = \cos\left(\theta + \frac{4\pi}{3}\right)$ are the three roots of $\lambda^3 t^3 + \lambda pt + q = 0$. Then show that the theorem follows by transforming this back to $y=\lambda t$ via part (b).
\end{enumerate}}
\begin{proof}
The discriminant of $f(y) = y^3+py+q$ is positive by hypothesis :
$$\Delta = -4p^3-27q^2>0.$$
Consequently, $f$ has three distinct real roots.
\begin{enumerate}
\item[(a)] 
If $p\geq 0$, then $\Delta =-4p^3-27q^2\leq 0$, which is false by hypothesis, so
$$p<0.$$

\item[(b)]
$g(t) = f(\lambda t) = \lambda^3t^3+\lambda pt + q, \quad \lambda \neq 0$.

$g(t) = 0 \iff  t^3 + \frac{p}{\lambda^2} t + \frac{q}{\lambda^3}=0 \iff  4t^3 -\left(-\frac{4p}{\lambda^2}\right) t - \left(-\frac{4q}{\lambda^3}\right)=0$.

Let $$h(t) = \frac{4}{\lambda^3} f(\lambda t) = 4t^3 -\left(-\frac{4p}{\lambda^2}\right) t - \left(-\frac{4q}{\lambda^3}\right)$$

Then $h(t)$ has the same roots as $g(t) = f(\lambda t)$. Moreover

$$-\frac{4p}{\lambda^2} = 3 \iff \lambda^2 = -\frac{4}{3} p.$$

If we take $\lambda = 2\sqrt{\frac{-p}{3}}$, then $-\frac{4q}{\lambda^3} = \frac{-4q}{-\frac{8p}{3}\sqrt{-\frac{p}{3}}} = \frac{3\sqrt{3}q}{2p\sqrt{-p}}$, and

 $$h(t) = 4t^3 -3t-\frac{3\sqrt{3}q}{2p\sqrt{-p}}.$$
So $h(t) = 4t^3 -3t - \cos(3\theta)$  if and only if
$$\lambda = 2 \sqrt{\frac{-p}{3}}\quad \mathrm{and}\quad  \cos(3\theta) = \frac{3\sqrt{3}q}{2p\sqrt{-p}}.$$
\item[(c)] Since $p<0$, 
 \begin{align*}
\left \vert \frac{3\sqrt{3}q}{2p\sqrt{-p}}  \right \vert <1 &\iff  -\frac{27}{4} \frac{q^2}{p^3} < 1\\
&\iff -27q^2 > 4p^3 \\
&\iff \Delta = -4p^3-27q^2>0.
\end{align*}

\item[(d)]
As $\Delta > 0$ by hypothesis, $ \left \vert \frac{3\sqrt{3}q}{2p\sqrt{-p}}  \right \vert <1$ : so there exists  $\beta \in ]0,\pi[$ such that $ \frac{3\sqrt{3}q}{2p\sqrt{-p}} = \cos \beta$.
Let $\theta = \beta/3$, then $\theta \in ]0,\pi/3[$ and
$$h(t) = 4t^3 - 3t -\cos(3\theta), $$
where
$$\theta = \frac{1}{3}\arccos\left(\frac{3\sqrt{3}q}{2p\sqrt{-p}} \right).$$
Since $3 \theta \in ]0,\pi[, \cos(3\theta) \neq\pm1$.

\item[(e)] We will solve the equation $h(t) = 4t^3 - 3t -\cos(3\theta)=0$.

Let $u(t) = 4t^3-3t$, where $t \in \R$.

 $$u'(t) <0 \iff 12t^2-3 = 3(4t^2-1)<0 \iff -\frac{1}{2} \leq t < \frac{1}{2} : $$ 
  $u$ is decreasing on $[-1/2,1/2]$, increasing on $]-\infty,-\frac{1}{2}]$ and on $[\frac{1}{2},+\infty[$.

$u(-1/2) = 1,u(0) = 0, u(1/2)=-1,u(1)=1,u(-1) = -1$. Thus $|t|>1 \Rightarrow |u(t)|>1$ and so
$$\forall t \in \R,\quad   u(t) \in [-1,1] \Rightarrow t \in [-1,1].$$


Let $t$ a root of $h(t)$, then $u(t) = 4t^3 - 3 t = \cos(3\theta) \in [-1,1]$. By the properties of the function $u$ on $[-1,1]$, $t \in [-1,1]$, so there exists $\alpha \in \mathbb{R}$ such that $\cos(\alpha) = t$.
\begin{align*}
h(t) = 0 &\iff 4 t^3 - 3t = \cos(3\theta)\\
&\iff 4 \cos^3(\alpha) - 3 \cos(\alpha) = \cos(3\theta)\\
&\iff \cos(3\alpha) = \cos(3\theta)\\
&\iff 3 \alpha = 3 \theta + 2k \pi \ \mathrm{or} \ 3 \alpha =-  3 \theta + 2k \pi, \quad k\in \mathbb{Z}\\
&\iff \alpha = \pm\theta + k \frac{2\pi}{3}, \quad k \in \Z\\
&\iff t = \cos\left (\theta + k \frac{2\pi}{3}\right),\quad  k = 0,1,2.\\
\end{align*}
Therefore the polynomial $$h(t) = 4t^3 - 3t -\cos(3\theta)$$ has three roots :  $\cos(\theta), \cos(\theta+ \frac{2\pi}{3}),\cos(\theta+ \frac{4\pi}{3})$. These roots are real, and distinct, since $\Delta>0$.

As  $h(t) = \frac{4}{\lambda^3} f(\lambda t) $, $f(t)=0 \iff h(t/\lambda)=0$.

The roots of $f$ are so  $\lambda \cos(\theta), \lambda \cos(\theta+ \frac{2\pi}{3}),\lambda \cos(\theta+ \frac{4\pi}{3})$, where $ \lambda = 2\sqrt{\frac{-p}{3}}$.

The Theorem  1.3.3 is proven:

{\it Let $y^3+py+q$ be a polynomial with real coefficients and positive discriminant. Then $p<0$, and the root of the equation are

$$y_1 = 2\sqrt{\frac{-p}{3}} \cos(\theta),2\sqrt{\frac{-p}{3}} \cos\left(\theta+\frac{2\pi}{3}\right),2\sqrt{\frac{-p}{3}} \cos\left(\theta+\frac{4\pi}{3}\right),$$
where $$ \theta = \frac{1}{3}\arccos\left(\frac{3\sqrt{3}q}{2p\sqrt{-p}} \right).$$
}
\end{enumerate}
\end{proof}

\paragraph{Ex. 1.3.11}

{\it Consider the equation $4t^3 - 3t - \cos(3\theta) = 0$, where $\cos(3\theta) \neq \pm 1$. In (1.25), we expresses the roots in terms of trigonometric functions. In this exercise, you will study what happens when we use Cardan's formulas.
\begin{enumerate}
\item[(a)] Show that Cardan's formulas give the root
$$t_1 = \frac{1}{2} \sqrt[3]{\cos(3\theta) + i \sin(3\theta)} + \frac{1}{2} \sqrt[3]{\cos(3\theta) - i \sin(3\theta)}.$$
\item[(b)] Explain why $\frac{1}{2} e^{i\theta} = \frac{1}{2} (\cos \theta + i \sin \theta)$ is a value of $\frac{1}{2} \sqrt[3]{\cos(3\theta) + i \sin(3\theta)}$, and use to show that $t_1$ is just $\cos \theta$.
\item[(c)] Similarly, show that Cardan's formulas also give the roots $t_2$ and $t_3$ as predicted by (1.25).
\end{enumerate}
}

\begin{proof}
\begin{enumerate}

\item[(a)] We apply the Cardan's formulas to $4 t^3 - 3 t -\cos(3\theta) = 0$, which is equivalent to
$$t^3 - \frac{3}{4} t -\frac{1}{4}\cos(3\theta) = 0.$$
$\frac{p}{3} = -\frac{1}{4}, \frac{q}{2} = -\frac{1}{8} \cos(3\theta)$, thus

$\left( \frac{q}{2}\right)^2 + \left ( \frac{p}{3} \right)^3 = \frac{1}{64} (\cos^2(3\theta) - 1) =\left( \frac{i \sin(3\theta)}{8} \right)^2 $



We choose $\sqrt{\left( \frac{q}{2}\right)^2 + \left ( \frac{p}{3} \right)^3} =  \frac{i \sin(3\theta)}{8}$,

then $-\frac{q}{2}+\sqrt{\left( \frac{q}{2}\right)^2 + \left ( \frac{p}{3} \right)^3} = \frac{1}{8}( \cos(3\theta) + i \sin(3 \theta)) = \left (\frac{1}{2} e^{i \theta} \right )^3$.


\item[(b,c)] 
We choose the cube root $z_1 = \frac{1}{2} e^{i \theta}$. 
Then $z_2 = \overline{z_1}$.

As $\omega =  e^{i 2 \pi/3} = e^{-i4\pi/3}$, the three real roots are given by
\begin{align*}
t_1 &= z_1 + \overline{z_1} = \frac{1}{2} (e^{i \theta} + e^{-i\theta}) = \cos (\theta),\\
t_2&= e^{i 2 \pi/3} z_1 + e^{-i2\pi/3} \overline{z_1}=\frac{1}{2} (e^{i (\theta + 2\pi/3)} + e^{-i(\theta+ 2\pi/3)}) = \cos\left(\theta + \frac{2\pi}{3}\right),\\
t_3&= e^{i 4 \pi/3} z_1 + e^{-i4\pi/3} \overline{z_1}=\frac{1}{2} (e^{i (\theta + 4\pi/3)} + e^{-i(\theta+ 4\pi/3)}) = \cos\left(\theta + \frac{4\pi}{3}\right).\\
\end{align*}
\end{enumerate}
\end{proof}

\paragraph{Ex. 1.3.12}

{\it Example 1.3.2 discusses Bombelli's discovery that $\sqrt[3]{2+11i} = 2+i$. But not all cube roots can be expressed so simply. This exercise will show that $\sqrt[3]{4+\sqrt{11}i}$ is not of the form $a+b\sqrt{11}i$ for $a,b \in \Z$.
\begin{enumerate}
\item[(a)] Suppose that $4 + \sqrt{11}i = (a+ b \sqrt{11}i)^3$ for some $a,b\in \Z$. Show that this implies that $4 = a^3 - 33 ab^2$ and $1 = 3a^2b - 11 b^3$.
\item[(b)] Show that the equations of part (a) imply that $b=\pm1$ and $a\mid 4$. Conclude that the equation $4 + \sqrt{11}i = (a+b \sqrt{11}i)^3$ has no solutions with $a,b\in \Z$.
\item[(c)] Find a cubic polynomial of the form $x^3+px+q$ with $p,q\in \Z$ which has the number $\sqrt[3]{4+\sqrt{11}i} + \sqrt[3]{4-\sqrt{11}i}$ as a root.
\end{enumerate}
}

\begin{proof}
\begin{enumerate}

\item[(a)]  Let $a,b\in \mathbb{Z}$ : 
\begin{align*}
(a+ib\sqrt{11})^3 &= a^3 + 3ia^2 b \sqrt{11} - 3 \times11a b^2 - i b^3 11\sqrt{11}\\
&=(a^3-33ab^2)+i\sqrt{11}(3a^2b-11b^3),
\end{align*}
so

$$4+i\sqrt{11} =(a+ib\sqrt{11})^3 \iff 
\left \{
\begin{array}{ccc}
  4&=   &  a^3-33ab^2,\\
  1& =  &    3a^2b - 11b^3.
\end{array}
\right.
$$

\item[(b)] 

The first equation show that $a \mid 4$, and the second that $b\mid 1$, thus $b = \pm1$.

As $b^3=b=\pm1$, the second equation gives $3a^2-11 = \pm1$, thus $3a^2 = 10$ or $3a^2 = 12$. $3a^2 = 10$ is  impossible since $3 \nmid 10$. $3a^2 = 12$ gives  $a = \pm 2$. But then $4 = a^3 -33 a b^2 = a(a^2-33) = \pm 2 \times 29$, which is false.

The equation $4+i\sqrt{11} =(a+ib\sqrt{11})^3$ has no solution.


\item[(c)] 
We must find $p,q$ such that 
$-\frac{q}{2} = 4,\left(\frac{q}{2}\right)^2 + \left(\frac{p}{3}\right)^3 = -11$.

$q=-8, p = -9$ is a solution.

An equation with solution $\sqrt[3]{4+\sqrt{11}i} + \sqrt[3]{4-\sqrt{11}i}$ is so

$$y^3 -9y-8 = 0.$$
\end{enumerate}
\end{proof}

\paragraph{Ex. 1.3.13}

{\it Suppose that a quartic polynomial $f = x^4+bx^3+cx^2+dx+e$ in $\R[x]$ has distinct roots $x_1,x_2,x_3,x_4 \in \C$. The discriminant of $f$ is defined by the equation
$$\Delta = (x_1-x_2)^2(x_1-x_3)^2(x_1-x_4)^2(x_2-x_3)^2(x_2-x_4)^2(x_3-x_4)^2.$$
The theory developed in Chapter 2 will imply that $\Delta \in \R$, and $\Delta \neq 0$, since the $x_i$ are distinct. Adapt the proof of Theorem 1.3.1 to show that

$\Delta<0 \iff x^4+bx^3+cx^2+dx+e = 0$ has exactly two real roots.
}

\begin{proof}
Suppose that $f$ has exactly 2 real roots $x_1,x_2$. Then the two others form a complex conjugate pair : $x_3 = a+bi, x_4 = a-bi, \ b\neq 0$.

Then
\begin{align*}
\Delta &= (x_1-x_2)^2 (x_1-x_3)^2 (x_1 - x_4)^2 (x_2-x_3)^2 (x_2-x_4)^2 (x_3-x_4)^2\\
&=(x_1-x_2)^2(x_1-a-bi)^2(x_1-a+bi)^2(x_2-a-bi)^2(x_2-a+bi)^2(2bi)^2\\
&=-4b^2(x_1-x_2)^2\vert x_1-a-bi\vert^4 \vert x_2 -a -bi \vert^4 <0.
\end{align*}

The only other possible cases are the case where the 4 roots are real, and then $\Delta >0$, and the case where the 4 roots are non real, forming two complex conjugate pairs.

$x_1 = a+ib, x_2 = a-ib, x_3= c+id, x_4 = c-id,\ a,b,c,d\in \mathbb{R}$.

Then
\begin{align*}
\Delta &= (x_1-x_2)^2 (x_1-x_3)^2 (x_1 - x_4)^2 (x_2-x_3)^2 (x_2-x_4)^2 (x_3-x_4)^2\\
&=(x_1-x_2)^2 (x_3-x_4)^2(x_1-x_3)^2(\overline{x_1-x_3})^2 (x_1-x_4)^2(\overline{x_1-x_4})^2\\
&=(2ib)^2(2id)^2 \vert x_1-x_3 \vert^4\vert x_1-x_4 \vert^4\\
&=16b^2d^2 \vert x_1-x_3 \vert^4\vert x_1-x_4 \vert^4 >0.\\
\end{align*}

Conclusion : $\Delta<0$ if and only if $f$ has exactly two real roots.
\end{proof}

\paragraph{Ex. 1.3.14}

{\it In Section 1.1, we discussed the equation $x^3+2x^2+10x = 20$ considered by Fibonacci.
\begin{enumerate}
\item[(a)] Show that this equation has precisely one real root. This is the root Fibonacci approximated so well.
\item[(b)] Use Cardan's formulas and a calculator to work out numerically the three roots of this polynomial.
\end{enumerate}
}

\begin{proof}

\begin{enumerate}
\item[(a)]
Let $f (x) = x^3+2x^2+10x$. 

For all $x\in \mathbb{R}$, $f'(x) = 3 x^2+4x+10 = 2x^2 +(x+2)^2 + 6 >0$, thus  $f$ is strictly increasing, $f$ is continuous, and $\lim\limits_{y\to +\infty} f(y) = +\infty$,  $\lim\limits_{y\to -\infty} f(y) = -\infty$. By the intermediate values theorem, $f : \mathbb{R} \to \mathbb{R}$ is a bijection. Therefore there exists $x\in \R$ such that $f(x) = 20$, so there exists a unique real root of $$g(x) = x^3+2x^2+10x - 20.$$
As $g(0) = -20$ and $\lim\limits_{x\to +\infty} g(x) = +\infty$, this root is positive.


\item[(b)] To remove the coefficient of $x^2$, let $h(y) = g\left(y-\frac{2}{3}\right)$.
\begin{align*}
h(y) &= g\left(y-\frac{2}{3}\right)\\
&=\left(y-\frac{2}{3}\right)^3 + 2\left(y-\frac{2}{3}\right)^2 + 10\left(y-\frac{2}{3}\right)-20\\
&=y^3-2y^2+\frac{4}{3}y-\frac{8}{27} +2y^2-\frac{8}{3}y+\frac{8}{9}+10y -\frac{20}{3}-20\\
&=y^3+\frac{26}{3} y - \frac{704}{27}.\\
\end{align*}
Define $p,q$ by
$$\frac{p}{3} = \frac{26}{9}, \frac{q}{2} = - \frac{352}{27}$$
\begin{align*}
 \left(\frac{q}{2}\right)^2+ \left(\frac{p}{3}\right)^3 &= \frac{123904}{3^6}+\frac{17576}{3^6}\\
 &=\frac{141480}{3^6}\\
 &=\frac{5240}{27}
 \end{align*}
Consequently, the real solution of the equation is given by

\begin{align*}
x_1 &= -\frac{2}{3} + \sqrt[3]{\frac{352}{27} + \sqrt{\frac{5240}{27}}}+ \sqrt[3]{\frac{352}{27} - \sqrt{\frac{5240}{27}}}\\
&=\frac{1}{3} \left( -2+\sqrt[3]{352 + 6\sqrt{3930}} +\sqrt[3]{352 - 6\sqrt{3930}}\right)\\
&=\frac{1}{3}\left(-2+\sqrt[3]{6\sqrt{3930}+352}-\sqrt[3]{6\sqrt{3930}-352}\right).\\
\end{align*}

\end{enumerate}

With a calculator, an approximation at $10^{-10}$ of the real root $x_1$ is $x_1 \simeq 1.3688081078$.

The two other roots are given by
\begin{align*}
x_2 &= \frac{1}{3}\left(-2+\omega \sqrt[3]{6\sqrt{3930}+352}-\omega^2 \sqrt[3]{6\sqrt{3930}-352}\right) \simeq -1.684404055 - i \ 3.431331350,\\
x_3 &=  \frac{1}{3}\left(-2+\omega^2 \sqrt[3]{6\sqrt{3930}+352}-\omega \sqrt[3]{6\sqrt{3930}-352}\right) = \overline{x_2}.\\
\end{align*}


To verify the Fibonacci's result  in 1225, we compute a decomposition of $x_1$ with basis 60 with the following Maple program (or Sage program):

\begin{verbatim}
 x1:=1/3*(-2+(6*sqrt(3930)+352)^(1/3)-(6*sqrt(3930)-352)^(1/3)):
 l:=[1]:Digits:=20:
 u:=evalf(x1):
 u:=u-1:
 for i to 6 do a:=floor(60*u):u:=60*(u-a/60.0):l:=l,a: od:
 [l];

           [[1], 22, 7, 42, 33, 4, 38]
\end{verbatim}

which gives

$$x_1= 1 + \frac{22}{60} + \frac{7}{60^2} + \frac{42}{60^3} + \frac{33}{60^4} + \frac{4}{60^5} + \frac{38}{60^6}+ \cdots$$

Only the last fraction is slightly different  in Fibonacci's result $\frac{40}{60^6}$, the difference being $2/60^6 \simeq 4. 10^{-11}$.

Idem with Sage :
\begin{verbatim}
x1 = 1/3*((6*sqrt(3930) + 352)^(1/3) - (6*sqrt(3930) - 352)^(1/3) - 2)
R = RealField(200)
u = R(x1)
n = floor(u)
l = [[n]]; u = u-n
for i in range(6):
    a = floor(60*u)
    u = 60*(u-a/60)
    l.append(a)    
print l
	[[1], 22, 7, 42, 33, 4, 38]
\end{verbatim}
\end{proof}

\paragraph{Ex. 1.3.15}

{\it Use a calculator and Theorem 1.3.3 to compute the roots of the cubic equation $y^3-7y+3 = 0$ to eight decimal places of accuracy.
}

\begin{proof}
We use the Vi\`ete's method to
$$y^3 - 7y +3 : \qquad p=-7,q=3.$$

By theorem 1.3.3, the roots are given by
$$y_k =2\sqrt{\frac{-p}{q}} \cos \left(\theta + \frac{2k\pi}{3}\right)= 2\sqrt{\frac{7}{3}} \cos\left (\theta + \frac{2k\pi}{3}\right), k=0,1,2,$$
where
$$\theta = \frac{1}{3}\arccos\left(  \frac{3\sqrt{3}q}{2p\sqrt{-p}}\right)= \frac{1}{3}\arccos\left(  -\frac{9\sqrt{3}}{14\sqrt{7}}\right).$$

$x =  -\frac{9\sqrt{3}}{14\sqrt{7}} \simeq -0.420848788312271,$

$\theta = \frac{1}{3}\arccos(x) \simeq 0.668392376195833,$
\begin{align*}
y_0 &\simeq 2.39766154089,\\
y_1 &\simeq-2.83846925239,\\
y_2&\simeq 0.44080771150.
\end{align*}
\end{proof}


\end{document}
