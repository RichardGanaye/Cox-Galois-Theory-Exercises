%&LaTeX
\documentclass[11pt,a4paper]{article}
\usepackage[frenchb,english]{babel}
\usepackage[applemac]{inputenc}
\usepackage[OT1]{fontenc}
\usepackage[]{graphicx}
\usepackage{amsmath}
\usepackage{amsfonts}
\usepackage{amsthm}
\usepackage{amssymb}
%\input{8bitdefs}

% marges
\topmargin 10pt
\headsep 10pt
\headheight 10pt
\marginparwidth 30pt
\oddsidemargin 40pt
\evensidemargin 40pt
\footskip 30pt
\textheight 670pt
\textwidth 420pt

\def\imp{\Rightarrow}
\def\gcro{\mbox{[\hspace{-.15em}[}}% intervalles d'entiers 
\def\dcro{\mbox{]\hspace{-.15em}]}}

\newcommand{\be} {\begin{enumerate}}
\newcommand{\ee} {\end{enumerate}}
\newcommand{\deb}{\begin{eqnarray*}}
\newcommand{\fin}{\end{eqnarray*}}
\newcommand{\ssi} {si et seulement si }
\newcommand{\D}{\mathrm{d}}
\newcommand{\Q}{\mathbb{Q}}
\newcommand{\Z}{\mathbb{Z}}
\newcommand{\N}{\mathbb{N}}
\newcommand{\R}{\mathbb{R}}
\newcommand{\C}{\mathbb{C}}
\newcommand{\F}{\mathbb{F}}
\newcommand{\re}{\,\mathrm{Re}\,}
\newcommand{\ord}{\mathrm{ord}}
\newcommand{\legendre}[2]{\genfrac{(}{)}{}{}{#1}{#2}}

\title{Solutions to David A.Cox  "Galois Theory''}
\author{Richard Ganaye}
\refstepcounter{section}


\begin{document}

\maketitle

\section{Chapter 2}

\subsection{POLYNOMIALS OF SEVERAL VARIABLES}
\paragraph{Ex. 2.1.1}

{\it Show that $\langle x, y \rangle = \{xg+yh\ \vert\  g,h \in F[x,y]\} \subset F[x,y]$ is not a principal ideal in $F[x,y]$.
}

\begin{proof}
We show first that $\langle x,y \rangle \neq F[x,y]$. If not, $1 \in \langle x,y \rangle$, so
$$1 = x u + y v,\ u,v \in F[x,y] .$$
If we evaluate this identity at  $x =0,y=0$, we obtain $1=0$, which is a contradiction, thus
$$\langle x,y \rangle \neq F[x,y].$$

If $\langle x,y \rangle$ was a principal ideal, generated by  $p \in F[x,y]$, then $\langle x,y \rangle = \langle p \rangle$, and
$$x = pq, y = pr, \quad q,r \in F[x,y].$$

$\deg(p)+ \deg(q) = \deg(x) = 1$, so $\deg(p)\leq 1$, and $p\neq 0$.

If $\deg(p)=0$, then $p =\lambda \in F^*$, and $\langle x,y \rangle = \langle \lambda \rangle = F[x,y]$, but we have proved that this is impossible.

Thus $\deg(p)=1$, so $p = \alpha x + \beta y + \gamma,\  \alpha,\beta,\gamma \in F$, and $\deg(q) = \deg(r)=0$, so $q = \lambda \in F^*, r = \mu \in F^*$:

\begin{align*}
x &= \lambda (\alpha x + \beta y + \gamma),\\
y &= \mu (\alpha x + \beta y + \gamma).\\
\end{align*}
This implies $\lambda \beta = 0$ and $\mu \alpha = 0$.

As $\lambda \neq 0, \mu \neq 0$, $\alpha = \beta = 0$, whitch is in contradiction with $\deg(p) = 1$.

We have proved that $\langle x,y \rangle$ is not a principal ideal, and thus $F[x,y]$ is not a principal ideal domain.
\end{proof}

\paragraph{Ex. 2.1.2}

{\it Express each the following polynomials as a polynomial in $y$ with coefficients that are polynomials in the remaining variables.
\begin{enumerate}
\item[(a)] $x^2y + 3y^2 -xy^2 + 3x +xy^2 +7x^2y^2$.
\item[(b)] $(y-(x_1+x_2))(y-(x_1+x_3))(y-(x_2+x_1))$.
\end{enumerate}
}

\begin{proof}
\begin{enumerate}
\item[(a)] 
\begin{align*}
p &= x^2 y + 3 y^2 - x y^2 + 3x + x y^2 + 7 x^3 y^3\\
&= ( 7 x^3) y^3 + 3 y^2 + x^2 y + 3x.
\end{align*}

\item[(b)] 
let 
\begin{align*}
q &= (y -(x_1+x_2))(y-(x_1+x_3))(y-(x_2+x_3)).\\
\end{align*}
Consider $p = (x+x_1)(x+x_2)(x+x_3) = x^3+ \sigma_1x^2+\sigma_2 x +\sigma_3$.

Then
\begin{align*}
q &= (y - \sigma_1 + x_3)(y-\sigma_1+x_2)(y - \sigma_1 + x_1)\\
&=p(y-\sigma_1)\\
&=(y-\sigma_1)^3 + \sigma_1 (y-\sigma_1)^2 + \sigma_2(y-\sigma_1) + \sigma_3\\
&=(y^3 -3 \sigma_1 y^2 + 3 \sigma_1^2 y - \sigma_1^3) + (\sigma_1y^2 - 2 \sigma_1^2y + \sigma_1^3)+ (\sigma_2 y - \sigma_1 \sigma_2) + \sigma_3\\
&=y^3 - 2 \sigma_1 y^2 + (\sigma_1^2+\sigma_2)y + (\sigma_3 - \sigma_1\sigma_2).
\end{align*}
\end{enumerate}
\end{proof}

\paragraph{Ex. 2.1.3}

{\it Given positive integers $n$ and $r$ with $1\leq r \leq n$, let $\binom{n}{r}$ be the number of ways of choosing $r$ elements from a set with $n$ elements. Recall that $\binom{n}{r} = \frac{n!}{r!(n-r)!}$.
\begin{enumerate}
\item[(a)] Show that the polynomial $\sigma_r$ is a sum of $\binom{n}{r}$ terms.
\item[(b)] Show that $\sigma_r(-\alpha,\ldots,-\alpha) = (-1)^r \binom{n}{r} \alpha^r$.
\item[(c)] Let $f = (x+\alpha)^n$. Use part (b) and Corollary 2.1.5 to prove that
$$(x+\alpha)^n = \sum_{r=0}^n \binom{n}{r} \alpha^r x^{n-r},$$
where $\binom{n}{0}=1$. This shows that the binomial theorem follows from Corollary 2.1.5.
\end{enumerate}
}

\begin{proof}
\begin{enumerate}
\item[(a)]
The number of terms in 
\begin{align}
\sigma_r = \sum_{1 \leq i_1 <  \cdots <i_r\leq n}x_{i_1}x_{i_2}\cdots x_{i_r} \label{eq2.1.3:1}
\end{align}
is the number of strictly increasing sequences $(i_1,i_2,\cdots,i_r)$ in the integer interval $\gcro1,n\dcro$. It is equal to the number of subsets with $r$ elements in the set $\gcro1,n\dcro$ with $n$ elements. Thus it is equal to $\binom{n}{r}$.


\item[(b)] Evaluating \eqref{eq2.1.3:1} with $x_1=x_2 = x_n = -\alpha$, we obtain
\begin{align*}
\sigma_r (-\alpha, \cdots,-\alpha)&= \sum_{1 \leq i_1 <  \cdots <i_r\leq n}(-\alpha)^r\\
&=(-1)^r \binom{n}{r} \alpha^r.
\end{align*}


\item[(c)]
By Corollary 2.1.5, with the substitution $x_1 = -\alpha, x_2 = -\alpha, \cdots, x_n = - \alpha$,
$$f = (x+\alpha)^n = x^n +a_1 x^{n-1}+\cdots +a_n,$$ where 
\begin{align*}
a_r &= (-1)^r \sigma_r(-\alpha,\cdots,-\alpha)\\
&=\binom{n}{r} \alpha^r.
\end{align*} 

Consequently, 
$$(x+ \alpha)^n = \sum_{i=1}^n \binom{n}{r} \alpha^r x^{n-r}.$$

With the substitution $x = \beta , \ \beta \in F$, we obtain the binomial formula

$$( \alpha+ \beta)^n = \sum_{i=1}^n \binom{n}{r} \alpha^r \beta^{n-r}.$$

\end{enumerate}
\end{proof}

\subsection{SYMMETRIC POLYNOMIALS}
\paragraph{Ex. 2.2.1}

{\it Show that the leading term of $\sigma_r$ is $x_1x_2\cdots x_r$.
}

\begin{proof}
We show that the leading term of $\sigma_r$ for the graded lexicographic order is $x_1x_2\cdots x_n$.

Let $x_{i_1}x_{i_2}\cdots x_{i_r}  (i_1 < i_2 < \cdots < i_r)$ any term of $\sigma_r$, distinct of $x_1x_2\cdots x_r$. We must show that $x_1x_2\cdots x_r > x_{i_1}x_{i_2}\cdots x_{i_r}$.

If $i_1>1$, then $x_1$ as no occurence in $x_{i_1}x_{i_2}\cdots x_{i_r}$. Its exponent is 0 in the right monomial, and $1$ in the left monomial, so $$x_1x_2\cdots x_r > x_{i_1}x_{i_2}\cdots x_{i_r},$$ and the proof is done in this case.

If $i_1 = 1$, let $j\ (1<j < n)$ the first subscript such that $i_j\neq j$. Then $$i_1=1, i_2=2,\cdots ,i_{j-1} = j-1, i_j\neq j.$$ Such a subscript exists, otherwise $x_1x_2\cdots x_r = x_{i_1}x_{i_2}\cdots x_{i_r}$. As $i_{j} > i_{j-1} = j-1$, $i_j \geq j$, and as $i_j\neq j, i_j > j$, so the exponent of $x_j$ is $0$ in the right monomial.

Therefore
$$x_1x_2\cdots  x_{j-1}x_j \cdots x_r > x_1x_2 \cdots x_{j-1} x_{i_j} \cdots x_{i_r} = x_{i_1}x_{i_2}\cdots x_{i_r}.$$

So the leading term of $\sigma_r$ is $x_1x_2\cdots x_r$.
\end{proof}

\paragraph{Ex. 2.2.2}

{\it This exercise will study the order relation defined in (2.5). Given an exponent vector $\alpha = (a_1,\ldots,a_n)$, where each $a_i \geq 0$ is an integer, let $x^\alpha$ denote the monomial
$$x^\alpha = x_1^{a_1}\cdots x_n^{a_n}.$$
If $\alpha$ and $\beta$ are exponent vectors, note that $x^{\alpha} x^{\beta} = x^{\alpha+\beta}$. Also, the leading term of a nonzero polynomial $f \in F[x_1,\ldots,x_n]$ will be denoted {\normalfont  \scriptsize LT}$(f)$.
\begin{enumerate}
\item[(a)] Suppose that $x^\alpha> x^\beta$, and let $x^\gamma$ be any monomial. Prove that $x^{\alpha+\gamma} > x^{\beta+\gamma}$.
\item[(b)] Suppose that $x^\alpha > x^\beta$ and $x^\gamma > x^\delta$. Prove that $x^{\alpha + \gamma} > x^{\beta + \delta}$.
\item[(c)] Let $f,g \in F(x_1,\ldots,x_n]$ be nonzero. Prove that {\normalfont  \scriptsize LT}$(fg) =$ {\normalfont  \scriptsize LT}$(f)${\normalfont  \scriptsize LT}$(g)$.
\end{enumerate}
}

\begin{proof}
\begin{enumerate}
\item[(a)]
Let $\alpha = (a_1,a_2,\cdots, a_n),\beta = (b_1,b_2,\cdots,b_n),\gamma = (c_1,c_2,\cdots, c_n)$ and suppose that $x^{\alpha}>x^{\beta}$.

Then $a_1+a_2+\cdots+a_n \geq b_1+b_2+\cdots+b_n$, otherwise $x^{\alpha}<x^{\beta}$.

If $a_1+a_2+\cdots+a_n > b_1+b_2+\cdots+b_n$, then $(a_1+c_1)+ \cdots+(a_n+c_n) > (b_1 + c_1)+\cdots + (a_n+c_n)$, thus $x^{\alpha+\gamma}>x^{\beta+\gamma}$.

We suppose now that $a_1+a_2+\cdots+a_n = b_1+b_2+\cdots+b_n$.

By definition of the graded lexicographical order, $a_1 \geq b_1$, otherwise $x^{\alpha}<x^{\beta}$. 

If $a_1>b_1$, then $a_1+c_1>b_1+c_1$, which implies $x^{\alpha+\gamma}>x^{\beta+\gamma}$.


It remains the case where $a_1=b_1$.

Let $j\ (j<n)$ the first subscript such that $a_i \neq b_j$ : $$a_1=b_1,a_2=b_2,\cdots,a_{j-1}=b_{j-1}, a_j\neq b_j.$$
As $x^{\alpha}>x^{\beta}$, such a subscript exists, otherwise $x^{\alpha}=x^{\beta}$.

If $a_j < b_j$, we would have $x^{\alpha}<x^{\beta}$, which is false by hypothesis, so $a_j > b_j$.

Then $a_1 +c_1 = b_1+c_1, \cdots, a_{j-1}+c_{j-1} = b_{j-1} + c_{j-1}$ and $a_j + c_j > b_j + c_j$, so 
$$x^{\alpha+\gamma}>x^{\beta+\gamma}.$$
Conclusion : $$x^{\alpha} > x^{\beta} \Rightarrow x^{\alpha+\gamma}>x^{\beta+\gamma}.$$

\item[(b)]
If $x^{\alpha}> x^{\beta}$ and $x^{\gamma} > x^{\delta}$, then by  (a), 
$$x^{\alpha+\gamma}> x^{\beta+\gamma},$$
$$x^{\beta+\gamma}>x^{\beta+\delta}.$$
So, by transitivity
$$x^{\alpha+\gamma}>x^{\beta+\delta}.$$

\item[(c)]
Let $c x^{\alpha} = $ { \scriptsize LT}$(f), d x^{\beta} = $ {  \scriptsize LT}$(g)$.
By definition of the leading term, for every term $ux^\gamma$ in $f$, distinct of {\scriptsize LT}$(f)$,
\begin{align*}
x^{\alpha}> x^{\gamma},
\end{align*}
and for every term $vx^\delta$ in $g$, distinct of {\scriptsize LT}$(g)$,
\begin{align*}
x^{\beta} > x^{\delta}.
\end{align*}
Every monomial in $fg$ distinct of $cd x^{\alpha+\beta}$ is a sum of terms of the form $gx^{\gamma+ \delta}$, where $\beta,\gamma$ verify $\alpha \geq \gamma, \beta > \delta$, or $\alpha>\gamma, \beta \geq \delta$. In both cases, by (a) and (b),
$$x^{\alpha+\beta}>x^{\gamma+\delta}.$$
Therefore $cd x^{\alpha+\beta}$ is the leading term of $fg$, so
\begin{center}
{ \scriptsize LT}$(fg)$ = { \scriptsize LT}$(f)${ \scriptsize LT}$(g)$.
\end{center}
\end{enumerate}
\end{proof}

\paragraph{Ex. 2.2.3}

{\it Prove (2.13)-(2.16). For (2.13), a computer will be helpful; the others can be proved by hand using the identity
$$(y_1+\cdots+y_m)^2 = y_1^2+\cdots+y_m^2 + 2 \sum_{i<j}y_i y_j.$$
}

\begin{proof}


Let $$ f= \Sigma_4x_1^3x_2^2x_3.$$
We must write  $f$ as a polynomial in $\sigma_1,\sigma_2,\sigma_3,\sigma_4$.

 The leading term of $f$ for the graded lexicographical order being  $x_1^3x_2^2x_3^1x_4^0$, the algorithm of section 2.2 asks to subtract to  $f$ the monomial $\sigma_1^{3-2}\sigma_2^{2-1} \sigma_3^{1-0} \sigma_4^0 = \sigma_1\sigma_2\sigma_3$.

\begin{enumerate}
\item[(a)]
\begin{align*}
\sigma_1\sigma_2\sigma_3 &=(x_1+x_2+x_3+x_4)\times(x_1x_2+x_1x_3+x_1x_4+x_2x_3+x_2x_4+x_3x_4)\\
&\qquad\times(x_1x_2x_3+x_1x_2x_4+x_1x_3x_4+x_2x_3x_4)\\
&= 3\,x_{{1}}{x_{{2}}}^{3}x_{{3}}x_{{4}}+8\,{x_{{1}}}^{2}{x_{{2}}}^{2}x_{
{3}}x_{{4}}+8\,{x_{{2}}}^{2}{x_{{4}}}^{2}x_{{1}}x_{{3}}+8\,{x_{{2}}}^{
2}{x_{{3}}}^{2}x_{{1}}x_{{4}}+8\,{x_{{1}}}^{2}x_{{2}}{x_{{4}}}^{2}x_{{
3}}\\
&+8\,{x_{{1}}}^{2}x_{{2}}{x_{{3}}}^{2}x_{{4}}+8\,x_{{2}}{x_{{3}}}^{2
}{x_{{4}}}^{2}x_{{1}}+3\,{x_{{1}}}^{3}x_{{2}}x_{{3}}x_{{4}}+3\,x_{{2}}
{x_{{4}}}^{3}x_{{1}}x_{{3}}+3\,x_{{2}}{x_{{3}}}^{3}x_{{1}}x_{{4}}\\
&+{x_{{1}}}^{3}{x_{{4}}}^{2}x_{{3}}+3\,{x_{{2}}}^{2}{x_{{3}}}^{2}{x_{{4}}}^{
2}+{x_{{1}}}^{2}{x_{{2}}}^{3}x_{{3}}+{x_{{3}}}^{2}{x_{{4}}}^{3}x_{{2}}
+{x_{{2}}}^{2}{x_{{3}}}^{3}x_{{1}}+3\,{x_{{1}}}^{2}{x_{{2}}}^{2}{x_{{3
}}}^{2}\\
&+{x_{{1}}}^{2}{x_{{3}}}^{3}x_{{4}}+{x_{{3}}}^{3}{x_{{4}}}^{2}x_
{{2}}+{x_{{2}}}^{3}{x_{{3}}}^{2}x_{{4}}+{x_{{2}}}^{2}{x_{{3}}}^{3}x_{{
4}}+{x_{{3}}}^{2}{x_{{4}}}^{3}x_{{1}}+{x_{{2}}}^{3}{x_{{3}}}^{2}x_{{1}
}\\
&+{x_{{2}}}^{3}{x_{{4}}}^{2}x_{{1}}+{x_{{2}}}^{2}{x_{{4}}}^{3}x_{{1}}+
{x_{{2}}}^{3}{x_{{4}}}^{2}x_{{3}}+{x_{{3}}}^{3}{x_{{4}}}^{2}x_{{1}}+{x
_{{1}}}^{3}{x_{{2}}}^{2}x_{{3}}+{x_{{1}}}^{3}{x_{{3}}}^{2}x_{{2}}\\
&+{x_{
{1}}}^{2}{x_{{4}}}^{3}x_{{2}}+3\,{x_{{1}}}^{2}{x_{{2}}}^{2}{x_{{4}}}^{
2}+{x_{{1}}}^{3}{x_{{3}}}^{2}x_{{4}}+{x_{{1}}}^{3}{x_{{2}}}^{2}x_{{4}}
+3\,{x_{{1}}}^{2}{x_{{3}}}^{2}{x_{{4}}}^{2}+{x_{{2}}}^{2}{x_{{4}}}^{3}
x_{{3}}\\
&+{x_{{1}}}^{3}{x_{{4}}}^{2}x_{{2}}+{x_{{1}}}^{2}{x_{{2}}}^{3}x_
{{4}}+{x_{{1}}}^{2}{x_{{4}}}^{3}x_{{3}}+{x_{{1}}}^{2}{x_{{3}}}^{3}x_{{
2}}\\
&=8\Sigma_4 x_1^2x_2^2x_3x_4+3\Sigma_4x_1^3x_2x_3x_4+3\Sigma_4x_1^2x_2^2x_3^2+\Sigma_4x_1^3x_2^2x_3.\\
\end{align*}
We find the 96 terms of the product $\sigma_1\sigma_2\sigma_3$ (see Ex. 2.2.12): 

$\Sigma_4 x_1^2x_2^2x_3x_4$  has $\frac{4!}{2!2!} = 6 $ terms, with the coefficient 8 : 48 terms.

$\Sigma_4x_1^3x_2x_3x_4$ has $\frac{4!}{1!3!} = 4$ terms,  with the coefficient 3 : 12 terms.

$\Sigma_4x_1^2x_2^2x_3^2$ has $\frac{4!}{3!1!} = 4$ terms,  with the coefficient 3 : 12 terms.

$\Sigma_4x_1^3x_2^2x_3$ has $\frac{4!}{1!1!1!1!} = 24$ terms,  with the coefficient 1 : 24 terms..

We obtain this product with the following Maple instructions :

$> P = (x+x_1).(x+x_2).(x+x_3)(x+x_4);$

$> p:=\mathrm{expand}(P);$

$> q:=\mathrm{collect}(p,x);$

$> \sigma_1:=\mathrm{coeff}(q,x,3); \sigma_2:=\mathrm{coeff}(q,x,2);\sigma_3:=\mathrm{coeff}(q,x,1);\sigma_4:=\mathrm{coeff}(q,x,1);$

$> \mathrm{expand}(\sigma_1.\sigma_2.\sigma_3);$

With sage :
\begin{verbatim}
e = SymmetricFunctions(QQ).e()
g = (e([1])* e([2])*e([3])).expand(4);g
\end{verbatim}
\item[(b)]
So
\begin{align*}
f_1 &= f - \sigma_1\sigma_2\sigma_3\\
&=-8\Sigma_4 x_1^2x_2^2x_3x_4-3\Sigma_4x_1^3x_2x_3x_4-3\Sigma_4x_1^2x_2^2x_3^2.\\
\end{align*}

The leading  of $f_1$ is $-3x_1^3x_2x_3x_4$, so we must subtract  $-3 \sigma_1^2 \sigma_4$ to $f_1$.

\begin{align*}
\sigma_1^2\sigma_4&=(\Sigma_4 x_1)^2  (x_1x_2x_3x_4)\\
&=(\Sigma_4 x_1^2 + 2 \Sigma_4 x_1x_2)x_1x_2x_3x_4\\
&= \Sigma_4 x_1^3x_2x_3x_4 + 2 \Sigma_4 x_1^2x_2^2x_3x_4,
\end{align*}
therefore $$f_2 = f - \sigma_1\sigma_2\sigma_3 +3 \sigma_1^2 \sigma_4  = -3\Sigma_4x_1^2x_2^2x_3^2-2\Sigma_4 x_1^2x_2^2x_3x_4.$$


\item[(c)] The leading term of $f_2$ is $-3x_1^2x_2^2x_3^2$, so must subtract  $-3\sigma_3^2$  to $f_2$.
\begin{align*}
\sigma_3^2&=(\Sigma_4 x_1x_2x_3)^2\\
&= \Sigma_4 x_1^2x_2^2x_3^2 + 2 \Sigma_4 x_1^2x_2^2x_3x_4,
\end{align*}
$$f_3 = f - \sigma_1\sigma_2\sigma_3 +3 \sigma_1^2 \sigma_4 +3\sigma_3^2 = 4 \Sigma_4 x_1^2x_2^2x_3x_4.$$

\item[(d)]  The leading term of $f_3$ is $4x_1^2x_2^2x_3x_4$, so we must subtract $ 4\sigma_2 \sigma_4$ to $f_3$.
\begin{align*}
\sigma_2 \sigma_4 &= (\Sigma_4 x_1x_2) (x_1x_2x_3x_4)\\
&=\Sigma_4x_1^2x_2^2x_3x_4,
\end{align*}
so $f_4 = f - \sigma_1\sigma_2\sigma_3 +3 \sigma_1^2 \sigma_4 +3\sigma_3^2 - 4\sigma_2 \sigma_4 = 0$.

$$f = \Sigma_4x_1^3x_2^2x_3 = \sigma_1\sigma_2\sigma_3 -3 \sigma_1^2 \sigma_4 -3\sigma_3^2 + 4\sigma_2 \sigma_4.$$ 
\end{enumerate}
\end{proof}

\paragraph{Ex. 2.2.4}

{\it Let $f = x^3+bx^2+cx+d \in F[x]$ have roots $\alpha_1,\alpha_2,\alpha_3$ in the field $L$ containing $F$, and let $g$ be the polynomial defined in (2.17). Show carefully that
$$g(x) = x^3 + 2bx^2 + (b^2+c)x +bc-d.$$
}

\begin{proof}
Let
\begin{align*}
f &= x^3+bx^2+cx+d\\
&=( x-\alpha)(x-\beta)(x-\gamma)\\
&= x^3 - \sigma_1(\alpha,\beta,\gamma)x^2 + \sigma_2(\alpha,\beta,\gamma)x - \sigma_3(\alpha,\beta,\gamma),\\
\end{align*}
which gives
\begin{align*}
& \sigma_1(\alpha,\beta,\gamma) = -b,\\
& \sigma_2(\alpha,\beta,\gamma)=+c,\\
& \sigma_3(\alpha,\beta,\gamma)=-d.
\end{align*}
Let
\begin{align*}
G(x) &= (x -(x_1+x_2))(x-(x_1+x_3))(x-(x_2+x_3)).\\
\end{align*}
Then $$g(x) = (x -(\alpha_1+\alpha_2))(x-(\alpha_1+\alpha_3))(x-(\alpha_2+\alpha_3))$$ is obtained from $G$ by the evaluation morphism which sends $x_1,x_2,x_3$ on $\alpha_1,\alpha_2,\alpha_3$.

Let  $p = (x+x_1)(x+x_2)(x+x_3) = x+ \sigma_1x^2+\sigma_2 x +\sigma_3$.

Then
\begin{align*}
G &= (x - \sigma_1 + x_3)(x-\sigma_1+x_2)(x - \sigma_1 + x_1)\\
&=p(x-\sigma_1)\\
&=(x-\sigma_1)^3 + \sigma_1 (x-\sigma_1)^2 + \sigma_2(x-\sigma_1) + \sigma_3\\
&=(x^3 -3 \sigma_1 x^2 + 3 \sigma_1^2 x - \sigma_1^3) + (\sigma_1x^2 - 2 \sigma_1^2x + \sigma_1^3)+ (\sigma_2 x - \sigma_1 \sigma_2) + \sigma_3\\
&=x^3 - 2 \sigma_1 x^2 + (\sigma_1^2+\sigma_2)x + (\sigma_3 - \sigma_1\sigma_2).
\end{align*}
The previous  evaluation morphism sends $\sigma_1$ on $\sigma_1(\alpha_1,\alpha_2,\alpha_3)=-b$, 
$\sigma_2$ on $\sigma_2(\alpha_1,\alpha_2,\alpha_3)=c$, $\sigma_3$ on $\sigma_3(\alpha_1,\alpha_2,\alpha_3)=-d$.

\begin{align*}
g(x) &= x^3 +2bx^2+(b^2+c)x+bc-d.
\end{align*}

In the example 2.2.6, 
$$f(x) = x^3+2x^2+x+7,$$
where
$$b=2,c=1,d=7,$$
$\alpha_1,\alpha_2,\alpha_3$ being the roots of $g$ in $\mathbb{C}$,
we obtain
\begin{align*}
g(x) &= (x -(\alpha_1+\alpha_2))(x-(\alpha_1+\alpha_3))(x-(\alpha_2+x\alpha_3))\\
&=x^3 +2bx^2+(b^2+c)x+bc-d\\
&=x^3 + 4x^2+5x-5.
\end{align*}
\end{proof}

\paragraph{Ex. 2.2.5}

{\it This exercise will complete the proof of Theorem 2.2.7. Let $h \in F[u_1,\ldots,u_n]$ be a nonzero polynomial. The goal is to prove that $h(\sigma_1,\ldots,\sigma_n)$  is not the zero polynomial in $x_1,\ldots,x_n$.
\begin{enumerate}
\item[(a)] If $cu_1^{b_1}\cdots u_n^{b_n}$ is a term of $h$, then use Exercise 2 to show that the leading term of $c\sigma_1^{b_1}\cdots \sigma_n^{b_n}$ is $cx_1^{b_1+\cdots+b_n} x_2^{b_2+\cdots+b_n}\cdots x_n^{b_n}$.
\item[(b)] Show that $(b_1,\ldots,b_n) \mapsto (b_1+\cdots+b_n, b_2+\cdots+b_n, \ldots,b_n)$ is one-to-one.
\item[(c)] To see why $h(\sigma_1,\ldots,\sigma_n)$ is nonzero, consider the term of $h(u_1,\ldots,u_n)$ for which the leading term of $c\sigma_1^{b_1}\cdots\sigma_n^{b_n}$ is maximal. Prove that this leading term is in fact the leading term of $h(\sigma_1,\ldots,\sigma_n)$, and explain how this proves what we want.
\end{enumerate}
}

\begin{proof}
\begin{enumerate}
\item[(a)]
Let $h \in F[u_1,u_2,\cdots,u_n],h\neq 0$, and $cu_1^{b_1}u_2^{b_2}\cdots u_n^{b_n}$ a term of $h$.

The leading term of a product is the product of the leading term of the factors (Ex 2.2.2), and the leading term of $\sigma_r$ is $x_1x_2\cdots x_r$ (Ex 2.2.1),  so the leading term of $c\sigma_1^{b_1}\sigma_2^{b_2}\cdots \sigma_n^{b_n}$ is
\begin{align*}
{\mathrm  {LT}}(c\sigma_1^{b_1}\sigma_2^{b_2}\cdots \sigma_n^{b_n})&=c(x_1)^{b_1}(x_1x_2)^{b_2}\cdots(x_1x_2\cdots x_n)^{b_n}\\
&=c x_1^{b_1+b_2+\cdots+b_n} x_2^{b_2+\cdots+ b_n} \cdots x_n^{b_n}.\\
\end{align*}

\item[(b)]
If $a_i,b_i \in \mathbb{Z}$, the system of equations
\begin{align*}
 b_1+b_2+\cdots+b_n=a_1,\\
b_2+\cdots + b_n=a_2,\\
\cdots\\
b_n=a_n,
\end{align*}
is equivalent to
\begin{align*}
b_1 &=a_1-a_2,\\
b_2&=a_2-a_3,\\
&\cdots\\
b_{n-1} &= a_{n}-a_{n-1},\\
b_n &= a_n.
\end{align*}
So the application $f : \mathbb{Z}^n \to \mathbb{Z}^n$ defined by $$(b_1,b_2,\cdots,b_n)\mapsto (b_1+b_2+\cdots+b_n, b_2+\cdots+b_n, \cdots,b_n)$$ is bijective (one-to-one and onto).

\item[(c)]

As $h\neq 0$, there exists a term  $cu_1^{b_1}u_2^{b_2}\cdots u_n^{b_n}$ of $h$ such that the leading term $cx_1^{a_1}\cdots x_n^{a_n}$ of  $c\sigma_1^{b_1}\sigma_2^{b_2}\cdots \sigma_n^{b_n}$ is maximal. Then every other term $c'u_1^{d_1}u_2^{d_2}\cdots u_n^{d_n}$ of $h$  verifies $(b'_1,b'_2,\cdots,b'_n) \neq (b_1,b_2,\cdots,b_n)$ and the leading term  $c'x_1^{a'_1}\cdots x_n^{a'_n}$ of $c'\sigma_1^{d_1}\sigma_2^{b_2}\cdots \sigma_n^{d_n}$ is less than $cx_1^{a_1}\cdots x_n^{a_n}$ : it can not be greater because this term is maximal, and $(a_1,a_2,\cdots,a_n) \neq (a'_1,a'_2,\cdots,a'_n)$, since the application $f$ in (b) is bijective. The graded lexicographic order defined on the monomials $x_1^{a_1}\cdots x_n^{a_n}$ being a total order, $x_1^{a_1}\cdots x_n^{a_n} > x_1^{a'_1}\cdots x_n^{a'_n}$.

So $cx_1^{a_1}\cdots x_n^{a_n}$ is greater than the leading terms of every other term $c'\sigma_1^{d_1}\sigma_2^{d_2}\cdots \sigma_n^{d_n}$ of $h(\sigma_1,\cdots,\sigma_n) \neq 0$, so is a fortiori greater than every other term of $h(\sigma_1,\cdots,\sigma_n)$.

 It can't be cancelled in the sum of these terms, and consequently $h(\sigma_1,\cdots,\sigma_n) \neq 0$.
\end{enumerate}
\end{proof}

\paragraph{Ex. 2.2.6}

{\it Here is an example of polynomials which are not algebraically independent. Consider $x_1^2,x_1x_2,x_2^2 \in F[x_1,x_2]$, and let $\phi : F[u_1,u_2,u_3] \to F[x_1,x_2]$ be defined by
$$\phi(u_1) = x_1^2,\phi(u_2)= x_1x_2, \phi(u_3) = x_2^2.$$
Show that $\phi$ is not one-to-one by finding a nonzero polynomial $h \in F[u_1,u_2,u_3]$ such that $\phi(h)=0$.
}

\begin{proof}
Let $h = u_1u_3 -u_2^2$.

Then the unique algebra morphism $\phi$ such that 
$$\phi(u_1) = x_1^2, \phi(u_2) = x_1x_2, \phi(u_3) = x_2^2$$
verifies
$$\phi(h) = \phi(u_1)\phi(u_3) - (\phi(u_2))^2 = x_1^2 x_2^2 - (x_1x_2)^2 = 0.$$
So $h \neq 0$ is in the kernel of $\phi$, and $\phi$ is not one-to-one. Thus $x_1^2,x_1x_2,x_2^2$ are not algebraically independent.
\end{proof}

\paragraph{Ex. 2.2.7}

{\it Given a polynomial $f \in F[x_1,\ldots,x_n]$ and a permutation $\sigma \in S_n$, let $\sigma \cdot f$ denote the polynomial obtained from $f$ by permuting the variables according to $\sigma$. Show that $\prod_{\sigma \in S_n} \sigma \cdot f$ and $\sum_{\sigma \in S_n} \sigma \cdot f$ are symmetric polynomials.
}

\begin{proof}
We use the relations (2.31) p. 48, (or (6.7) p. 138) proved in Exercises 6.4.3 and 6.4.4 :  for all $\sigma,\tau \in S_n$, and all $f,g \in F[x_1,x_2,\cdots,x_n]$ : 
\begin{align}
\sigma \cdot (f+g) &= \sigma \cdot f + \sigma \cdot g, \label{eq2.2.7:2}\\
\sigma \cdot (fg) &= (\sigma \cdot f)(\sigma \cdot g),\label{eq2.2.7:3}\\
\tau\cdot (\sigma \cdot f) &= (\tau \circ \sigma) \cdot f.\label{eq2.2.7:4}
\end{align}
(We will use the notation $\tau \circ \sigma = \tau  \sigma$.)

Let $g = \prod_{\sigma \in S_n} \sigma \cdot f$.

Then, if $\tau \in S_n$, using \eqref{eq2.2.7:3} and \eqref{eq2.2.7:4}
\begin{align*}
\tau \cdot g &= \tau \cdot \prod_{\sigma \in S_n} \sigma \cdot f\\
&=\prod_{\sigma \in S_n}  \tau \cdot (\sigma \cdot f)\\
&=\prod_{\sigma \in S_n} (\tau \sigma) \cdot f.
\end{align*}

As the application $S_n \to S_n, \sigma \mapsto \tau \sigma$ is bijective, the index change $\sigma' = \tau \sigma$ gives 
$$\prod_{\sigma \in S_n} (\tau \sigma) \cdot f = \prod_{\sigma' \in S_n} \sigma' \cdot f = \prod_{\sigma \in S_n} \sigma \cdot f= g$$.

So,  for all $\tau \in S_n, \tau \cdot g = g$ : thus $g$ is a symmetric polynomial.

Same proof for $\tau. \sum _{\sigma \in S_n} \sigma \cdot f = \sum _{\sigma \in S_n} \sigma \cdot f $ : use (2) in place of (3).

Conclusion : $\prod\limits_{\sigma \in S_n} \sigma \cdot f$ and $\sum\limits_{\sigma \in S_n} \sigma \cdot f$ are symmetric polynomials.
\end{proof}

\paragraph{Ex. 2.2.8}

{\it In this exercise, you will prove that if $\varphi \in F(x_1,\ldots,x_n)$ is symmetric, then $\varphi$ is a rational function in $\sigma_1,\ldots,\sigma_n$ with coefficients in $F$. To begin the proof, we know that $\varphi = A/B$, where $A$ and $B$ are in $F[x_1,\ldots,x_n]$. Note that $A$ and $B$ need not be symmetric, only their quotient $\varphi = A/B$ is. Let
$$C = \prod_{\sigma \in S_n \setminus \{e\} } \sigma \cdot B,$$
where we are using the notation of Exercise 7.
\begin{enumerate}
\item[(a)] Use Exercise 7 to show that $BC$ is a symmetric polynomial.
\item[(b)] Then use the symmetry of $\varphi = A/B$ to show that $AC$ is a symmetric polynomial.
\item[(c)] Use $\varphi = (AC)/(BC)$ and theorem 2.2.2 to conclude that $\varphi$ is a rational function in the elementary symmetric polynomials with coefficients in $F$.
\end{enumerate}
}

\begin{proof}
Let $\varphi = A/B\in F(x_1,\cdots,x_n)$ a symmetric rational function:
$$\forall \sigma \in S_n,\  \sigma \cdot \varphi = \sigma \cdot A/\sigma \cdot B = \varphi = A/B.$$
\begin{enumerate}
\item[(a)]
Let $$C =  \prod_{\sigma \in S_n\setminus \{e\}} \sigma \cdot B.$$
Then $$ BC = \prod_{\sigma \in S_n}\sigma \cdot B.$$
By Exercise 2.2.7, $BC$ is then a symmetric polynomial.


\item[(b)]
Note that the rules (2.31) for polynomials extend to rational functions. In particular, if $\varphi = A/B,\psi =A_1/B_1 \in F(x_1,\cdots,x_n)$, and $\sigma \in S_n$, 
$$\sigma \cdot (\varphi \psi) = (\sigma \cdot \varphi)\  (\sigma \cdot \psi).$$
Indeed, 

$$ (\sigma \cdot \varphi)\  (\sigma \cdot \psi) = \frac{\sigma \cdot A}{\sigma \cdot B} \frac{\sigma \cdot A_1}{\sigma \cdot B_1}=\frac{\sigma \cdot (AA_1)}{\sigma \cdot (BB_1)} = \sigma \cdot (\varphi \psi). $$
Using this property, for all $\sigma \in S_n$, from $AC = \varphi BC$, we obtain

$$\sigma \cdot (AC) = (\sigma \cdot  \varphi)(\sigma \cdot (BC)) = \varphi BC = AC.$$
So  $AC$ is a symmetric polynomial.


\item[(c)]
So $\varphi = \frac{AC}{BC}$ is  the quotient of two symmetric polynomials, thus there exists $h,k \in F[x_1,\cdots,x_n]$ such that
$$\varphi = \frac{AC}{BC} = \frac{h(\sigma_1,\cdots,\sigma_n)}{k(\sigma_1,\cdots,\sigma_n)} =\left( \frac{h}{k}\right)(\sigma_1,\cdots,\sigma_n).$$
$\varphi \in F(\sigma_1,\cdots,\sigma_n)$ is a rational function in the elementary symmetric polynomials with coefficients in $F$.
\end{enumerate}
\end{proof}

\paragraph{Ex. 2.2.9}

{\it In the Historical Notes, we gave Gauss's definition of lexicographic order.
\begin{enumerate}
 \item[(a)] Give a definition (in English) of lexicographic order.
 \item[(b)] In the proof of Theorem 2.2.2, we showed that grade lexicographic order has the property that there are only finitely many monomials less than a given monomial. In contrast this property fails for lexicographic order. Give an explicit example to illustrate this.
 \item[(c)] In spite of part (b), lexicographic order does have an interesting finiteness property. Namely, prove that there is no infinite sequence of polynomials $f_1,f_2,f_3,\ldots$ that have strictly decreasing terms according to lexicographic  order.
 \item[(d)] Explain how part (c) allows one to prove Theorem 2.2.2 using lexicographic order.
\end{enumerate}
}

\begin{proof}
\begin{enumerate}
\item[(a)]
For the lexicographic order, $x_1^{a_1}\cdots x_n^{a_n} < x_1^{b_1}\cdots x_n^{b_n}$ is equivalent by definition to

$$\exists j \in [1,n], (\forall i \in \N,\ 1\leq  i< j \Rightarrow a_i = b_i) \ \mathrm{and}\ a_j<b_j.$$
(The property $(\forall i \in \N ,\ 1\leq  i< j \Rightarrow a_i = b_i) $ is automatically verified for  $j=1$, since $1\leq  i< j $ is false, so the implication is true.)

In informal terms :

$a_1<b_1$ or ($a_1=b_1$ and $a_2<b_2$) or ($a_1=b_1,a_2=b_2$ and $a_3<b_3$) or $\ldots$

In other words, $x_1^{a_1}\cdots x_n^{a_n} < x_1^{b_1}\cdots x_n^{b_n}$ iff the first subscript $i$ such that $a_i \neq b_i$ exists and verifies $a_i < b_i$.

This relation $\leq$ is a total order.


\item[(b)]
The monomials less than  $x_1 = x_1^1 x_2^0 \cdots x_n^0$ for the lexicographic order contain the monomials $x_1^0 x_2^{a_2}\ldots x_n^{a_n}$, where $a_2,\cdots, a_n$ are arbitrary integers in $\N = \Z_{\geq 0}$. There are infinitely many such monomials.


\item[(c)]
We show this property by induction on the numbers of variables $x_i$.

If there is a unique variable, say $x_1$, then a strictly decreasing sequence of monomial $x_1^{n_0} > x_1^{n_2} >\cdots$, with $n_i \in \N$,  is such that  $n_0 > n_1 > \cdots$: such a sequence is necessary finite. This is a property of the natural order in $\N$: Every non empty subset of $\N$ has a smallest element, so a strictly decreasing infinite sequence in $\N$ doesn't exist.

Suppose that this property is true for  $n-1$ variables, say $x_2, \cdots,x_n$. Consider the sequence

$$x_1^{i_{1,1}}\cdots x_n^{i_{1,n}}   > x_1^{i_{2,1}}\cdots x_n^{i_{2,n}}  > \cdots >x_1^{i_{k,1}}\cdots x_n^{i_{k,n}} >\cdots .$$
By the induction hypothesis, for each fixed exponent $i_{k,1}$ of  $x_1$, there exists only finitely monomial in this sequence  with this exponent for $x_1$. As these exponents are at most $i_{1,1}$, the sequence is finite and the induction is done.


\item[(d)]
The beginning of the demonstration of Theorem 2.2.2 remains unchanged with the lexicographic order. Then we builds a sequence

$$f,f_1 = f -cg, f_2= f -cg -c_1g_1, \ldots$$

of polynomials whose  leading terms constitute  a strictly decreasing sequence for this order, until $f_i = 0$.  By (c), this sequence is finite, so one polynomial $f_i$ is zero, which completes the algorithm.

\end{enumerate}
\end{proof}

\paragraph{Ex. 2.2.10}

{\it Apply the proof of theorem 2.2.2 to express $\sum_3 x_1^2x_2$ in terms of $\sigma_1, \sigma_2, \sigma_3$.
}

\begin{proof} Explicitly, 
$$f = \Sigma_3 x_1^2 x_2 = x_1^2x_2+x_1^2 x_3 + x_1 x_2^2+x_1 x_3^2 +x_2^2x_3+x_2 x_3^2.$$

Note that $x_1^2 x_2 = x_1^2 x_2^1 x_3^0$ is the leading term for the graded lexicographic order, so the following term in the sequence is $g = f - \sigma_1^{2-1} \sigma_2^{1-0} \sigma_3^0 = f - \sigma_1 \sigma_2$.

\begin{align*}
\sigma_1 \sigma_2 &= (x_1+x_2+x_3) (x_1 x_2+x_1x_3+x_2x_3)\\
&= x_1^2x_2 + x_1^2 x_3+x_1x_2x_3+x_1x_2^2+x_2^2x_3+x_1x_2x_3+x_1x_3^2+x_2x_3^2+x_1x_2x_3\\
&=f+ 3x_1x_2x_3,
\end{align*}
thus
$$f = \Sigma_3 x_1^2 x_2  = \sigma_1\sigma_2 - 3 \sigma_3.$$
\end{proof}

\paragraph{Ex. 2.2.11}

{\it Let the roots of $y^3+2y^2-3y+5$ be $\alpha,\beta,\gamma \in \C$. Find polynomials with integers coefficients that have the following roots:
\begin{enumerate}
\item[(a)] $\alpha \beta, \alpha \gamma$ and $\beta \gamma$.
\item[(b)] $\alpha +1, \beta + 1$, and $\gamma +1$.
\item[(c)] $\alpha^2, \beta^2$, and $\gamma^2$.
\end{enumerate}
}

\begin{enumerate}
\item[(a)]
$f = y^3+2y^2-3y+5 = (y-\alpha)(y-\beta)(y-\gamma) = y^3-\sigma_1y^2+\sigma_2 y -\sigma_3,$

so
$\sigma_1 = -2, \sigma_2=-3,\sigma_3 = -5.$
\begin{align*}
g&=(y-\alpha\beta)(y-\alpha\gamma)(y-\beta \gamma)\\
&=y^3-(\alpha \beta + \alpha \gamma+\beta \gamma) y^2 + (\alpha^2 \beta \gamma+\alpha \beta^2 \gamma+\alpha \beta \gamma^2) y +\alpha^2\beta^2\gamma^2\\
&=y^3-\sigma_2y^2+\sigma_3\sigma_1 y +\sigma_3^2\\
&=y^3+3y^2+10y+25.
\end{align*}
$y^3+3y^2+10y+25$ is the polynomial whose roots are $\alpha \beta, \alpha \gamma, \beta \gamma$.


\item[(b)]
\begin{align*}
g &=(y - \alpha-1)(y-\beta-1)(y - \gamma-1)\\
 &= f(y-1)\\
&= (y-1)^3+2(y-1)^2-3(y-1)+5\\
&= y^3-3y^2+3y-1+2y^2-4y+2-3y+3+5\\
&=y^3-y^2-4y+9.
\end{align*}


\item[(c)]
Let $h(y) = (y-\alpha^2)(y-\beta^2)(y-\gamma^2)$.
Then
\begin{align*}
h(y^2) &= (y^2-\alpha^2)(y^2 - \beta^2)(y^2-\gamma^2)\\
&=(y-\alpha)(y-\beta)(y-\gamma)(y+\alpha)(y+\beta)(y+\gamma)\\
&=(y^3+2y^2-3y+5)(y^3-2y^2-3y-5)\\
&=(y^3-3y)^2-(2y^2+5)^2\\
&=y^6-6y^4+9y^2 - 4 y^4-20y^2-25\\
&=y^6-10y^4-11y^2-25.
\end{align*}
Thus $$h(y) = (y-\alpha^2)(y-\beta^2)(y-\gamma^2) = y^3-10y^2-11y-25.$$
\end{enumerate}
(In particular, $\sigma_2(\alpha^2,\beta^2,\gamma^2 )= -11$, which we can verify directly :

$\alpha^2\beta^2+\alpha^2\gamma^2+\beta^2\gamma^2 = \sigma_2^2- 2 \sigma_1\sigma_3 = 9-20 = -11$.)

\paragraph{Ex. 2.2.12}

{\it Consider the symmetric polynomial $f = \sum_n x_1^{a_1}\cdots x_n^{a_n}$.
\begin{enumerate}
\item[(a)] Prove that $f$ has $n!$ terms when $a_1,\ldots,a_n$ are distinct.
\item[(b)] (More challenging) Suppose that the exponents $a_1,\ldots, a_n$ break up into $r$ disjoint groups so that exponent within the same group are equal, but exponents from different groups are unequal. Let $l_i$ denote the number of elements in the $i$th group, so that $l_1+l_2+\cdots+l_r  = n$. Prove that the number of terms in $f$ is
$$\frac{n!}{l_1!\cdots l_r!}.$$
\end{enumerate}
}

\begin{proof}
\begin{enumerate}
\item[(a)]
Here we suppose that the exponents $a_i$ are distinct

If  $\sigma,\tau \in S_n$ and $\sigma \neq \tau$, then $x_{\sigma(1)}^{a_1}\cdots x_{\sigma(n)}^{a_n} \neq x_{\tau(1)}^{a_1}\cdots x_{\tau(n)}^{a_n}$.

Then $\Sigma_n x_{1}^{a_1}\cdots x_{n}^{a_n} = \sum\limits_{\sigma \in S_n} x_{\sigma(1)}^{a_1}\cdots x_{\sigma(n)}^{a_n}$ has $n! = |S_n|$ terms.


\item[(b)]
Now we suppose that the exponents have same value on $I_1 = \gcro1,l_1\dcro$ and on each interval  $I_k  = \gcro l_1+\cdots+l_{k-1}+1, l_1+\cdots+l_k\dcro , (k=2,\cdots,r)$, with distinct constants on each interval.

The terms of $\Sigma_n x_{1}^{a_1}\cdots x_{n}^{a_n}$ are the terms of the image of the application

$$\varphi : 
\begin{array}{ccc}
S_n & \to  &F[x_1,\cdots,x_n]   \\
\sigma  &  \mapsto  &   x_{\sigma(1)}^{a_1}\cdots x_{\sigma(n)}^{a_n} = \sigma \cdot(x_{1}^{a_1}\cdots x_{n}^{a_n}).
\end{array}
$$
This image is the orbit ${\cal O}_t $ of $t=x_{1}^{a_1}\cdots x_{n}^{a_n}$ for the group operation defined by $(\sigma,f) \mapsto\sigma \cdot f$.

As $\vert {\cal O}_t \vert = \vert S_n \vert /  \vert \mathrm{Stab}_{S_n}(t) \vert$, it is sufficient to compute the cardinality of this stabilizer $S = \mathrm{Stab}_{S_n}(t)$, stabilizer in $S_n$ of $x_{1}^{a_1}\cdots x_{n}^{a_n}$: 

$$S = \{ \sigma \in S_n \ \vert \  x_{\sigma(1)}^{a_1}\cdots x_{\sigma(n)}^{a_n}=x_{1}^{a_1}\cdots x_{n}^{a_n} \}.$$

Note that $\sigma \in S$ iff $\sigma$  applies $I_k$ on itself : $$\sigma(I_k) = I_k, k=1,\ldots,r.$$

Let $\psi$ the application
$$\psi: 
\begin{array}{ccc}
S  & \to  &S(I_1) \times S(I_2)\times \cdots S(I_r)  \\
\sigma  &  \mapsto  &   (\sigma_1,\sigma_2, \cdots,\sigma_r)
\end{array}
$$
where  $\sigma_k = \sigma \vert_{I_k}$ is the restriction of $\sigma$ to $I_k$.

$\psi $ is bijective, so $$\vert S \vert = l_1!l_2!\cdots l_r!.$$

So the number of terms in  $\Sigma_n x_{1}^{a_1}\cdots x_{n}^{a_n}$, equal to the cardinality of the orbit of the monomial $t$, is equal to
$$\vert {\cal O}_t \vert = \vert S_n \vert /  \vert \mathrm{Stab}_{S_n}(x_{1}^{a_1}\cdots x_{n}^{a_n}) \vert = \frac{n!}{l_1!l_2!\cdots l_r!}$$

\end{enumerate}
\end{proof}

\paragraph{Ex. 2.2.13}

{\it Let $g_1,g_2 \in F[x_1,\ldots,x_n]$ be homogeneous of total degree $d_1,d_2$.
\begin{enumerate}
\item[(a)] Show that $g_1g_2$ is homogeneous of total degree $d_1+d_2$.
\item[(b)] When is $g_1+g_2$ homogeneous ?
\end{enumerate}
}

\begin{proof}
\begin{enumerate}
\item[(a)]
Every term $m$ of $g_1g_2$ is a product of a term $m_1$ of $g_1$ with a term $m_2$ of $g_2$.
$\deg(m) = \deg(m_1m_2) = \deg(m_1) + \deg(m_2) = d_1+d_2$.
So $g_1g_2$ is homogeneous of degree $d_1+d_2$.

\item[(b)]

$g_1+g_2$ is homogeneous iff $d_1=d_2$.
\end{enumerate}
\end{proof}

\paragraph{Ex. 2.2.14}

{\it We define the weight of $\sigma_1^{a_1}\cdots \sigma_n^{a_n}$ to be $a_1+2a_2+\cdots+na_n$.
\begin{enumerate}
\item[(a)] Prove that $\sigma_1^{a_1}\cdots \sigma_n^{a_n}$ is homogeneous and that its weight is the same as its total degree when considered as a polynomial in $x_1,\ldots,x_n$.
\item[(b)] Let $f = F(x_1,\ldots,x_n]$ be symmetric and homogeneous of total degree $d$. Show that $f$ is a linear combination of products $\sigma_1^{a_1}\cdots \sigma_n^{a_n}$ of weight $d$.
\end{enumerate}
}

\begin{proof}
\begin{enumerate}
\item[(a)] By Ex. 2.2.13, each $\sigma_k$ being homogeneous of degree $k$, the product $\sigma_1^{a_1}\cdots \sigma_n^{a_n}$ is homogeneous. As $\deg(\sigma_k) = k$, $\deg(\sigma_1^{a_1}\cdots \sigma_n^{a_n}) = a_1+2a_2+\cdots+na_n$ is equal to the weight of $\sigma_1^{a_1}\cdots \sigma_n^{a_n}$.
\item[(b)] Since $f$ is symmetric, $f$ is a linear combination of products $\sigma_1^{a_1}\cdots \sigma_n^{a_n}$. These products being homogeneous of degree $a_1+2a_2+\cdots+na_n$, and $f$ being homogeneous, by Ex 2.2.13(b), each term of this sum has degree $d$.

Conclusion : $f$ is a linear combination of products $\sigma_1^{a_1}\cdots \sigma_n^{a_n}$ of weight $d$.

\end{enumerate}
\end{proof}

\paragraph{Ex. 2.2.15}

{\it Given a polynomial $f \in F[x_1,\ldots,x_n]$, let $\deg_i(f)$ be the maximal exponent of $x_i$ which appears in $f$. Thus $f =x_1^3x_2 +x_1x_2^4$ has degree $\deg_1(f) = 3$ and $\deg_2(f) = 4$.
\begin{enumerate}
\item[(a)] If $f$ is symmetric, explain why the $\deg_i(f)$ are the same for $i=1,\ldots,n$.
\item[(b)] Show that $\deg_i(\sigma_1^{a_1}\cdots \sigma_n^{a_n}) = a_1+a_2+\cdots+a_n$ for $i=1,\ldots,n$.
\end{enumerate}
}

\begin{proof}
\begin{enumerate}
\item[(a)]  If $x_1$ appears in a term $c x_1^{a_1}x_2^{a_2}\cdots x_n^{a_n}$ of $f$, then the transposition $\tau=(1,2)$ applied to $f$ show that $c x_2^{a_1}x_1^{a_2}\cdots x_n^{a_n}$ is a term of $f$, so $x_2$ appears in a term of $f$ with the same exponent. Thus the maximal exponent is the same for the two variables : $$\deg_1(f) = \deg_2(f),$$
and the same is true for any pair of variables.

\item[(b)] 
As $\sigma_k = \sum\limits_{1\leq i_1 < \cdots<i_k\leq n} x_{i_1}x_{i_2}\cdots x_{i_k}$, $\deg_i(\sigma_k) = 1$.
For polynomial of one variable $x$, $\deg(pq) = \deg(p) + \deg(q)$, and $\deg_1(f)$ is the degree in $x_1$ of $f$ as an element of $k[x_2,\ldots,x_n][x_1]$, so
$$\deg_i(fg) = \deg_i(f)+ \deg_i(g).$$
Therefore
$\deg_i(\sigma_1^{a_1}\cdots \sigma_n^{a_n}) = a_1 \deg_i(\sigma_1) + \cdots + a_n\deg_i(\sigma_n) = a_1+\cdots+a_n$.
\end{enumerate}
\end{proof}

\paragraph{Ex. 2.2.16}

{\it This exercise is based on [7, pp. 110-112] and will express the discriminant $\Delta = (x_1-x_2)^2(x_1-x_3)^2(x_2-x_3)^2$ in terms of the elementary symmetric functions without using a computer. We will use the terminology of Exercises 14 and 15. Note that $\Delta$ is homogeneous of total degree 6 and $\deg_i(\Delta) = 4$ for $i=1,2,3$.
\begin{enumerate}
\item[(a)] Find all products $\sigma_1^{a_1}\sigma_2^{a_2}\sigma_3^{a_3}$ of weight 6 and $\deg_i(\sigma_1^{a_1} \sigma_2^{a_2}\sigma_3^{a_3}) \leq 4$.
\item[(b)] Explain how part (a) implies that there are constants $l_1,\ldots,l_5$ such that
$$\Delta = l_1\sigma_3^2+l_2\sigma_1\sigma_2\sigma_3+l_3\sigma_1^3\sigma_3+l_4\sigma_2^3+l_5\sigma_1^2\sigma_2^2.$$
\item[(c)] We will compute the $l_i$ by using the universal property of the elementary symmetric polynomial. For example, to determine $l_1$, use the cube roots of unity $1,\omega,\omega^2$ to show that $x^3-1$ has coefficient $-27$. By applying the ring homomorphism defined by $x_1\mapsto 1,x_2\mapsto\omega,x_3\mapsto\omega^2$ to part (b), conclude that $l_1 = -27$.
\item[(d)] Show that $x^3-x$ has roots $0,\pm1$ and discriminant 4. By adapting the argument of part (c), conclude that $l_4 = -4$.
\item[(e)] Similarly, use $x^3-2x^2+x$ to show that $l_5 = 1$.
\item[(f)] Next, note that $x^3-2x^2-x+2$ has roots $\pm1,2$ and use this (together with the known values of $l_1,l_4,l_5)$) to conclude that $l_2-4l_3 = 34$.
\item[(g)] Finally use $x^3-3x^2+3x-1$ to show $l_2+3l_3 = 6$. Using part (f), this implies $l_2 = 18,l_3 = -4$ and gives the usual formula for $\Delta$.

\end{enumerate}
}

\begin{proof}
\begin{enumerate}
\item[(a)]
By Ex. 14,15, to find all products $\sigma_1^{a_1} \sigma_2^{a_2} \sigma_3^{a_3}$ of weight 6 verifying $\deg_i(\sigma_1^{a_1} \sigma_2^{a_2} \sigma_3^{a_3}) \leq 4$, it suffices to solve the system of equations
$$
\left\{
\begin{array}{ccc}
a_1+2a_2+3a_3  & =  &  6 \\
 a_1+a_2+a_3 &\leq &   4   
\end{array}
\right.
$$
The solutions of the first equation are
$$(0,0,2),  (1,1,1), (3,0,1),(0,3,0) , (2,2,0), (4,1,0) (6,0,0).$$
Only the two last solutions don't verify the second condition. So the solutions of the system are
$$(0,0,2), (1,1,1), (3,0,1),(0,3,0) , (2,2,0),$$
which correspond to the symmetric polynomials
$$\sigma_3^2,  \sigma_1\sigma_2\sigma_3, \sigma_1^3 \sigma_3, \sigma_2^3, \sigma_1^2 \sigma_2^2.$$


\item[(b)]
As $\Delta$ is homogeneous of total degree $\deg(\Delta) = 6 $ and  as $\deg_i(\Delta) = 4, \ i=1,2,3$, by Ex. 14,15, $\Delta$ is a linear combination of  products $\sigma_1^{a_1} \sigma_2^{a_2} \sigma_3^{a_3}$ of weight 6. 

Moreover, the relative degree to the $i$-th variable of each of these products is at most 4 : if $f$ has the form
\begin{align*}
f &= f_1 + c \sigma_1^4 \sigma_2 + d \sigma_1^6\\
&= f_1+ c(x_1+x_2+x_3)^4(x_1x_2+x_1x_3+x_2x_3) + d (x_1+x_2+x_3)^6,\\
\end{align*}
where $\deg_i(f_1) \leq 4$, then the comparison of degree of  $x_1^6$ gives $d=0$, and the term in  $x_1^5$ gives $c=0$.

So there exists coefficients $l_i \in \Z$ such that 

$$\Delta = l_1\sigma_3^2+l_2\sigma_1\sigma_2\sigma_3+l_3\sigma_1^3\sigma_3+ l_4\sigma_2^3+l_5\sigma_1^2\sigma_2^2.$$


\item[(c)]
The discriminant of $x^3-1$ is equal to

$$\Delta(1,\omega,\omega^2) = (1-\omega)^2(1-\omega^2)^2(\omega-\omega^2)^2$$

\begin{align*}
\sqrt{\Delta} &= (1-\omega)(1-\omega^2)(\omega - \omega^2)\\
&= - 
\left \vert
\begin{array}{ccc}
 1&   1 & 1 \\
 1&   \omega  & \omega^2\\
1&   \omega^2  & \omega
\end{array}
\right \vert\\
&= -(3\omega^2 - 3 \omega) = 3(\omega-\omega^2)\\
&= 3i\sqrt{3}.\\
\end{align*}
Therefore $$\Delta(1,\omega,\omega^2) = -27.$$

The ring homomorphism defined by $x_1 \mapsto 1, x_2 \mapsto \omega, x_3 \mapsto \omega^2$ sends $\Delta$ on $\Delta(1,\omega,\omega^2)$ and $\sigma_k$ on $\sigma_k(1,\omega,\omega^2)$. As $$\sigma_1(1,\omega,\omega^2) = \sigma_2(1,\omega,\omega^2) = 0,\sigma_3(1,\omega,\omega^2)=1,$$
$$l_1 = \Delta(1,\omega,\omega^2)  = -27.$$


\item[(d)] 
$x^3-x = x(x-1)(x+1)$ has roots $0,1,-1$.

$\Delta(0,1,-1) = (0-1)^2(0+1)^2(1+1)^2 = 4$ and $\sigma_1=0,\sigma_2=-1,\sigma_3=0$, so
$l_4 \sigma_2^3 = -l_4 = 4$.
$$l_4 = -4.$$


\item[(e)] 
$x^3-2x^2+x = x(x-1)^2$ has a discriminant equal to 0, and $\sigma_1 = 2,\sigma_2=1,\sigma_3=0$, so
$l_4+4l_5=0$, with $l_4=-4$.
$$l_5=1.$$


\item[(f)]
$x^3-2x^2-x+2 = x^2(x-2) - (x-2) = (x^2-1)(x-2)$ has roots $1,-1,2$. Its discriminant is 
$\Delta = 2^2 1^2 3^2 = 36$, with $\sigma_1=2,\sigma_2=-1,\sigma_3=-2$.

Thus
\begin{align*}
36 &= l_1\sigma_3^2+l_2\sigma_1\sigma_2\sigma_3+l_3\sigma_1^3\sigma_3+l_4\sigma_2^3+l_5\sigma_1^2\sigma_2^2\\
&= 4l_1+4l_2-16l_3-l_4+4l_5\\
&=-4\times 27 +4 l_2-16l_3+4+4.\\
\end{align*}
With a division by 4,
$ l_2-4l_3 = \frac{36+ 4\times 27 - 8}{4} = 9 + 27 - 2 = 34.$
$$l_2 - 4 l_3 = 34.$$


\item[(g)]
$x^3-3x^2+3x-1 = (x-1)^3$ has a discriminant equal to 0, with $\sigma_1 = 3,\sigma_2=3,\sigma_3=1$.
\begin{align*}
0&=l_1+9l_2+27l_3+27l_4+81l_5\\
&=-27+9l_2+27l_3-27\times4+81.
\end{align*}
With a division by 9,
$l_2+3l_3 = 3+12-9=6$. So $l_2,l_3$ are solutions of the system of equations
$$\left\{
\begin{array}{ccc}
l_2-4l_3  &  = & 34,  \\
 l_2+3l_3 & =   &  6. 
\end{array}
\right.
$$
Thus $l_2 = 18,l_3=-4$, and

$$\Delta = -27\sigma_3^2+18\sigma_1\sigma_2\sigma_3-4\sigma_1^3\sigma_3-4\sigma_2^3+\sigma_1^2\sigma_2^2.$$

\end{enumerate}
\end{proof}

\paragraph{Ex. 2.2.17}
{\it Use the Newton identities (2.22) to express the power sum $s_2,s_3,s_4$ in terms of the elementary symmetric polynomials $\sigma_1,\sigma_2,\sigma_3,\sigma_4$.
}

\begin{proof}
$s_r = x_1^r+x_2^r+\cdots+x_n^r$.

 We suppose here that the number $n$  of variables is at least 4.
Then $$s_r = \sigma_1 s_{r-1}-\sigma_2 s_{r-2}+\cdots+(-1)^r\sigma_{r-1}s_1 + (-1)^{r-1} r \sigma_r.$$
\begin{align*}
s_1 &= \sigma_1,\\
\\
s_2 &= \sigma_1 s_1 - 2 \sigma_2\\
&= \sigma_1^2-2\sigma_2,\\
\\
s_3 &= \sigma_1 s_2 -  \sigma_2s_1 +3\sigma_3\\
&=\sigma_1(\sigma_1^2 - 2 \sigma_2) - \sigma_2 \sigma_1 + 3 \sigma_3\\
&=\sigma_1^3-3\sigma_1\sigma_2+3\sigma_3,\\
\\
s_4 &= \sigma_1s_3 - \sigma_2 s_2 + \sigma_3 s_1 - 4 \sigma_4\\
&=\sigma_1(\sigma_1^3-3\sigma_1\sigma_2+3\sigma_3) - \sigma_2(\sigma_1^2-2\sigma_2)+ \sigma_3\sigma_1 - 4 \sigma_4\\
&=\sigma_1^4-4\sigma_1^2\sigma_2+4\sigma_1\sigma_3+2\sigma_2^2-4\sigma_4.
\end{align*}

Verification with Sage:

\begin{verbatim}
e = SymmetricFunctions(QQ).e()
e1, e2, e3, e4 = e([1]).expand(4),e([2]).expand(4),e([3]).expand(4),e([4]).expand(4)
R.<x0,x1,x2,x3,y1,y2,y3,y4> = PolynomialRing(QQ, order = 'lex')
J = R.ideal(e1-y1,e2-y2,e3-y3,e4-y4)
G = J.groebner_basis()
s2 = x0^2 + x1^2 + x2^2 + x3^2
s3 = x0^3 + x1^3 + x2^3 + x3^3
s4 = x0^4 + x1^4 + x2^4 + x3^4
g2, g3, g4 = s2.reduce(G),s3.reduce(G),s4.reduce(G)
var('sigma_1,sigma_2,sigma_3,sigma_4')
h2 = g2.subs(y1=sigma_1,y2=sigma_2,y3=sigma_3,y4=sigma_4)
h3 = g3.subs(y1=sigma_1,y2=sigma_2,y3=sigma_3,y4=sigma_4)
h4 = g4.subs(y1=sigma_1,y2=sigma_2,y3=sigma_3,y4=sigma_4)
\end{verbatim}

\begin{verbatim}
h2, h3, h4
\end{verbatim}
$$\left(\sigma_{1}^{2} - 2 \, \sigma_{2}, \sigma_{1}^{3} - 3 \, \sigma_{1}
\sigma_{2} + 3 \, \sigma_{3}, \sigma_{1}^{4} - 4 \, \sigma_{1}^{2}
\sigma_{2} + 2 \, \sigma_{2}^{2} + 4 \, \sigma_{1} \sigma_{3} - 4 \,
\sigma_{4}\right).$$
\end{proof}

\paragraph{Ex. 2.2.18}

{\it Suppose that complex numbers $\alpha,\beta,\gamma$ satisfy the equations
\begin{align*}
\alpha+\beta+\gamma &= 3,\\
\alpha^2+\beta^2+\gamma^2 &= 5,\\
\alpha^3+\beta^3+\gamma^3 &=12.
\end{align*}
Show that $\alpha^n+\beta^n+\gamma^n \in \Z$ for all $n\geq 4$. Also compute $\alpha^4+\beta^4+\gamma^4$.
}

\begin{proof}
$\alpha,\beta,\gamma$ are the root of $$p = (x-\alpha)(x-\beta)(x-\gamma) = x^3-\sigma_1x^2+\sigma_2x-\sigma_3.$$
(We write $\sigma_i$ in place of $\sigma_i(\alpha,\beta,\gamma)$.)

By Exercise 17, with $n=3$ :
$$
\left\{
\begin{array}{cccl}
 3 &  = & s_1 &=\sigma_1  \\
  5&  = &  s_2 &= \sigma_1^2 - 2 \sigma_2 \\
  12&=   &   s_3& = \sigma_1^3-3\sigma_1 \sigma_2 + 3 \sigma_3.
\end{array}
\right.
$$
Thus $\sigma_1 = 3, \sigma_2 = \frac{1}{2}(\sigma_1^2-5) = \frac{1}{2}(9-5) = 2$.

$\sigma_3 = \frac{1}{3}(12 - \sigma_1^3+3\sigma_1\sigma_2) = 4 +\sigma_1\sigma_2 -\frac{\sigma_1^3}{3} = 4 + 6 - 9 = 1$.

$\alpha,\beta,\gamma$ are the roots of $p = x^3-3x^2+2x-1$.

If $n \geq 4$, $\alpha^n = 3 \alpha^{n-1} - 2 \alpha^{n-2}+\alpha^{n-3}$, and similar equations for $\beta,\gamma$. Summing these equations, we obtain
\begin{align}
s_n = 3 s_{n-1} -2 s_{n-2}+s_{n-3}. \label{eq2.2.18:1}
\end{align}
(This is a particular case of Newton identities (2.22).)

$s_0 = 3, s_1,s_2,s_3$ are in $\mathbb{Z}$.  If we suppose that $s_k \in \mathbb{Z}$ for all $k, 1\leq k<n$, then \eqref{eq2.2.18:1} show that $s_n \in \mathbb{Z}$, and the induction is done.

$$\forall k \in \mathbb{N}, \ s_n \in \mathbb{Z}.$$

In particular, $s_4 = 3 s_3 - 2 s_2+3 s_1= 3\times 12- 2 \times 5 + 3 \times 3 = 35$.
\end{proof}

\paragraph{Ex. 2.2.19}

{\it Suppose that $F$ is a field of characteristic 0.
\begin{enumerate}
\item[(a)] Use the Newton identities (2.22) and Theorem 2.2.2 to prove that every symmetric polynomial in $F[x_1,\ldots,x_n]$ can be expressed as a polynomial in $s_1,\ldots,s_n$.
\item[(b)] Show how to express $\sigma_4 \in F[x_1,x_2,x_3,x_4]$ as a polynomial in $s_1,s_2,s_3,s_4$.
\end{enumerate}
}

\begin{proof}
For all $r, 1\leq r \leq n$,
 $$s_r = \sigma_1 s_{r-1}-\sigma_2 s_{r-2}+\cdots+(-1)^r\sigma_{r-1}s_1 + (-1)^{r-1} r \sigma_r,$$
 and $\sigma_1 = s_1$.
 
 If we suppose that $\sigma_1,\sigma_2,\cdots,\sigma_{r-1}$ are polynomials in $s_1,s_2,\cdots,s_n$, the characteristic of the field $F$ being 0 (this allows the division by $r$), then
 
 $\sigma_r =\frac{ (-1)^{r-1}}{r}(s_r - \sigma_1 s_{r-1}+\sigma_2 s_{r-2}+\cdots+(-1)^{r-1} \sigma_{r-1}s_1)$
is a polynomial in $s_1,\cdots,s_n$.
 
 Conclusion : for all $r,1\leq r \leq n$, $\sigma_r$ can be expressed as a polynomial in $s_1,\cdots,s_n$.
 
By Ex. 2.2.17, we obtain
 \begin{align*}
\sigma_1 &= s_1,\\
\\
 \sigma_2 &= -\frac{1}{2}(s_2 - \sigma_1 s_1 )\\
 &=\frac{1}{2}(s_1^2-s_2),\\
 \\
 \sigma_3 &= \frac{1}{3}(s_3-\sigma_1s_2+\sigma_2s_1)\\
 &=\frac{1}{3}\left[s_3-s_1s_2+\frac{1}{2}s_1(s_1^2-s_2) \right]\\
 &=\frac{1}{6}(2s_3+s_1^3-3s_1s_2),\\
 \\
 \sigma_4 &=-\frac{1}{4}(s_4-\sigma_1s_3+\sigma_2s_2-\sigma_3s_1)\\
 &=-\frac{1}{4} \left[  s_4 - s_1s_3 +\frac{1}{2} s_2(s_1^2-s_2) -\frac{s_1}{6}(2s_3-3s_1s_2+s_1^3) \right]\\
 &=-\frac{1}{24}\left[   6s_4 - 6 s_1s_3 +3 s_2(s_1^2-s_2)-s_1(2s_3-3s_1s_2+s_1^3)\right]\\
 &=\frac{1}{24}(-6s_4+8s_1s_3-6s_1^2s_2+3s_2^2+s_1^4).
 \end{align*}
\end{proof}

\paragraph{Ex. 2.2.20}

{\it Let $\F_2$ be the field with two elements. Show that in $\F_2[x_1,\ldots,x_n]$, it is impossible to express $\sigma_2$ as a polynomial in $s_1,\ldots, s_n$ when $n\geq 2$.
}

\begin{proof}
Suppose that $\sigma_2 = f(s_1,s_2,\cdots,s_n)$, where $f$ is a polynomial with coefficients in $\mathbb{F}_2$.
If we use the evaluation defined by $x_1=x_2=\cdots = x_n =0$, we obtain $0 = f(0,\cdots,0)$.

With the evaluation defined by $x_1=x_2 = 1$ and $x_i = 0, i>2$, as $\sigma_2 = \sum_{i<j} x_i x_j$, then $\sigma_2(1,1,0,\ldots,0) = 1\times 1 = 1$ and $s_k(1,1,0,\ldots,0)  = 1^k + 1^k = 1+1 =  0$, so $1 = f(0,\cdots,0)$. As $1\neq 0$ in $\mathbb{F}_2$, this is a contradiction. So  it is impossible to express $\sigma_2$ as a polynomial in $s_1,\ldots, s_n$ when $n\geq 2$.
\end{proof}

\subsection{COMPUTING WITH SYMMETRIC POLYNOMIALS}
\paragraph{Ex. 2.3.1}

{\it Examples 2.3.1 and 2.3.2 showed that the roots of $y^3+41y^2+138y+125$ are the cubes of the roots of $y^3+2y^2-3y+5$. Verify this numerically.
}

\begin{proof}
We repeat Examples 2.3.1 and 2.3.2 with Sage :

$\bullet$ We build the Groebner basis of the ideal $\langle e_1 - y_1, e_2-y_2,e_3-y_3\rangle$, where $e_1,e_2,e_3$ are the elementary symmetric polynomials in $x_0,x_1,x_2$ :

\begin{verbatim}
e = SymmetricFunctions(QQ).e()
e1, e2, e3 = e([1]).expand(3),e([2]).expand(3),e([3]).expand(3)
R.<x0,x1,x2,y1,y2,y3> = PolynomialRing(QQ, order = 'degrevlex')
J = R.ideal(e1-y1, e2-y2, e3-y3)
G = J.groebner_basis()
\end{verbatim}

$\bullet$ We compute the coefficients of $f = (x-x_0^3) (x-x_1^3) (x-x_2^3)$ as  polynomials in $x_1,x_2,x_3$:
\begin{verbatim}
f = (x-x0^3) * (x-x1^3) * (x-x2^3)
coeffs = f.coefficients(x, sparse = False)
coeffs = map(lambda c : R(c), coeffs)
coeffs
\end{verbatim}
$$\left[- x_{0}^{3} x_{1}^{3} x_{2}^{3},\ x_{0}^{3} x_{1}^{3} + x_{0}^{3} x_{2}^{3} + x_{1}^{3} x_{2}^{3},\ - x_{0}^{3} -  x_{1}^{3} -  x_{2}^{3}, 1\right]$$

$\bullet$ The same coefficients as polynomials in $\sigma_1, \sigma_2, \sigma_3$:
\begin{verbatim}
var('sigma_1,sigma_2,sigma_3')
ncoeffs = [c.reduce(G) for c in coeffs]
nncoeffs = [c.subs(y1 = sigma_1,y2 = sigma_2,y3 = sigma_3) for c in ncoeffs]
nncoeffs
\end{verbatim}
$$\left[-\sigma_{3}^{3},\ \sigma_{2}^{3} - 3 \, \sigma_{1} \sigma_{2}
\sigma_{3} + 3 \, \sigma_{3}^{2},\  -\sigma_{1}^{3} + 3 \, \sigma_{1}
\sigma_{2} - 3 \, \sigma_{3},\ 1\right]$$

$\bullet$ We apply the substitution $\sigma_1 \mapsto -2, \sigma_2 \mapsto -3, \sigma_3 \mapsto -5$ and compute the polynomial $p$ whose roots are $\alpha_1^3,\alpha_2^3,\alpha_3^3$, where $\alpha_1, \alpha_2,\alpha_3$ are the roots of $y^3+2y^2-3y+5$.

\begin{verbatim}
nncoeffs = [c.subs(sigma_1 = -2, sigma_2 = -3, sigma_3 = -5) for c in nncoeffs]
p = sum(nncoeffs[i]*y^i for i in range(1+f.degree(x)))
p
\end{verbatim}
$$	y^{3} + 41 \, y^{2} + 138 \, y + 125$$


$\bullet$ Numerical verification:
\begin{verbatim}
S.<y>  = PolynomialRing(ComplexField(prec = 40))
[c[0] for c in S(p).roots()]
\end{verbatim}
$$\left[-37.399476110, -1.8002619448 - 0.31835473525\ i, -1.8002619448 + 0.31835473525\ i\right]$$
\begin{verbatim}
q = y^3+2*y^2-3*y+5
l = [c[0]^3 for c in q.roots()]
l
\end{verbatim}
$$\left[-37.399476110, -1.8002619448 - 0.31835473525\ i, -1.8002619448 + 0.31835473525\ i\right]$$
\end{proof}

\paragraph{Ex. 2.3.2}

{\it Use the method of Example 2.3.1 or 2.3.2 to find the cubic polynomial whose roots are the fourth powers of the roots of the polynomial $y^3+2y^2-3y+5$.
}

\begin{proof}
Same method in Sage as in Ex.2.3.1
\begin{verbatim}
e = SymmetricFunctions(QQ).e()
e1, e2, e3 = e([1]).expand(3),e([2]).expand(3),e([3]).expand(3)
R.<x0,x1,x2,y1,y2,y3> = PolynomialRing(QQ, order = 'degrevlex')
J = R.ideal(e1-y1, e2-y2, e3-y3)
G = J.groebner_basis()
f = (x-x0^4) * (x-x1^4) * (x-x2^4)
coeffs = f.coefficients(x, sparse = False)
coeffs = map(lambda c : R(c), coeffs)
coeffs
\end{verbatim}
$$\left[- x_{0}^{4} x_{1}^{4} x_{2}^{4}, x_{0}^{4} x_{1}^{4} + x_{0}^{4}
x_{2}^{4} + x_{1}^{4} x_{2}^{4}, - x_{0}^{4} -  x_{1}^{4} -  x_{2}^{4},
1\right]
$$
\begin{verbatim}
var('sigma_1,sigma_2,sigma_3,y')
ncoeffs = [c.reduce(G) for c in coeffs]
nncoeffs = [c.subs(y1 = sigma_1,y2 = sigma_2,y3 = sigma_3) for c in ncoeffs]
nncoeffs
\end{verbatim}
$$\left[- x_{0}^{4} x_{1}^{4} x_{2}^{4}, x_{0}^{4} x_{1}^{4} + x_{0}^{4}
x_{2}^{4} + x_{1}^{4} x_{2}^{4}, - x_{0}^{4} -  x_{1}^{4} -  x_{2}^{4},
1\right]
$$
\begin{verbatim}
nnncoeffs = [c.subs(sigma_1 = -2, sigma_2 = -3, sigma_3 = -5) for c in nncoeffs]
p = sum(nnncoeffs[i]*y^i for i in range(1+f.degree(x)))
p
\end{verbatim}
$$y^{3} - 122 \, y^{2} - 379 \, y - 625.$$

So the cubic polynomial whose roots are the fourth powers of the roots of the polynomial $y^3+2y^2-3y+5$ is
$$y^{3} - 122 \, y^{2} - 379 \, y - 625.$$
\end{proof}

\paragraph{Ex. 2.3.4}

{\it Given a cubic $x^3+bx^2+cx+d$, what condition must $b,c,d$ satisfy in order that one root be the average of the other two ?
}

\begin{proof}
$\bullet$ Suppose that the polynomial $ f = x^3 +bx^2+cx+d =(x-x_1)(x-x_2)(x-x_3)$ has one root which is the average of the other two. We choose a numbering of  the roots such that $$x_3 = \frac{x_1+x_2}{2}.$$
Then 
\begin{align*}
-b = \sigma_1 &= x_1+x_2 + \left(\frac{x_1+x_2}{2}\right)\\
&= \frac{3}{2}(x_1+x_2),\\
c = \sigma_2 &= x_1x_2+x_2x_3+x_1x_3\\
&=x_1x_2 +\left(\frac{x_1+x_2}{2}\right)(x_1+x_2)\\
&=x_1x_2 + \frac{1}{2}(x_1+x_2)^2,\\
-d = \sigma_3 &=x_1x_2\left(\frac{x_1+x_2}{2}\right)\\
&=\frac{1}{2}(x_1+x_2)x_1x_2.
\end{align*}

Let $s=x_1+x_2,p=x_1x_2$. The preceding equations give
\begin{align}
b &= -\frac{3}{2} s, \label{eq2.3.4:6}\\
c &= p +\frac{1}{2} s^2,\label{eq2.3.4:7}\\
d&=-\frac{1}{2} sp.\label{eq2.3.4:8}
\end{align}

We eliminate $s,p$ from these equations :
\begin{align*}
s &= -\frac{2}{3} b,\\
p &= c - \frac{1}{2}\left(-\frac{2}{3} b\right)^2\\
&=c - \frac{2}{9} b^2,\\
d&=-\frac{1}{2} \left(-\frac{2}{3} b + \frac{4}{27} b^3 \right)\\
&= \frac{1}{3}bc -\frac{2}{27} b^3.
\end{align*}
So the coefficients $b,c,d$ verify $$2b^3 - 9bc + 27d = 0.$$

\vspace{0.5cm}
$\bullet$ Conversely, suppose that $b,c,d$ verify 
\begin{align}
2b^3 - 9bc + 27d = 0.\label{eq2.3.4:9}
\end{align}
Let  $s =-\frac{2}{3} b, p = c - \frac{2}{9} b^2$. Then  $b = -\frac{3}{2} s, c = p +\frac{2}{9} b^2 = p + \frac{1}{2} (\frac{2}{3} b)^2 = p + \frac{1}{2} s^2$ : (6) and (7) are valid.

By the equation \eqref{eq2.3.4:9}, 
\begin{align*}
d &=  \frac{1}{3}bc -\frac{2}{27} b^3\\
&=-\frac{1}{2} \left ( -\frac{2}{3} b\right) \left (c - \frac{2}{9} b^2\right)\\
&= -\frac{1}{2} sp.
\end{align*}
So $s,p$ verify the system \eqref{eq2.3.4:6},\eqref{eq2.3.4:7},\eqref{eq2.3.4:8} :

\begin{align*}
b &= -\frac{3}{2} s,\\
c &= p +\frac{1}{2} s^2,\\
d&=-\frac{1}{2} sp.
\end{align*}

Let $x_1,x_2$ the complex roots of $x^2-sx+p$. Then $x_1+x_2 = s,x_1x_2=p$. Let $x_3 = \frac{x_1+x_2}{2} = \frac{1}{2}s$.
Then
\begin{align*}
\sigma_1 &= x_1+x_2+x_3\\
&= \frac{3}{2} s\\
&=-b\\
\sigma_2 &=x_1x_2+x_2x_3+x_1x_3\\
&=x_1x_2 +\left(\frac{x_1+x_2}{2}\right)(x_1+x_2)\\
&=x_1x_2 + \frac{1}{2}(x_1+x_2)^2\\
&= p +\frac{1}{2} s^2\\
&=c\\
\sigma_3 &= x_1x_2 x_3\\
&=\frac{1}{2}sp\\
&=-d
\end{align*}
Thus $x_1,x_2,x_3$ are the roots of $(x-x_1)(x-x_2)(x-x_3) = x^3 -\sigma_1x^2+\sigma_2x-\sigma_3 = x^3+bx^2+cx+d$, and $x_3 = \frac{x_1+x_2}{2}$.


Conclusion : one of the roots of $x^3+bx^2+cx+d$ the average of the other two iff $2b^3 - 9bc + 27d = 0.$
\end{proof}

\paragraph{Ex. 2.3.5}

{\it Given a quartic $x^4+bx^3+cx^2+dx+e$, what condition must $b,c,d,e$ satisfy in order that one root be the negative of another ?
}

\begin{proof}
The polynomial $$f = x^4+bx^3+cx^2+dx+e = (x-\alpha_1)(x-\alpha_2)(x-\alpha_3)(x-\alpha_4)$$ has two opposite roots iff

$$(\alpha_1+\alpha_2)(\alpha_1+\alpha_3)(\alpha_1+\alpha_4)(\alpha_2+\alpha_3)(\alpha_2+\alpha_4)(\alpha_3+\alpha_4)=0$$

Let $$u = (x_1+x_2)(x_1+x_3)(x_1+x_4)(x_2+x_3)(x_2+x_4)(x_3+x_4).$$ 
$u$ is symmetric, so  is a polynomial in $\sigma_1,\sigma_2,\sigma_3,\sigma_4$.

We obtain this polynomial with the following Sage instructions
\begin{verbatim}
e = SymmetricFunctions(QQ).e()
e1,e2,e3,e4 = e([1]).expand(4),e([2]).expand(4),e([3]).expand(4),e([4]).expand(4)
R.<x0,x1,x2,x3,y1,y2,y3,y4> = PolynomialRing(QQ, order = 'lex')
J = R.ideal(e1-y1,e2-y2,e3-y3,e4-y4)
G = J.groebner_basis()
u = (x0+x1)*(x0+x2)*(x0+x3)*(x1+x2)*(x1+x3)*(x2+x3)
var('sigma_1,sigma_2,sigma_3,sigma_4')
u.reduce(G).subs(y1=sigma_1, y2 = sigma_2,y3=sigma_3,y4=sigma_4)
\end{verbatim}
$$\sigma_{1} \sigma_{2} \sigma_{3} - \sigma_{1}^{2} \sigma_{4} -\sigma_{3}^{2}.$$

So 
$$u = \sigma_1\sigma_2\sigma_3 - \sigma_1^2\sigma_4 - \sigma_3^2.$$

The evaluation ring homomorphism defined by $x_i \mapsto \alpha_i, i=1,2,3,4$ verifies $$\sigma_1\mapsto -b, \sigma_2 \mapsto c, \sigma_3 \mapsto -d, \sigma_4\mapsto e.$$

So $(\alpha_1+\alpha_2)(\alpha_1+\alpha_3)(\alpha_1+\alpha_4)(\alpha_2+\alpha_3)(\alpha_2+\alpha_4)(\alpha_3+\alpha_4) = bcd - b^2e-d^2$.


Conclusion : 
$f = x^4+bx^3+cx^2+dx+e$ is such that one root is the negative of another iff $bcd - b^2e-d^2=0$.
\end{proof}

\paragraph{Ex. 2.3.6}

{\it Find the quartic polynomial whose roots are obtained by adding 1 to each of the roots of $x^4+3x^2+4x+7$.
}

\begin{proof}
Let $f  =  x^4+3x^2+4x+7 = (x-x_1)(x-x_2)(x-x_3)(x-x_4)$.

The polynomial whose roots are $1+x_1,1+x_2,1+x_3,1+x_4$ is
\begin{align*}
g &= (x-1-x_1)(x-1-x_2)(x-1-x_3)(x-1-x_4)\\
&=f(x-1)\\
&=(x-1)^4 + 3(x-1)^2 + 4(x-1)+7\\
&= x^4-4x^3+6x^2-4x+1+3x^2-6x+3 +4x-4+7\\
&=x^4-4x^3+9x^2-6x+7.
\end{align*}

If $x_1,x_2,x_3,x_4$ are the roots of $f$,  then  $x_1+1,x_2+1,x_3+1,x_4+1$ are the roots of
$$g = x^4-4x^3+9x^2-6x+7.$$
\end{proof}

\subsection{THE DISCRIMINANT}

\paragraph{Ex. 2.4.1}

{\it Let $M$ be the $n\times n$ matrix appearing on the right-hand side of the Vandermonde formula given in Proposition 2.4.5. Prove that (2.32) follows from the fact that $M$ and its transpose both have determinant $\sqrt{\Delta}$.
}

\begin{proof}
Let $a_1,a_2,\cdots,a_n$ be elements of a field $F$, and 
$$
A_n = 
\begin{pmatrix}
1    & 1     &   \cdots    & 1\\
a_1&  a_2 &   \cdots   & a_n\\
a_1^2&  a_2^2 &   \cdots   & a_n^2\\
\vdots & \vdots & \ddots & \vdots\\
a_1^{n-1}&  a_2^{n-1} &   \cdots   & a_n^{n-1}\\
\end{pmatrix}
$$

We show by induction on $n,n\geq 2$ that $$\det(A_n) = \prod\limits_{1\leq i < j \leq n} (a_j - a_i).$$

$\det(A_2) = 
\begin{vmatrix}   1 & 1\\ a_1 & a_2 \end{vmatrix} = a_2 - a_1 = \prod\limits_{1 \leq i < j \leq 2} (a_j - a_i).
$

Suppose that this formula is true for the integer $n-1, n\geq 3$. We will show that it is true for the integer $n$.

If there exists a pair  $(i,j), i\neq j$ such that $a_i = a_j$, then two columns in $A_n$ are identical, so $\det(A_n) = 0=\prod\limits_{1\leq i < j \leq n} (a_j - a_i).$

We can so suppose that the $a_i, 1 \leq i \leq n$ are distinct.

Let the polynomial  $P \in F[X]$ given by 
$$
P= 
\begin{pmatrix}
1    & 1     &   \cdots    & 1& 1\\
a_1&  a_2 &   \cdots   & a_{n-1}& X\\
a_1^2&  a_2^2 &   \cdots & a_{n-1}^2  & X^2\\
\vdots & \vdots & \ddots &\vdots& \vdots\\
a_1^{n-1}&  a_2^{n-1} &   \cdots & a_{n-1}^{n-1}  & X^{n-1}\\
\end{pmatrix}
$$

Then $\det(A_n) = P(a_n)$, and $P(a_1) = P(a_2)=\cdots = P(a_{n-1}) = 0$.
As $a_1,a_2,\cdots,a_{n-1}$ are distinct roots of $P$, with $\deg(P) = n-1$, $P$ is factored as
$$P = k(X-a_1)\cdots(X-a_{n-1}), k \in F,$$
where $k$ is the coefficient of $X^{n-1}$ in $P$, so $k$ is the cofactor of $X^{n-1}$ in $\det(P)$:  so $$k = \det(A_{n-1} )= \prod\limits_{1\leq i < j \leq n-1} (a_j - a_i)$$
by the induction hypothesis.

Therefore $$ \det(A_n) = P(a_n) = \prod_{1\leq i < j \leq n-1} (a_j - a_i) \prod_{i=1}^n(a_n - a_i) = \prod_{1\leq i < j \leq n} (a_j - a_i),$$
which completes the induction.


The matrix 
$$
B_n = 
\begin{pmatrix}
a_1^{n-1}&  a_2^{n-1} &   \cdots   & a_n^{n-1}\\
\vdots & \vdots & \ddots & \vdots\\
a_1&  a_2 &   \cdots   & a_n\\
1    & 1     &   \cdots    & 1\\
\end{pmatrix}
$$
is obtained from $A_n$ by $\frac{n(n-1)}{2}$ transpositions of rows : $n-1$ to put the last row in first position, then $n-2$ to put which is now the last row  in second position, and so on.

Thus $\det(B_n) = (-1)^{(n(n-1))/2}\det(A_n)$.

As the number of factors in $\prod\limits_{1 \leq i < j \leq n} (a_j - a_i) $ is $\frac{n(n-1)}{2}$,
$$\prod\limits_{1 \leq i < j \leq n} (a_j - a_i) =(-1)^{(n(n-1))/2} \prod\limits_{1 \leq i < j \leq n} (a_i- a_j).$$

Consequently,
$$\det(B_n) = \prod\limits_{1 \leq i < j \leq n} (a_i- a_j).$$

Applying this result in the field $F(x_1,\cdots,x_n)$, we obtain that

$$\sqrt{\Delta} =\prod\limits_{1 \leq i < j \leq n} (x_i- x_j) =
\begin{vmatrix}
x_1^{n-1}&  x_2^{n-1} &   \cdots   & x_n^{n-1}\\
x_1^{n-2}&  x_2^{n-2} &   \cdots   & x_n^{n-2}\\
\vdots & \vdots & \ddots & \vdots\\
x_1&  x_2 &   \cdots   & x_n\\
1    & 1     &   \cdots    & 1\\
\end{vmatrix}
$$

If  $A = 
\begin{pmatrix}
x_1^{n-1}&  x_2^{n-1} &   \cdots   & x_n^{n-1}\\
x_1^{n-2}&  x_2^{n-2} &   \cdots   & x_n^{n-2}\\
\vdots & \vdots & \ddots & \vdots\\
x_1&  x_2 &   \cdots   & x_n\\
1    & 1     &   \cdots    & 1\\
\end{pmatrix}
$, then $\,A^t= 
\begin{pmatrix}
x_1^{n-1}&  x_1^{n-2} &   \cdots  &x_1 & 1\\
x_2^{n-1}&  x_2^{n-2} &   \cdots   &x_2 & 1\\
\vdots & \vdots & \ddots & \vdots\\
x_{n-1}^{n-1}   &x_{n-1}^{n-2}   &   \cdots  &x_{n-1}   & 1\\
x_n^{n-1}   &x_n^{n-2}   &   \cdots  &x_n   & 1\\
\end{pmatrix}$

thus $$\Delta = \det(A)^2 = \det(A^t A)= 
\begin{vmatrix}
s_{2n-2} &  s_{2n-3} &   \cdots   &s_{n}    & s_{n-1}\\
s_{2n-3}&  s_{2n-4} &   \cdots   &s_{n-1}  & s_{n-2}\\
\vdots     & \vdots     & \ddots      & \vdots      &       \\
s_{n}      & s_{n-1}   &   \cdots    &s_{2}    & s_{1} \\
s_{n-1}    &s_{n-2}    &   \cdots    &s_{1}   & s_{0} \\
\end{vmatrix}
$$
\end{proof}

\paragraph{Ex. 2.4.2}

{\it Let $F$ have characteristic $\ne 2$, and let $f \in F[x_1,\ldots,x_n]$ satisfy $\tau \cdot f = -f$ for all transpositions $\tau \in S_n$. Prove that $f = B \sqrt{\Delta}$ for some $B \in F[\sigma_1,\ldots,\sigma_n]$.
}

\begin{proof}
 Here, the field $F$ have characteristic $\ne 2$.
 
Let $f\in F[x_1,\cdots,x_n]$  such that $\tau \cdot f = -f$ for all transpositions $\tau \in S_n$.
 
If $\sigma \in A_n$ is an even permutation, then $\sigma$ is product of an even number of permutations :
 $$\sigma = \tau_1 \tau_2\cdots \tau_{2k}.$$
As the group $S_n$ acts on $F[x_1,\cdots,x_n]$,  $\sigma \cdot f = \tau_1 \cdot ( \tau_2 \cdot ( \cdots (\tau_{2k} \cdot f)\cdots)) = (-1)^{2k} f = f$. Therefore $f$ is invariant under $A_n$ and so the theorem  2.4.4 applies:

There exist $A,B \in F[\sigma_1,\cdots,\sigma_n]$ such that
$$f = A+B\sqrt{\Delta}.$$

Therefore $-f = \tau \cdot f = \tau \cdot A + (\tau\cdot B)(\tau \cdot \sqrt{\Delta}) = A - B\sqrt{\Delta}$ (by 2.31).

So $f = A + B\sqrt{\Delta}$ and  $f = -A+B\sqrt{\Delta}$, thus $2A=0$. Since the characteristic is not 2, $A=0$, therefore
$$f = B\sqrt{\Delta}, B \in F[\sigma_1,\cdots,\sigma_n].$$
\end{proof}

\paragraph{Ex. 2.4.3}

{\it Let $f = x^2+bx+c \in F[x]$. Use the definition of discriminant given in the text to show that $\Delta(f) = b^2 - 4c$.
}

\begin{proof}
Let $f=x^2+bx+c, \ b,c \in F$.

$\Delta = (x_1-x_2)^2 = x_1^2+x_2^2-2x_1x_2 = (x_1+x_2)^2 -4 x_1x_2 = \sigma_1^2-4\sigma_2$.

The ring homomorphism which sends $\sigma_1$ on $-b$ and $\sigma_2$ on $c$ send $\Delta$ on
$$\Delta(-b,c) = b^2 -4c,$$
which is by definition the discriminant of $x^2+bx+c$.
\end{proof}

\paragraph{Ex. 2.4.4}

{\it Let $f \in F[x]$ be monic, and suppose that $f = (x-\alpha_1)\cdots(x-\alpha_n)$ in some field $L$ containing $F$. Prove that $\Delta(f) \ne 0$ if and only if $\alpha_1,\ldots,\alpha_n$ are distinct. This shows that $f$ has distinct roots if and only if its discriminant is nonvanishing.
}

\begin{proof}
Let $f \in F[x]$ such that $f = (x-\alpha_1)\cdots(x-\alpha_n)$ in an  extension $L$ of $F$.

By Proposition 2.4.3, 
\begin{align}
\Delta(f) = \prod_{1\leq i < j \leq n} (\alpha_i - \alpha_j)^2. \label{eq2.4.4:10}
\end{align}

$\bullet$ If $\Delta(f) \neq 0$, by \eqref{eq2.4.4:10}, for all pairs $(i,j), 1\leq i < j \leq n, \alpha_i - \alpha_j \neq 0$. The roots $\alpha_i$ are so distinct. 

$\bullet$ If the roots $\alpha_i,1 \leq i \leq n,$ are distinct roots, then $\alpha_i - \alpha_j \neq 0$ for all $(i,j)$ such that $1\leq i < j \leq n$, thus $\Delta(f) \neq 0$.

\end{proof}

\paragraph{Ex. 2.4.5}

{\it Show that $\sqrt{\Delta} \in F[x_1,\ldots,x_n]$ is symmetric if and only if $F$ is a field of characteristic 2.
}

\begin{proof}
By Proposition 2.4.1, if $\tau$ is a transposition in $S_n$,
$$\tau\cdot \sqrt{\Delta} =  - \sqrt{\Delta}.$$

$\bullet$ If the field $F$ is of characteristic 2, $- \sqrt{\Delta} = +  \sqrt{\Delta}$, so for all transpositions $\tau$,

$$\tau \cdot \sqrt{\Delta} =   \sqrt{\Delta}.$$
Therefore $\sqrt{\Delta} $ is a symmetric polynomial.

$\bullet$ If the field $F$ is not of characteristic 2, as $\sqrt{\Delta} \neq 0$,
$$\tau \cdot \sqrt{\Delta} =  - \sqrt{\Delta} \neq \sqrt{\Delta},$$
so $\sqrt{\Delta}$ is not symmetric.
\end{proof}

\paragraph{Ex. 2.4.6}

{\it This exercise will describe how to solve quadratic equations over a field $F$ of characteristic 2.
\begin{enumerate}
\item[(a)] Given $b \in F$, we will assume there is a larger field $F\subset L$ such that $b=\beta^2$ for some $\beta \in L$.
Show that $\beta$ is unique and that $\beta$ is the unique root of $x^2+b$. Because of this, we denote $\beta$ by $\sqrt{b}$.
\item[(b)] Now suppose that $f = x^2+ax+b$ is a quadratic polynomial in $F[x]$ with $a\ne 0$. Suppose also that $f$ is irreducible over $F$, so that it has no roots in $F$. We will see in Chapter 3 that $f$ has a root $\alpha$ in a field $L$ containing $F$. Prove that $\alpha$ cannot be written in the form $\alpha = u + v \sqrt{w}$, where $u,v,w \in F$.
\item[(c)] Part (b) shows that solving a quadratic equation with nonzero $x$-coefficient requires more than square roots. We do this as follows. If $b\in F$, let $R(b)$ denote a root of $x^2+x+b$ (possibly lying in some larger field). We call $R(b)$ and $R(b)+1$ the $2$-roots of $b$. Prove that the roots of $x^2+x+b$ are $R(b)$ and $R(b)+1$, and explain why adding 1 to the second $2$-root gives the first. 
\item[(d)] Show that the roots of $f = x^2 + ax+b, a\ne 0$, are $aR(b/a^2)$ and $a(R(b/a^2)+1$.
\end{enumerate}
}

\begin{proof}
\begin{enumerate}
\item[(a)]
Let $L$ an extension of $F$ and $\beta \in L$ such that $\beta^2 = b$.

As $x^2 - b = x^2 - \beta^2 = (x - \beta)^2$, $\beta$ is the unique root of $x^2-b = x^2+b$. We write $\beta = \sqrt{b} \in L$.

\item[(b)]
Suppose that  $f = x^2+ax+b, \ a\ne 0$ is irreducible on $F$. As $\deg(f) = 2$, this is equivalent to the fact that $f$ has no root in $F$.
 $f $ has a root $\alpha$ in an extension $L \supset F$.
 
If $\alpha = u+v\sqrt{w},\ u,v,w \in F$, then $v\neq 0$, otherwise $\alpha \in F$, in contradiction with the irreducibility of $f$.
 
 Then
 \begin{align*}
 0&= \alpha^2 + a \alpha + b\\
 &= u^2+wv^2 + a(u+v \sqrt{w})+ b\\
 &= u^2+wv^2+au+b + av \sqrt{w}\\
 &=s + t \sqrt{w},
 \end{align*}
 
where $s = u^2+wv^2+au+b  \in F, t = av \in F, t\neq 0$.

Thus $\sqrt{w} = -s/t \in F$, so $\alpha \in F$, in contradiction with the irreducibility of $f$.

Conclusion : $\alpha = u+v\sqrt{w},\ u,v,w \in F$ is impossible.

\item[(c)]
Write $R(b)$ a root of $x^2+x+b$ in an extension of $F$.

As $R(b)^2+R(b)+b=0$, $(R(b)+1)^2 + (R(b) +1)+b = R(b)^2+1 + R(b)+1+b = R(b)^2+R(b)+b=0$.

As $R(b)+1+1 = R(b)$, the two (distinct) roots of  $x^2+x+b$ are $R(b),R(b+1)$, and  $\sigma : x\mapsto x+1$ exchanges the two roots.

\item[(d)]

For all $y \in L$,
\begin{align*}
f(y) =0 &\iff y^2+ay+b=0\\
&\iff \left(\frac{y}{a}\right ) ^2 + \left(\frac{y}{a}\right ) + \frac{b}{a^2} = 0\\
&\iff \frac{y}{a} \in \left \{R \left(\frac{b}{a^2}\right ) , R \left(\frac{b}{a^2}\right ) +1 \right \}\\
&\iff y \in \left \{aR \left(\frac{b}{a^2}\right ) , a\left [R \left(\frac{b}{a^2}\right ) +1\right] \right \}.\\
\end{align*}
The roots of $x^2+ax+b,a\neq 0$ are so $aR \left(\frac{b}{a^2}\right ) , a\left [R \left(\frac{b}{a^2}\right ) +1\right] $.
\end{enumerate}
\end{proof}

\paragraph{Ex. 2.4.7}

{\it Explain how the third property of (2.31) was used (implicitly) in (2.28) in the proof of Proposition 2.4.1.
}

\begin{proof}
Knowing that $\tau \cdot \sqrt{\Delta} = -\sqrt{\Delta}$ for a transposition $\tau\in S_n$, we show by induction on  $l$ that
$$(\tau_l  \cdots \tau_1) \cdot \sqrt{\Delta} = (-1)^l\sqrt{\Delta}.$$
By the induction hypothesis $(\tau_l  \cdots \tau_1). \sqrt{\Delta}= (-1)^l\sqrt{\Delta}$, we deduce, using 2.31
\begin{align*}
(\tau _{l+1} \tau_l  \cdots \tau_1) \cdot \sqrt{\Delta}&= \tau _{l+1} \cdot [( \tau_l  \cdots \tau_1) \cdot \sqrt{\Delta}]\\
&= \tau _{l+1}\cdot ((-1)^l \sqrt{\Delta})\\
&=(-1)^l\tau_{l+1} \cdot \sqrt{\Delta}\\
&= (-1)^{l+1}\sqrt{\Delta}.
\end{align*}
\end{proof}

\paragraph{Ex. 2.4.8}

{\it In this exercise, you will prove that although $\Delta$ factors in $F[x_1,\ldots,x_n]$, it is irreducible in $F[\sigma_1,\ldots,\sigma_n]$ when $F$ has characteristic different from 2. To begin the proof, assume that $\Delta = AB$, where $A,B \in F[\sigma_1,\ldots,\sigma_n]$ are nonconstant.
\begin{enumerate}
\item[(a)] Using the definition of $\Delta$ and unique factorization in $F[x_1,\ldots,x_n]$, show that $A$ is divisible in $F[x_1,\ldots,x_n]$ by $x_i-x_j$ for some $1\leq i < j \leq n$.
\item[(b)] Given $1\leq i < j \leq n$ and $1\leq l < m \leq n$, show that there is a permutation $\sigma \in S_n$ such that $\sigma(i)= l$ and $\sigma(j)=m$.
\item[(c)]Use part (a) and (b) to show that $A$ is divisible by $x_l-x_m$ for all $1\leq l < m \leq n$.
\item[(d)] Conclude that $A$ is a multiple of $\sqrt{\Delta}$ and that the same is true for $B$.
\item[(e)] Show that part (d) implies that $A$ and $B$ are constant multiples of $\sqrt{\Delta}$ and explain why this contradicts $A,B \in F[\sigma_1,\ldots,\sigma_n]$.
\item[(f)] Finally, suppose that $F$ has characteristic 2. Prove that $\Delta$ is not irreducible.

\end{enumerate}
}

\begin{proof}
\begin{enumerate}
\item[(a)] 
Suppose that $\Delta = A B$, where $A,B \in F[\sigma_1,\cdots,\sigma_n]$ are nonconstant.

As $A$ is not a constant, it is divisible by an irreducible factor $h \in F[x_1,\cdots,x_n]$. This irreducible factor $h$ divides $\Delta$, whose only irreducible factors are associate to $x_i-x_j,1\leq i<j\leq n$. $F[x_1,\cdots,x_n]$ being a factorial domain, there exists a pair of subscripts $(i,j)$ and $\lambda \in F^*$ such that $h =\lambda ( x_i-x_j),1\leq i<j\leq n$.

Conclusion : 

$A$ is divisible in $k[x_1,\ldots,x_n]$ by a factor $x_i-x_j$, for some $(i,j), 1 \leq i <j\leq n$.


\item[(b)] 
The set $U = \gcro1,n\dcro \setminus \{i,j\}$ and $V = \gcro1,n\dcro \setminus \{l,m\}$ have same cardinality $n-2$, so there exists a bijection $f : U \to V$.

Let $\sigma :\gcro1,n\dcro \to \gcro1,n\dcro$ defined by $\sigma(k) = f(k)$ if $k\in U, \sigma(i) = l, \sigma(j) = m$. Then $\sigma$ is bijective (the application $\tau$  defined by $\tau(m) = f^{-1}(m)$ if $m \in V, \tau(l)=i,\tau(m) = j$ satisfies $\tau \circ \sigma = \sigma \circ \tau = e$).

There exists $\sigma \in S_n$ such that $\sigma(i) = l, \sigma(j)=m$.


\item[(c)] 
By (a),
$A = (x_i-x_j) C, C \in k[x_1,\cdots x_n]$. 

As $A$ is symmetric, using the permutation $\sigma$ of (b), 
\begin{align*}
A &=\sigma\cdot A\\
&=\sigma\cdot[(x_i-x_j) C]\\
&= \sigma \cdot(x_i-x_j)\ \sigma\cdot C\\
&=(x_l-x_m) (\sigma\cdot C).
\end{align*}

So $A$ is divisible by $x_l-x_m,\ 1\leq l<m\leq m$.


\item[(d)] 
As these factors are irreducible and not associate, their product divides $A$, thus 
$$\sqrt{\Delta} = \prod_{1\leq l <m\leq n} (x_l-x_m) \mid A.$$

The same reasoning applies to $B$, which is also divisible by $\sqrt{\Delta}$.


\item[(e)] $A = A_1 \sqrt{\Delta}, B = B_1 \sqrt{\Delta}$, where $A_1,B_1 \in F[x_1,\cdots,x_n]$.

Thus $\Delta = AB = A_1B_1 \Delta$, with $\Delta \neq 0$, therefore $A_1B_1=1$, which implies that $A_1=a\in F^*,B_1=b\in F^*$ :

$$A = a \sqrt{\Delta}, B = b \sqrt{\Delta},\ a,b\in F^*.$$
But $A\in F[\sigma_1,\ldots,\sigma_n]$, thus for all transposition $\tau$ in $S_n$, $$A =\tau\cdot A = \tau\cdot(a\sqrt{\Delta})=a\, \tau\cdot \sqrt{\Delta} = -a\sqrt{\Delta}=-A.$$
So $2A = 0$, and as the characteristic of $F$ is not 2, $A=0$, so $\Delta = 0$, which is a contradiction.

Conclusion : $\Delta$ is irreducible in  $F[\sigma_1,\cdots,\sigma_n]$.


\item[(f)] 
If the characteristic of $F$ is 2, then $\sqrt{\Delta}$ is symmetric, since for all transposition $\tau$, $\tau.\sqrt{\Delta} = - \sqrt{\Delta} = \sqrt{\Delta}$.

Thus $\Delta = (\sqrt{\Delta})^2 = D^2$, where $D = \sqrt{\Delta} \in F[\sigma_1,\cdots,\sigma_n]$ : therefore $\Delta$ is not irreducible in $F[\sigma_1,\cdots,\sigma_n]$ if the characteristic of $F$ is 2.
\end{enumerate}
\end{proof}

\paragraph{Ex. 2.4.9}

{\it For $n=4$, the variables $x_1,x_2,x_3,x_4$ have discriminant
$$\Delta = (x_1-x_2)^2(x_1-x_3)^2(x_1-x_4)^2(x_2-x_3)^2(x_2-x_4)^2(x_3-x_4)^2.$$
Let $y_1 = x_1x_2+x_3x_4,y_2 = x_1x_3+x_2x_4,y_3=x_1x_4+x_2x_3$, and consider
$$\theta(y) = (y-y_1)(y-y_2)(y-y_3).$$
This is a cubic polynomial in $y$. As in the text, the discriminant of $\theta$ will be denoted $\Delta(\theta)$. Show that $\Delta(\theta) = \Delta$.
}

\begin{proof}
\begin{align*}
y_1-y_2 &= x_1x_2+x_3x_4-x_1x_3-x_2x_4=x_1(x_2-x_3)-x_4(x_2-x_3) =(x_1-x_4)(x_2-x_3)\\
y_1-y_3&= x_1x_2+x_3x_4-x_1x_4-x_2x_3=x_1(x_2-x_4)-x_3(x_2-x_4) =(x_1-x_3)(x_2-x_4)\\
y_2-y_3&= x_1x_3+x_2x_4-x_1x_4-x_2x_3=x_1(x_3-x_4)-x_2(x_3-x_4) =(x_1-x_2)(x_3-x_4),\\
\end{align*}
Therefore
\begin{align*}
\Delta(\theta) &= (y_1-y_2)^2(y_1-y_3)^2(y_2-y_3)^2\\
&=[(x_1-x_4)(x_2-x_3)(x_1-x_3)(x_2-x_4)(x_1-x_2)(x_3-x_4)]^2\\
&=\Delta.
\end{align*}
\end{proof}

\paragraph{Ex. 2.4.10}

{\it Let $C,D \in F[\sigma_1,\ldots,\sigma_n]$ be nonzero and relatively prime. This exercise will show that $C$ and $D$ remain relatively prime when regarded as elements of $F[x_1,\ldots,x_n]$.
\begin{enumerate}
\item[(a)] Show that $C^m,D^m$ are relatively prime in $F[\sigma_1,\ldots,\sigma_n]$ for any positive integer $m$.
\item[(b)] Suppose that $p\in F[x_1,\ldots,x_n]$ is a nonconstant polynomial dividing $C$ and $D$. Prove that $\sigma\cdot p$ divides $C$ and $D$ for all $\sigma \in S_n$.
\item[(c)] As in Exercise 7 of Section 2.2, let $P = \prod_{\sigma \in S_n} \sigma \cdot p$. Show that $P$ divides $C^{n!}$ and $D^{n!}$, and then use part (a) and Exercise 7 of Section 2.2 to obtain a contradiction.
\end{enumerate}
}

\begin{proof}
\begin{enumerate}
\item[(a)]
If $p$ is an irreducible factor in $F[\sigma_1,\cdots,\sigma_n]$ which divides $C^m$ and $D^m$ ($m\in \mathbb{N}^*$), as $F[\sigma_1,\cdots,\sigma_n] \simeq F[u_1,\cdots,u_n]$ is a factorial domain, $p$ divides $C$ and $p$ divides $D$, which is in contradiction with the fact that $C,D$ are relatively prime in $F[\sigma_1,\cdots,\sigma_n]$. Consequently $C^m,D^m$ are relatively prime in $F[\sigma_1,\cdots,\sigma_n]$.


\item[(b)]
If $p$ is an irreducible factor in $F[x_1,\cdots,x_n]$ which divides $C$ and $D$ , then $C = p E, E \in F[x_1,\cdots,x_n]$.
As $C$ is symmetric, we obtain, using 2.31:
\begin{align}
C = \sigma\cdot C = (\sigma\cdot p) (\sigma\cdot E).\label{eq2.4.10:1}
\end{align}
Therefore $\sigma \cdot p$ divides $C$ for all $\sigma \in S_n$, and it is the same for $D$.


\item[(c)] The product, for all $\sigma \in S_n$ of the relations \eqref{eq2.4.10:1} gives :
$$C^{n!} =  \prod_{\sigma \in S_n} \sigma \cdot p \  \prod_{\sigma \in S_n} \sigma \cdot  E.$$
Therefore $P= \prod\limits_{\sigma \in S_n} \sigma \cdot  p$ divides  $C^{n!} $ in $F[x_1,\cdots,x_n]$, and similarly for $D$.


\item[(d)]
By Exercise 2.2.7, $P$ is symmetric, and $C^{n!} = P Q, D^{n!} = P S,\ Q,S \in F[x_1,\cdots,x_n]$.

As $C^{n!},D^{n!},P$ are symmetric, $Q,S$ are also symmetric. Indeed, for all $\sigma \in S_n$, 
$P Q = C^{n!} = \sigma \cdot C^{n!} = (\sigma \cdot P)( \sigma \cdot Q )= P (\sigma \cdot Q),$ thus $Q = \sigma\cdot Q$.

Therefore $P=P_1(\sigma_1,\cdots,\sigma_n)$, and $P_1 \in F[\sigma_1,\cdots,\sigma_n]$ divides $C^{n!},D^{n!}$ in $F[\sigma_1,\cdots,\sigma_n]$. As the irreducible polynomial  $p$ divides $P$, $P_1$ is not a constant. Therefore the two polynomials  $C^{n!},D^{n!}$ are not relatively prime in $F[\sigma_1,\ldots,\sigma_n]$, and by (a), $C,D$ are not relatively prime in $F[\sigma_1,\ldots,\sigma_n]$, in contradiction with the hypothesis.

Conclusion : two relatively prime polynomials in  $F[\sigma_1,\cdots,\sigma_n]$ are also relatively prime in  $F[x_1,\cdots,x_n]$.
\end{enumerate}
\end{proof}

\paragraph{Ex. 2.4.11}

{\it Exercise 8 of section 2.2 showed that if $\varphi \in F(x_1,\ldots,x_n)$ is symmetric, then $\varphi \in F(\sigma_1,\ldots,\sigma_n)$. In this exercise, you will refine this result as follows. Suppose that $\varphi \in F(x_1,\ldots,x_n)$ is symmetric, and write $\varphi = A/B$, where $A,B \in F[x_1,\ldots,x_n]$ are relatively prime. The claim is that A,B are themselves symmetric and hence lie in $F[\sigma_1,\ldots,\sigma_n]$. We can assume that $A$ and $B$ are nonzero.
\begin{enumerate}
\item[(a)] Use the previous exercise and Exercise 8 of section 2.2 to show that $\varphi = C/D$ where $C,D \in F[\sigma_1,\ldots,\sigma_n]$ are relatively prime in $F[x_1,\ldots,x_n]$.
\item[(b)] Show that $AD = BC$ and then use unique factorization in $F[x_1,\ldots,x_n]$ to show that $A$ and $B$ are constant multiples of $C$ and $D$ respectively.
\item[(c)] Conclude that $A,B \in F[\sigma_1,\ldots,\sigma_n]$ as claimed.
\end{enumerate}
}

\begin{proof}
\begin{enumerate}
\item[(a)]
As $\varphi \in F(\sigma_1,\cdots,\sigma_n)$, by Exercise 2.2.8,
$$\varphi = C/D, \ C,D \in F[\sigma_1,\cdots,\sigma_n].$$
Reducing this fraction, we can suppose that $C,D$ are relatively prime in $F[\sigma_1,\cdots,\sigma_n]$, thus relatively prime in  $F[x_1,\cdots, x_n]$ by Exercise 2.4.10.


\item[(b)]
$\varphi = A/B = C/D$, so $AD = BC$, where $C,D$ are symmetric and relatively prime in  $F[x_1,\cdots,x_n]$ , and also $A,B$  relatively prime in  $F[x_1,\cdots,x_n]$.

As $F[x_1,\ldots,x_n]$ is a unique factorisation domain, as $A \mid BC$ and $A,B$ are relatively prime, $A \mid C$. Similarly, $C \mid AD$, and $C,D$ are relatively prime,  so $C \mid A$ : $A$ and $C$ are associate, therefore
$$A = k C, B = k D, k\in F^*.$$


\item[(c)]

Since $C,D$ are symmetric, $A,B$ are also symmetric.

Conclusion : if $\varphi=A/B$ is symmetric, where $A,B \in F[x_1,\ldots,x_n]$ are relatively prime, then $A,B$ are symmetric.
\end{enumerate}
\end{proof}

\end{document}
