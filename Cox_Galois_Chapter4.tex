%&LaTeX
\documentclass[11pt,a4paper]{article}
\usepackage[frenchb,english]{babel}
\usepackage[applemac]{inputenc}
\usepackage[OT1]{fontenc}
\usepackage[]{graphicx}
\usepackage{amsmath}
\usepackage{amsfonts}
\usepackage{amsthm}
\usepackage{amssymb}
%\input{8bitdefs}

% marges
\topmargin 10pt
\headsep 10pt
\headheight 10pt
\marginparwidth 30pt
\oddsidemargin 40pt
\evensidemargin 40pt
\footskip 30pt
\textheight 670pt
\textwidth 420pt

\def\imp{\Rightarrow}
\def\gcro{\mbox{[\hspace{-.15em}[}}% intervalles d'entiers 
\def\dcro{\mbox{]\hspace{-.15em}]}}

\newcommand{\be} {\begin{enumerate}}
\newcommand{\ee} {\end{enumerate}}
\newcommand{\deb}{\begin{eqnarray*}}
\newcommand{\fin}{\end{eqnarray*}}
\newcommand{\ssi} {si et seulement si }
\newcommand{\D}{\mathrm{d}}
\newcommand{\Q}{\mathbb{Q}}
\newcommand{\Z}{\mathbb{Z}}
\newcommand{\N}{\mathbb{N}}
\newcommand{\R}{\mathbb{R}}
\newcommand{\C}{\mathbb{C}}
\newcommand{\F}{\mathbb{F}}
\newcommand{\re}{\,\mathrm{Re}\,}
\newcommand{\ord}{\mathrm{ord}}
\newcommand{\legendre}[2]{\genfrac{(}{)}{}{}{#1}{#2}}

\title{Solutions to David A.Cox  "Galois Theory''}
\author{Richard Ganaye}
\refstepcounter{section} \refstepcounter{section} \refstepcounter{section}


\begin{document}

\maketitle


\section{Chapter 4}

\subsection{FIELDS}

\paragraph{Ex. 4.1.1}

{\it Let $\alpha \in L \setminus \{0\}$ be algebraic over a subfield $F$. Prove that $1/\alpha$ is also algebraic over $F$.
}

\begin{proof}
Suppose that $\alpha \in L \setminus \{0\}$ be algebraic over a subfield $F$ of $L$. Then there exists a polynomial $p = \sum\limits_{k=0}^d a_k x^k \in F[x]$, with $a_d \neq 0$, whose $\alpha$ is a root:
$$\sum\limits_{k=0}^d a_k \alpha^k = 0.$$
Dividing by $\alpha^d$, we obtain $\sum\limits_{k=0}^d a_k\left(\frac{1}{\alpha}\right)^{d-k} = 0$, which we can write
$\sum\limits_{i=0}^d a_{d-i}\left(\frac{1}{\alpha}\right)^{i} = 0$.

So $1/\alpha$ is a root of the polynomial $q = \sum\limits_{i=0}^d a_{d-i}x^{i} \in F[x]$, and $q\neq 0$ since $a_d \neq0$, thus $1/\alpha$ is algebraic over $F$.
\end{proof}

\paragraph{Ex. 4.1.2}

{\it Complete the proof of Lemma 4.1.3 by showing that if $f$ and $g$ are monic polynomials in $F[x]$ each of which divides the other, then $f=g$.
}

\begin{proof}
Suppose that $f,g \in F[x]$  are monic, and $f\mid g, g\mid f$. 

$f = g h , h\in F[x]$ and $g = f l , l \in F[x]$, so $f = f hl$, where $f\neq 0$ since $f$ is monic, thus $hl=1$, and so $\deg(h)+ \deg(l)=0$, $\deg(h) = \deg(l) = 0$.

Therefore $h = \lambda \in F^*$, $f = \lambda g$. In particular, $f,g$ have the same degree $d$.

Write $f = \sum\limits_{k=0}^d a_k x^k, g =  \sum\limits_{k=0}^d b_k x^k$.

As $f,g$ are monic, $a_d=b_d=1$, and $a_d = \lambda b_d$, so $\lambda=1$, and $f = g$.

Conclusion: If $f$ and $g$ are monic polynomials in $F[x]$ each of which divides the other, then $f=g$
\end{proof}

\paragraph{Ex. 4.1.3}

{\it Suppose that $F\subset L$ is a field extension and that $\alpha_1,\ldots,\alpha_n \in L$. Show that $F[\alpha_1,\ldots,\alpha_n]$ is a subring of $L$ and that $F(\alpha_1,\ldots,\alpha_n)$ is a subfield of $L$.
}

\begin{proof}
$\bullet$ By hypothesis,  $F\subset L$ and $\alpha_1,\ldots,\alpha_n \in L$.

$1 \in F[\alpha_1,\ldots,\alpha_n],$ so $F[\alpha_1,\ldots,\alpha_n] \neq \emptyset$.

Let $x,y \in F[\alpha_1,\ldots,\alpha_n]$. By definition, there exist polynomials $p,q \in F[x_1,\ldots,x_n]$ such that
$$x = p(\alpha_1,\ldots,\alpha_n),\quad y = q(\alpha_1,\ldots,\alpha_n).$$
As $p-q,pq \in F[x_1,\ldots,x_n]$, and as $x-y = (p-q)(\alpha_1,\ldots,\alpha_n), xy = pq(\alpha_1,\ldots,\alpha_n)$, so $x-y \in F[\alpha_1,\ldots,\alpha_n], xy \in F[\alpha_1,\ldots,\alpha_n]$.

Conclusion: $F[\alpha_1,\ldots,\alpha_n]$ is a subring of $L$.

$\bullet$  The same argument, where we take rational fractions $p,q$ in place of polynomials show that $p,q \in F(x_1,\ldots,x_n) \Rightarrow p-q,pq \in F(x_1,\ldots,x_n)$, so $x-y = (p-q)(\alpha_1,\ldots,\alpha_n), xy = pq(\alpha_1,\ldots,\alpha_n) \in F(\alpha_1,\ldots,\alpha_n)$. Thus $F(\alpha_1,\ldots,\alpha_n)$ is a subring of $L$.

Moreover, if $x \in F(\alpha_1,\ldots,\alpha_n), x\neq 0$, then $x = \frac{p(\alpha_1,\ldots,\alpha_n)}{q(\alpha_1,\ldots,\alpha_n)}$, where $p,q \in F[x_1,\ldots,x_n]$, and $q(\alpha_1,\ldots,\alpha_n)\neq 0$. Since $x\neq 0$, we have also $p(\alpha_1,\ldots,\alpha_n) \neq 0$.

Hence $\frac{1}{x} = \frac{q(\alpha_1,\ldots,\alpha_n)}{p(\alpha_1,\ldots,\alpha_n)} \in F(\alpha_1,\ldots,\alpha_n)$.

Conclusion: $F(\alpha_1,\ldots,\alpha_n)$ is a subfield of $L$.
\end{proof}

\paragraph{Ex. 4.1.4}

{\it Complete the proof of Corollary 4.1.11 by showing that
$$F(\alpha_1,\ldots,\alpha_r)(\alpha_{r+1},\ldots,\alpha_n) \subset F(\alpha_1,\ldots,\alpha_n).$$
}

\begin{proof}  $F(\alpha_1,\ldots,\alpha_r) \subset F(\alpha_1,\ldots,\alpha_n),\ 1\leq r \leq n$, since $F(\alpha_1,\ldots,\alpha_n)$ contains $F$ and $\alpha_1,\ldots,\alpha_r$, and since $F(\alpha_1,\ldots,\alpha_r)$ is the smallest subfield of $L$ containing $F$ and $\alpha_1,\ldots,\alpha_r$.

Moreover $F(\alpha_1,\ldots,\alpha_n)$ contains $\alpha_{r+1}, \ldots, \alpha_n$.

By Lemma 4.1.9, $F(\alpha_1,\ldots,\alpha_r)(\alpha_{r+1},\ldots,\alpha_n)$ is the smallest subfield of $L$ containing $F(\alpha_1,\ldots,\alpha_r)$ and $\alpha_{r+1},\ldots,\alpha_n$, thus 
$$F(\alpha_1,\ldots,\alpha_r)(\alpha_{r+1},\ldots,\alpha_n) \subset F(\alpha_1,\ldots,\alpha_n).$$

From the reciprocal inclusion proved in section 4.1, we conclude that
$$F(\alpha_1,\ldots,\alpha_r)(\alpha_{r+1},\ldots,\alpha_n) = F(\alpha_1,\ldots,\alpha_n).$$
\end{proof}

\paragraph{Ex. 4.1.5}

{\it Prove carefully that $F[\alpha_1,\ldots,\alpha_{n-1}][\alpha_n] = F[\alpha_1,\ldots,\alpha_n]$.
}

\begin{proof}
$\bullet$ Let $\gamma \in F[\alpha_1,\ldots,\alpha_{n-1}][\alpha_n]$. Write  $R =  F[\alpha_1,\ldots,\alpha_{n-1}]$. By definition, there exists a polynomial $p = \sum_{k=0}^d a_k x_n^k \in R[x_n]$ such that $\gamma= p(\alpha_n)$, and for every  $a_k \in R, 0 \leq k \leq d$, there exists $f_k  \in F[x_1,\ldots,x_{n-1}]$ such that $a_k = f_k(\alpha_1,\ldots,\alpha_{n-1})$.

Thus $$\gamma = \sum\limits_{k=0}^d  f_k(\alpha_1,\ldots,\alpha_{n-1})\alpha_n^k.$$

Let $f = \sum\limits_{k=0}^d f_k(x_1,\ldots,x_{n-1}) x_n^k$. Then $f \in F[x_1,\ldots,x_n]$, and $\gamma = f(\alpha_1,\ldots,\alpha_n)$, so $\gamma \in F[\alpha_1,\ldots,\alpha_n]$. We have proved
$$F[\alpha_1,\ldots,\alpha_{n-1}][\alpha_n] \subset F[\alpha_1,\ldots,\alpha_n].$$


$\bullet$ Conversely, let $\gamma \in F[\alpha_1,\ldots,\alpha_n]$. 

There exists $f \in F[x_1,\ldots,x_n]$ such that $x = f(\alpha_1,\ldots,\alpha_n)$.

As $F[x_1,\ldots,x_n] = F[x_1,\ldots,x_{n-1}][x_n]$, $f = \sum\limits_{k=0}^d f_k(x_1,\ldots,x_{n-1})x_n^k$, where $f_k \in F[x_1,\ldots,x_{n-1}]$.

So  $\gamma = \sum\limits_{k=0}^d f_k(\alpha_1,\ldots,\alpha_{n-1})\alpha_n^k = \sum\limits_{k=0}^d a_k x_n^k$, with $a_k = f_k(\alpha_1,\ldots,\alpha_{n-1}) \in F[\alpha_1,\ldots,\alpha_{n-1}]=R$.

Let $p = \sum_{k=0}^d a_k x_n^k$. Then $p \in R[x_n]$ and $x = p(\alpha_n)$, thus $x \in R[\alpha_n] = F[\alpha_1,\ldots,\alpha_{n-1}][\alpha_n]$.

The reciprocal inclusion
$$F[\alpha_1,\ldots,\alpha_n] \subset F[\alpha_1,\ldots,\alpha_{n-1}][\alpha_n]$$
is proved, and so 
$$F[\alpha_1,\ldots,\alpha_n] =F[\alpha_1,\ldots,\alpha_{n-1}][\alpha_n].$$

Note: in an alternative way, we could write a lemma analogous to Lemma 4.1.9 and show that $F[\alpha_1,\ldots,\alpha_n]$ is the smallest subring of $L$ containing $ \alpha_1,\ldots,\alpha_n$ (where $L$ is a ring containing $F$ and $\alpha_1,\ldots,\alpha_n$), and prove as in Exercise 4 that
$$F[\alpha_1,\ldots,\alpha_r][\alpha_{r+1},\ldots,\alpha_n] = F[\alpha_1,\ldots,\alpha_n].$$
\end{proof}

\paragraph{Ex. 4.1.6}

{\it Suppose that $F\subset L$ and that $\alpha_1,\ldots,\alpha_n\in L$ are algebraically independent over $F$ (as defined in the Mathematical Notes to section 2.2). Prove that there is an isomorphism of fields
$$F(\alpha_1,\ldots,\alpha_n) \simeq F(x_1,\ldots,x_n),$$
where $F(x_1,\ldots,x_n)$ is the field of rational functions in variables $x_1,\ldots,x_n$.
}

\begin{proof}

Let $f \in F(x_1,\ldots,x_n)$, $f =p/q,\ p,q \in F[x_1,\ldots,x_n], q \neq 0$. Since $\alpha_1,\ldots,\alpha_n$ are algebraically independent over $F$, $q(\alpha_1,\ldots,\alpha_n) \neq 0$. We can so define
\begin{align*}
\varphi : F(x_1,\ldots,x_n) &\to F(\alpha_1,\ldots,\alpha_n)\\
 f = p/q  &\mapsto f(\alpha_1,\ldots,\alpha_n) = p(\alpha_1,\ldots,\alpha_n)/q(\alpha_1,\ldots,\alpha_n).
\end{align*}
(this quotient doesn't depend on the choice of the representative $p/q$ of $f$).

$\varphi$ is a ring homomorphism.

By definition of $F(\alpha_1,\ldots,\alpha_n)$, $\varphi$ is surjective.

Let $f = p/q \in F(x_1,\ldots,x_n)$, with $p,q \in F[x_1,\ldots,x_n], q\neq 0$. If $f \in \ker(\varphi)$, then $p(\alpha_1,\ldots,\alpha_n)/q(\alpha_1,\ldots,\alpha_n) =0$, thus $p(\alpha_1,\ldots,\alpha_n)=0$. Since $\alpha_1,\ldots,\alpha_n$ are algebraically independent, $p=0$. Consequently $\ker(\varphi) = \{0\}$, and so $\varphi$ is a ring isomorphism between two fields: it is a field isomorphism.

Conclusion: If $\alpha_1,\ldots,\alpha_n \in L$ are algebraically independent over $F$,then
$$F(\alpha_1,\ldots,\alpha_n) \simeq  F(x_1,\ldots,x_n).$$
\end{proof}

\paragraph{Ex. 4.1.7}

{\it In the proof of Proposition 4.1.14, we used the quotient ring $F[x]/\langle p\rangle$ to show that $F[\alpha]$ is a field when $\alpha$ is algebraic over $F$ with minimal polynomial $p \in F[x]$.Here, you will prove that $F[\alpha]$ is a field without using quotient rings. Since we know that $F[\alpha]$ is a ring, it suffices to show that every nonzero element $\beta \in F[\alpha]$ has a multiplicative inverse in $F[\alpha]$. So pick $\beta \ne 0$ in $F[\alpha]$ Then $\beta = g(\alpha)$ for some $g \in F[x]$.
\begin{enumerate}
\item[(a)] Show that $g$ and $p$ are relatively prime in $F[x]$.
\item[(b)] By part (a) and the Euclidean algorithm, we have $Ap+Bg = 1$ for some $A,B \in F[x]$. Prove that $B(\alpha) \in F[\alpha]$ is the multiplicative inverse of $g(\alpha)$.
\end{enumerate}
Do you see how this exercise relates to Exercise 5 of section 3.1?
}

\begin{proof}
As in Proposition 4.1.14, we assume that $F\subset L$ is a field extension, and that $\alpha \in L$. Suppose that $\alpha\in L$ is algebraic over $F$, where $p \in F[x]$ is the minimal polynomial of $\alpha$ over $F$,  and $\beta \in F[\alpha],\beta \neq 0$.

There exists $g \in F[x]$ such that $\beta = g(\alpha)$.

\begin{enumerate}
\item[(a)] The minimal polynomial $p$ of $\alpha$ is irreducible over $F$ (Prop. 4.1.5).

Let $u\in F[x]$ such that $u \mid p, u \mid g$. Then
$p = uq,\ q\in F[x]$, and since $p$ is irreducible over $F$,  $u$ or $q$ is a constant of $F^*$.

If $q =\lambda \in F^*$, then $u = \lambda^{-1} p $ and $p$ 
divides $u$, which divides $g$, thus $p$ divides $g$. In this case, since $p(\alpha) = 0$, $\beta = g(\alpha) = 0$, in contradiction with the hypothesis $\beta \neq 0$.

So $u = \mu \in F^*$ , $u \mid 1$. Consequently, for all $u \in F[x]$, $(u \mid p, u \mid g) \Rightarrow u\mid 1$: $p,g$ are relatively prime.

\item[(b)]  Then there exists a B\'ezout's relation between these two polynomials:
$$Ap + Bg = 1, \ A,B \in F[x].$$
The evaluation of these polynomials in $\alpha$, since $p(\alpha)=0$, gives
$$B(\alpha) g(\alpha) = 1, B(\alpha) \in F[\alpha]$$
So $B(\alpha)$ is the multiplicative inverse of $\beta = g(\alpha)\neq 0$ in $F[\alpha]$: $F[\alpha]$ is a field.

Note: We have proved in Exercise 3.5.1 that $F[x]/\langle f\rangle$, where $f$ is irreducible over $F$,  is a field with the same argumentation. Here $f = p$ is the minimal polynomial  of $\alpha$ over $F$, so it is irreducible over $f$.
\end{enumerate}
\end{proof}

\paragraph{Ex. 4.1.8}

{\it If a polynomial is irreducible over a field $F$, it may or may not remain irreducible over a large field. Here are examples of both types of behavior.
\begin{enumerate}
\item[(a)] Prove that $x^2-3$ is irreducible over $\Q(\sqrt{2})$.
\item[(b)] In Example 4.1.7, we showed that $x^4 -10x^2+1$ is irreducible over $\Q$ (it is the minimal polynomial of $\alpha = \sqrt{2}+\sqrt{3}$). Show that $x^4-10x^2+1$ is not irreducible over $\Q(\sqrt{3})$.
\end{enumerate}
}

\begin{proof}
\begin{enumerate}
\item[(a)]$x^2-3$ is irreducible over $\mathbb{Q}$. We show that it remains irreducible over $\mathbb{Q}[\sqrt{2}]$.

Suppose on the contrary that $f$ is reducible over $F$:  $f = x^2 - 3 = u v,\ u v \in \mathbb{Q}[\sqrt{2}][x]$, where $u,v$ are nonconstant polynomials. Then $\deg(u)\geq 1, \deg(v) \geq 1$, and as $\deg(u) + \deg(v) = \deg(f) = 2$, $\deg(u) = \deg(v)=1$, 

$$u = ax+b, a,b\in \mathbb{Q}[\sqrt{2}], a\neq 0.$$

Then $\alpha = -b/a \in \mathbb{Q}[\sqrt{2}]$ is a root of $u$, thus is a root of $f = x^2-3$. Since $\sqrt{2}^{2n} = 2^n$ and $\sqrt{2}^{2n+1} = 2^n \sqrt{2}$, every element of $\mathbb{Q}[\sqrt{2}]$  is of the form $c+ d\sqrt{2} , \ c,d \in \mathbb{Q}$.

We should have $\alpha = c + d \sqrt{2} = \pm \sqrt{3}. $
Then $$\alpha^2 = c^2+2d^2  +2cd\sqrt{2}=3.$$

If $cd\neq 0$, $\sqrt{2} = (c^2+2d^2 -3)/(2cd) \in \mathbb{Q}$, in contradiction with the irrationality of $\sqrt{2}$. Thus $c=0$ or $d=0$.

$d=0$ gives $\sqrt{3} = \pm c \in \mathbb{Q}$: this is in contradiction with the irrationality of $\sqrt{3}$.

$c=0$ implies $\sqrt{\frac{3}{2}} = \pm d \in \mathbb{Q}$. But then $\sqrt{\frac{3}{2}} =\frac{p}{q}, (p,q) \in \mathbb{Z} \times \mathbb{N}^*, p\wedge q=1$.

$3q^2 = 2 p^2$, $q^2 \mid 2 p^2$ and $q^2 \wedge p^2 = 1$. By Gauss Lemma, $q^2 \mid 2, q\in \mathbb{N}^*$, hence $q=1, 3 = 2p^2$, thus $3$ is even: this is absurd.

Conclusion: $x^2-3$ is irreducible $\mathbb{Q}[\sqrt{2}]$.


\item[(b)]
\begin{align*}
 f&= [(x-\sqrt{2} - \sqrt{3})(x+\sqrt{2} -\sqrt{3})] [(x-\sqrt{2} + \sqrt{3})(x+\sqrt{2} + \sqrt{3})]\\
&=[(x-\sqrt{3})^2-2][(x+\sqrt{3})^2-2]\\
&= (x^2 -2\sqrt{3} x +1)(x^2 -2\sqrt{3} x +1)\\
&= (x^2+1)^2 - (2\sqrt{3}x)^2\\
&=x^4 -10x^2 +1
\end{align*}

The equality $f = x^4 -10x^2 +1 = (x^2 -2\sqrt{3} x +1)(x^2 -2\sqrt{3} x +1)$ show that $f$ is not irreducible over $\mathbb{Q}[\sqrt{3}]$.

{\bigskip}

Factorisation with Sage:
\begin{verbatim}
K = NumberField(x^2-3, 'a');L.<X> = PolynomialRing(K)
p = X^4-10*X^2+1
factor(p)
\end{verbatim}
$$(X^2-2aX+1).(X^2+2aX+1).$$
\end{enumerate}
\end{proof}

\subsection{IRREDUCIBLE POLYNOMIALS}
\paragraph{Ex. 4.2.1}

{\it This exercise will study the Lagrange interpolation formula. Suppose that $F$ is a field and that $b_0,\ldots,b_d,c_0,\ldots,c_d \in F$, where $b_0,\ldots,b_d$ are distinct and $d\geq 1$. Then consider the polynomial
$$g(x) = \sum_{i=0}^d c_i \prod_{j\ne i} \frac{x-b_j}{b_i-b_j} \in F[x].$$
\begin{enumerate}
\item[(a)] Explain why $\deg(g) \leq d$, and give an example for $F=\R$ and $d=2$ where $\deg(g)<2$.
\item[(b)] Show that $g(b_i) = c_i$ for $i=0,\ldots,d$.
\item[(c)] Let $h$ be a polynomial in $F[x]$ with $\deg(h) \leq d$ such that $h(b_i) = c_i$ for $i=0,\ldots,d$. Prove that $h=g$.
\end{enumerate}
}

\begin{proof}
Let $p_i(x) = \prod\limits_{j\neq i} \frac{x-b_j}{b_i-b_j}\ , 0 \leq i \leq d$.
Then $g(x) = \sum\limits_{i=0}^d c_i p_i(x)$.

\begin{enumerate}
\item[(a)]
$p_i$ is product of $d$ linear polynomials, thus $\deg(p_i) = d$. Consequently $\deg(g) \leq \max(\deg(p_0), \ldots, \deg(p_d))=d$: 
$$\deg(g)\leq d.$$

This inequality can be a strict inequality: We show such an example for $d=2$.

 $(b_0,c_0) = (0,0), (b_1,c_1) = (1,1), (b_2,c_2) = (2,2)$.

Then $p_0(x) = \frac{1}{2}(x-1)(x-2),p_1(x) = -x(x-2),p_2(x) = \frac{1}{2}x(x-1)$. So
\begin{align*}
g(x)&= 0.p_0(x)+1.p_1(x)+2.p_2(x)\\
&=-x(x-2)+x(x-1)\\
&=x.
\end{align*}
Here $\deg(g)=1 < d=2$.

\item[(b)] $p_i(b_i)=1$ and $p_i(b_j)=0$ if $j\neq i$, so $p_i(b_j) =\delta_{i,j}$.
$$g(b_j) = \sum\limits_{i=0}^dc_i \delta_{i,j} = c_j, \ j=0,\ldots,d.$$
The graph of the polynomial  $g$ with degree at most $d$ contains the $d+1$ points $(b_0,c_0), \ldots,(b_d,c_d)$.

\item[(c)]
Suppose that the polynomial $h \in F[x]$ satisfies the same conditions as $g$: 
\begin{center}
 $h(b_i) = c_i,\  0\leq i\leq d$, with $\deg(h)\leq d$. \end{center}
 Let $p = g-h$. Then $\deg(p) \leq \max(\deg(g),\deg(h)) \leq d$, and $p(b_i) = g(b_i) - h(b_i) = c_i-c_i=0, i = 0,\ldots,d$.
 
 $p$ is a polynomial with degree at most $d$ and has $d+1$ roots, hence $p=0$, so 
 $$g=h.$$
 
Conclusion: There exists one and only one polynomial $g$ with degree at most $d$ such that $g(b_i) = c_i,\  i=0,\ldots,d$ (where $b_0,\ldots,b_d$ are distinct, $d\geq 1$)
\end{enumerate}
\end{proof}

\paragraph{Ex. 4.2.2}

{\it This exercise deals with Sch\"onemann's version of the irreducibility criterion.
\begin{enumerate}
\item[(a)] Let $f(x) = (x-a)^n + pF(x)$, where $a\in \Z$ and $F(x) \in \Z[x]$ satisfy $\deg(F) \leq n$, and $p\nmid F(a)$. Prove that $f$ is irreducible over $\Q$.
\item[(b)] More generally, let $g(x) \in \Z[x]$ be irreducible modulo $p$ (i.e., reducing its coefficients modulo $p$ gives an irreducible polynomial in $\F_p[x]$). Then let $f(x) = g(x)^n + pF(x)$, where $F[x] \in \Z[x]$ and $g(x)$ and $F(x)$ are relatively prime modulo $p$. Also assume that $\deg(F) \leq n \deg(g)$. Prove that $f$ is irreducible over $\Q$.
\end{enumerate}
}

\begin{proof}
\begin{enumerate}
\item[(a)]
Let $f(x) = (x-a)^n + pF(x)$, where $a\in \mathbb{Z}$, and $p$ is prime. We show that $f$ is irreducible. If we suppose on the contrary that $f$ is reducible over $\Q$, then by Corollary 4.2.1  $$f = g h,\ g,h \in \mathbb{Z}[x], k = \deg(g)\geq 1,l=\deg(h) \geq 1.$$

As $\deg(F) \leq n$, $\deg(f)\leq n$, and as the coefficient of $x^n$ in $f$ is congruent to 1 modulo $p$, it is nonzero, so $\deg(f) = n$, and $k+l=n$.

Write $\overline{f} \in \mathbb{F}_p[x]$ the reductio modulo $p$ of $f$, and write $\overline{a} = [a]_p$ the class of $a\in \mathbb{Z}$ modulo $p$. 

The application 
\begin{align*}
\varphi : \mathbb{Z}[x] &\to \mathbb{F}_p[x]\\
q= \sum_{i=0}^d a_i x^i &\mapsto \overline{q} =  \sum_{i=0}^d \overline{a_i} x^i 
 \end{align*}
is a ring homomorphism, so  $\overline{f} = \overline{gh} = \overline{g}\,\overline{h}$.
 
Thus $$\overline{f} = (x- \overline{a})^n = \overline{g}\overline{h}$$
 
As $\deg(\overline{g}) \leq \deg(g), \deg(\overline{h}) \leq \deg(h)$ and as $\deg(\overline{g})+\deg(\overline{h}) = \deg((x- \overline{a})^n)=n = \deg(g)+ \deg(h)$, we conclude that $\deg(\overline{g}) = \deg(g) = k, \deg(\overline{h})=\deg(h)=l$.
 
 $x-\overline{a}$ is irreducible in $\mathbb{F}_p[x]$, as every polynomial of degree 1. $\mathbb{F}_p$ being a field, the unicity of the decomposition in irreducible factors in  the principal ideal domain $\mathbb{F}_p[x]$ shows that the only irreducible factors of $\overline{g}, \overline{h}$ are associate to powers of $x-\overline{a}$: 
 
$$\overline{g} = \overline{u}( x-\overline{a})^k, \overline{h} = \overline{v}( x-\overline{a})^l, \ \overline{u},\overline{v} \in \mathbb{F}_p^*.$$
Hence there exist polynomials $G,H \in \mathbb{Z}[x]$ such that
 $$g = u(x-a)^k + p G(x), h =v(x-a)^l + pH(x).$$
Consequently
 $$f(x) = (x-a)^n + pF(x) = [u(x-a)^k + p G(x)][v(x-a)^l + pH(x)].$$
As $k\geq 1, l \geq 1$, $(x-a)^k$ and $(x-a)^l$ have $a$ as a root, thus
 $$f(a) = pF(a) = p^2 G(a) H(a).$$
Then $F(a) = p G(a)H(a)$ is divisible by $p$, in contradiction with the hypothesis $p\nmid F(a)$.
 
 Conclusion: $f\in \mathbb{Z}[x]$ is not a product of nonconstant polynomials  in $\mathbb{Z}[x]$. By Corollary 4.2.1, $f$ is irreducible over $\mathbb{Q}$.
 

 \item[(b)]
More generally, suppose that $u \in \mathbb{Z}[x]$ is such that $\overline{u}$ is irreducible over $\F_p$, and that $f(x) = u(x)^n + pF(x)$, $F(x) \in \mathbb{Z}[x], \overline{u}\wedge \overline{F} =1$ and $\deg(F) \leq n \deg(u)$.

We must suppose also that the leading coefficient of $u$ is not divisible by $p$, so $\deg(\overline{u}) = \deg(u) $.

Then $\deg(f) \leq n \deg(u)$, and the coefficient of the monomial of degree $n \deg(u)$ being nonzero modulo $p$, $\deg(f) = n \deg(u)=n\deg(\overline{u}) =  \deg(\overline{f})$.

If we suppose $f$ reducible, then  $f = gh, k = \deg(g)\geq 1, l = \deg(h) \geq 1$, which implies as in (a) 
$$\overline{f} = \overline{u}^n = \overline{g}\overline{h}.$$
Since $\overline{u}$ is irreducible,
 
 $$\overline{g} = \overline{c} \, \overline{u}^i, \overline{h} =\overline{d} \,  \overline{u}^j,\  i,j \in \mathbb{N}, \ \overline{c}, \overline{d} \in \F_p$$
 
As $\deg(\overline{g}) \leq \deg(g), \deg(\overline{h}) \leq \deg(g)$, and $\deg(\overline{g}) + \deg(\overline{h}) = \deg(\overline{f}) = \deg(f) = \deg(g) + \deg(h)$, we conclude $\deg(\overline{g})  = \deg(g)\geq 1, \deg(\overline{h})= \deg(h)\geq 1$.  Consequently $i\geq 1, j \geq 1$.

There exist polynomials $G,H \in \mathbb{Z}[x]$ such that
$$g= cu^i + p G, h = du^j+pH.$$
Thus
$$f = u^n +pF = (cu^i+pG)(du^j+pH).$$

As $i\geq 1, j\geq 1$, $u$ divides $pF - p^2 GH$ in $\mathbb{Z}[x]$, so there exists $v \in \mathbb{Z}[x]$ such that 
$$uv = p(F - pGH).$$

As $\overline{u}\, \overline{v} = 0$, and $ \overline{u} \neq 0$ in the integral domain $\F_p[x]$, then $\overline{v} = 0$: all the coefficients of $v$ are divisible by $p$, thus $w= v/p \in \mathbb{Z}[x]$, and
$$uw = F-pGH,\qquad  \overline{u}\, \overline{w} = \overline{F}.$$
Hence $\overline{u} \mid \overline{F}$, in contradiction with the hypothesis $\overline{u} \wedge \overline{F} = 1$.

$f = u^n +pF$ is so irreducible.
\end{enumerate}
\end{proof}

\paragraph{Ex. 4.2.3}

{\it Use part (a) of Exercise 2 with $a=1$ to give another proof of Proposition 4.2.5.
}

\begin{proof}
{\bf Lemma}: {\it If $p$ is prime, then for all $k, 0\leq k \leq p-1$,
$$\binom{p-1}{k} \equiv (-1)^k \pmod p.$$}

Proof by induction on $k$.

$\bullet$ If $k=0$, $\binom{p-1}{0} = 1 = (-1)^0$.

$\bullet$ Suppose that this property is true for $k-1$ ($1\leq k \leq p-1$): 
$$\binom{p-1}{k-1} \equiv (-1)^{k-1} \pmod p$$
 Then, as $1\leq k \leq p-1$, we know that $\binom{p}{k} \equiv 0\  \pmod p$, thus from Pascal's formula,
$$\binom{p-1}{k} = \binom{p}{k} - \binom{p-1}{k-1} \equiv 0 - (-1)^{k-1} \equiv (-1)^{k} \pmod p,$$
which concludes the induction. $\qed$

If $p = 2$, $\Phi_2 = 1+x$ is irreducible. Suppose now that $p$ is an odd prime. 

Applying the lemma, we obtain
\begin{align*}
\Phi_p(x) - (x-1)^{p-1} &= \sum_{k=0}^{p-1} x^k -\sum_{k=0}^{p-1}(-1)^{p-1-k} \binom{p-1}{k} x^k\\
&=\sum_{k=0}^{p-1} \left[ 1 - (-1)^{p-1-k} \binom{p-1}{k} \right] x^k\\
&=\sum_{k=0}^{p-1} \left[ 1 - (-1)^{k} \binom{p-1}{k} \right] x^k\\
&=p \sum_{k=0}^{p-1} a_k x^k \qquad (a_k \in \Z)
\end{align*}
since every coefficient $\left[ 1 - (-1)^{k} \binom{p-1}{k} \right]$ is divisible by $p$, of the form $p a_k, a_k \in \mathbb{Z}$.

Consequently $$\Phi_p(x) = (x-1)^{p-1} + pF(x), F(x) = \sum_{k=0}^{p-1} a_k x^k \in \mathbb{Z}[x], \deg(F) \leq p-1.$$
Moreover 
$$F(1) = \sum_{k=0}^{p-1} a_k =  \sum_{k=0}^{p-1} \frac{ 1 - (-1)^{k} \binom{p-1}{k} }{p} = 1 -  \frac{1}{p} \sum_{k=0}^{p-1}(-1)^{k} \binom{p-1}{k} = 1 - \frac{1}{p}(1-1)^{p-1} = 1.$$

$F(1)\not \equiv 0 \pmod p$. By Exercise 2, $\Phi_p$ is irreducible.
\end{proof}

\paragraph{Ex. 4.2.4}

{\it For each of the following polynomials, use a computer to determine whether it is irreducible over the given field.
\be
\item[(a)] $x^4 + x^3 + x^2 + x + 2$ over $\Q$.
\item[(b)] $3x^6 + 6 x^5 + 9x^4 + 2x^3 + 3x^2 + 1$ over $\Q$ and $\Q(\sqrt[3]{2})$.
\ee
}

\begin{proof}
\begin{enumerate}
\item[(a)]
With Sage, the instructions
\begin{verbatim}
factor(x^4+x^3+x^2+x+2)
factor(3*x^6+6*x^5+9*x^4+2*x^3+3*x^2+1);
\end{verbatim}
give the same polynomials.

So $x^4+x^3+x^2+x+2$ and $3x^6+6x^5+9x^4+2x^3+3x^2+1$ are irreducible over $\Q$.

\item[(b)]
The instructions

\begin{verbatim}
K = NumberField(x^3-2, 'a'); L.<X> = PolynomialRing(K)
p =  3*x^6 + 6*x^5 + 9*x^4 + 2*x^3 + 3*x^2 + 1
u = factor(p)
\end{verbatim}
give the following decomposition, where $a=\sqrt[3]{2}$:

$3x^6+6x^5+9x^4+2x^3+3x^2+1 =$ 

$\frac{1}{3} (3x^2 + (-a^2 + a + 2)x + a^2 - a + 1)\times$

$(3x^4 + (a^2 - a + 4)x^3 + (a + 4)x^2 + (-a^2 - a)x + a + 1)$.

Thus $3x^6+6x^5+9x^4+2x^3+3x^2+1$ is not irreducible over $\Q(\sqrt[3]{2})$.
\end{enumerate}
\end{proof}

\paragraph{Ex. 4.2.5}

{\it Find the minimal polynomial of the 24th root of unity $\zeta_{24}$ as follows.
\begin{enumerate}
\item[(a)] Factor $x^{24} - 1$ over $\Q$.
Determine which of the factors is the minimal polynomial of $\zeta_{24}$.
\end{enumerate}
}

\begin{proof}
\begin{enumerate}
\item[(a)] The instruction Sage  'factor' gives the decomposition

$x^{24}-1 = (x^8 - x^4 + 1)(x^4 - x^2 + 1)(x^4 + 1)(x^2 + x + 1)(x^2 - x +
1)(x^2 + 1)(x + 1)(x - 1)$


\item[(b)]
The Sage instructions
\begin{verbatim}
zeta = exp(2*i*pi/24)
(x^8 - x^4 + 1).subs(x=zeta).expand()
\end{verbatim}
return the value 0.

Thus $\zeta _{24} = e^{2i \pi}/24$ is a root of $x^8-x^{4}+1$, irreducible over $\mathbb{Q}$ by (a).

$x^8 - x^4+1$ is so the minimal polynomial over $\mathbb{Q}$ of $\zeta_{24}$.

Verification: $\zeta_{24}^8 - \zeta_{24}^4+1 = e^{2i\pi/3}-e^{i\pi/3}+1 = \omega+\omega^2+1 = 0$.


Note: If we know the cyclotomic polynomials, since $3$ is prime:
\begin{align*}
\Phi_3(x) &= x^2+x+1,\\
\Phi_6(x) &= \Phi_3(-x) = x^2-x+1,\\
\Phi_{24}(x) &= \Phi_{\mathrm{rad}(24)} (x^{\frac{24}{\mathrm{rad}(24)}}) = \Phi_6(x^4) = x^8 - x^4+1,
\end{align*}
($24 = 3\times2^3, \mathrm{rad}(24) = 3\times 2 = 6$).
\end{enumerate}
$\Phi_{24}$ is the minimal polynomial of $\zeta_{24}$ over $\mathbb{Q}$.
The decomposition in (a) is the decomposition
$$x^{24}-1 =\prod_{d\mid 24 }  \Phi_d(x) = \Phi_{24} \,\Phi_{12} \,\Phi_{8} \,\Phi_{3}\, \Phi_{6} \,\Phi_{4} \,\Phi_{2}\, \Phi_{1}.$$
\end{proof}

\paragraph{Ex. 4.2.6}

{\it Let $F$ be a finite field. Explain why there is an algorithm for deciding whether $f \in F[x]$ is irreducible.
}

\begin{proof}
If $f$ is reducible, of degree $n$, $f=gh, g,h\in F[x]$, where $1\leq \deg(g)\leq \deg(h) \leq n-1$.

As $\deg(g)+\deg(h) = n$, $2 \deg(g) \leq n, \deg(g) \leq n/2$.
If we multiply $g,h$ by appropriate constants, we can suppose that $g$ is monic.

So $f$ is reducible iff there exists a monic factor of $f$ of degree $d$,  $d, 1 \leq d \leq n/2$.

As $F$ is finite, with cardinality $q$, we can list all monic polynomials of degree  $k$, of the form $p=x^k+a_{k-1}x^{k-1}+\cdots+a_0$, by listing all $q^k$ $k$-plets $(a_0,\cdots,a_{k-1})$, and test the divisibility of $f$  by each such polynomial, for every value of $k, 1\leq k \leq n/2$.

If $f$ is irreducible, the number of polynomial division to prove the irreducibility is so

$$q+q^2+\cdots q^r=  q\,\frac{q^{r}-1}{q-1}, \qquad r = \lfloor n/2 \rfloor.$$
\end{proof}

\paragraph{Ex. 4.2.7}

{\it For each of the following polynomials, determine, without using a computer, whether it is irreducible over the given field.
\begin{enumerate}
\item[(a)] $x^3+x+1$ over $\F_5$.
\item[(b)] $x^4+x+1$ over $\F_2$.
\end{enumerate}
}

\begin{proof}

\begin{enumerate}
\item[(a)]
$f= x^3+x+1$ being of degree 3, it is reducible iff it has a linear factor (see Ex. 6), iff  it has a root in $\mathbb{F}_5$, which request 5 verifications:

$f(0)=1, f(1)=3, f(2)=1, f(-2) = 1, f(-1) = -1$, all nonzero, so $f$ is irreducible over $\F_5$.

\item[(b)]
$f = x^4+x+1$ has no root in  $\F_2$.

It is so sufficient to test the divisibility of $f$ by quadratic polynomials, which are

$$x^2,x^2+1, x^2+x, x^2+x+1.$$

$x^2$ and $x^2+x$ are not irreducible, can be excluded of the list. It remains to test two divisions by

$$x^2+1,x^2+x+1$$.

\begin{align*}
x^4+x+1 &= (x^2+1)(x^2+1)+x\\
&=(x^2+x+1)(x^2+x)+1
\end{align*}
The remainders of these divisions being nonzero, $x^4+x+1$ is so irreducible over $\mathbb{F}_2$.

Note: the factorization of $\Phi_{15}$ over the field $\mathbb{F}_2$, gives the list of irreducible polynomials over $\mathbb{F}_2$ of degree 4.
\begin{verbatim}
S.<t> = GF(2)['t']
phi15 =( (x^15-1)*(x-1)*(x-1))/((x-1)*(x^3-1)*(x^5-1)); phi15
	x^8 + x^7 + x^5 + x^4 + x^3 + x + 1
factor(phi15)
	(x^4 + x + 1) * (x^4 + x^3 + 1)
\end{verbatim}
\end{enumerate}
\end{proof}

\paragraph{Ex. 4.2.8}

{\it Let $a\in \Z$ be a product of distinct prime numbers. Prove that $x^n-a$ is irreducible over $\Q$ for any $n\geq 1$. What does this imply about $\sqrt[n]{a}$ when $n\geq 2$.
}

\begin{proof}
Let $a= p_1\cdots p_r$ a product of distinct prime numbers.

We show that $f = x^n -a$ is irreducible over $\Q$. Suppose on the contrary that $f = x^n -a$ is reducible. By Gauss Lemma $f$ has a monic factor $g \in \mathbb{Z}[x], 1\leq \deg(g) <n$. 

The decomposition of $f$ in $\C[x]$ is
 $$f =\prod_{\zeta \in \mathbb{U}_n} ( x - \zeta \sqrt[n]{a}).$$ 
 $\mathbb{C}[x]$ being a unique factorization domain, $$g = \prod_{\zeta \in A} ( x - \zeta \sqrt[n]{a}),\ \emptyset \neq A \varsubsetneq \mathbb{U}_n,$$
where $\vert A \vert =s$ satisfies $1\leq s <n$.

As $g \in \mathbb{Z}[x]$, the constant term is an integer $N$, given by
$$N = \xi \sqrt[n] {a}^s,$$ where $\xi = \prod_{\zeta \in A} \zeta \in \mathbb{U}_n$ is a $n$-th root of unity.

Moreover $\xi = N/\sqrt[n] {a}^s \in \mathbb{R}$, thus $\xi = \pm1$, and $\sqrt[n] {a}^s = \pm N =M\in \mathbb{Z}$.

But then $p_1^s\cdots p_r^s = M^n$.

The unicity of the decomposition in prime factors shows that the $p_i$ are the only prime divisors of $M$: $M = p_1^{k_1}\cdots p_r^{k_r}$, and $s = nk_i, i=1,\ldots,r$.

Thus $n\mid s$, in contradiction with $1\leq s <n$.

Conclusion: $x^n -a$ is irreducible over $\mathbb{Q}$, if $a = p_1\cdots p_r$ is a product of distinct prime numbers.

\bigskip

The easy part of Proposition 4.2.6 shows that $x^n-a, n\geq 2$ has no root in  $\mathbb{Q}$, in other words $\sqrt[n]{a}$ is irrational, for every $a$ being a product of distinct prime numbers.
\end{proof}

\paragraph{Ex. 4.2.9}

{\it Let $k$ be a field, and let $F = k(t)$ be the field of rational functions in $t$ with coefficients in $k$. Then consider $f = x^p-t \in F[x]$, where $p$ is prime. By Proposition 4.2.6, $f$ is irreducible provided we can show that $f$ has no roots in $F$. Prove this.
}

\begin{proof}
If $f$ has a root in $k(t)$, then there exists a rational function $u/v,\ u,v \in k[t], u \wedge v = 1$ such that
$$t =\left( \frac{u(t)}{v(t}\right)^p,$$
which is equivalent to the equality in $k[t]$:
$$u(t)^p = t v(t)^p.$$
As $u\wedge v=1$, then $u \wedge v^p=1$, and $u$ divides $t v^p$, thus $u$ divides $t$.

Since $t$ is irreducible (as every polynomial of degree 1), $u(t) = \lambda$, or $u(t) = \lambda t$, $\lambda \in k^*$.

The case $u(t) = \lambda$ implies $t \mid 1$, which is false.

The case $u(t) = \lambda t$ gives $\lambda^p t^p = t v(t)^p$, thus $\lambda^p t^{p-1} = v(t)^p$, and as  $p>1$, $t$ divides also $v$, which contradicts $u \wedge v=1$.

Conclusion: If $p$ is prime, $f =x^p -t$ is irreducible over $F= k(t)$.
\end{proof}

\subsection{THE DEGREE OF AN EXTENSION}
\paragraph{Ex. 4.3.1}

{\it In (4.9) we represented elements of $F(\alpha)$ uniquely using remainders on division by the minimal polynomial of $\alpha$. In the exercise you will adapt the proof of Proposition 4.3.4 to the case of quotient rings. Suppose that $f \in F[x]$ has degree $n>0$. Prove that every coset on $F[x]/\langle f \rangle$ can be written as
$$a_0+a_1x+\cdots+a_{n-1} x^{n-1} + \langle f \rangle,$$
where $a_0,a_1,\ldots,a_{n-1} \in F$ are unique.
}

\begin{proof}
Let $f \in F[x],\deg(f) = n>0$, and $y \in F[x]/\langle f \rangle$. There exists $g \in F[x]$ such that $y = g +\langle f \rangle$.

The division of $g$ by $f$ gives 
$$g = qf+r,\  \deg(r)  < \deg(f)=n.$$
Thus $g-r =qf \in \langle f \rangle$, and consequently $y=g+\langle f \rangle = r+ \langle f \rangle$.

As $\deg(r)<n$, $r = a_0+a_1x+\cdots+a_{n-1}x^{n-1},\ a_0,a_1,\ldots,a_{n-1} \in F$.

Every $y \in F[x]/\langle f \rangle$ can be written as

$$y = a_0+a_1x+\cdots+a_{n-1}x^{n-1} + \langle f \rangle, \ a_0,a_1,\ldots,a_{n-1} \in F.$$

{\bigskip}

Unicity: 

Suppose that $y \in g+\langle f \rangle$ is written as
\begin{align*}
y &= a_0+a_1x+\cdots+a_{n-1}x^{n-1} + \langle f \rangle\\
&=b_0+b_1x+\cdots+b_{n-1}x^{n-1} + \langle f \rangle\\
(a_i, b_i \in F, i = 0,\ldots,n-1).
\end{align*}
Then there exist two polynomials $a,b \in \langle f \rangle$ such that $$p = \sum\limits_{k=0}^{n-1}{a_kx^k} + a = \sum\limits_{k=0}^{n-1}{b_kx^k} + b.$$
Let $r = \sum\limits_{k=0}^{n-1}{a_kx^k}, s = \sum\limits_{k=0}^{n-1}{b_kx^k}$.
By definition of $\langle f \rangle$,  there exists $q_1,q_2\in F[x]$ such that $$p = q_1f + r=q_2f+s, \qquad \deg(r)<n, \deg(s)<n.$$ 
The unicity of the remainder in the division of $p$ by $f$ shows that  $r=s$, so $a_i=b_i, i=0,\ldots,n-1$.

{\bigskip}

Conclusion: Every element in  $F[x]/\langle f \rangle$ is written as
$$a_0+a_1x+\cdots+a_{n-1}x^{n-1} + \langle f \rangle, \qquad a_0,a_1,\ldots,a_{n-1} \in F.$$
where $a_0,a_1,\ldots, a_{n-1}$ are unique.
\end{proof}

\paragraph{Ex. 4.3.2}

{\it Compute the degree of the following extensions:
\begin{enumerate}
\item[(a)] $\Q \subset \Q(i,\sqrt[4]{2})$.
\item[(b)] $\Q \subset \Q(\sqrt{3},\sqrt[3]{2})$.
\item[(c)] $\Q \subset \Q(\sqrt{2 + \sqrt{2}})$
\item[(d)] $\Q \subset \Q(i,\sqrt{2 + \sqrt{2}})$.
\end{enumerate}
}

\begin{proof}
\begin{enumerate}
\item[(a)]
Note that $\sqrt[4]{2}$ is a root of $p=x^4-2 \in \Q[x]$, and $p$ is irreducible over $\Q$ by Exercise 4.2.8 (or Sch\"onemann-Eisenstein Criterion for the prime 2). Thus $$[\Q(\sqrt[4]{2}) : \Q]=4.$$
$i$ is a root of $x^2+1$, which has no root in $\mathbb{R}$, a fortiori in $\Q[\sqrt[4]{2}]$. As its degree is 2, it is irreducible over $\Q[\sqrt[4]{2}]$, thus
$$[\Q(\sqrt[4]{2},i) : \Q(\sqrt[4]{2})]  =2 .$$

Moreover $\Q(i,\sqrt[4]{2}) = \Q(\sqrt[4]{2},i)$. The Tower Theorem gives
$$[\Q(i,\sqrt[4]{2}) : \Q] = [\Q(\sqrt[4]{2},i) : \Q(\sqrt[4]{2})] \times [\Q(\sqrt[4]{2}) : \Q] = 8.$$

\item[(b)] $\sqrt[3]{2}$ is irrational, so $f = x^3-2$ has no root in $\Q$, and $\deg(f)= 3$, thus $f$ is irreducible over $\Q$ and $f$ is the minimal polynomial over $\Q$ of $\sqrt[3]{2}$, and so
$$[\Q(\sqrt[3]{2}) : \Q] = 3.$$
The roots of $x^2 - 3$ are $\pm\sqrt{3}$ and are irrational. As $\deg(x^2-3)=2$, and as $x^2-3$ has no root in $\Q$, $x^2-3$ is irreducible over $\Q$. It is the minimal polynomial of $\sqrt{3}$ over $\Q$, thus
$$[\Q(\sqrt{3}) : \Q] = 2.$$
Moreover 
\begin{align*}
[\Q(\sqrt{3},\sqrt[3]{2}):\Q ]&=[\Q(\sqrt{3},\sqrt[3]{2}):\Q(\sqrt[3]{2}) ] \times [\Q(\sqrt[3]{2}):\Q) ] \\
&= [\Q(\sqrt{3},\sqrt[3]{2}):\Q(\sqrt{3}) ] \times [\Q(\sqrt{3}) : \Q] ,
\end{align*}
thus, if we write $d = [\Q(\sqrt{3},\sqrt[3]{2}):\Q ]$, then $2 \mid d,3 \mid d$, with $2\wedge 3=1$, thus $6\mid d$, $6\leq d$.

$\sqrt{3}$ is a root of  $x^2-3 $, and the degree of $x^2-3 $ is 2. Its coefficients are in $\Q$, a fortiori in $\Q(\sqrt[3]{2})$. Thus the minimal polynomial $p$ of $\sqrt{3}$ over $\Q(\sqrt[3]{2})$ divides $x^2-3$. Its degree $\delta = \deg(p) = [\Q(\sqrt{3},\sqrt[3]{2}):\Q(\sqrt[3]{2}) ]$ satisfies then $\delta \leq 2$.

As $d = 3 \delta \geq 6 $, and so $\delta \leq 2$, $d \leq 6$. Therefore $d=6$.
$$[\Q(\sqrt{3},\sqrt[3]{2}):\Q ]=6.$$

\item[(c)]
Let $\alpha= \sqrt{2+\sqrt{2}}$. 

Then $\alpha^2 = 2 + \sqrt{2}, \alpha^2-2 = \sqrt{2}, (\alpha^2-2)^2 - 2= 0, \alpha^4 - 4 \alpha^2 + 2 = 0$.

 $\alpha$ is a root of
$$f= x^4-4x^2+2.$$

We show that $f$ is irreducible $\Q$.
$f= x^4-4x^2+2 = a_4x^4+a_3x^3+a_2x^2+a_1x+a_0$ satisfies
$2 \nmid a_4 =1, 2 \mid a_3=0,2\mid a_2=-4, 2 \mid a_1=0,2 \mid a_0=2,  2^2 \nmid a_0 = 2$, so 
the Sch\"onemann-Eisenstein Criterion with $p=2$ implies that $f$ is irreducible over $\mathbb{Q}$.


Conclusion: $f = x^4 -4x^2+2$ is irreducible over $\Q$. $f$ is the minimal polynomial of $\alpha = \sqrt{2+\sqrt{2}}$, thus
$$\left[\Q\left(\sqrt{2+\sqrt{2}}\right):\Q\right]=4.$$

\item[(d)]
$x^2+1$ has no real root, a fortiori no root in $\Q(\sqrt{2+\sqrt{2}})$, and $\deg(x^2+1) = 2$. Thus $x^2+1$ is irreducible over $\Q(\sqrt{2+\sqrt{2}})$, it is the minimal polynomial of $i$ over $\Q(\sqrt{2+\sqrt{2}})$, thus
$$\left [\Q\left(i,\sqrt{2+\sqrt{2}}\right):\Q\left(\sqrt{2+\sqrt{2}}\right)\right]=2.$$
Consequently
$$\left [\Q\left(i,\sqrt{2+\sqrt{2}}\right):\Q\right]=\left [\Q\left(i,\sqrt{2+\sqrt{2}}\right):\Q\left(\sqrt{2+\sqrt{2}}\right)\right] \times \left[\Q\left(\sqrt{2+\sqrt{2}}\right):\Q\right] = 8.$$
\end{enumerate}
\end{proof}

\paragraph{Ex. 4.3.3}

{\it For each of the extensions in Exercise 2, find a basis over $\Q$ using the method of Example 4.3.9.
}

\begin{proof}
\begin{enumerate}
\item[(a)]
$(1,\sqrt[4]{2},\sqrt[4]{2}^2,\sqrt[4]{2}^3)$ is a basis of $\Q(\sqrt[4]{2})$ over $\Q$, and $(1,i)$ a basis of $\Q(i,\sqrt[4]{2})$ over $\Q(\sqrt[4]{2})$, thus
$$(1,\sqrt[4]{2},\sqrt[4]{2}^2,\sqrt[4]{2}^3, i, i\sqrt[4]{2},i\sqrt[4]{2}^2,i\sqrt[4]{2}^3)$$
is a basis of $\Q(i,\sqrt[4]{2})$ over $\Q$

\item[(b)]
$(1,\sqrt[3]{2},\sqrt[3]{2}^2)$ is a basis of $\Q(\sqrt[3]{2})$ over $\Q$, and $(1,\sqrt{3})$ a basis of $\Q(\sqrt{3},\sqrt[3]{2})$ over $\Q(\sqrt[3]{2})$, thus
$$ ( 1,\sqrt[3]{2},\sqrt[3]{2}^2, \sqrt{3},\sqrt{3}\sqrt[3]{2},\sqrt{3}\sqrt[3]{2}^2)$$
is a basis of $\Q(\sqrt{3},\sqrt[3]{2})$ over $\Q$.

\item[(c)] The minimal polynomial of $\sqrt{2+\sqrt{2}}$ over $\Q$ being of degree 4, 
$$\left(1,\sqrt{2+\sqrt{2}},\sqrt{2+\sqrt{2}}^{\, 2}= 2+\sqrt{2}, \sqrt{2+\sqrt{2}}^{\, 3} =(2+\sqrt{2})\sqrt{2+\sqrt{2}} \right)$$
is a basis of $\Q\left(\sqrt{2+\sqrt{2}}\right)$ over $\Q$.

\item[(d)]
A basis of $\Q\left(i,\sqrt{2+\sqrt{2}}\right)/ \Q\left(\sqrt{2+\sqrt{2}}\right)$ being $(1,i)$, 
$$(1,\alpha,\alpha^2,\alpha^3,i,i\alpha,i\alpha^2,i\alpha^3),\qquad \mathrm{where}\ \alpha = \sqrt{2+\sqrt{2}},$$
is a basis of $\Q\left(i,\sqrt{2+\sqrt{2}}\right)$ over $\Q$.

\end{enumerate}
\end{proof}

\paragraph{Ex. 4.3.4}

{\it Suppose that $F\subset L$ is a finite extension with $[L:F]$ prime.
\begin{enumerate}
\item[(a)] Show that the only subfields of $L$ containing $F$ are $F$ and $L$.
\item[(b)] Show that $L = F(\alpha)$ for any $\alpha \in L \setminus F$.
\end{enumerate}
}

\begin{proof}
\begin{enumerate}
\item[(a)]
If a subfield  $K$ of $L$ satisfies  $F \subset K \subset L$, then
$$[L:F] = [L:K][K:F],$$ 
so $[K:F]$ divides $p = [L:F]$, where $p$ is a prime.

If $[K:F] = 1$, then $K = F$, and if $[K:F]=p$, then $[L:K]= 1$, thus $K = L$.

Conclusion: If  $[L:F]$ is a prime number, the only intermediate subfields of the extension $F \subset L$  are $L$ and $F$.


\item[(b)] Since $\alpha \in L$, $F \subset F(\alpha) \subset L$.
If $\alpha \not \in F$, then $F(\alpha) \neq F$, thus by (a), $F(\alpha) = L$.

\end{enumerate}
\end{proof}

\paragraph{Ex. 4.3.5}

{\it Consider the extension $\Q \subset L = \Q(\sqrt[4]{2},\sqrt[3]{3})$. We will compute $[L:\Q]$.
\begin{enumerate}
\item[(a)] Show that $x^4 - 2$ and $x^3 - 3$ are irreducible over $\Q$.
\item[(b)] Use $\Q \subset \Q(\sqrt[4]{2}) \subset L$ to show that $4 \mid [L:\Q]$ and $[L:\Q] \leq 12$.
\item[(c)] Use $\Q \subset \Q(\sqrt[3]{3}) \subset L$ to show that $[L : \Q]$ is also divisible by 3.
\item[(d)] Explain why parts (b) and (c) imply that $[L:\Q] = 12$. This works because 3 and 4 are relatively prime. Do you see why ?
\end{enumerate}
}

\begin{proof}
\begin{enumerate}
\item[(a)]
The Sch\"onemann-Eisenstein Criterion with $p=2$ shows that $x^4-2$ is irreducible over $\Q$, and with $p=3$ shows that $f = x^3-3$ is irreducible over $\Q$ (alternatively, we can use Exercise 4.2.8).

\item[(b)]

As $x^4-2$ is irreducible over $\Q$ by (a), $x^4 - 2$ is the minimal polynomial over $\Q$ of $\sqrt[4]{2}$.
$$ [L : \Q] = [L : \Q(\sqrt[4]{2})] \times [\Q(\sqrt[4]{2}):\Q], $$ 
thus $4 = \deg(x^4-2) =  [\Q(\sqrt[4]{2}):\Q]$ divides $ [L : \Q] $.

As $x^3 - 3 \in \mathbb{Q}[x]$ is a fortiori in $\mathbb{Q}(\sqrt[4]{2})[x]$, the minimal polynomial $P$ of $\sqrt[3]{3}$ over $\Q(\sqrt[4]{2})$ divides $x^3-3$, so its degree satisfies $\deg(P) \leq 3$. Consequently, $[L : \Q(\sqrt[4]{2})] = \deg(P) \leq 3$ (et $[\Q(\sqrt[4]{2}):\Q] = 4)$, thus
$$ [L : \Q] = [L : \Q(\sqrt[4]{2})] \times [\Q(\sqrt[4]{2}):\Q] \leq 12$$


\item[(c)]
Similarly, $x^3-3$ is the minimal polynomial of $\sqrt[3]{3}$ over $\Q$.
$$ [L : \Q] = [L : \Q(\sqrt[3]{3})] \times [\Q(\sqrt[3]{3}):\Q], $$
thus $3 = \deg(x^3-3) =  [\Q(\sqrt[3]{3}):\Q]$ divides $ [L : \Q] $.


\item[(d)]
As $3\mid [L : \Q]$,  and as $4 \mid [L : \Q] $, where 3 and 4 are relatively prime, $$12 = 3 \times 4 \mid [L : \Q] .$$ In particular, $12 \leq [L : \Q] $. By (b), $12 \geq [L : \Q] $, thus
$$ [L : \Q] =12.$$

\end{enumerate}
\end{proof}

\paragraph{Ex. 4.3.6}

{\it Suppose that $\alpha$ and $\beta$ are algebraic over $F$ with minimal polynomials $f$ and $g$ respectively. Prove the {\bf Reciprocity theorem}: $f$ is irreducible over $F(\beta)$ if and only if $g$ is irreducible over $F(\alpha)$.
}

\begin{proof}
Write $d_1 =[F(\alpha) : F], \delta_1= [F(\alpha,\beta):F(\alpha)], d_2 = [F(\beta):F],\delta_2 = [F(\alpha,\beta):F(\beta)]$.

The tower Theorem  gives the two relations
\begin{align}
[F(\alpha,\beta) :F] = \delta_1 d_1 = \delta_2 d_2. \label{eq:4.3.6}
\end{align}
Suppose that $f$ is irreducible over $F(\beta)$ (this makes sense because $f \in F[x]$ has a fortiori its coefficients in $F(\beta)$).

Then $f$ is the minimal polynomial of  $\alpha$ over $F(\beta)$, thus $$ \delta_2  = [F(\alpha,\beta),F(\beta)] = \deg(f) =d_1.$$

$\delta_2 = d_1$, combined with the relation \eqref{eq:4.3.6}, gives $\delta_1= d_2$.

Let $G$ the minimal polynomial of  $\beta$ over $F(\alpha)$.

As $g \in F[x] \subset F(\alpha)[x]$, and $g(\beta)=0$, then $G \mid g$, and $\deg(g) = d_2 = \delta_1 = \deg(G)$, where $g$ and  $G$ are monic, thus $g=G$.

As $G$ is irreducible over $F(\alpha)$, $g$ is also irreducible over $F(\alpha)$.

We have proved:
\begin{center}
$f$ is irreducible over $F(\beta)$ $\Rightarrow$ $g$ is irreducible over $F(\alpha)$.
\end{center}
The proof of the converse is similar, by exchange of $\alpha,\beta$.

\begin{center}
$f$ is irreducible over $F(\beta)$ $\iff$ $g$ is irreducible over $F(\alpha)$.
\end{center}
\end{proof}

\paragraph{Ex. 4.3.7}

{\it Suppose we have extensions $L_0 \subset L_1 \subset \cdots \subset L_m$. Use induction to prove the following generalization of Theorem 4.3.8:
\begin{enumerate}
\item[(a)] If $[L_i:L_{i-1}] = \infty$ for some $1\leq i \leq m$, then $[L_m:L_0] = \infty$.
\item[(b)] If $[L_i:L_{i-1}] < \infty$ for all $1\leq i \leq m$, then
$$[L_m:L_0] = [L_m:L_{m-1}][L_{m-1}:L_{m-2}]\cdots[L_2:L_1][L_1:L_0].$$
\end{enumerate}
}

\begin{proof}
\begin{enumerate}
\item[(a)]
The Tower Theorem shows that (a) and (b) are true for $m=2$. Suppose that (a) and (b) are true for an integer $m\geq 2$. We prove that they remain true for the integer $m+1$.

$\bullet$  If  $[L_i : L_{i-1}] = \infty$ for some $i, 1 \leq i \leq m$, the induction hypothesis show that $[L_m : L_0] = \infty$. As  $L_0 \subset L_m \subset L_{m+1}$, the part (a) of Theorem 4.3.8 (Tower Theorem), shows that $[L_{m+1} : L_0] = \infty$.

Moreover, if $[L_{m+1} : L_{m}] = \infty$, this same part (a) of Tower Theorem gives also $[L_{m+1} : L_0] = \infty$.

For all $i, 1 \leq i \leq m+1$,  $$[L_i : L_{i-1}] = \infty \Rightarrow [L_{m+1} : L_0] = \infty,$$  so the part (a) is proved for the integer $m+1$.

$\bullet$ Suppose that $[L_i : L_{i-1}] < \infty$ for all $i, 1\leq i \leq m+1$.
Then the induction hypothesis gives 

$$[L_m : L_0] = \prod_{1\leq i \leq m} [L_i : L_{i-1}]$$
The part (b) of theorem 4.3.8 implies that
\begin{align*}
 [L_{m+1}:L_0] &=  [L_{m+1}:L_{m}] \times  [L_{m}:L_0] \\
 &=  [L_{m+1}:L_{m}] \times \prod_{1\leq i \leq m} [L_i : L_{i-1}] \\
 &=  \prod_{1\leq i \leq m+1} [L_i : L_{i-1}].
 \end{align*}
So the induction is done.
\end{enumerate}
\end{proof}

\subsection{ALGEBRAIC EXTENSIONS}

\paragraph{Ex. 4.4.1}

{\it Lemma 4.4.2 shows that a finite extension is algebraic. Here we will give an example to show that the converse is false. The field of algebraic numbers $\overline{\Q}$ is by definition algebraic over $\Q$. You will show that $[\overline{\Q} : \Q] = \infty$ as follows
\begin{enumerate}
\item[(a)] Given $n\geq 2$ in $\Z$, use Example 4.2.4 from section 4.2 to show that $\overline{\Q}$ has a subfield $L$ such that $[L:\Q] = n$.
\item[(b)] Explain why part (a) implies that $[\overline{\Q} : \Q] = \infty$.
\end{enumerate}
}

\begin{proof}
\begin{enumerate}
\item[(a)]
In Example 4.2.4, we have seen that the Sch\"onemann-Eisenstein Criterion implies that, for all  $n\geq 2$, and $p$ prime,
$$f=x^n+px+p$$ is irreducible over $\mathbb{Q}$.
Let $\alpha$ a root of $f$ in $\C$. Since $f$ is irreducible over $\Q$, the minimal polynomial of $\alpha$ over $\Q$ is $f$, and
$$[\Q(\alpha) : \Q] = \deg(f) = n.$$
As $[\Q(\alpha) : \Q] < \infty$, every element of $\Q(\alpha)$ is algebraic, so 
$$\Q \subset \Q(\alpha) \subset \overline{\Q}.$$
$L = \Q(\alpha)$ is so an answer to the question.


\item[(b)]

Suppose on the contrary that $[\overline{\Q} : \Q] < \infty$. The tower theorem gives then
$$[\overline{\Q} : \Q] = [\overline{\Q} : \Q(\alpha)] \times [\Q(\alpha) : \Q] \geq [\Q(\alpha) : \Q] \geq n.$$
Then for all integer $n \geq 2$, $[\overline{\Q} : \Q] \geq n$, thus $[\overline{\Q} : \Q] = \infty$, which is a contradiction.

Conclusion :  $[\overline{\Q} : \Q] = \infty$.

$\overline{\Q}$ is an algebraic extension of $\Q$, with infinite dimension.
\end{enumerate}
\end{proof}


\paragraph{Ex. 4.4.2}

{\it Let $\alpha \in \C$ be a solution of (4.14). We will show that the minimal polynomial of $\alpha$ over $\Q$ has degree at most $1760$. Let $L = \Q(\sqrt{2},\sqrt{5},\sqrt[4]{12},i,\sqrt[5]{17},\alpha)$. 
\begin{enumerate}
\item[(a)] Show that $[L:\Q] \leq 1760$.
\item[(b)] Use Lemme 4.4.2 to show that the minimal polynomial polynomial of $\alpha$ has degree at most $1760$.
\end{enumerate}
}

\begin{proof}
\begin{enumerate}
\item[(a)]
Let $\alpha \in \C$ a root of 
$$f = x^{11}-(\sqrt{2}+\sqrt{5}) x^5+3\sqrt[4]{12}x^3+(1+3i) x +\sqrt[5]{17}.$$
Let $L = \Q(\sqrt{2},\sqrt{5},\sqrt[4]{12},i,\sqrt[5]{17},\alpha)$, and $K = \Q(\sqrt{2},\sqrt{5},\sqrt[4]{12},i,\sqrt[5]{17})$.

$f \in K[x]$, and $\alpha$ is a root of $f$. The minimal polynomial $p$ of $\alpha$ over $K$ divides $f$, thus  $[L : K] = [K(\alpha) : K] =\deg(p) \leq \deg(f)=11$:
$$[L : K]\leq 11.$$
Moreover, if we write 

$K_4  =  \Q(\sqrt{2},\sqrt{5},\sqrt[4]{12},i), K_3 =  \Q(\sqrt{2},\sqrt{5},\sqrt[4]{12}),K_2 =  \Q(\sqrt{2},\sqrt{5}), K_1 = \Q(\sqrt{2})$,

then
\begin{align*}
[K : \Q] &= [K : K_4] .[K_4:K_3] .[K_3:K_2] .[K_2:K_1] .[K_1:\Q]\\
&  = [K_4( \sqrt[5]{17}): K_4]. [K_3(i):K_3]. [K_2(\sqrt[4]{12}):K_2] .[K_1(\sqrt{5}):K_1]. [\Q(\sqrt{2}):\Q]\\
\end{align*}
 The minimal polynomial $P$ of $ \sqrt[5]{17}$ over $K_4$ divides $x^5-17 \in \Q[x] \subset K_4[x]$, thus $[K_4(\sqrt[5]{17}) : K_4] = \deg(P) \leq 5$. With similar arguments, 
 $$[K_3(i):K_3] \leq 2, [K_2(\sqrt[4]{12}):K_2] \leq 4, [K_1(\sqrt{5}):K_1] \leq 2,[\Q(\sqrt{2}):\Q] \leq 2,$$
Consequently
\begin{align*}
[K : \Q] &\leq 5\times 2 \times 4 \times 2 \times 2 = 160
\end{align*}
and
 $$[L : \Q] = [L:K][K:\Q] \leq 11 \times 160 = 1760.$$

\item[(b)]

By Lemma 4.4.2(b),  as $\alpha \in L$, the degree of the minimal polynomial of  $\alpha$ over $\Q$ divides ${[L:\Q] \leq 1760}$, hence $\deg(p) \leq 1760$.
\end{enumerate}
\end{proof}

\paragraph{Ex. 4.4.3}

{\it In the Mathematical Notes, we defined an algebraic integer to be a complex number $\alpha \in \C$ that is a root of a monic polynomial in $\Z[x]$.
\begin{enumerate}
\item[(a)] Prove that $\alpha \in \C$ is an algebraic integer if and only if $\alpha$ is an algebraic number whose minimal polynomial over $\Q$ has integer coefficients.
\item[(b)] Show that $\omega/2$ is not an algebraic integer, where $\omega = (-1+i\sqrt{3})/2$.
\end{enumerate}
}

\begin{proof}
\begin{enumerate}
\item[(a)]

$\bullet$ Following this definition, suppose that $p(\alpha) = 0$, where $p\in \mathbb{Z}[x]$ is monic.

Write $P\in \Q[x]$ the minimal polynomial of $\alpha$ over $\Q$. Then $P$ divides $p$ in $\Q[x]$: there exists $q \in \Q[x]$ such that $p = P q$.

By Gauss Lemma, Proposition A.3.2 of appendix A, there exists $\delta \in \mathbb{\Q}^*$ such that $\tilde{P} =\delta P$ and $\tilde{q} = \delta^{-1} q$ have integer coefficients. So $p = \tilde{P} \tilde{q}, \tilde{P}, \tilde{q} \in \Z[x]$.

As $p$ is monic, $\pm \tilde{P},\pm \tilde{q}$ are also monic. Possibly by multiplying  $\delta$ by $-1$, we can so suppose that $ \tilde{P}, \tilde{q}$ are monic. Thus $P = \tilde{P}$, and so $P \in \mathbb{Z}[x]$.

$\bullet$ The converse is straightforward: If the minimal polynomial $P$ of $\alpha$ over $\Q$ has integer coefficients, $P$ is an example of monic polynomial such that $P(\alpha) = 0$ , so $\alpha$ is an algebraic integer.

Conclusion:  $\alpha$ is an algebraic integer iff the minimal polynomial of $\alpha$ over $\Q$ has integer coefficients.


\item[(b)]
$\omega/2$ is a root of $  x^2 +\frac{1}{2}x + \frac{1}{4}$, and $f = \omega /2 \not \in \mathbb{Q}$, thus $  x^2 +\frac{1}{2}x + \frac{1}{4}$ is the minimal polynomial of $\alpha$ over $\Q$. Since $f \not \in \Z[x]$, by part (a), $\omega/2$ is not an algebraic integer.

\end{enumerate}
\end{proof}

\paragraph{Ex. 4.4.4}

{\it Use (4.10) and (4.11) to prove the following weak form of Lemma 4.4.2: if $n = [L:F] < \infty$, then every $\alpha \in L$ is a root of a nonzero polynomial of degree $\leq n$.
}

\begin{proof}
If $n = [L:F] < \infty$, and $\alpha \in L$, then $(1, \alpha,\alpha^2, \cdots,\alpha^n)$ has $n+1$ elements in a space of dimension $n$.
Thus there exists $(a_0,\cdots,a_n) \neq (0,\cdots,0)$ such that $a_0 + a_1\alpha+\cdots+a_n \alpha^n = 0$. If we write $P = \sum_{i=0}^n a_ix^i$, then $P\neq 0$, and $P(\alpha) = 0, \deg(P)\leq n$.

Conclusion: If $n = [L:F] < \infty$, every $\alpha \in L$ is a root of a nonzero polynomial of degree at most $n$.

\end{proof}

\paragraph{Ex. 4.4.5}

{\it In 1873 Hermite proved that the number $e$ is transcendental over $\Q$, and in 1882, Lindemann showed that $\pi$ is transcendental over $\Q$. It is unknown whether $\pi+e$ and $\pi-e$ are transcendental. Prove that {\bf at least} one of these numbers is transcendental over $\Q$.
}

\begin{proof}
If $\pi+e$ and $\pi -e$ were both algebraic, then  $\pi+e,\pi -e \in \overline{\Q}$.  As $\overline{\Q}$ is a field containing $\Q$, we should have

$$\pi = \frac{1}{2}\left( (\pi+e) + (\pi-e)\right) $$
element of $\overline{\Q}$, which is false.

At least one of the numbers $\pi+e,\pi-e$ is transcendental over $\Q$.

\end{proof}

\paragraph{Ex. 4.4.6}

{\it Let $F$ be a field. Show that other than the elements of $F$ itself, no elements of $F(x)$ are algebraic over $F$.
}

\begin{proof}
Let $f \in F(x), f\neq 0$. Then $f =p/q,\  p,q \in F[x], \ p \wedge q = 1,\  p\neq 0, q\neq 0$.

If $f$ is algebraic over $F$, let $P  = \sum_{i=0}^n a_i x^i \in F[x]$ the minimal polynomial $f$ over $F$, of degree $n$. Then $a_n = 1 \neq 0$, and $a_0 \neq 0$ (if $a_0 = 0$,  $P/x$ has the root $f$ and so $P$ should not be the minimal polynomial). Then 
$$a_{n}\left ( \frac{p}{q}\right )^n+a_{n-1}\left ( \frac{p}{q}\right )^{n-1}+\cdots+a_0 = 0,$$
thus   $$a_n p^n + a_{n-1} p^{n-1}q+\cdots+a_0 q^n = 0.$$

This equality, with $a_0 \neq 0, a_n \neq 0$, shows that $p \mid q^n$, with $p\wedge q= 1$, so $p \wedge q^n=1$ shows that $p \mid 1$. Similarly $q\mid 1$. Thus $\deg(p) =\deg(q) =0$, and so $f=p/q \in F$.

The only elements of $F(x)$ which are algebraic over $F$ are the elements of $F$.
\end{proof}

\paragraph{Ex. 4.4.7} 

{\it Suppose that $F$ is an algebraically closed field, and let $F\subset L$ be an algebraic extension. Prove that $F=L$.
}

\begin{proof}
Let $\alpha \in L$. As $L$ is algebraic over $F$, $\alpha$ is algebraic over $F$. Let $f\in F[x]$ the minimal polynomial of $\alpha$ over $F$.

As $F$ is an algebraically closed field, $f$ is a product of linear factors in $F[x]$, thus all the roots of $f$ are in $F$. In particular, $\alpha \in F$ (and so $f$ has degree 1). This proves the inclusion $L \subset F$, and as  $F \subset L$, $F=L$.

An algebraically closed field has no proper algebraic extension.
\end{proof}

\paragraph{Ex. 4.4.8}

{\it In this exercise you will show that every algebraic extension of $\R$ is finite of degree at most 2. To prove this, consider an extension $\R\subset L$.
\begin{enumerate}
\item[(a)] Explain why we can find an extension $L\subset K$ such that $x^2+1$ has a root $\alpha \in K$.
\item[(b)] Prove that $L(\alpha)$ is algebraic over $\R(\alpha)$ and that $\R(\alpha) \simeq \C$.
\item[(c)] Now use the previous exercise to conclude that $[L:\R] \leq 2$ and that equality occurs if and only if $L\simeq \C$
\end{enumerate}
}

\begin{proof}
\begin{enumerate}
\item[(a)]
Let $\R \subset L$ is an algebraic extension. 

If  $x^2+1$ has a root $\alpha$ in $L$, we can take  $K=L$. Otherwise $x^2+1$, being of degree 2, is irreducible over $L$, thus $K = L[x]/\langle x^2 +1\rangle$ is an extension of $L$ containing $\alpha = \overline{x} = x +\langle x^2 +1\rangle $, root of $x^2+1$ in $K$. 

In the two cases, there exists  an extension $L \subset K$ such that $x^2+1$ has a root $\alpha$ in $K$ (and $ [L[\alpha]:L] \leq \deg(x^2+1) = 2$).

\item[(b)] Let $\beta \in L(\alpha)$. As $L[\alpha]$ is algebraic over $L$ (since $[L(\alpha) : L] \leq 2)$, and as $L$ is algebraic over $\R$, the Theorem 4.4.7 shows that $\beta$ is algebraic over $\R$. As the coefficients of the minimal polynomial of $\beta$ over $\R$ are real, these coefficients are a fortiori in $\R(\alpha)$, thus $L(\alpha)$ is algebraic over $\R(\alpha)$.

As  $\alpha$ is a root of $x^2+1$, irreducible over $\R$, $\R(\alpha) = \R[\alpha] \simeq \R[x]/\langle x^2+1\rangle \simeq \C$.

\item[(c)] 
As $\R(\alpha)$ is isomorphic to $\C$, $\R(\alpha)$ is an algebraically closed field. Moreover $L(\alpha)$ is algebraic over $\R(\alpha)$. By Exercise 4.4.7, $L(\alpha)=\R(\alpha)$.

Since
\begin{align}
2={[ \R(\alpha):\R]} = [L(\alpha) : \R] =  [L(\alpha) : L] \times [L:\R], \label{eq:4.4.8}
\end{align}
$[L:\R]$ divides 2, thus  $[L:\R] =  1$ or $2$.

Conclusion: Every algebraic extension of $\R$ is finite of degree at most 2.

By~\eqref{eq:4.4.8}, 
\begin{align*}
[L:\R] = 2 &\iff [L(\alpha) : L] = 1\\
&\iff L(\alpha) = L\\
&\ \ \Rightarrow\  \C\simeq L
\end{align*}
Conversely, if $\C \simeq L$, then $L(\alpha) \simeq L$. Let $\varphi : L(\alpha) \to L$ an isomorphism. Then $\beta = \varphi(\alpha)\in L$ satisfies $\beta^2 +1 = 0$, thus $\beta \not \in \R$. Consequently $\R \subsetneq L$, $1<[L : \R]\leq 2$, thus $[L:\R]= 2$.

$$[L:\R] = 2 \iff L \simeq \C.$$
\end{enumerate}
\end{proof}

\paragraph{Ex. 4.4.9}

{\it Prove that $\alpha \in \Q$ is an algebraic integer if and only if $\alpha \in \Z$.
}

\begin{proof}
$\bullet$ If $\alpha \in \Z$, $\alpha$ is a root of  the monic polynomial $x-\alpha \in \Z[x]$, thus  $\alpha$ is an algebraic integer.

$\bullet$ Conversely, let $\alpha \in \Q$ an algebraic integer.
$$\alpha = p/q, \qquad(p,q) \in \Z\times \N^*, \ p\wedge q = 1.$$
 
 $\alpha$ is a root of $f = x^n+a_{n-1}x^{n-1}+\cdots+a_0$, where the coefficients $a_i$ are integers. Thus
 $$\left(\frac{p}{q}\right)^n+ a_{n-1} \left(\frac{p}{q}\right)^{n-1} + \cdots+a_0 = 0,$$
that is
 $$p^n +a_{n-1} p^{n-1}q +\cdots+a_0q^n = 0.$$
This implies $q \mid p^n$, where $q\wedge p = 1$, thus $q \wedge p^n=1$. Hence $q\mid 1$, where $q>0$, thus $q=1$, and $\alpha = p/q = p \in \Z$.
 
 Conclusion: For all $\alpha \in \Q$, $\alpha$ is an algebraic integer iff $\alpha \in \Z$.
 
 $$\overline{\Q} \cap \Q = \Z.$$
\end{proof}

\end{document}
