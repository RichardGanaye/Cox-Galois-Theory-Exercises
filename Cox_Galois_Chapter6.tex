%&LaTeX
\documentclass[11pt,a4paper]{article}
\usepackage[frenchb,english]{babel}
\usepackage[applemac]{inputenc}
\usepackage[OT1]{fontenc}
\usepackage[]{graphicx}
\usepackage{amsmath}
\usepackage{amsfonts}
\usepackage{amsthm}
\usepackage{amssymb}
\usepackage{tikz}
%\input{8bitdefs}

% marges
\topmargin 10pt
\headsep 10pt
\headheight 10pt
\marginparwidth 30pt
\oddsidemargin 40pt
\evensidemargin 40pt
\footskip 30pt
\textheight 670pt
\textwidth 420pt

\def\imp{\Rightarrow}
\def\gcro{\mbox{[\hspace{-.15em}[}}% intervalles d'entiers 
\def\dcro{\mbox{]\hspace{-.15em}]}}

\newcommand{\be} {\begin{enumerate}}
\newcommand{\ee} {\end{enumerate}}
\newcommand{\deb}{\begin{eqnarray*}}
\newcommand{\fin}{\end{eqnarray*}}
\newcommand{\ssi} {si et seulement si }
\newcommand{\D}{\mathrm{d}}
\newcommand{\Q}{\mathbb{Q}}
\newcommand{\Z}{\mathbb{Z}}
\newcommand{\N}{\mathbb{N}}
\newcommand{\R}{\mathbb{R}}
\newcommand{\C}{\mathbb{C}}
\newcommand{\F}{\mathbb{F}}
\newcommand{\U}{\mathbb{U}}
\newcommand{\re}{\,\mathrm{Re}\,}
\newcommand{\ord}{\mathrm{ord}}
\newcommand{\Gal}{\mathrm{Gal}}
\newcommand{\legendre}[2]{\genfrac{(}{)}{}{}{#1}{#2}}

\title{Solutions to David A.Cox  "Galois Theory''}
\author{Richard Ganaye}
\refstepcounter{section} \refstepcounter{section} \refstepcounter{section} \refstepcounter{section}
\refstepcounter{section}
\begin{document}

\maketitle

\section{Chapter 6 : THE GALOIS GROUP}

\subsection{DEFINITION OF THE GALOIS GROUP}

\paragraph{Ex. 6.1.1}

{\it Let $L = F(\alpha_1,\ldots,\alpha_n)$, and let $p_i \in F[x]$ be a nonzero polynomial vanishing at $\alpha_i$. Explain why the proof of Corollary 6.1.5 implies that $|\Gal(L/F)| \leq \deg(p_1)\cdots\deg(p_n)$.
}

\begin{proof}
$L=F(\alpha_1,\cdots,\alpha_n)$, where $\alpha_i$ is algebraic over $F$. Then $\alpha_i$ is the root of a polynomial $p_i \in F(x)$.

By Proposition 6.1.4, every $\sigma \in \mathrm{Gal}(L/F)$ is uniquely determined by the images of $\alpha_i,\  i=1,\cdots,n$. $\alpha_i$ being a root of $p_i \in F[x]$, $\sigma(\alpha_i)$ is also a root of $p_i$. So there exist only $\deg(p_i)$ possibilities for the choice of $\sigma(\alpha_i)$.

More formally, write $R_i$ the set of the roots of  $p_i$ in $L$, then $\sigma(\alpha_i) \in R_i$, with $\vert R_i \vert \leq \deg(p_i)$, and the map
$$
\left\{
\begin{array}{ccc}
 \mathrm{Gal}(L/F) & \to  &  R_1\times \cdots \times R_n\\
\sigma  & \mapsto  &   (\sigma(\alpha_1), \ldots,\sigma(\alpha_n))
\end{array}
\right.
$$
is injective (one-to-one), since $\sigma \in \mathrm{Gal}(L/F)$ is uniquely determined by the images of $\alpha_i,\ i=1,\cdots,n$.

Therefore $$\vert \mathrm{Gal}(L/F) \vert \leq \vert R_1 \vert \times \cdots  \times \vert R_n\vert \leq \deg(p_1)\cdots \deg(p_n). $$
\end{proof}

\paragraph{Ex. 6.1.2}

{\it Consider the extension $\Q \subset L = \Q(\sqrt{2},\sqrt{3})$. In Exercise 13 of Section 5.1, you used Proposition 5.1.8 to construct an automorphism of $L$ that takes $\sqrt{3}$ to $-\sqrt{3}$ and is the identity on $\Q(\sqrt{2})$. By interchanging the roles of 2 and 3 in this construction, explain why all possible signs in (6.1) can occur. This shows that $|\Gal(L/\Q)| = 4$.
}

\begin{proof}
As $L = \Q(\sqrt{2},\sqrt{3})$ is the splitting field of $x^2-3$ over $\mathbb{Q}(\sqrt{2})$, and as $x^2 - 3$ is irreducible over $\Q(\sqrt{2})$ (see Exercise 5.1.13), there exists by Proposition 5.1.8 a field isomorphism $\sigma : L\to L$ which is identity on $\Q(\sqrt{2})$ and which takes $\sqrt{3}$ on $-\sqrt{3}$. As $\sigma$ is identity on $\Q(\sqrt{2})$, we have also $\sigma(\sqrt{2}) = \sqrt{2}$. As the restriction of $\sigma$ to $\Q(\sqrt{2})$ is identity, the restriction of $\sigma$ to $\Q$ is the identity on $\Q$, so $\sigma \in \mathrm{Gal}(L/\Q)$.

Similarly $\Q(\sqrt{2},\sqrt{3})$ is the splitting field of $x^2-2$ over $\mathbb{Q}(\sqrt{3})$, and $x^2 - 2$ is irreducible over $\Q(\sqrt{3})$ by the Reciprocity Theorem (see Exercise 4.3.6), so there exists by Proposition 5.1.8 a field isomorphism $\tau : L\to L$ which is identity on $\Q(\sqrt{3})$ and which takes $\sqrt{2}$ on $-\sqrt{2}$. As $\tau$ is identity on  $\Q(\sqrt{3})$, we have also $\sigma(\sqrt{3}) = \sqrt{3}$, and $\tau \in \mathrm{Gal}(L/\Q)$.

Moreover $1_L(\sqrt{2})=\sqrt{2}, 1_L(\sqrt{3})=\sqrt{3}$, with $1_L\in \mathrm{Gal}(L/\Q)$.

Finally $\sigma \tau = \sigma \circ \tau \in \mathrm{Gal}(L/\Q)$ satisfies $(\sigma  \tau) (\sqrt{2}) = -\sqrt{2}, (\sigma  \tau) (\sqrt{3}) = -\sqrt{3}$.

All possibilities in Example 6.1.10 can occur. Consequently $ \vert \mathrm{Gal}(L/\Q) \vert \geq 4$. As it is proved in Example 6.1.10 that $ \vert \mathrm{Gal}(L/\Q) \vert \leq  4$, then $ \vert \mathrm{Gal}(L/\Q) \vert=  4$, and

$$  \mathrm{Gal}(\Q(\sqrt{2},\sqrt{3})/\Q) =\{1_L, \sigma, \tau, \sigma \tau\}.$$
\end{proof}

\paragraph{Ex. 6.1.3}

{\it This exercise will prove a generalized form of Proposition 6.1.11.
\begin{enumerate}
\item[(a)] Let $\varphi:L_1\to L_2$ be an isomorphism of fields. Given a subfield $F_1 \subset L_1$, set $F_2 = \varphi(F_1)$, which is a subfield of $L_2$. Prove that the map sending $\sigma \in \Gal(L_1/F_1)$ to $\varphi \circ \sigma \circ \varphi^{-1}$ induces an isomorphism $\Gal(L_1/F_1) \simeq \Gal(L_2/F_2)$.
\item[(b)] Explain why Proposition 6.1.11 follows from part (a).
\end{enumerate}
}

\begin{proof}
\begin{enumerate}
\item[(a)]
If $\varphi : L_1 \to L_2$ is a field isomorphism, and $\sigma \in \mathrm{Gal}(L_1/F_1)$, then ${\sigma : L_1 \to L_1}$, and so $\varphi \circ \sigma \circ \varphi^{-1}$ is a map from $L_2$ to $L_2$, composed of three field isomorphisms. Therefore $\varphi \circ \sigma \circ \varphi^{-1}$ is an automorphism of $L_2$.

Moreover, if $\alpha \in F_2$, then $\varphi^{-1}(\alpha) \in F_1$, since $F_2 = \varphi(F_1)$. As $\sigma \in \mathrm{Gal}(L_1/F_1)$, $\sigma$ is identity on $F_1$, thus $\sigma ( \varphi^{-1})(\alpha))  = \varphi^{-1}(\alpha)$, and $(\varphi \circ \sigma \circ \varphi^{-1}) (\alpha) = \alpha$. Consequently $$\varphi \circ \sigma \circ \varphi^{-1} 
\in  \mathrm{Gal}(L_2/F_2).$$

Let
$$
\chi : 
\left\{
\begin{array}{ccc}
 \mathrm{Gal}(L_1/F_1) & \to  &  \mathrm{Gal}(L_2/F_2) \\
 \sigma  &\mapsto    &  \varphi \circ \sigma \circ \varphi^{-1}
\end{array}
\right.
$$

If $\sigma,\tau \in  \mathrm{Gal}(L_1/F_1)$, $$\chi(\sigma) \chi(\tau) = \varphi \circ \sigma \circ \varphi^{-1} \circ \varphi \circ \tau  \circ \varphi^{-1} =  \varphi \circ \sigma \circ \tau  \circ \varphi^{-1} = \chi(\sigma \circ \tau).$$
$\chi$ is so a group homomorphism.

Moreover, if $\chi(\sigma) = \mathrm{id}$, then $\varphi \circ \sigma \circ \varphi^{-1} = \mathrm{id}$, 
then $\sigma = \varphi^{-1} \circ \varphi = \mathrm{id}$ : $\ker(\chi) = \{\mathrm{id}\}$, so $\chi$ is injective.

If $\tau \in  \mathrm{Gal}(L_2/F_2)$, let $\sigma = \varphi^{-1} \circ \tau \circ \varphi$, then $\sigma \in  \mathrm{Gal}(L_1/F_1)$ with the same arguments, and $\chi(\sigma) = \tau$, thus $\chi$ is surjective.

Conclusion : $\chi : \mathrm{Gal}(L_1/F_1)  \to    \mathrm{Gal}(L_2/F_2)$ is a group isomorphism.

\item[(b)]

Suppose as in  Proposition 6.1.11 that the restriction of $\varphi$ to $F$ is identity, and let $F_1 = F$. Then $F_2 = \varphi(F_1) = F_1 = F$, and part (a) shows that

$\chi : \mathrm{Gal}(L_1/F)  \to    \mathrm{Gal}(L_2/F),\sigma  \mapsto      \varphi \circ \sigma \circ \varphi^{-1}$ is a group isomorphism: this is Proposition 6.1.11.

\end{enumerate}
\end{proof}

\paragraph{Ex. 6.1.4}

{\it In the Historical Notes, we saw that Dedekind defined a "permutation" $\alpha \to \alpha'$ to be a map $\Omega \to \omega'$ satisfying $(\alpha + \beta)' = \alpha' + \beta'$ and $(\alpha \beta)' = \alpha' \beta'$ for all $\alpha \beta \in \Omega$. Dedekind also assumes that $\Omega' =\{\alpha' \ \vert \ \alpha \in \Omega]$ and that the $\alpha'$ are not all zero.
\begin{enumerate}
\item[(a)] Show that $1 \in \Omega$ maps to $1 \in \Omega'$. Once this is proved, it follows that $\alpha \mapsto \alpha'$ is a ring homomorphism (Recall that sending 1 to 1 is part of the definition of ring homomorphism given in Appendix A.)
\item[(b)] Show that the map $\alpha \to \alpha'$ is one-to-one.
This shows that Dedekind's definition of field is equivalent to ours.
\end{enumerate}
}

\begin{proof}

Let $\varphi : \alpha \to \alpha'$. By hypothesis, for all  $\alpha, \beta \in \Omega,$

$$\varphi(\alpha + \beta) =\varphi(\alpha)+ \varphi(\beta), \varphi(\alpha \beta) =\varphi(\alpha)\varphi(\beta).$$
\begin{enumerate}
\item[(a)]
By hypothesis, there exists $\alpha \in \Omega$ such that $\alpha' = \varphi(\alpha) \neq 0$. Then $\varphi(\alpha) = \varphi(\alpha.1) = \varphi(\alpha) \varphi(1)$, and since $\varphi(\alpha) \neq 0$, $\alpha' = \varphi(\alpha)$ has a inverse in  $\Omega$, thus 
$$\varphi(1) = 1.$$
$\varphi$ is so a ring homomorphism between two fields.

\item[(b)]
We show that $\varphi$ is injective: 

If $a \neq 0$, there exists an inverse $b$ of $a$: $ab=1$, thus $\varphi(a) \varphi(b) = \varphi(ab) = \varphi(1) = 1$, therefore $\varphi(a) \neq 0$. The kernel of $\varphi$ is null, thus $\varphi$ is injective.

As $\Omega' = \{\varphi(\alpha) , \alpha \in \Omega\}$, $\varphi$ is surjective. So $\varphi : \Omega \to \Omega'$ is a field isomorphism.
\end{enumerate}
\end{proof}

\paragraph{Ex. 6.1.5}

{\it Prove the following inequalities:
\begin{enumerate}
\item[(a)]  $ | \Gal(\Q(\sqrt{2},\sqrt{3},\sqrt{5})/\Q)| \leq 8$
\item[(b)]$ | \Gal(\Q(\sqrt{p_1},\ldots,\sqrt{p_n})/\Q)| \leq 2^n$, where $p_1,\ldots,p_n$ are the first $n$ primes.
In each case, one can show that these are actually equalities.
\end{enumerate}
}

\begin{proof}
\begin{enumerate}
\item[(a)]
As $\sqrt{2}$ is a root of $f_1=x^2-2$, $\sqrt{3}$ a root of $f_2=x^2-3$, and  $\sqrt{5}$ a root of $f_3=x^2-5$, Exercise 1 shows that
$$\vert \mathrm{Gal}(F(\sqrt{2},\sqrt{3},\sqrt{5})/F) \vert \leq \deg(f_1)\deg(f_2)\deg(f_3) = 8.$$


\item[(b)]
As $\sqrt{p_i}$ is a root of $f_i=x^2-p_i$, the same Exercise 1 shows that 
$$\vert \mathrm{Gal}(F(\sqrt{p_1},\ldots,\sqrt{p_n})/F) \vert \leq \deg(f_1)\cdots\deg(f_n) = 2^n.$$

\end{enumerate}
\end{proof}

\paragraph{Ex. 6.1.6}

{\it If we apply Exercise 1 to the extension $\Q \subset L = \Q(\sqrt{6},\sqrt{10},\sqrt{15})$, we get the inequality $|\Gal(L/\Q)| \leq 8$. Show that $|\Gal(L/\Q)| \leq 4$.
}

\begin{proof}
$L = \Q(\sqrt{6},\sqrt{10}, \sqrt{15})$.

$\sqrt{15} = \sqrt{3\cdot 5} = 3 \frac{\sqrt{10}}{\sqrt{6}} \in \Q(\sqrt{6},\sqrt{10})$, therefore
$$L = \Q(\sqrt{6},\sqrt{10}, \sqrt{15}) = \Q(\sqrt{6},\sqrt{10}).$$
Then Exercise 1 shows that $$|\Gal(L/\Q)| \leq 4.$$

Note : moreover, $x^2-10$ is irreducible over $\Q(\sqrt{6})$, otherwise the roots $\pm\sqrt{10}$ of $f$ would be in $\Q(\sqrt{6})$, and then
$$\sqrt{10} = a+b\sqrt{6}, \ a,b \in \Q(\sqrt{6}).$$
By squaring, we obtain $10  = a^2 + 6 b^2 + 2ab \sqrt{6}$. The irrationality of $\sqrt{6}$ shows that $ab=0, a^2+6b^2=10$. Since $\sqrt{10}$ and $\sqrt{\frac{5}{3}}$ are irrational, this system has no solution in $\Q \times \Q$.

$x^2-10$ is irreducible over $\Q(\sqrt{6})$, thus 
$$ [\Q(\sqrt{6},\sqrt{10}):\Q] = [\Q(\sqrt{6},\sqrt{10}):\Q(\sqrt{6}]\cdot[\Q(\sqrt{6}):\Q] = 4.$$
Using section 6.2, as $L$ is the splitting field of the separable polynomial {$(x^2-6)(x^2-10)$} over $\Q$, we obtain
$$|\Gal(L/\Q)| = [L : \Q] =  4.$$
\end{proof}

\paragraph{Ex. 6.1.7} 
{\it Let $F \subset L$ be a finite extension, and let $\sigma : L \to L$ be a ring homomorphism that is the identity on $F$. This exercise will show that $\sigma$ is an automorphism.
\begin{enumerate}
\item[(a)] Show that $\sigma$ is one-to-one.
\item[(b)] Show that $\sigma$ is onto.
\end{enumerate}
}

\begin{proof}
\begin{enumerate}
\item[(a)]
Let $a \in L, a\neq 0$. Then $a$ has an inverse $b$ in the field $L$, so $ab=1$, $\sigma(a) \sigma(b) = \sigma(1) = 1 $, $\sigma(a) \neq 0$. Therefore $\ker(\sigma) =\{0\}$, thus $\sigma$ is injective.

$\sigma : L \to L$ is an injective field homomorphism.

\item[(b)]
As $K \subset L$ is a finite extension, $L$ is a finite dimensional vector space over $F$. As $\sigma$ is identity on $F$, $\sigma : L \to L$ is an injective linear application on a finite dimensional vector space, thus $\sigma$ is also surjective :   $$\sigma \in \mathrm{Gal}(L/F).$$
\end{enumerate}
\end{proof}

\subsection{GALOIS GROUPS OF SPLITTING FIELDS}
\paragraph{Ex. 6.2.1}

{\it Complete Example 6.2.2 by showing that $\Gal(L/\Q) = \{1_L,\sigma, \tau, \sigma \tau\}$ and that
$\Gal(L/\Q)\simeq \Z/2\Z \times \Z/2\Z.$
}

\begin{proof}
We proved in Exercise 6.1.2 that $$ G:= \mathrm{Gal}(\Q(\sqrt{2},\sqrt{3})/\Q) =\{1_L, \sigma, \tau, \sigma \tau\}.$$
Every group of order 4 is abelian, and isomorphic to $\Z/4\Z$ or $\Z/2\Z \times \Z/2\Z$.

As $G$ has at least 2 elements of order 2, since $\sigma^2 = \tau^2 = 1_L$. This is not the case in  $\Z/4\Z$. Thus 
$$\mathrm{Gal}(\Q(\sqrt{2},\sqrt{3})/\Q) \simeq \Z/2\Z \times \Z/2\Z.$$
\end{proof}

\paragraph{Ex. 6.2.2}

{\it Consider $\Q \subset L = \Q(\omega, \sqrt[3]{2})$, where $\omega = e^{2 \pi i/3}$.
\be
\item[(a)] Explain why $\sigma \in \Gal(L/\Q)$ is uniquely determined by $\sigma(\omega) \in \{\omega, \omega^2 \}$ and $\sigma(\sqrt[3]{2}) \in \{\sqrt[3]{2},\omega \sqrt[3]{2}, \omega^2 \sqrt[3]{2} \}$.
\item[(b)] Explain why all possible combinations for $\sigma(\omega)$ and $\sigma(\sqrt[3]{2})$ actually occur.
\ee
In the next section we will show that $\Gal(L/Q) \simeq S_3$.
}

\begin{proof}
\begin{enumerate}
\item[(a)]
As $L = \Q(\omega,\sqrt[3]{2})$, Proposition 6.1.4(b) shows that $\sigma \in \mathrm{Gal}(L/\Q)$ is uniquely determined by $\sigma(\omega),\sigma(\sqrt[3]{2})$.

Moreover, by theorem 6.1.4 (a), $\sigma(\omega)$ is a root of $f = x^2+x+1$, whose roots are $\omega,\omega^2$, and $\sigma(\sqrt[3]{2})$ is a root of $g = x^3-2$ whose roots are $\sqrt[3]{2}, \omega\sqrt[3]{2},\omega^2\sqrt[3]{2}$.

Then Exercise 6.1.1 shows that $$\vert \mathrm{Gal}(L/\Q)\vert \leq \deg(f)\deg(g)=6.$$

\item[(b)]
$L$ is the splitting field of the separable irreducible polynomial $g = x^3-2\in \Q[x]$. Indeed, $g$ is irreducible over $\Q$ since $\deg(g) = 3$ and $g$ has no root in $\Q$. Moreover $g$ is separable since its roots in $\C$ are $\sqrt[3]{2}, \omega\sqrt[3]{2},\omega^2\sqrt[3]{2}$ which are distinct.

By theorem 6.2.1, $\vert \mathrm{Gal}(L/\Q)\vert = [L:\Q]$, and by Exercise 5.1.8, $[L:\Q] = 2 \times 3 = 6$, therefore

$$\vert \mathrm{Gal}(L/\Q)\vert = [L:\Q]=6.$$

If all possible combinations for $\sigma(\omega)$ and $\sigma(\sqrt[3]{2})$ don't actually occur, then $\vert \mathrm{Gal}(L/\Q)\vert <6$, which is false, so all possible combinations occur.
\end{enumerate}
\end{proof}

\paragraph{Ex. 6.2.3}

{\it Consider $\Q \subset L = \Q(\zeta_5, \sqrt[5]{2})$, where $\zeta_5 = e^{2 \pi i /5}$. By proposition 4.2.5, the minimal polynomial of $\zeta_5$ over $\Q$ is $x^4+x^3+x^2+x+1$.
\be
\item[(a)] Show that $[L:\Q] = 20$.
\item[(b)] Show that $L$ is the splitting field of $x^5-2$ over $\Q$, and conclude that $\Gal(L/\Q)$ is a group of order 20.
\ee
We will describe the structure of this Galois group in section 6.4.
}

\begin{proof}
Write $\zeta = \zeta_5$.
\begin{enumerate}
\item[(a)]
as $L = \Q(\zeta,\sqrt[5]{2})$, Proposition 6.1.4(b) shows that $\sigma \in \mathrm{Gal}(L/\Q)$ is uniquely determined by $\sigma(\zeta),\sigma(\sqrt[3]{2})$.

Moreover by Proposition 6.4.1(a), $\sigma(\zeta)$ is a root of $f = x^4+x^3+x^2+x+1$, whose roots are $\zeta^i,\ 1\leq i \leq 4$, and $\sigma(\sqrt[5]{2})$ is a root of $g = x^5-2$ , whose roots are $\zeta^j \sqrt[5]{2},\ 0 \leq j \leq 4$.

Then Exercice 6.1.1 shows that $$\vert \mathrm{Gal}(L/\Q)\vert \leq \deg(f)\deg(g)=20.$$

\item[(b)]
$L$ is the splitting field of the separable irreducible polynomial $g = x^5-2\in \Q[x]$ over $\Q$. Indeed,  $g$ is  irreducible over $\Q$ by Sch�nemann-Eisenstein Criterion with $p=2$, and separable since its roots in $\C$ are $\sqrt[5]{2}, \zeta\sqrt[5]{2},\zeta^2 \sqrt[5]{2},\zeta^3 \sqrt[5]{2},\zeta^4\sqrt[5]{2}$ which are distinct.

By theorem 6.2.1, $\vert \mathrm{Gal}(L/\Q)\vert = [L:\Q]$, and by Exercise 5.1.8, $[L:\Q]=4 \times 5 = 20$, therefore

$$\vert \mathrm{Gal}(L/\Q)\vert = [L:\Q]=20.$$
\end{enumerate}
\end{proof}

\paragraph{Ex. 6.2.4}

{\it Consider the $n$th root of unity $\zeta_n = e^{2\pi i/n}$. We call $\Q \subset \Q(\zeta_n)$ a cyclotomic extension of $\Q$.
\be
\item[(a)]  Show that $\Q \subset \Q(\zeta_n)$ is a splitting field of a separable polynomial.
\item[(b)] Given $\sigma \in \Gal(\Q(\zeta_n)/\Q)$, show that $\sigma(\zeta_n) = \zeta_n^i$ for some integer $i$.
\item[(c)] Show that the integer $i$ in part (b) is relatively prime to $n$.
\item[(d)] The set of congruence classes modulo $n$ relatively prime to $n$ form a group under multiplication, denoted $(\Z/n\Z)^*$. Show that the map $\sigma \mapsto [i]$, where $\sigma(\zeta_n) = \zeta_n^i$, define a one-to-one group homomorphism $\Gal(\Q(\zeta_n)/\Q) \to (\Z/n\Z)^*$.
\item[(e)] The order of $(\Z/n\Z)^*$ is $\vert (\Z/n\Z)^* \vert = \phi(n)$, where $\phi(n)$ is the Euler $\phi$-function from number theory. Prove that the homomorphism of part (d) is an isomorphism if and only if $[\Q(\zeta_n) : \Q ] = \phi(n)$.
\item[(f)] Let $p$ be prime. Use part (e) and Proposition 4.2.5 to show that $\Gal(\Q(\zeta_p)/\Q) \simeq (\Z/p\Z)^*$.
\ee
In chapter 9 we will prove that $[\Q(\zeta_n):\Q] = \phi(n)$. By part (e), this will imply that there is an isomorphism $\Gal(\Q(\zeta_n)/\Q) \simeq (\Z/n\Z)^*$ for all $n$.
}

\begin{proof}
\begin{enumerate}
\item[(a)]
 $\zeta_n$ is a root of $x^n-1\in \Q[x]$. Write $\U_n$ the set of $n$th  roots of unity in $\C$ : $$\U_n = \{\zeta_n^k,\  0\leq k \leq n-1\}$$ and $|\U_n| = n$.

As $x^n-1 = \prod\limits_{\zeta \in \U_n}(x-\zeta)$, $x^n-1$ is separable, and the splitting field of $x^n-1$ over $\Q$ is 
$\Q(\zeta,\ldots, \zeta^{n-1}) = \Q(\zeta)$

Conclusion: $\Q(\zeta_n)$ is the splitting field of the separable polynomial $x^n-1\in \Q[x]$ over $\Q$.

\item[(b)]
Let $\sigma \in \Gal(\Q(\zeta_n):\Q)$.

As $\zeta_n$ is a root of $x^n - 1 \in \Q[x]$, by Proposition 6.1.4(a), $\sigma(\zeta_n) $ is a root of $x^n-1$, thus  $\sigma(\zeta_n)  \in \U_n$, so
$$\sigma(\zeta_n) = \zeta_n^i,\  i \in \N.$$

\item[(c)] Note that $\zeta_n = e^{2i\pi/n}$ is an element of order $n$ in the group $\U_n$. Indeed, for all $k\in \Z$,

$$\zeta_n^k = 1     \iff e^{2i\pi k /n} = 1 \iff k/n \in \Z \iff n\mid k.$$   

$\sigma$ being a field isomorphism, $\sigma(\zeta_n) \in \U_n$ is also of order $n$. Indeed, for all $k\in \Z$,

$$\sigma(\zeta_n) ^k=1 \iff \sigma(\zeta_n^k)=1 \iff \zeta_n^k = 1 \iff n \mid k.$$

If the order of an element $\zeta $ is $| \zeta| =n$, then for all integer $j$,  the order of $\zeta^j$ in  $\U_n$ is

$$|\zeta^j |= \frac{n}{n\wedge j}.$$

Indeed for all $k\in \Z$, 

$(\zeta^j)^k = 1 \iff n \mid jk \iff \frac{n}{n\wedge j} \mid \frac{j}{n\wedge j} k \iff \frac{n}{n\wedge j} \mid k$ (since $\frac{n}{n\wedge j} \wedge \frac{j}{n\wedge j}=1$).

If we apply this result to $\zeta_n^i =\sigma(\zeta_n)$, we obtain

  $$\frac{n}{n\wedge i} = | \zeta_n^i |= | \sigma(\zeta_n |= n,$$ thus $$n\wedge i = 1.$$
                                                 
\item[(d)]
Let
\begin{center}
$
\varphi : 
\left\{
\begin{array}{ccc}
 \Gal(\Q(\zeta_n)/\Q) &  \to  & ( \Z/n\Z)^* \\
  \sigma &  \mapsto  &  [i] : \ \sigma(\zeta_n) = \zeta_n^i
\end{array}
\right.
$
\end{center}
Note that $\varphi$ is well defined, since $\zeta_n^i = \zeta_n^j$ implies $i\equiv j \pmod n$ and so $[i] = [j]$.

We show that $\varphi$ is a group homomorphism. 

If $\sigma,\tau \in  \Gal(\Q(\zeta_n)/\Q)$, and $\varphi(\sigma) = [i], \varphi(\tau) = [j]$, then $\sigma(\zeta_n) = \zeta_n^i, \tau(\zeta_n) = \zeta_n^j,$ thus
$$(\sigma \circ \tau)(\zeta_n) = \sigma((\zeta_n)^j)= (\sigma(\zeta_n))^j = (\zeta_n^i)^j = \zeta_n^{ij},$$therefore$$\varphi(\sigma \circ \tau) = [ij] = [i][j] = \varphi(\sigma)\varphi(\tau).$$

$\varphi$ is injective :

If $\varphi(\sigma) = [1]$, then $\sigma(\zeta_n) = \zeta_n$. Since $\sigma \in \Gal(\Q(\zeta_n)/\Q)$, $\sigma$ is uniquely determined by the image of $\zeta_n$, thus $\sigma = 1_{\Q(\zeta_n)}$. The kernel of $\varphi$ is trivial, thus $\varphi$ is injective.

Conclusion: there exist an injective group homomorphism 
$$\varphi : \Gal(\Q(\zeta_n)/\Q)   \to   ( \Z/n\Z)^*.$$


\item[(e)]
As $\Q(\zeta_n)$ is the splitting field of a separable polynomial over  $\Q$, $$\vert  \Gal(\Q(\zeta_n)/\Q)\vert = [\Q(\zeta_n):\Q].$$

If we suppose that $ [\Q(\zeta_n):\Q]= \phi(n)$,  $\varphi$ is an injection between two set of same cardinality, thus $\varphi$ is a bijection, and so $\varphi$ is a group isomorphism. Conversely, if $\varphi$ is a group isomorphism, then $ [\Q(\zeta_n):\Q] = \Gal(\Q(\zeta_n)/\Q)\vert = \phi(n)$

Conclusion: $ [\Q(\zeta_n):\Q] = \phi(n)$ if and only if $ \Gal(\Q(\zeta_n)/\Q)   \simeq   ( \Z/n\Z)^*$.


\item[(f)]
If $p$ is prime, we know  that $f = 1+x+\cdots+x^{p-1}$ is irreducible over $\Q$, so $f$ is the minimal polynomial of $\zeta_p$ over $\Q$. This implies that $[\Q(\zeta_p) : \Q] = p-1 = \phi(p)$.

By part (e), we know then that $ \Gal(\Q(\zeta_p)/\Q)   \simeq   ( \Z/p\Z)^*$ (and so this group is cyclic).
\end{enumerate}
\end{proof}


\paragraph{Ex. 6.2.5}
{\it Let $F$ have characteristic $p$, and assume that $f = x^p -x +a \in F[x]$ is irreducible over $F$. Then let $L = F(\alpha)$, where $\alpha$ is a root of $f$ in some splitting field. In Exercise 16 of Section 5.3, you showed that $F\subset L$ is a normal extension.
\be
\item[(a)] Show that $|\Gal(L/F) | = p$, and use this to prove that $\Gal(L/F)\simeq \Z/p\Z$.
\item[(b)] Exercise 15 of Section 5.3 showed that $\alpha +1$ is a root of $f$. For $i=0,\ldots,p-1$, show that there is a unique element of $\Gal(L/F)$ that takes $\alpha$ to $\alpha + i$.
\item[(c)] Use part (b) to describe an explicit isomorphism $\Gal(L/F) \simeq \Z/p\Z$.
\ee

}

\begin{proof}
\begin{enumerate}
\item[(a)]
$L=F(\alpha)$ and $\alpha$ has for minimal polynomial $f=x^p-x+a$, thus $[L:F]=p$.

By Exercice 5.3.16, we know that $L = F(\alpha) = F(\alpha, \alpha+1,\ldots,\alpha+p-1)$ is the splitting field of
$$f =x^p-x-a =  (x-\alpha)(x-\alpha-1)\cdots(x-\alpha- p+1).$$

Therefore $F(\alpha)$ is the splitting field of a separable polynomial $f \in F[x]$, and by theorem 6.2.1 $$\vert \Gal(L/F) \vert = [L:F] = p.$$
Every group of order $p$, where $p$ is prime, is cyclic and isomorphic to $\Z/p\Z$ : 
$$\Gal(L/F) \simeq \Z/p\Z.$$

\item[(b)]
 $F \subset L$ is by part (a) a normal extension, and $f \in F[x]$  is irreducible over $F$ by hypothesis. The roots of $f$ in L are $\alpha,\alpha +1, \ldots, \alpha+p-1$. By Proposition 5.1.8 , there exists a field isomorphism $\sigma_i : L \to L$ which is identity on $F$ and which takes $\alpha$ on $\alpha+i, i \in \F_p$. Then $\sigma_i \in \Gal(L/F), \sigma(\alpha) = \alpha+i$. As $L=F(\alpha)$, $\sigma$ is uniquely determined by the image of $\alpha$.
 
 Conclusion: $\alpha$ being a fixed root of $f$, and $i\in \F_p$, there exist a unique $\sigma_i \in \Gal(L/F)$ such that $ \sigma_i(\alpha) = \alpha+i$.

\item[(c)]
 
Let
$$
\varphi
\left\{
\begin{array}{ccc}
  \Gal(L/F)&\to    &\F_p  \\
  \sigma& \mapsto   &   \sigma(\alpha) - \alpha    
\end{array}
\right.
$$

$\bullet$ For all $\sigma \in \Gal(L/F)$, $\varphi(\sigma) \in \F_p$ since $\sigma(\alpha)$ is a root of $f$, so $\sigma(\alpha) - \alpha = i \in \F_p$.

 $\bullet$ $\varphi$ is bijective by part(b), since for all $i\in\F_p$, there exists a unique $\sigma \in \Gal(L/F)$ such that $\varphi(\sigma) = \sigma(\alpha) - \alpha = i$.
 
 $\bullet$ $\varphi$ is a group homomorphism : if $\sigma,\tau \in  \Gal(L/F)$, and $\varphi(\sigma) = i, \varphi(\tau) = j$, then $\sigma(\alpha) = \alpha+i, \tau(\alpha) = \alpha+j (\ i,j \in \F_p)$.
 
 $(\sigma \circ \tau)(\alpha) = \sigma(\alpha+j) = \sigma(\alpha) + \sigma(j) = (\alpha +i) +j = \alpha+(i+j)$ ($ \sigma(j) = j$ since $\sigma$ is identity on $F$, a fortiori on  $\F_p\subset F$).
 
 As $(\sigma \circ \tau)(\alpha) = \alpha+(i+j)$, $\varphi(\sigma \circ \tau) = i+j = \varphi(\sigma) + \varphi(\tau)$.
 
 $\varphi : \Gal(L/F) \to \F_p  $ is so a group isomorphism.
\end{enumerate}
\end{proof}

\paragraph{Ex. 6.2.6}

{\it Let $f \in F[x]$ be irreducible and separable of degree $n$, and let $F\subset L$ be a splitting field of $f$. Prove that $n$ divides $\vert \Gal(L/F)\vert$.
}

\begin{proof}
Let $L$ a splitting field of  $f$ over $F$, where $f$ is a separable irreducible polynomial.


By Proposition 6.2.1 (using the separability of $f$) :
$$\vert \Gal(L/F) \vert =[L:F].$$

Let $\alpha$ a root of  $f$ in  $L$. As $f$ is irreducible, $f$ is the minimal polynomial of $\alpha$ over $F$, thus $[F(\alpha) : F] = \deg(f) = n$, and
$$[L:F] = [L:F(\alpha)]\, [F(\alpha) : F] = n [L:F(\alpha)] :$$
So $n$ divides $\vert \Gal(L/F) \vert$.




\end{proof}

\subsection{PERMUTATION OF THE ROOTS}

\paragraph{Ex. 6.3.1}

{\it Consider $\Gal(L/\Q)$, where $L = \Q(\omega,\sqrt[3]{2}), \omega = e^{2 \pi i/3}$. By Exercise 2 of section 6.2, there are $\sigma, \tau \in \Gal(L/\Q)$ such that
$$\sigma(\sqrt[3]{2}) = \omega \sqrt[3]{2}, \sigma(\omega) = \omega \quad \mathrm{and} \quad \tau(\sqrt[3]{2}) = \sqrt[3]{2}, \tau (\omega) = \omega^2.$$
Find the permutations in $S_3$ corresponding to $\sigma$ and $\tau$.
}

\begin{proof}
 $L = \Q(\omega,\sqrt[3]{2})$ is the splitting field over $\Q$ of $f=x^3-2$.

By Exercise 6.2.2, there exist $\sigma,\tau \in \Gal(L/\Q)$ such that
$$\sigma(\sqrt[3]{2}) = \omega\sqrt[3]{2}, \sigma(\omega) = \omega \quad \mathrm{and} \quad \tau(\sqrt[3]{2}) = \sqrt[3]{2}, \tau(\omega) = \omega^2.$$

Number the roots of $f$ by $\alpha_1 = \sqrt[3]{2}, \alpha_2 = \omega\sqrt[3]{2}, \alpha_3 = \omega^2\sqrt[3]{2}$.

Then $\sigma(\alpha_1) = \alpha_2, \sigma(\alpha_2) =\alpha_3, \sigma(\alpha_3) = \alpha_1$. If we write $\tilde{\sigma} = (1,2,3)$, then for $i=1,2,3, \sigma(\alpha_i) = \sigma(\alpha_{\tilde{\sigma}(i)})$, so the 3-cycle $\tilde{\sigma} =(1,2,3)$ corresponds to $\sigma$.

$\tau(\alpha_1) = \alpha_1, \tau(\alpha_2) = \alpha_3, \tau(\alpha_3) = \alpha_2$, so $\tilde{\tau} = (2,3)$  corresponds to $\tau$.

As $S_3$ is generated by $\tilde{\sigma}, \tilde{\tau}$, $ \Gal(L/\Q)$ is generated by $\sigma,\tau$.
\end{proof}

\paragraph{Ex. 6.3.2}

{\it For each of the following Galois groups, find an explicit subgroup of $S_4$ that is isomorphic to the group. Also, the Galois group is isomorphic to which known group? 
\be
\item[(a)] $\Gal(\Q(i,\sqrt{2})/\Q)$.
\item[(b)] $\Gal(\Q(i,\sqrt[4]{2})/\Q)$.
\ee
}

\begin{proof}
\begin{enumerate}
\item[(a)]
$\Gal(\Q(i,\sqrt{2})/\Q) = \{1_{\Q}, \sigma, \tau, \sigma\tau\}$, where
$$\sigma(i) = -i, \sigma(\sqrt{2}) = \sqrt{2},$$
$$\tau(i) = i, \tau(\sqrt{2}) = -\sqrt(2).$$
As every  $g \in \Gal(\Q(i,\sqrt{2})/\Q)$  satisfies $g^2 = 1_{\Q}$, 
$$\Gal(\Q(i,\sqrt{2})/\Q) \simeq \Z/2\Z \times \Z/2\Z$$
(Klein's ViertelGruppe: cf Exercise 6.2.1 and example 6.2.2 for more details).

If we number the roots by $\alpha_1 = i, \alpha_2 = -i, \alpha_3 = \sqrt{2}, \alpha_4 = -\sqrt{2}$, then $(1,2)$ corresponds to $\sigma$, and $(3,4)$ to $\tau$.

As subgroup of $S_4$, $\Gal(Q(i,\sqrt{2})/\Q)$ is represented by $$\{ (), (1,2), (3,4), (1,2)(3,4)\} = \langle (1,2),(3,4) \rangle \simeq \Gal(\Q(i,\sqrt{2})/\Q).$$

\item[(b)]
 \begin{align*}
 f&=x^4-2\\
 &=(x^2-\sqrt{2})(x^2+\sqrt{2})\\
 &=(x-\sqrt[4]{2})(x+\sqrt[4]{2})(x+i\sqrt[4]{2})(x-i\sqrt[4]{2})
\end{align*}
The splitting root of $f$ over $\Q$ is so $L = \Q(i,i\sqrt[4]{2}) = \Q(i,\sqrt[4]{2})$. $f$ is separable, since $f$ has simple roots in its splitting field. $L$ is so the splitting field over $\Q$ of a separable  polynomial, therefore by Theorem 6.2.1,$$\vert \Gal(L:\Q) \vert= [L:\Q].$$

$f$ is irreducible over $\Q$ by the Sch�nemann-Eisenstein Criterion with $p=2$.
As $f$ is irreducible over $\Q$, $$[\Q(\sqrt[4]{2}) : \Q] = \deg(f) = 4,$$ and $x^2+1$ is irreducible over $\Q(\sqrt[4]{2})$, since it is of degree 2, without root in $\Q(\sqrt[4]{2}) \subset \R$, thus $${[\Q(i,\sqrt[4]{2}) : \Q(\sqrt[4]{2})] = 2}.$$

Consequently, 
$$[\Q(i,\sqrt[4]{2}) : \Q] = [\Q(i,\sqrt[4]{2}) : \Q(\sqrt[4]{2})] \ [\Q(\sqrt[4]{2}) : \Q]  = 8,$$
and so
$$\vert \Gal(L:\Q) \vert = 8.$$

If $\sigma \in \Gal(L/\Q)$, as $i$ is a root of $x^2+1 \in \Q[x]$, and $\sqrt[4]{2}$ a root of $x^4-2 \in \Q[x]$, then $\sigma(i) = \pm i$, et $\sigma(\sqrt[4]{2}) = i^k \sqrt[4]{2},k=0,1,2,3$.

As $\sigma$ is uniquely determined by the images of $ i,\sqrt[4]{2}$, and as $\vert \Gal(L:\Q) \vert = 8$, these 8 possibilities occur, thus $G=\Gal(L/\Q) = \{\sigma_{j,k} \ \vert \ 0\leq j \leq1, 0 \leq k \leq 3 \}$, where $\sigma_{j,k}$, which is identity on $\Q$, is determined by
$$\sigma_{j,k}(i) = (-1)^j i, \sigma_{j,k}(\sqrt[4]{2}) = i^k \sqrt[4]{2}.$$

Write $\tau :L \to L, z \mapsto \overline{z}$ the complex conjugation restricted to $L$. $\tau$ is a ring homomorphism and an involution, thus $\tau$ is a field automorphism of $L$, which is identity on $\Q$, so $\tau \in G$. Moreover 

$$ \tau(i) = -i, \tau(\sqrt[4]{2}) = \sqrt[4]{2},$$

Let $\sigma \in \Gal(L/\Q)$ defined by $$\sigma(i)=i, \sigma(\sqrt[4]{2}) = i \sqrt[4]{2}.$$
Then  $\tau = \sigma_{1,0}, \sigma = \sigma_{0,1}$.

As $\tau^2 = 1_L$ and $\tau \neq e$, the order of $\tau$ is 2.

$\sigma^4(i) = i$ and $\sigma^4(\sqrt[4]{2}) = \sqrt[4]{2}$, thus $\sigma^4 = e$. As $\sigma^2(\sqrt[4]{2}) = i^2 \sqrt[4]{2}=-\sqrt[4]{2},\sigma_2 \neq e$, thus the order of $\sigma$ is 4.
$$|\tau| = 2, \qquad |\sigma| = 4.$$

As $\tau(i) = -i, \tau \not \in \langle \sigma \rangle$. Thus the subgroup $\langle \sigma, \tau \rangle$ of $G$ contains at least 5 elements, so is equal to $G$ by Lagrange's Theorem:
$$G = \langle \sigma, \tau \rangle.$$

As the index of  $H = \langle \sigma \rangle$ in $G$ is 2, and $\tau \not \in H$, $G = H \cup \tau H$ :
 $$ G=\Gal(L/\Q) = \{1_{L}, \sigma, \sigma^2, \sigma^3,\tau,  \tau \sigma, \tau \sigma^2, \tau \sigma^3\}.$$

If we number the roots of $f$ by  $\alpha_k=  i^{k-1} \sqrt[4]{2}$, for $k=1,2,3,4$, then $\tau$ corresponds to the transposition $(2,4)$, and $\sigma$ to the cycle $(1,2,3,4)$ :
$$G\simeq \langle(1,2,3,4),(2,4)\rangle \subset S_4.$$

If we number the 4 summits of a square by 1,2,3,4 in the direct orientation, then $\sigma$ corresponds to a rotation of angle $\pi/2$, and $\tau$ to a symmetry with respect to the diagonal $[1,3]$. They generate the group of isometry of the square, which is the dihedral group $D_8$, defined also by generators and relations :

$$G = \langle \sigma, \tau \rangle, \sigma^4 = \tau^2 = e, \tau \sigma = \sigma^{-1} \tau.$$

(Since $\tilde{\sigma}^{-1} \tilde{\tau} = (1,4,3,2) (2,4)  = (4,3,2,1) =(2,4) (1,2,3,4)= \tilde{\tau} \tilde{\sigma}$.)

As a verification, the following GAP instruction confirm the result $D_8$ : 
\begin{verbatim}
G:= Group((1,2,3,4),(2,4));
StructureDescription(G);
	  "D8"
\end{verbatim}
\end{enumerate}
\end{proof}

\paragraph{Ex. 6.3.3}

{\it In the terminology of Exercise 2, $\Gal(\Q(i,\sqrt{2},\sqrt{3})/\Q)$ is isomorphic to which known group? Explain your reasoning in detail.
}

\begin{proof}
We have already proved (Ex. 5.1.13) that $f= x^2-3$ is irreducible over $\Q[\sqrt{2}]$. $f$ is so the minimal polynomial of $\sqrt{3}$ over $\Q(\sqrt{2})$, thus $[\Q(\sqrt{2},\sqrt{3}):\Q(\sqrt{2})]=\deg(f) = 2$.

As $g = x^2 - 2$ is irreducible over $\Q$, $[\Q(\sqrt{2}):\Q] =\deg(g) =  2$, therefore
$$[\Q(\sqrt{2},\sqrt{3}):\Q] =  [\Q(\sqrt{2},\sqrt{3}):\Q(\sqrt{2})]\ [\Q(\sqrt{2}):\Q] = 4.$$
Moreover, $h = x^2+1$ has no real root, a fortiori $h$ has no root in  $\Q(\sqrt{2},\sqrt{3})$. As $\deg(h)=2$, $h$ is irreducible over $\Q(\sqrt{2},\sqrt{3})$, $h$ is the minimal polynomial of  $i$ over $\Q(\sqrt{2},\sqrt{3})$, thus $[\Q(\sqrt{2},\sqrt{3},i): \Q(\sqrt{2},\sqrt{3})]=2$, and by the Tower Theorem, and the equality $\Q(\sqrt{2},\sqrt{3},i) = \Q(i,\sqrt{2},\sqrt{3})$,

$$[\Q(i,\sqrt{2},\sqrt{3}):\Q] = 8.$$

$L = \Q(i,\sqrt{2},\sqrt{3})$ is the splitting field of $p = (x^2+1)(x^2-2)(x^2-3)$ over $\Q$, and $p =(x-i)(x+i)(x-\sqrt{2})(x+\sqrt{2})(x-\sqrt{3})(x+\sqrt{3})$ is separable. By theorem 6.2.1, we obtain

$$\vert \Gal(L / \Q) \vert = [L:\Q] = 8.$$

If $\sigma \in \Gal(L /\Q)$, $\sigma(i) = \pm i, \sigma(\sqrt{2}) = \pm \sqrt{2}, \sigma(\sqrt{3}) = \pm \sqrt{3}$. As $\vert \Gal(L / \Q) \vert = 8$, all of these  possibilities occur: there exist 8 $\Q$-automorphisms of $L$ satisfying these equalities. As $L = \Q(i,\sqrt{2},\sqrt{3})$, $\sigma \in \Gal(L : \Q)$ is uniquely determined by the images of these 3 elements.

In particular, there exist  $\sigma_1,\sigma_2,\sigma_3 \in \Gal(L : \Q)$ defined by
$$
\begin{array}{lll}
\sigma_1(i) = -i,& \sigma_1(\sqrt{2}) = \sqrt{2},& \sigma_1(\sqrt{3}) = \sqrt{3}\\
\sigma_2(i) = i, &\sigma_2(\sqrt{2}) =- \sqrt{2},& \sigma_2(\sqrt{3}) = \sqrt{3}\\
\sigma_3(i) = i,& \sigma_3(\sqrt{2}) = \sqrt{2}, &\sigma_3(\sqrt{3}) =- \sqrt{3}
\end{array}
$$
and $1_{L}, \sigma_1,\sigma_2,\sigma_3,\sigma_1\sigma_2, \sigma_1\sigma_3,\sigma_2\sigma_3,\sigma_1\sigma_2\sigma_3$ give distinct images to $i,\sqrt{2},\sqrt{3}$, thus
$$G := \Gal(L/\Q) = \{1_{L}, \sigma_1,\sigma_2,\sigma_3,\sigma_1\sigma_2, \sigma_1\sigma_3,\sigma_2\sigma_3,\sigma_1\sigma_2\sigma_3\}.$$

Therefore $$G = \langle \sigma_1,\sigma_2,\sigma_3\rangle.$$

Note that $\sigma_1 \sigma_2 = \sigma_2 \sigma_1$ since they give the same images to $i,\sqrt{2},\sqrt{3}$. Similarly $\sigma_1 \sigma_3 = \sigma_3 \sigma_1$ and $\sigma_2 \sigma_3 = \sigma_3 \sigma_2$. Thus $G$ is abelian, generated by  3 elements of order 2, with $\sigma_2 \not \in \langle \sigma_1 \rangle, \sigma_3 \not \in \langle \sigma_1,\sigma_2 \rangle$. Therefore $G$ is the direct sum of the 3 subgroups $\{e,\sigma_i\},\ i=1,2,3$, d'ordre 2 :

$$\Gal(L:\Q) \simeq (\Z/2\Z)^3.$$

Some instructions Sage and Gap to verify these results :
\begin{verbatim}
f=(x-i-sqrt(2)-sqrt(3))*(x-i-sqrt(2)+sqrt(3))*(x-i+sqrt(2)-sqrt(3))
	*(x-i+sqrt(2)+sqrt(3))*(x+i-sqrt(2)-sqrt(3))*(x+i-sqrt(2)+sqrt(3))
	*(x+i+sqrt(2)-sqrt(3))*(x+i+sqrt(2)+sqrt(3));f
\end{verbatim}
${\left(x + \sqrt{3} + \sqrt{2} + i\right)} {\left(x + \sqrt{3} +\sqrt{2} - i\right)} {\left(x + \sqrt{3} - \sqrt{2} + i\right)} {\left(x+ \sqrt{3} - \sqrt{2} - i\right)}$\\
$ {\left(x - \sqrt{3} + \sqrt{2} +i\right)} {\left(x - \sqrt{3} + \sqrt{2} - i\right)} {\left(x - \sqrt{3}- \sqrt{2} + i\right)} {\left(x - \sqrt{3} - \sqrt{2} - i\right)}$

\begin{verbatim}
g=f.expand();g
\end{verbatim}
$x^{8} - 16 \, x^{6} + 88 \, x^{4} + 192 \, x^{2} + 144$
\begin{verbatim}
g.factor()
\end{verbatim}
$x^{8} - 16 \, x^{6} + 88 \, x^{4} + 192 \, x^{2} + 144$

\begin{verbatim}
x=polygen(QQ,'x')
K.<z> = NumberField(x^8-16*x^6+88*x^4+192*x^2+144)
G = K.galois_group();G
\end{verbatim}
$\langle (1,2)(3,4)(5,6)(7,8), (1,3)(2,4)(5,7)(6,8), (1,5)(2,6)(3,7)(4,8)
\rangle$

\bigskip

With Gap : 
\begin{verbatim}
G:=Group((1,2)(3,4)(5,6)(7,8),(1,3)(2,4)(5,7)(6,8),(1,5)(2,6)(3,7)(4,8));
StructureDescription(G);
\end{verbatim}
$C_2\times C_2\times C_2$.

\bigskip
As $x^{8} - 16 \, x^{6} + 88 \, x^{4} + 192 \, x^{2} + 144$ is irreductible over $\Q$, $[\Q(i+\sqrt{2}+\sqrt{3}) : \Q] = 8 = [L:\Q]$, and since $\Q(i+\sqrt{2}+\sqrt{3})  \subset L$, $L = \Q(i+\sqrt{2}+\sqrt{3}) $.

These results imply that $L= \Q(i,\sqrt{2},\sqrt{3})$ is the splitting field of the irreducible polynomial $x^{8} - 16 \, x^{6} + 88 \, x^{4} + 192 \, x^{2} + 144$, that $i+\sqrt{2}+\sqrt{3}$ is a primitive element of $\Q \subset L$, and that $\Gal(L/\Q) \simeq C_2\times C_2\times C_2.$
\end{proof}

\paragraph{Ex. 6.3.4}

{\it Consider the extension $\Q\subset L = \Q(\alpha)$, where $\alpha = \sqrt{2+\sqrt{2}}$. In Exercise 6 of Section 5.1, you showed that $f = x^4 - 4x^2+2$ is the minimal polynomial of $\alpha$ over $\Q$ and that $L$ is the splitting field of $f$ over $\Q$. Show that $\Gal(L/\Q) \simeq \Z/4\Z$.
}

\begin{proof}
$L = \Q(\alpha), \alpha= \sqrt{2+\sqrt{2}}$.
We have already proved (Ex. 5.1.6) that
\begin{align*}
f&= x^4-4x^2+2\\
&= \left(x-\sqrt{2+\sqrt{2}}\right)\left(x+\sqrt{2+\sqrt{2}}\right)\left(x-\sqrt{2-\sqrt{2}}\right)\left(x+\sqrt{2-\sqrt{2}}\right)
\end{align*}
is the minimal polynomial of $\alpha$ over $\Q$, and that $L = \Q(\alpha)$ is the splitting field of $f$ over $\Q$.

$L =\Q(\alpha)$ is so the splitting field of the irreducible separable polynomial $f$. By theorem 6.2.1,
$$\vert \Gal(L/\Q) \vert =[L:\Q] = 4.$$

Write $\beta = \sqrt{2-\sqrt{2}}$. If $\sigma \in \Gal(L/\Q), \sigma(\alpha)$ is a root of $f$, thus
$$\sigma(\alpha) \in \{\alpha,\beta,-\alpha,-\beta\}.$$
Moreover, since $L = \Q(\alpha)$, an automorphism of $\Gal(L/\Q)$ is uniquely determined by the image of $\alpha$, and since $\vert \Gal(L/\Q) \vert = 4$, all of these possibilities occur, so there exist one and only one $\sigma \in \Gal(L/\Q)$ such that $\sigma(\alpha)= \gamma$, where $\gamma \in \{\alpha,\beta,-\alpha,-\beta\}$ (alternatively, since $f$ is irreducible over $\Q$, we can use Theorem 5.1.8).

$$ \Gal(L/\Q) = \{\sigma_0 = e,\sigma_1,\sigma_2,\sigma_3\} ,\sigma_1(\alpha) = \beta, \sigma_2(\alpha) = -\alpha,\sigma_3(\alpha) = -\beta.$$
In particular, there exists $\sigma (= \sigma_1) \in \Gal(L/\Q)$ defined by 
$\sigma(\alpha) = \beta$.

Recall that $\alpha \beta= \sqrt{2}, \alpha^2 = 2+\sqrt{2}, \beta^2 = 2-\sqrt{2}$ (see Ex. 5.1.6), thus
$$\alpha^2 - \beta^2 = 2\alpha \beta.$$
From this equality we obtain
$$\alpha^2 - \frac{2}{\alpha^2} = 2 \alpha \beta,\ \frac{2}{\beta^2} - \beta^2 = 2\alpha \beta,$$
therefore
$$\beta = \frac{1}{2}\left(\alpha - \frac{2}{\alpha^3}\right),\ \alpha =  -\frac{1}{2}\left(\beta - \frac{2}{\beta^3}\right).$$


As $\sigma(\alpha) = \beta$,
\begin{align*}
\sigma(\beta) &=  \frac{1}{2}\left(\sigma(\alpha) - \frac{2}{\sigma(\alpha)^3}\right)\\
&=\frac{1}{2}\left(\beta - \frac{2}{\beta^3}\right)\\
&=-\alpha
\end{align*}
Finally $\sigma(-\alpha) =\sigma(-1) \sigma(\alpha) = -\sigma(\alpha) = -\beta$, so

$$ \sigma(\alpha) = \beta, \sigma^2(\alpha) = -\alpha, \sigma^3(\alpha) = -\beta.$$
As every element in $\Gal(L/\Q)$ is uniquely determined by the image of $\alpha$,  $$\sigma^0 = e = \sigma_0, \sigma^1 =\sigma_1,\sigma^2 = \sigma_2,\sigma^3 = \sigma_3,$$ and
$$\Gal(L/\Q) = \{e,\sigma,\sigma^2,\sigma^3\} = \langle \sigma \rangle.$$
So $\Gal(L/\Q)$ is cyclic, generated by  $\sigma$ :

$$\Gal(L/\Q) \simeq \Z/4\Z.$$
\end{proof}

\paragraph{Ex. 6.3.5}

{\it Let $f \in F[x]$ be separable, where $f = g_1\cdots g_s$ for $g_i \in F[x]$ of degree $d_i>0$, and let $L$ be the splitting field of $f$ over $F$. Show that $\Gal(L/F)$ is isomorphic to a subgroup of the product group $S_{d_1} \times \cdots \times S_{d_s}$.
}

\begin{proof}
We show the proposition for $s=2$ to have lighter notations.

Suppose that $f = gh\in F[x]$ is separable, with $g,h \in F[x]$, $\deg(g)=r,\deg(h)=s$. Then $g,h$ are separable.

 Write $\alpha_1,\cdots,\alpha_{r}$ the roots of $g$ in $M$, and $\beta_1,\cdots,\beta_{s}$ the roots of $h$ in $N$. 

Let $M$ be a splitting field of $g$ over $F$, and $N$ be a splitting field of $h$ over $F$.
Then $M=F(\alpha_1,\cdots,\alpha_r), N = F(\beta_1,\cdots,\beta_{s})$, and  $L=F(\alpha_1,\cdots,\alpha_{r}, \beta_1,\cdots,\beta_{s})$. As $f$ is separable, the $d=r+s$ roots of $f$, $\alpha_1, \cdots,\alpha_{r},\beta_1,\cdots,\beta_{s}$ are distinct.

Write $A$ the set of the roots of $g$ in $L$, $B$ the set of roots of $h$ in $L$ : $\vert A \vert = r, \vert B \vert = s$, and write $S(A)$ the set of bijections of $A$ (and the same for $B$) : $S(A)\simeq S_r, S(B) \simeq S_s$.

Let $\sigma \in \Gal(L/F)$. As $g,h \in F[x]$, $\sigma$ induces a permutation of the roots of $g$ and of the roots of $h$, so the maps
\begin{center}
$\sigma_1 : 
\left\{
\begin{array}{ccc}
A  &\to   &  A \\
  \alpha&  \mapsto  &   \sigma(\alpha) 
\end{array}
\right.
$ and
$\sigma_2 : 
\left\{
\begin{array}{ccc}
B  &\to   &  B \\
  \beta&  \mapsto  &   \sigma(\beta) 
\end{array}
\right.
$
\end{center}
restrictions  of $\sigma$ � $A,B$, satisfy $\sigma_1 \in S(A), \sigma_2 \in S(B)$.

The map
\begin{center}
$\varphi : 
\left\{
\begin{array}{ccc}
\Gal(L/F) &\to   &  S(A)\times S(B) \\
  \sigma&  \mapsto  &  (\sigma_1,\sigma_2)
\end{array}
\right.
$
\end{center}
is a group homomorphism: if $\varphi(\sigma) = (\sigma_1,\sigma_2)$ and $\varphi(\tau) = (\tau_1,\tau_2)$ (with $\sigma,\tau \in \Gal(L/F)$), and also $\eta = \sigma \circ \tau, \varphi(\eta) = (\eta_1,\eta_2)$, then for all $\alpha$ in $A$ and $\beta \in B$, 
\begin{align*}
\eta_1(\alpha) = \eta(\alpha) = (\sigma  \tau)(\alpha) = (\sigma_1  \tau_1)(\alpha) , \eta_2(\beta) = \eta(\beta) = (\sigma  \tau)(\beta) =(\sigma_2  \tau_2)(\beta),
\end{align*}
thus $\eta_1 = \sigma_1\tau_1, \eta_2 = \sigma_2\tau_2$. Consequently 

$\varphi(\sigma \circ \tau) = \varphi(\eta) = (\eta_1,\eta_2) = (\sigma_1  \tau_1, \sigma_2  \tau_2) = (\sigma_1, \sigma_2) (\tau_1,\tau_2) =  \varphi(\sigma)\varphi(\tau)$.

$\varphi$ is injective : if $\varphi(\sigma) = (\sigma_1,\sigma_2) = (1_A,1_B)$, then $$\sigma(\alpha_i) = \alpha_i,\ i=1,\cdots r \ \mathrm{et}\ \sigma(\beta_j) = \beta_j,\ j=1,\cdots s.$$

As $L=F(\alpha_1,\cdots,\alpha_{r}, \beta_1,\cdots,\beta_{s})$, $\sigma = 1_L$.

 $\Gal(L/F)$ is isomorphic to a subgroup of $S(A) \times S(B)$, and as $S(A)\times S(B) \simeq S_r\times S_s$,  $\Gal(L/F)$ is isomorphic to a subgroup of $S_r\times S_s$.

We can generalize to $s$ polynomials similarly, or by induction.
\end{proof}

\paragraph{Ex. 6.3.6}

{\it Let $H$ be a transitive subgroup of $S_n$. Prove that $|H|$ is a multiple of $n$.
}

\begin{proof}
A subgroup $H$ of $S_n$ defines an action on $\gcro 1,n \dcro$ by $h\cdot x = h(x), h\in H, x \in \gcro 1,n \dcro$. By definition $H$ is a transitive subgroup of $S_n$ if this action is transitive, i.e. if the only orbit is ${\cal O}_i =\gcro 1,n \dcro,\ i=1,\cdots,n$.
If we write $H_i = \mathrm{Stab}_H(i)$ the stabilizer in  $H$ of a fixed element $i$, then $(H:H_i) = \vert {\cal O}_i \vert =n$, thus $\vert H \vert = \vert H_i \vert \times n$:
$$n \ \mathrm{divides} \  \vert H \vert.$$
\end{proof}

\paragraph{Ex. 6.3.7}

{\it Let $f \in F[x]$ be irreducible and separable of degree $n$ and let $F\subset L$ be a splitting field of $f$ over $F$. Use Exercise 6 and Proposition 6.3.7 to prove that $n$ divides $|\Gal(L/F) |$. This gives an alternate proof of Exercise 6 of Section 6.2.
}

\begin{proof}

We define a left action of the Galois group $G = \Gal(L/F)$ on the set $S$ of the roots of $f$ by $\sigma \cdot \alpha = \sigma(\alpha)$, where $\sigma \in G,  \alpha \in S$ (we know that $\sigma(\alpha) \in S$).

For a fixed $\alpha \in  S$, define $G_{\alpha}= \mathrm{Stab}_G(\alpha) = \{\sigma \in G \ \vert \sigma(\alpha) = \alpha\}$ the stabilizer of $\alpha$ in $G$, and ${\cal O}_\alpha = \{\sigma \cdot \alpha \ \vert \  \sigma \in G \}$ its orbit.

As $f$ is irreducible, by proposition 5.8.1, if $\alpha,\beta$ are two roots of $f$, there exists a field isomorphism $\sigma : L \to L$, which is identity on $F$ (so $\sigma \in \Gal(L/F)$,  and such that $\sigma(\alpha) = \beta$.

Therefore the action of $G$ on $S$ is transitive, so there exists a unique orbit : for all $\alpha \in S$, $\cal{O}_\alpha = S$, thus
$$|{\cal O}_\alpha| = |S| = n.$$
Indeed the separability of $f$ implies that $f$ has $n = \deg(f)$ distinct roots in $L$.

As $(G : G_\alpha) = |{\cal O}_\alpha|$, we obtain
$$|G | = |\Gal(L/F) | = n \times |G_\alpha|.$$
So $n = \deg(f)$ divides $ |\Gal(L/F) | $.
\end{proof}

\subsection{EXAMPLES OF GALOIS GROUPS}
\paragraph{Ex. 6.4.1}

{\it Given $a,b\in \F_p$, define $\gamma_{a,b}:\F_p \to \F_p$ by $\gamma_{a,b}(u) = au +b$.
\be
\item[(a)] Prove that $\gamma_{a,b}$ is one-to-one and onto if and only if $a\neq 0$.
\item[(b)] Suppose that $a\ne 0$. Prove that the inverse function of $\gamma_{a,b}$ is $\gamma_{a^{-1},-a^{-1}b}$.
\item[(c)] Show that
$$\mathrm{AGL}(1,\F_p) = \{\gamma_{a,b}\ \vert \ (a,b) \in \F_p^* \times \F_p\}$$
is a group under composition.
\ee
}

\begin{proof}

Let $a,b\in \mathbb{F}_p$, and $\gamma_{a,b} : \F_p\to \F_p, u\mapsto \gamma_{a,b}(u) = au+b$.
\begin{enumerate}
\item[(a)]  If $a=0$, $\gamma_{a,b}$ is the constant function  $b$, thus $\gamma_{0,b}$ is not a bijection.

Suppose that $a\neq 0$. Then , for all $u,v \in \F_p$, 
$$v=au+b \iff u = a^{-1}v - a^{-1} b.$$
So every $v \in \F_p$ has a unique antecedent, therefore $\gamma_{a,b}$ is bijective.

\item[(b)]
If $a\neq 0$, by part (a), $\gamma_{a,b}$ is bijective, and the unique antecedent $u$ of any $v\in \F_p$ is given by $u = a^{-1}v - a^{-1} b = \gamma_{a^{-1}, -a^{-1}b}(v)$.
Consequently
$$\gamma_{a,b}^{-1} =  \gamma_{a^{-1}, -a^{-1}b}.$$

\item[(c)]
We show that $\mathrm{AGL}(1,\F_p)$ is a subgroup of $(S(\F_p),\circ)$.

$\bullet$ By part (a), if $f \in\mathrm{AGL}(1,\F_p)$, then $f = \gamma_{a,b}$, where $a\neq 0$, thus $f$ est bijective :  $\mathrm{AGL}(1,\F_p) \subset S(\F_p)$, and $1_{\F_p} = \gamma_{1,0} \in \mathrm{AGL}(1,\F_p)$, so $\mathrm{AGL}(1,\F_p) \ne \emptyset$.

$\bullet$ If $f,g \in \mathrm{AGL}(1,\F_p)$, then $f = \gamma_{a,b}, g=\gamma_{c,d},\ a,b,c,d \in \F_p, a\neq 0, c\neq 0$.

For all $u \in \F_p$, 

$(g\circ f)(u) = \gamma_{c,d}(\gamma_{a,b}(u) = \gamma_{c,d}(au+b) = c(au+b)+d = ac u +(bc+d) = \gamma_{ac,bc+d}$.

Therefore $g \circ f =  \gamma_{c,d} \circ \gamma_{a,b} = \gamma_{ac,bc+d}$ and $ac \neq 0$, so $g\circ f \in \mathrm{AGL}(1,\F_p)$.

$\bullet$ If  $f \in \mathrm{AGL}(1,\F_p)$, $f = \gamma_{a,b},a\neq 0$, then $f^{-1} =  \gamma_{a^{-1}, -a^{-1}b} \in \mathrm{AGL}(1,\F_p)$.

$\mathrm{AGL}(1,\F_p)$ is a group under composition.
\end{enumerate}
\end{proof}

\paragraph{Ex. 6.4.2}

{\it Consider the map $\mathrm{AGL}(1,\F_p) \to \F_p^*$ defined by $\gamma_{a,b} \mapsto a$.
\be
\item[(a)] Show that this map is an onto group homomorphism with kernel $T = \{\gamma_{1,b} \ | \ b\in \F_p\}$. Then use this to prove (6.6).
\item[(b)] Show that $T \simeq \F_p$.
\ee
}

\begin{proof}
Let $\varphi :  \mathrm{AGL}(1,\F_p) \to \F_p^*, \gamma_{a,b} \mapsto  \varphi(\gamma_{a,b}) =a$.
\begin{enumerate}
\item[(a)] This map is well defined, since 
$$f = \gamma_{a,b} = \gamma_{c,d} \in \mathrm{AGL}(1,\F_p) \Rightarrow \forall u \in \F_p, au+b = cu+d \Rightarrow a=c.$$
$\varphi$ is a group homomorphism: if $f = \gamma_{a,b} , g = \gamma_{c,d} \in \mathrm{AGL}(1,\F_p)$, then
$$\varphi(g\circ f) = \varphi(\gamma_{c,d} \circ \gamma_{a,b}) = \varphi(\gamma_{ac,bc+d}) = ac =\varphi(g) \varphi(f).$$
This homomorphism is surjective, since every  $a\in \F_p^*$ satisfies $a = \varphi(\gamma_{a,0})$, with $\gamma_{a,0} \in \mathrm{AGL}(1,\F_p)$.

$\gamma_{a,b} \in \ker(\varphi) \iff a = 1$: the kernel of $\varphi$ is $T = \{\gamma_{1,b} \ \vert  \ b \in \F_p\}$, so $T$ is a normal subgroup.

As the image of the group homorphism $\varphi$ is $\F_p^*$, and its kernel $T$, the Isomorphism Theorem shows that
$$\mathrm{AGL}(1,\F_p)/T \simeq \F_p^*.$$

\item[(b)]
The map $\psi : T \to \F_p, \gamma_{1,b} \mapsto b$ is bijective, and satisfies $$\psi(\gamma_{1,b} \circ \gamma_{1,d}) = \psi(\gamma_{1,b+d}) = b+d = \psi(\gamma_{1,b})\psi(\gamma_{1,d}),$$ 
So $\psi$ is a group homomorphism: $T \simeq \F_p$.

\end{enumerate}
\end{proof}

\paragraph{Ex. 6.4.3}

{\it This exercise is concerned with the proof of (6.7). Given $\tau \in S_n$, observe that $f\mapsto \tau \cdot f$ can be regarded as the evaluation map from $F[x_1,\ldots,x_n]$ to itself that evaluates $f(x_1,\ldots,x_n)$ at $(x_{\tau(1)},\ldots,x_{\tau(n)})$.
\be
\item[(a)] Explain why Theorem 2.1.2 implies that $f \mapsto \tau \cdot f$ is a ring homomorphism. This proves the first two bullets of (6.7).
\item[(b)] Prove the third bullet of (6.7).
\ee
}

\begin{proof}
\begin{enumerate}
\item[(a)]
Let $\tau \in S_n$. As $f \mapsto \tau.f$ is the evaluation map that evaluates $f(x_1,\ldots,x_n)$ at $(x_{\tau(1)},\ldots,x_{\tau(n)})$, Theorem 2.1.2 shows that this application is a ring homomorphism, thus
\begin{align*}
\tau \cdot (f+g) &= \tau\cdot  f+ \tau\cdot  g\\
\tau \cdot (fg) &= (\tau \cdot  f) (\tau \cdot  g)
\end{align*}

\item[(b)]
Let $f = f(x_1,\cdots,x_n) \in F(x_1,\cdots,x_n)$, and $\tau,\gamma \in S_n$. Define $g$ by

$$g(x_1,\cdots,x_n)  = \gamma \cdot  f = f(x_{\gamma(1)},\cdots,x_{\gamma(n)}).$$

Then $\tau \cdot (\gamma \cdot f) = \tau \cdot g = g(x_{\tau(1)},\ldots,x_{\tau(n)})$ is obtained by substituting each $x_i$ in the expression of $g$ by $x_{\tau(i)}$, thus $x_{\gamma(j)}$ becomes $x_{\tau(\gamma(j))} = x_{(\tau \gamma)(j)}$:

$$\tau\cdot(\gamma \cdot f) = f(x_{(\tau\gamma)(1)},\ldots,x_{(\tau\gamma)(n)}) = (\tau\gamma) \cdot f.$$

Conclusion: $$\tau\cdot (\gamma \cdot f) = (\tau\gamma) \cdot f.$$
\end{enumerate}
\end{proof}

\paragraph{Ex. 6.4.4}

{\it Let $\tau \in S_n$. Prove that $f\mapsto \tau \cdot f$ is a ring isomorphism from $F[x_1,\ldots,x_n]$ to itself.
}

\begin{proof}
We know (Exercise 6.4.3 (a)) that $\varphi : f\mapsto \tau \cdot f$ is a ring isomorphism. As $\tau \in S_n$, $\tau$ is bijective and so $\tau^{-1}$ exists.  Let $\psi : f\mapsto \tau^{-1} \cdot f$. Then for all $f \in F[x_1,\ldots,x_n]$, by Exercise 6.4.3 (b)
\begin{align*}
(\psi \circ \varphi)(f) &= \tau^{-1} \cdot (\tau \cdot f) = (\tau^{-1} \tau)\cdot f = 1_{[1,n]} \cdot f = f\\
(\varphi \circ \psi)(f) &= \tau \cdot (\tau^{-1} \cdot f) = (\tau \tau^{-1})\cdot f = 1_{[1,n]} \cdot f = f\\
\end{align*}
Therefore $\psi \circ \varphi = \varphi \circ \psi = 1_{F[x_1,\ldots,x_n]}$, so $\varphi$ is a bijection.

Conclusion: $\varphi$ is a ring isomorphism.
\end{proof}

\paragraph{Ex. 6.4.5}

{\it Let $R$ be an integral domain, and let $K$ be its field of fractions. Prove that every ring isomorphism $\phi : R \to R$ extends uniquely to an automorphism $\tilde{\phi} : K \to K$.
}

\begin{proof}

If $f = p/q \in K$, then the fraction $\phi(p)/\phi(q)$ doesn't depends of the choice of the representative $(p,q)$ of the fraction: if $f = p/q = r/s$, then $ps=qr$, thus $\phi(p)\phi(s) = \phi(ps) = \phi(qr) =\phi(q)\phi(r)$, and so $\phi(p)/\phi(q) = \phi(r)/\phi(s)$. Therefore there exists a map $\tilde{\phi} : K \to K$ defined for all $p/q\in  K$ by
$$\tilde{\phi}(p/q) = \phi(p)/\phi(q).$$
In particular, if $p\in R$, $\tilde{\phi}(p) = \tilde{\phi}(p/1) = \phi(p)/\phi(1) = \phi(p)$ : $\tilde{\phi}$ extends $\phi$.

$\tilde{\phi}$ is a ring homomorphism: $\tilde{\phi}(1) = 1$ since $1 \in R$ and $\phi(1)=1$.
\begin{align*}
\tilde{\phi}\left(\frac{p}{q} \frac{r}{s}\right) &= \tilde{\phi}\left(\frac{pr}{qs}\right) =\frac{ \phi(pr)}{\phi(qs)}
 =\frac{\phi(p) \phi(r)}{\phi(q)\phi(s)}\\
 & =\tilde{\phi}\left(\frac{p}{q}\right) \tilde{\phi}\left(\frac{r}{s}\right).\\
\tilde{\phi}\left(\frac{p}{q} +\frac{r}{s}\right) &=  \tilde{\phi}\left(\frac{ps+qr}{qs}\right)= \frac{ \phi(ps+qr)}{\phi(qs)}\\
&=\frac{\phi(p)\phi(s)+\phi(q) \phi(r)}{\phi(q)\phi(s)} = \frac{\phi(p)}{\phi(q)}+ \frac{\phi(r)}{\phi(s)} \\
&= \tilde{\phi}\left(\frac{p}{q}\right) + \tilde{\phi}\left(\frac{r}{s}\right) 
\end{align*}

If $\tilde{\phi}(p/q) =0$, then $\phi(p)/\phi(q) =0$, thus $\phi(p) = 0$, $p=0$, $p/q = 0$ : $\tilde{\phi}$ is injective.

If $g = u/v \in K$, as $\phi$ is surjective, $u = \phi(p),v=\phi(q),\  p,q\in R$. Then $g = \phi(p)/\phi(q) = \tilde{\phi}(p/q),\  p/q \in K$ : $\tilde{\phi}$ est surjective.

$\tilde{\phi} : K \to K$ is a field automorphism.

If $\psi : K \to K$ is any field automorphism which extends $\phi$, then for any fraction $p/q \in K$,
$$\psi \left(\frac{p}{q} \right) = \frac{\psi(p)}{\psi(q)} = \frac{\phi(p)}{\phi(q)} = \tilde{\phi}\left(\frac{p}{q}\right),$$
so $\psi = \tilde{\phi}$:

every ring isomorphism $\phi : R \to R$ extends uniquely to an automorphism of $K$.
\end{proof}


\paragraph{Ex. 6.4.6}

{\it As in the text, let $f = x^5-6x+3$.
\be
\item[(a)] Use the hints given in the text to show that every element of $S_5$ of order 5 is a 5-cycle.
\item[(b)] Use curve graphing from calculus to show that $f$ has exactly three real roots.
\ee
}

\begin{proof}
Let $f = x^5-6x+3$.
\begin{enumerate}
\item[(a)]
Let $\sigma \in S_5$ a permutation of order 5. Write $\sigma = \sigma_1\cdots\sigma_r \  (\sigma_i\neq e)$ the cycle decomposition of $\sigma$. Let $d_i = |\sigma_i|$ the order of $\sigma_i$ in $S_n$. As the cycles are disjoint, for all integer $k$, 
$\sigma ^k = \sigma_1^k \cdots \sigma_r^k$ and
\begin{align*}
\sigma^k = e &\iff \sigma_1^k = \cdots = \sigma_r^k = e\\
&\iff d_1 \mid k,\ d_2 \mid k,\ldots,\ d_r \mid k\\
&\iff \mathrm{lcm}(d_1,\ldots,d_r) \mid k
\end{align*}
So the order of $\sigma$ is the lcm of the orders $d_i$.
$$5 = \mathrm{lcm}(d_1,\ldots,d_r).$$
As $d_i \mid 5, \ i=1,\ldots,r$, and $d_i\neq 1$, where $5$ is prime, $d_i = 5$. The cycles $\sigma_i$ being disjoint, as $d_i = |\sigma_i| =  \mathrm{length}(\sigma_i)$, $d_1+\cdots+d_r\leq 5$, thus $rd_1 = 5r\leq 5$, so $r=1$.

Conclusion:  $\sigma = \sigma_1$ is a 5-cycle.

\item[(b)]

Let $f : \R \to \R, x\mapsto f(x) = x^5-6x+3$.

If $x\in \R$, $f'(x) = 5x^4-6 <  0 \iff x^4 < \frac{6}{5} \iff -x_0 < x < x_0,\quad \mathrm{where} \ x_0 =  \sqrt[4]{\frac{6}{5}}$.

$f$ is so strictly increasing on $]-\infty,-x_0]$, strictly decreasing on  $[-x_0,x_0]$, and strictly increasing on $[x_0,+\infty[$.

$f(x_0) = x_0(x_0^4-6)+3 = x_0(\frac{6}{5} - 6) + 3 = -\frac{24}{5}x_0 + 3 = \frac{3}{5}(5-8x_0)<0$: indeed $x_0=\sqrt[4]{\frac{6}{5}}> 1 > \frac{5}{8}$, so $5 - 8x_0 <0 $.

$f(-x_0) = -x_0(x_0^4-6)+3 = \frac{24}{5}x_0+3> 0$.

As $f$ is continuous, $\lim\limits_{x\to -\infty}f(x) = -\infty, f(-x_0)>0$, and $f$ is strictly increasing on $]-\infty, -x_0]$, the Intermediate Values Theorem shows that $f$ has a unique root in $]-\infty, -x_0]$.

With a similar reasoning on  $[-x_0,x_0]$ and and on $[x_0,+\infty[$, with $f(-x_0)<0,f(x_0)>0,\lim\limits_{x\to +\infty}f(x) = +\infty$, we prove that $f$ has a unique root in $[-x_0,x_0]$, and also in $[x_0,+\infty[$.

Conclusion: $f$ has exactly three real roots.
\end{enumerate}
\end{proof}

\paragraph{Ex. 6.4.7}

{\it Show that $S_n$ is generated by the transposition $(1\, 2)$ and the $n$-cycle $(1\, 2 \ldots\, n)$.
}

\begin{proof}
Let $G_n$  be the subgroup of $S_n$ generated by the transpositions $(1,2),(2,3),\cdots,{(n-1,n)}$:
$$G_n=\langle (1,2), (2,3),\ldots,(n-1,n) \rangle$$
 For all $i\in \{1,\cdots,n\}$, there exists $g\in G_n$ such that $g(n) = i$.

Indeed, if  $g = (i,i+1) \circ (i+1,i+2)\circ \cdots \circ (n-1,n)  =  (i,i+1) (i+1,i+2)\cdots (n-1,n)$  (with the convention $g=e$ if $i=n$), then  $g \in G_n$ and $g(n) = i$
(as ${\cal O}_n = [1,n]$, $G_n$ is a transitive subgroup of $S_n$).

Conclusion: for all $i \in \{1,\ldots,n\}$, there exists $g \in G_n$ tel que $g(n) = i$.

\bigskip

We show that $G_n = S_n$.

$S_2 = \{e,(1,2)\}$ is equal to  $G_2=\langle (1,2) \rangle$.

By induction, we suppose that $S_{n-1} = G_{n-1}\ (n\geq 3)$.

The subgroup of  $S_n$ of the permutations fixing $n$ is identified with $S_{n-1}$, so
$$\mathrm{Stab}_{S_n}(n) = S_{n-1}.$$

Let $\sigma \in S_n$ and $i = \sigma(n)$.
By part (a), there exist $g \in G_n$ such that $g(n) = i$. 
Then $$(g^{-1} \circ \sigma)(n) = n :  g' = g^{-1} \circ \sigma \in S_{n-1}.$$

Thus $\sigma = g \circ g'$, where $g \in G_n, g' \in G_{n-1} \subset G_n$, therefore $\sigma \in G_n$.
So $S_n \subset G_n$, and by definition $G_n \subset S_n$, thus $G_n = S_n$.

Conclusion: for all $n\geq 2$, $S_n = \langle (1,2), (2,3),\ldots,(n-1,n) \rangle$

\bigskip

We recall the following lemma :

{\bf Lemma. }{\it If $g=(a_1,\ldots,a_k)$ is a cycle in $S_n$, and $\sigma \in S_n$, then

$$\sigma \circ g \circ \sigma^{-1} = (\sigma(a_1),\ldots, \sigma(a_k)).$$}

Indeed,

$\bullet$ If $1 \leq i <k, (\sigma \circ g \circ \sigma^{-1})(\sigma(a_i)) = \sigma(g(a_i)) = \sigma(a_{i+1}))$.

$\bullet$ If $i = k, (\sigma \circ g \circ \sigma^{-1})(\sigma(a_k)) = \sigma(g(a_k)) = \sigma(a_1))$.

$\bullet$ if $x \not \in \{\sigma(a_1),\ldots,\sigma(a_k)\}$, then $\sigma^{-1}(x) \not \in \{a_1,\cdots,a_k\}$,

therefore $g(\sigma^{-1}(x)) = \sigma^{-1}(x), (\sigma \circ g \circ \sigma^{-1})(x) = x$. $\qed$

\bigskip

Let $\tau = (1,2), \sigma = (1,2,\ldots,n)$.

We apply the Lemma to  $ \tau$ and $\sigma^{k-1}, 1 \leq k <n$ :

$$\sigma^{k-1} \circ \tau \circ \sigma^{-(k-1)} = (\sigma^{k-1}(1), \sigma^{k-1}(2)) = (k,k+1).$$

Thus $\langle \sigma, \tau \rangle \supset G_n = S_n$.

Conclusion :  $S_n$ is generated by the transposition $(1, 2)$ and the $n$-cycle $(1, 2 \ldots,n)$.
\end{proof}

\paragraph{Ex. 6.4.8}

{\it Let $G$ and $H$ be groups where $G$ acts on $H$ by group homomorphisms. As in the text, we let $H \rtimes G$ denote the set $H \times G$ with the binary operation given by (6.9).
\be
\item[(a)] Prove that $H \rtimes G$ is a group.
\item[(b)] Prove that the map $H \rtimes G \to G$ defined by $(h,g) \mapsto g$ is an onto homomorphism with kernel $H \times \{e\}$.
\item[(c)] Prove that $h \mapsto (h,e)$ defines an isomorphism $H \simeq H \times \{e\}$ (where the group structure on $H \times \{e\}$ comes from $H \rtimes G$).
\ee
}

\begin{proof}
By definition of an action by group homomorphisms, there exists a group homomorphism $\varphi:G \to \mathrm{Aut}(H)$ such that for all $(g,h) \in G \times H$,
$$g \cdot h = \varphi(g)(h),$$
so $ h\mapsto g\cdot h$ is a group automorphism of $H$ for all $g\in G$.
\begin{enumerate}
\item[(a)] 
\boxed{I} If $h,h' \in H, g,g' \in G$, then $g\cdot h' \in H$, thus $(h(g\cdot h'),gg') \in H\times G$ , so this law defines a binary operation on $H\times G$.

\boxed{A} Let $(h,g), (h',g'), (h'',g'') \in H\times G$. Then
\begin{align*}
((h,g).(h',g')).(h'',g'') &= (h(g\cdot h'),gg').(h'',g'')\\
&=(h(g\cdot h')((gg')\cdot h''),gg'g'')\\
&=(h(g\cdot h')(g\cdot (g'\cdot h'')),gg'g'').
\end{align*}
The last equality is true because $G$ acts on $H$.
\begin{align*}
(h,g).((h',g').(h'',g'')) &= (h,g)((h'(g'\cdot h''),g'g'')\\
&=(h (g\cdot(h'(g'\cdot h''))),gg'g'')\\
&=(h (g\cdot h') (g\cdot(g' \cdot h'')), gg'g'').
\end{align*}
The last equality is true because $G$ acts on $H$ by group homomorphism.

Therefore $((h,g).(h',g')).(h'',g'') = (h,g).((h',g').(h'',g''))$: the law is associative.

\boxed{N} Write $e_H,e_G$ the identity of $H$ and the identity of $G$.
\begin{align*}
(f,g).(e_H,e_G) = (f (g\cdot e_H), g e_G) =  (f e_H, g e_G) = (f,g),\\
(e_H,e_G).(f,g) = (e_H(e_G\cdot f), e_G g) = (e_H f, e_Gg) = (f,g).
\end{align*}
So $(e_H,e_G)$ is the identity of $H \rtimes G$, which we will write now $(1,1)$.

\boxed{S}

Analysis: if $(h',g')$ is the inverse of $(h,g)$, then $(h(g\cdot h'),gg') = (1,1)$. Thus $g' = g^{-1}$, and $g\cdot h' = h^{-1}$, therefore $h' = g^{-1} \cdot (h^{-1})$. 

Synthesis: we show that $( g^{-1} \cdot (h^{-1}),g^{-1})$ is the inverse of $(h,g)$ :
\begin{align*}
(h,g).( g^{-1} \cdot (h^{-1}),g^{-1}) &= (h (g.(g^{-1}\cdot (h^{-1}))), g g^{-1})\\
&= (h (g g^{-1}\cdot (h^{-1})), 1)\\
&=(hh^{-1},1) = (1,1)\\
( g^{-1} \cdot (h^{-1}),g^{-1})(h,g) &= ((g^{-1} \cdot (h^{-1})) (g^{-1}.h), g^{-1}g)\\
&=(g^{-1}\cdot(h^{-1}h),1)\\
&=(g^{-1}\cdot 1,1) = (1,1)
\end{align*}
Every element of $H \rtimes G$ has an inverse.
\begin{center}
$H\rtimes G$ is a group.
\end{center}

\item[(b)]
Let
$
\psi : 
\left\{
\begin{array}{ccc}
  H\rtimes G &   \to & G  \\
  (h,g)& \mapsto   &  g 
\end{array}
\right.
$. 

$\psi((h,g).(h',g')) = \psi(h(g\cdot h'),gg') = gg' = \psi(h,g) \psi(h',g')$. $\psi$ is so a group homomorphism.

As every $g$ in $G$ is the image of $(1,g) \in   H\rtimes G$ by $\psi$, $\psi$ is surjective.

$\ker(\psi) = \{(h,g) \in   H\rtimes G \ \vert \ g=e\} = H \times \{e\}$.

\item[(c)]
Let
$
\chi : 
\left\{
\begin{array}{ccc}
  H &   \to & H \times \{e\}  \\
h& \mapsto   &  (h,e)
\end{array}
\right.
$. 

$\chi(h)\chi(h') = (h,e)(h',e) = (h (e\cdot h'), e) = (hh',e) = \chi(hh')$ : $\chi$ is a group homorphisme from $H$ on the subgroup $H\times \{e\}$ of $H \rtimes G$.

$\chi(h) = (e,e) \iff h=e$, thus $\chi$ est injective, and surjective since every element of $H\times\{e\}$ is of the form $(h,e) = \chi(h)$ : $H \simeq H\times \{e\}$.

Therefore the sequence  $\{e\} \to H \to H\rtimes G \to G \to \{e\}$ is a short exact sequence, so $H\rtimes G$ is an extension of $H$ by $G$.
\end{enumerate}
\end{proof}

\paragraph{Ex. 6.4.9}

{\it Explain how (6.6) and (6.10) relate to the last paragraph of the discussion of semidirect products in the Mathematical Notes.
}

\begin{proof}
The group homomorphism $\psi$ of Exercise 8 shows that $(H\rtimes G) / (H \times\{e\}) \simeq G$.

Moreover the isomorphism (6.10)
$
\phi : 
\left\{
\begin{array}{ccc}
  \mathrm{AGL}(1,\F_p)  &   \to & \F_p \rtimes \F_p^*  \\
 \gamma_{a,b}& \mapsto   &  (b,a) 
\end{array}
\right.
$

maps $T = \{\gamma_{1,b} \ \vert \ b \in \F_p\}$ on $\F_p \times \{1\}$.
 
Therefore $\mathrm{AGL}(1,\F_p)/T \simeq (\F_p \rtimes \F_p^*)/(\F_p \times \{1\}) \simeq \F_p^*$ : we obtain so (6.6) :
 \begin{center}
  $\mathrm{AGL}(1,\F_p)/T \simeq \F_p^*$.
\end{center}
\end{proof}

\paragraph{Ex. 6.4.10}

{\it Let $p\geq 3$ be prime, and let $\F_p \rtimes \F_p^*$ be the semidirect product described in the Mathematical Notes.
\be
\item[(a)] Show that $\F_p \rtimes \F_p^*$ is not Abelian.
\item[(b)] Show that the product group $\F_p \times \F_p^*$ is Abelian.
\item[(c)] Show that $\F_p \times \F_p^*$ is an extension of $\F_p$ by $\F_p^*$.
\ee
Since we already know that $\F_p \rtimes \F_p^*$ is an extension of $\F_p$ by $\F_p^*$, we see that (a) and (b) give nonisomorphic extensions.
}

\begin{proof}
\begin{enumerate}
\item[(a)]
As $p\geq 3$, there exist in  $\F_p$ an element $2$ with $2\neq 0,2\neq 1$, so $(0,2) \in \F_p \rtimes \F_p^*$, and also $(1,1) \in \F_p \rtimes \F_p^*$.
\begin{align*}
(0,2) . (1,1) = (0+2\times 1,2\times 1)=(2,2)\\
(1,1). (0,2) = (1+1\times0,1\times 2) = (1,2) \\
\end{align*}
Since $2\neq 1$,  $(0,2) . (1,1) \neq (1,1). (0,2)$. So if $p\geq 3$, then $\F_p \rtimes F_p^*$ is not Abelian.

\item[(b)]
By definition of the product in $\F_p \times \F_p^*$, $(a,b)(c,d) = (ac,bd) = (ca,db) = (c,d)(a,b)$: $\F_p \times \F_p^*$ is Abelian.

\item[(c)]The sequence
$$\{0\} \to \F_p \to \F_p \times \F_p^* \to \F_p^* \to \{1\}$$
is a short exact sequence (the first arrow is the injective map $x \mapsto (x,1)$, and the second one is the surjective map $(x,y) \mapsto y$).  Actually, a direct product is a special case of semidirect product, where $\varphi : G \to \mathrm{Aut}(h)$ is the trivial action defined by $\phi(g) = 1_H$ for all $g\in G$, so $\varphi(g)(h) = g\cdot h = h$ for all $h \in H$. By part (a) and (b), these two extensions are not isomorphic.

\end{enumerate}
\end{proof}

\paragraph{Ex. 6.4.11}

{\it The goal of this exercise is to show that the group $G$ of permutations (6.11) is metacyclic in the sense that $G$ has a normal subgroup $H$ such that $H$ and $G/H$ are cyclic. Show that this follows from $G \simeq AGL(1,\F_p)$ together with (6.6) and proposition A.5.3.
}

\begin{proof}
If $L=\Q(\zeta_p,\sqrt[p]{2})$, and $G = \Gal(L/\Q)$, then by (6.4), $G \simeq \mathrm{AGL}(1,\F_p)$.
By (6.6) and Exercise 9, $ \mathrm{AGL}(1,\F_p)/T \simeq \F_p^*$, and $T \simeq \F_p$. As $\F_p$ is a cyclic (additive) group, and $\F_p^*$ a cyclic (multiplicative) group by Proposition A.5.3, $G = \Gal(L/\Q)$ is metacyclic.
\end{proof}

\paragraph{Ex. 6.4.12}

{\it Let $p$ be prime. Generalize part (a) of Exercise 6 by showing that every element of $S_p$ of order $p$ is a $p$-cycle.
}

\begin{proof}
Let $\sigma \in S_p$ a permutation of order p. Write $\sigma = \sigma_1\cdots\sigma_r \ (\sigma_i\neq e)$ the cycle decomposition of $\sigma$. Let $d_i = |\sigma_i|$ the order of $\sigma_i$ in $S_n$.  The order of $\sigma$ is the lcm of the orders $d_i$ (see Ex. 6).
$$p = \mathrm{lcm}(d_1,\ldots,d_r).$$
As $d_i \mid p, \ i=1,\ldots,r$, and $d_i\neq 1$, where $p$ is prime, $d_i = p$. The cycles $\sigma_i$ being disjoint, as $d_i = |\sigma_i| =  \mathrm{length}(\sigma_i)$, $d_1+\cdots+d_r\leq p$, thus $rd_1 = pr\leq p$, so $r=1$.

Conclusion :  $\sigma = \sigma_1$ is a $p$-cycle.
\end{proof}

\paragraph{Ex. 6.4.13}

{\it Let $L$ be the splitting field of $2x^5 - 10x+ 5$ over $\Q$. Prove that ${\Gal(L/\Q) \simeq S_5}$.
}

{\bf Lemma} : {\it  Let $p$ be a prime number. Let $\alpha = (i,j)\in S_p$ a transposition, and $\beta\in S_p$ a $p$-cycle. 
Then $S_p = \langle \alpha,\beta \rangle$}.

{\it Proof of lemma.}  $\beta \in S_p$ is a $p$-cycle, so $\beta = (a_1,a_2,\ldots,a_p) = (a , \beta(a), \ldots, \beta^{p-1}(a))$, 

where $1 \leq a=a_1\leq p$ is fixed.

The $\beta^i(a)$ are distinct, otherwise $\beta^i(a) = \beta^j(a), i<j$ implies  $ \beta^{j-i}(a) = a$, so the cycle would have an order at most equal to $j-i <p$, thus not equal to $p$.

The support of $\beta$, $\mathrm{Supp}(\beta) = \{a_1,\cdots,a_p\}$ has so $p$ elements, therefore 

$$\mathrm{Supp}(\beta) =\{1,2,\ldots,p\}.$$

So there exists $r<p, s<p$ such that $i = \beta^r(a), j = \beta^s(a)$, thus $j = \beta^{s-r}(i)$.

Let $k$ the remainder of the division of $s-r$ by $p$. Then $\beta^k(i) = j, 0 \leq k \leq p-1,$

and as $i\neq j$ since $\alpha = (i j)$ is a transposition, $k\neq 0$, so
$$\beta^k(i) = j, 1 \leq k \leq p-1.$$
As $p$ is prime, and $1 \leq k \leq p-1$, $\beta^k$ is also a $p$-cycle.

 Indeed,  $H = \{n \in \mathbb{Z} \ \vert  \ (\beta^k)^n(a) = a\}$ is a subgroup of $\mathbb{Z}$, therefore it is of the form $H=d\mathbb{Z}, d>0$.

As $p \in H$, $d$ divides $p$, and $d \neq 1$ (otherwise $\beta^k(a) = a, k<p $), thus $d = p$.

Consequently $\beta^k = (a, \beta^k(a), \beta^{2k}(a), \ldots,\beta^{(p-1)k}(a))$ is a $p$-cycle, thus $i$ is in the support of $\beta^k$.

$\alpha= (i,j), \beta^k = (i , j = \beta^k(i), \cdots,\beta^{(p-1)k}(i) )$ generate  $S_n$ as in Exercice 7 where we have proved that $\sigma = (1,2,\ldots,p)$ and $\tau = (1,2)$ generate $Sp$.

 There is a simple relabeling of the roots. More formally, let 
 $$\gamma =
\left(
\begin{array}{cccc}
  1 & 2  & \ldots & p \\
   i &  j = \beta^k(i) & \ldots & \beta^{k(p-1)}(i) 
\end{array}
\right).
$$

Let $g $ be any permutation in $S_n$. Then $\gamma^{-1} g \gamma \in S_n = \langle \sigma, \tau \rangle$. 

So  $\gamma^{-1} g \gamma = \sigma_1 \sigma_2 \cdots \sigma_l$, where $\sigma_i = \tau$, or $\sigma_i =\sigma$ (we can avoid negative powers since each element is of finite order).

Then $g = (\gamma \sigma_1 \gamma^{-1})(\gamma \sigma_2 \gamma^{-1})\cdots(\gamma \sigma_l \gamma^{-1})$, and $\gamma \sigma_i \gamma^{-1} \in \{ \alpha, \beta^k\}$, since by the Lemma of Exercise 6.4.1: $\gamma \tau \gamma^{-1} = \alpha, \gamma \sigma \gamma^{-1} = \beta^k$.

$S_n$ is generated by $\alpha, \beta^k$, a fortiori by $\alpha, \beta$.

Conclusion: if $p$ is prime, a $p$-cycle $\beta$, and any transposition $(i,j)$ generate $S_n$.
$\qed$

\bigskip 

\begin{proof}
Let $f=2x^5-10x+5 \in \Q[x]$, and $L$ the splitting field of $f$, $G = \Gal(L,\Q)$, and $G' \subset S_5$ the corresponding subgroup of $S_5$ isomorphic to $G$.

The  Sch�nemann-Eisenstein Criterion with $p=5$ shows that $f$ is irreducible over $\Q$ (if  $f= \sum_{k=0}^5 a_i x^i$, $5\nmid a_5=2, 5\mid a_i, i=0,\ldots 4, 5^2\nmid a_0 = 5$).

Thus $G$ acts transitively on the roots of $f$ (Proposition 6.3.7). By Exercise 6.3.6, 5 divides $ \vert G \vert$.

By Cauchy's Theorem, there exists an element $\sigma$ of order 5 in $G$, thus an element  $\tilde{\sigma}$ of order 5 in $G' \subset S_5$. Exercice 6.4.6(a) shows that $\tilde{\sigma}$ is a 5-cycle.

For all $t \in \R, f'(t) <  0 \iff \vert t \vert < 1$, $f$ is strictly decreasing on $[-1,1]$,  strictly increasing on $]-\infty,-1]$ and on $[1,+\infty[$. As $f$ is continuous, $f(1) = -3<0, f(-1) = 13>0$, and $\lim_{+\infty} f= +\infty, \lim_{-\infty} f= -\infty$, the Intermediate Values Theorem shows that the polynomial $f$ has exactly 3 real roots, thus 2 non real conjugate complex roots. The restriction $\tau$ of complex conjugation to $L$ is a $\Q$-automorphism of $L$ (thus $\tau \in G$) who fixes three roots and exchanges the two others. The corresponding element  $\tilde{\tau}$ in $G' \subset S_5$ is so a transposition. By the above Lemma, $G' = S_5$, and so $$G = \Gal(L/\Q) \simeq S_5.$$
\end{proof}

\paragraph{Ex. 6.4.14}

{\it Let $L = \Q(\zeta_p,\sqrt[P]{2})$.  Prove that $L = \Q(\sqrt[p]{2}, \zeta_p \sqrt[p]{2})$, i.e. the splitting field of $x^p-2$ over $Q$ can be generated by two of its roots.
}

\begin{proof}
Let $L = \Q(\zeta_p, \sqrt[p]{2})$.

$\sqrt[p]{2} \in L$, and $\zeta_p \sqrt[p]{2} \in L$, thus $\Q(\sqrt[p]{2},\zeta_p \sqrt[p]{2}) \subset L$.

$\zeta_p = \zeta_p \sqrt[p]{2}/\sqrt[p]{2} \in \Q(\sqrt[p]{2},\zeta_p \sqrt[p]{2})$, and $\sqrt[p]{2} \in \Q(\sqrt[p]{2},\zeta_p \sqrt[p]{2})$.
As $L$ is the smallest subfield of $\C$ containing $\Q$, $\zeta_p, \sqrt[p]{2} $, then  $L \subset \Q(\sqrt[p]{2},\zeta_p \sqrt[p]{2})$.

Conclusion : $$\Q(\zeta_p, \sqrt[p]{2}) = \Q(\zeta_p, \zeta_p\sqrt[p]{2}).$$

The splitting field of $x^p-2$ over $\Q$ is generated by two of its roots.
\end{proof}

\paragraph{Ex. 6.4.15}

{\it Let $L = \Q(\zeta_p,\sqrt[p]{2})$. The description of $\Gal(L/\Q)$ given in the text enables one to construct some elements of $\Gal(L/\Q(\zeta_p))$. Use these automorphisms and Proposition 6.3.7 to prove that $x^p - 2 $ is irreducible over $\Q(\zeta_p)$.
}

\begin{proof}
Let $L = \Q(\zeta_p,\sqrt[p]{2})$, the splitting field of $f = x^p-2$ over $\Q$. Then $\Gal(L/\Q) \simeq \mathrm{AGL}(1,\F_p)$.

We show that $x^p-2$ is irreducible over $\Q(\zeta_p)$.


$\Phi_p=1+x+\cdots+x^{p-1}$ is irreducible over $\Q$, thus $[\Q(\zeta_p):\Q] = p-1$.

$[L:\Q] = p(p-1)$ by Section 6.4.  We deduce of $[L:\Q] = [L:\Q(\zeta_p)]\ [\Q(\zeta_p):\Q]$ that

$$[\Q(\zeta_p,\sqrt[p]{2}) : \Q(\zeta_p)]  = p.$$

(If $g$ is the minimal polynomial of $\sqrt[p]{2}$ over $\Q(\zeta_p)$, then ${\deg(g) = [\Q(\zeta_p,\sqrt[p]{2}) : \Q(\zeta_p)] } = p$. Moreover $\sqrt[p]{2}$ is a root of $f=x^p-2\in \Q[x] \subset\Q(\zeta_p)[x]$, thus $g \mid f$ in $\Q(\zeta_p)[x]$, where $f,g$ are of the same degree  $p$ and monic, thus $g=f = x^p-2$. Therefore $x^p-2$ is irreducible over $\Q(\zeta_p)$.)

Following the wording, we note that $\sigma = \sigma_{1,1} \in \Gal(L/\Q(\zeta_p))$ defined by 
$$\sigma(\zeta_p) = \zeta_p, \sigma(\sqrt[p]{2}) = \zeta_p  \sqrt[p]{2}$$
is of order $p$, and corresponds to the $p$-cycle $\tilde{\sigma} = (1, 2, \cdots, p)$, if we number the roots by $z_1 = \sqrt[p]{2}, \ldots,  z_p= \zeta_p^{p-1} \sqrt[p]{2}$. Since $\tilde{\sigma}^{k-1}(1) = k, k = 1, \ldots, p-1$, the subgroup $\langle \tilde{\sigma} \rangle \subset S_n$ is transitive, and so is $\langle \sigma \rangle$. Since $\Gal(L/\Q(\zeta_p)) \supset \langle \sigma \rangle$, $\Gal(L/\Q(\zeta_p))$ acts transitively on the roots of $x^p - 2$. So, by Proposition 6.3.7, $x^p - 2$ is irreducible over $\Q(\zeta_p)$.

\end{proof}

\subsection{ABELIAN EQUATION (OPTIONAL)}
\paragraph{Ex. 6.5.1}

{\it Assume that $f\in F[x]$ is nonconstant and has roots $\alpha_1 = \alpha,\alpha_2,\ldots,\alpha_n$ in a splitting field $L$. Prove that $L = F(\alpha)$ if and only if there are rational functions $\theta_i \in F(x)$ such that $\alpha_i = \theta_i(\alpha)$. Can we assume that the $\theta_i$ are polynomials?
}

\begin{proof}

$\bullet$ Suppose that $L=F(\alpha)$. As $\alpha_i \in L$, $\alpha_i \in F(\alpha)$. By definition of $F(\alpha)$, there exist $\theta_i \in F(x)$ such that $\alpha_i = \theta_i(\alpha)$.

$\bullet$ Conversely, suppose that for all $i$, $1\leq i \leq n$, $\alpha_i = \theta_i(\alpha), \theta_i \in F(x)$.
Thus $\alpha_i\in F(\alpha)$. Consequently $L=F(\alpha_1,\ldots,\alpha_n)\subset F(\alpha)$. 

As $F(\alpha) = F(\alpha_1)\subset F(\alpha_1,\ldots,\alpha_n)$, $L =F(\alpha_1,\ldots,\alpha_n) = F(\alpha)$.

Conclusion: $L=F(\alpha) \iff \alpha_i = \theta_i(\alpha), \theta_i \in F(x)\  (1\leq i \leq n)$.

Moreover, as $\alpha$ is algebraic over $F$, $F(\alpha) = F[\alpha]$, therefore every $\alpha_i \in F(\alpha) = F[\alpha]$ is of the form $\alpha_i = \theta_i(\alpha)$, where the $\theta_i \in F[x]$ are polynomials.
\end{proof}

\paragraph{Ex. 6.5.2}

{\it Show that the equation $x^4-10x^2+1 = 0$ discussed in Example 6.5.1 is Abelian.
}

\begin{proof}
As in Example 6.5.1, let $\theta_1(x) = x, \theta_2(x) = -x, \theta_3(x) = 10x-x^3,\theta_4(x) = -10x+x^3$, so the solutions of the equations are $\alpha_i = \theta_i(\alpha), \ i=1,2,3,4$.

The roots of $f$ being polynomials in $\alpha$, the splitting field of $f$ is $F(\alpha)$ (See  Exercise 1).

Moreover, as $\theta_1 = x, \theta_2 = -x, \theta_4 = -\theta_3$ and $\theta_3,\theta_4$ are odd functions,

 $\theta_1(\theta_i(\alpha)) = \theta_i(\alpha) = \theta_i(\theta(\alpha)),\ i=2,3,4$.

$\theta_2(\theta_i(\alpha)) = - \theta_i(\alpha) = \theta_i(-\alpha) = \theta_i(\theta_2(\alpha)),\ i=3,4$.

$\theta_3(\theta_4(\alpha)) = \theta_3(-\theta_3(\alpha)) = -\theta_3^2(\alpha) = -\theta_4^2(\alpha) = \theta_4(-\theta_4(\alpha)) = \theta_4(\theta_3(\alpha))$.

Thus $\theta_i(\theta_j(\alpha)) = \theta_j(\theta_i(\alpha))$, for $1\leq i <j\leq 4$, thus also for $1\leq i,j\leq 4$.
\begin{center}
$x^4-10x^2+1 = 0$ is an Abelian equation.
\end{center}
\end{proof}

\paragraph{Ex. 6.5.3}

{\it Complete the proof of theorem 6.5.3.
}

\begin{proof}
We show that the Galois group $G$ of an Abelian equation is Abelian. 

Let $L=F(\alpha_1,\ldots,\alpha_n)$ a splitting field of $f \in F[x]$, et $\alpha=\alpha_1$.

By definition of an Abelian equation,  there exists $\theta_i \in F(x)$ tels que $\alpha_i = \theta_i(\alpha)\ (i=1,\ldots,n)$, so $L = F(\alpha)$ (see Exercise 1).

$\bullet$ 
$\sigma \in \Gal(L/F)$, et $f\in F[x]$, thus $\sigma(\alpha)$ is also a root $\alpha_i,1\leq i \leq n$ of $f$:

 $\sigma(\alpha) = \alpha_i = \theta_i(\alpha)$. Similarly $\tau(\alpha) = \theta_j(\alpha),1\leq j\leq n$.

$\bullet$ 
if $\sigma\tau = \tau \sigma$, then $\sigma( \tau (\alpha)) = (\sigma \tau)(\alpha) = (\tau \sigma)(\alpha) = \tau(\sigma(\alpha))$.

R�ciproquement, si $\sigma( \tau (\alpha))  = \tau(\sigma(\alpha))$, then $(\sigma \tau)(\alpha) = (\tau \sigma)(\alpha)$.

As $L = F(\alpha)$, and as  $\sigma \tau$ and $\tau \sigma$ are identity over $f$ $F$ and send $\alpha$ on the same element, $\sigma\tau = \tau \sigma$.
$$\sigma \tau = \tau \sigma \iff \sigma( \tau (\alpha))  = \tau(\sigma(\alpha)).$$

$\bullet$ $\sigma(\tau(\alpha)) = \sigma(\theta_j(\alpha))$.
Moreover $\sigma$ is a $F$-automorphism of fields, et $\theta_j \in F(x)$ a polynomial, thus $\sigma(\theta_j(\alpha)) = \theta_j(\sigma(\alpha)) = \theta_j(\theta_i(\alpha)$. Therefore 
$\sigma(\tau(\alpha)) = \theta_j(\theta_i(\alpha))$. Similarly $\tau(\sigma(\alpha)) = \theta_i(\theta_j(\alpha))$.

The equation $f=0$ being Abelian, $\theta_j(\theta_i(\alpha)) = \theta_i(\theta_j(\alpha))$, thus $\sigma(\tau(\alpha)) = \tau(\sigma(\alpha))$, so $\sigma \tau=\tau \sigma$, and this is true for all  $\sigma,\tau \in \Gal(L/F)$: $\Gal(L/F)$ is Abelian.

Conclusion: the Galois group of an Abelian equation is Abelian. 
\end{proof}

\paragraph{Ex. 6.5.4}

{\it Show that $x^n-1$ is an Abelian equation over $\Q$.
}

\begin{proof}
The roots of $f=x^n-1$ in $\C$ are $\zeta^k, 0\leq k < n$, where $\zeta = e^{2i\pi/n}$.

The splitting field of $f$ over $\Q$ is $Q(1,\zeta,\cdots,\zeta^{n-1}) = \Q(\zeta)$. Moreover, every root $\zeta^k$ is of the form $\zeta^k = \theta_k(\zeta)$, where $\theta_k = x^k,0\leq k \leq n-1$.

$$\theta_i(\theta_j(\zeta)) = (\zeta^j)^i = \zeta^{ji} = (\zeta^i)^j = \theta_j(\theta_i(\zeta)),\ 0\leq i,j \leq n-1,$$ so by definition $x^n-1=0$ is an Abelian equation.
\end{proof}

\paragraph{Ex. 6.5.5}

{\it Let $f$ the minimal polynomial of $\sqrt{2+\sqrt{2}}$ over $\Q$. Show that $f=0$ is an Abelian equation.
}

\begin{proof}

By Exercises 5.1.6 and 6.3.4,
\begin{align*}
f&= x^4-4x^2+2\\
&= \left(x-\sqrt{2+\sqrt{2}}\right)\left(x+\sqrt{2+\sqrt{2}}\right)\left(x-\sqrt{2-\sqrt{2}}\right)\left(x+\sqrt{2-\sqrt{2}}\right)\\
&=(x-\alpha)(x+\alpha)(x-\beta)(x+\beta)\\
\end{align*}
where $\beta= \frac{1}{2}\left(\alpha- \frac{2}{\alpha^3}\right)  = \alpha^3-3\alpha$.

The 4 roots of $f$ are of the form $\alpha = \theta_1(\alpha), -\alpha = \theta_2(\alpha), \beta=\theta_3(\alpha), -\beta = \theta_4(\alpha)$, where
$$\theta_1(x) = x, \theta_2(x) = -x, \theta_3(x) = x^3-3x,\theta_4(x) = -x^3+3x.$$

As $\theta_1 = x, \theta_2 = -x, \theta_4 = -\theta_3$ and $\theta_3,\theta_4$ are odd functions, as in Exercise 2, 

 $\theta_1(\theta_i(\alpha)) = \theta_i(\alpha) = \theta_i(\theta_1(\alpha)),\ i=2,3,4$.

$\theta_2(\theta_i(\alpha)) = - \theta_i(\alpha) = \theta_i(-\alpha) = \theta_i(\theta_2(\alpha)),\ i=3,4$.

$\theta_3(\theta_4(\alpha)) = \theta_3(-\theta_3(\alpha)) = -\theta_3^2(\alpha) = -\theta_4^2(\alpha) = \theta_4(-\theta_4(\alpha)) = \theta_4(\theta_3(\alpha))$.

Thus $\theta_i(\theta_j(\alpha)) = \theta_j(\theta_i(\alpha))$, for $1\leq i <j\leq 4$, thus also for $1\leq i,j\leq 4$.

 $\theta_i(\theta_j(\alpha)) = \theta_j(\theta_i(\alpha)),\ 1\leq i,j\leq 4$. 
 \begin{center}
  $x^4-4x^2+2=0$ is an Abelian equation.
  \end{center}
\end{proof}

\paragraph{Ex. 6.5.6}

{\it In this exercise, you will prove a partial converse to Theorem 6.5.3. Suppose that a finite extension $F\subset L$ is normal ans separable and has an Abelian Galois group.
\be
\item[(a)] Explain why $F\subset L$ has a primitive element.
\item[(b)] By part (a), we can find $\alpha \in L$ such that $L = F(\alpha)$. Let $f$ be the minimal polynomial of $\alpha$. Prove that $f=0$ is an Abelian equation over $f$.
\ee
}

\begin{proof}
Suppose that $F \subset L$ is normal and separable and that $G = \Gal(L/F)$ is an Abelian group.
\begin{enumerate}
\item[(a)]
As $F \subset L$ is separable, the Theorem of the Primitive Element shows that there exists a separable element $\alpha \in L$ such that $L=F(\alpha)$.

\item[(b)]
Let $f$ the minimal polynomial of $\alpha$ over $F$. Then $f$ is irreducible and separable. As $F \subset L$ is normal, the roots $\alpha_1=\alpha,\ldots,\alpha_n$ of $f$ are all in $L$, so  $L = F(\alpha) = F(\alpha_1,\ldots,\alpha_n)$ is the splitting field of $f$. By Exercise 1, there exist polynomials $\theta_i\in F[x]$ such that $\alpha_i = \theta_i(\alpha),\ i= 1,\ldots,n$.

Let $1\leq i,j \leq n$. As $f$ is separable and irreducible, by Proposition 6.3.7, the Galois group $G$ acts transitively on the set of the roots of $f$, so there exists $\sigma, \tau \in G$ such that $\theta_i(\alpha) = \sigma(\alpha)$ and $\theta_j(\alpha) = \tau(\alpha)$.

Exercise 3 shows that $(\sigma \tau)(\alpha) = \theta_j(\theta_i(\alpha))$ and $(\tau \sigma)(\alpha) = \theta_i(\theta_j(\alpha))$ . As $G$ is Abelian by hypothesis, $\sigma \tau = \tau \sigma$, so

$$ \theta_j(\theta_i(\alpha)) =  \theta_i(\theta_j(\alpha)), 1\leq i,j \leq n.$$
The equation $f=0$ is Abelian.

Conclusion: if the finite extension $F\subset L$ is normal and separable and has an Abelian Galois group, and if $f$ is the minimal polynomial of a primitive element $\alpha$, then $f=0$ is an Abelian equation.
\end{enumerate}
\end{proof}
\paragraph{Ex. 6.5.7}

{\it Show that the implication $(a) \Rightarrow (b)$ of Theorem 6.5.5 is equivalent to Kronecker's assertion that the roots of an Abelian equation over $\Q$ can be expressed rationally in terms of a root of unity.
}

\begin{proof}
Suppose that the implication (a)$\Rightarrow$(b) of Theorem 6.5.5 is true.

Let $f \in \Q[x]$ such that the equation $f=0$ is Abelian. Then $f$ has a root $\alpha$ such that $L = F(\alpha)$ is the splitting field of $F$, so the extension $F \subset L$ is normal. By Theorem 6.5.3 (and Exercise 3), as the equation $f=0$ is Abelian, $\Gal(L/\Q)$ is an Abelian group. The hypothesis (a) is so satisfied, and  (b) follows : $L \subset \Q(\zeta_n)$, where $\zeta_n =e^{2i\pi/n}$. As the roots of $f$ are in $L$, these roots can be expressed rationally in terms of a root of unity.

Conversely, suppose that the roots $\alpha_1,\ldots,\alpha_n$ of any Abelian equation $f=0$ in the splitting field of $f$ can be expressed rationally in terms of a root of unity
 $\zeta_n$, and suppose also (a) : $\Q\subset L \subset \C$, the extension $\Q\subset L$ is normal, and $\Gal(L/\Q)$ is an Abellan group.
 
 As the characteristic of $\Q$ is 0, $\Q\subset L$  is also separable, and there exists a primitive element $\alpha$ for the extension  $\Q \subset L$. Let $f$ the minimal polynomial of $\alpha$ over $\Q$. By Exercise 6, since $\Q\subset L$ is normal and separable, the equation $f=0$ is Abelian. By hypothesis, the roots  $\alpha_1=\alpha,\ldots,\alpha_n$ of $f$ can be expressed rationally in terms of a root of unity $\zeta_n$, therefore $\alpha_i \in \Q(\zeta_n), 1\leq i \leq n$. In particular $\alpha \in \Q(\zeta_n)$, thus $L = \Q(\alpha) \subset \Q(\zeta_n)$. (b) is so proved under the hypothesis (a).


Conclusion : (a)$\Rightarrow$(b) is equivalent to the assertion of Kronecker : the roots of an Abelian equation over $\Q$ can be expressed rationally in terms of a root of unity.
\end{proof}

\end{document}