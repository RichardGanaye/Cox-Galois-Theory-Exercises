%&LaTeX
\documentclass[11pt,a4paper]{article}
\usepackage[frenchb,english]{babel}
\usepackage[applemac]{inputenc}
\usepackage[OT1]{fontenc}
\usepackage[]{graphicx}
\usepackage{amsmath}
\usepackage{amsfonts}
\usepackage{amsthm}
\usepackage{amssymb}
\usepackage{tikz}
%\input{8bitdefs}

% marges
\topmargin 10pt
\headsep 10pt
\headheight 10pt
\marginparwidth 30pt
\oddsidemargin 40pt
\evensidemargin 40pt
\footskip 30pt
\textheight 670pt
\textwidth 420pt

\def\imp{\Rightarrow}
\def\gcro{\mbox{[\hspace{-.15em}[}}% intervalles d'entiers 
\def\dcro{\mbox{]\hspace{-.15em}]}}

\newcommand{\be} {\begin{enumerate}}
\newcommand{\ee} {\end{enumerate}}
\newcommand{\deb}{\begin{eqnarray*}}
\newcommand{\fin}{\end{eqnarray*}}
\newcommand{\ssi} {si et seulement si }
\newcommand{\D}{\mathrm{d}}
\newcommand{\Q}{\mathbb{Q}}
\newcommand{\Z}{\mathbb{Z}}
\newcommand{\N}{\mathbb{N}}
\newcommand{\R}{\mathbb{R}}
\newcommand{\C}{\mathbb{C}}
\newcommand{\F}{\mathbb{F}}
\newcommand{\U}{\mathbb{U}}
\newcommand{\re}{\,\mathrm{Re}\,}
\newcommand{\im}{\,\mathrm{Im}\,}
\newcommand{\ord}{\mathrm{ord}}
\newcommand{\Gal}{\mathrm{Gal}}
\newcommand{\legendre}[2]{\genfrac{(}{)}{}{}{#1}{#2}}

\title{Solutions to David A.Cox  "Galois Theory''}
\author{Richard Ganaye}
\refstepcounter{section} \refstepcounter{section} \refstepcounter{section} \refstepcounter{section}
\refstepcounter{section}\refstepcounter{section}\refstepcounter{section}\refstepcounter{section}
\refstepcounter{section}\refstepcounter{section}\refstepcounter{section}\refstepcounter{section}

\begin{document}
\maketitle

\section{Chapter 13 : LAGRANGE, COMPUTING GALOIS GROUPS}

\subsection{QUARTIC POLYNOMIALS}
\paragraph{Ex. 13.1.1}

{\it Let $f \in F[x]$ be separable of degree $n$, and let $\alpha_1,\ldots,\alpha_n$ be the roots of $f$ in a splitting field $F\subset L$ of $f$. In Section 6.3 we used the action of the Galois group on the roots to construct a one-to-one group homomorphism $\phi_1:\Gal(L/F) \to S_n$. Now let $\beta_1,\ldots,\beta_n$ be the same roots, possibly written in a different order. This gives $\phi_2 : Gal(L/F) \to S_n$. To relate $\phi_1$ and $\phi_2$, note that there is $\gamma \in S_n$ such that $\beta_i = \alpha_{\gamma(i)}$ for $1\leq i \leq n$. Now define the conjugation map $\hat{\gamma}:S_n \to S_n$ by $\hat{\gamma}(\tau) = \gamma^{-1} \tau \gamma$.
\be
\item[(a)] Prove that $\phi_2 = \hat{\gamma} \circ \phi_1$.
\item[(b)] Let $G \subset S_n$ be the image of $\phi_1$. Explain why part (a) justifies the assertion made in the text that "if we change the labels, then $G$ gets replaced with a conjugate subgroup".
\ee
}

\begin{proof}
\be
\item[(a)]
By definition of the isomorphism $\phi_1 : \Gal(L/F) \to S_n$ in Section 6.3, if $\tau_1 = \phi_1(\sigma)$, then 
$$\sigma(\alpha_i) = \alpha_{\tau_1(i)},\qquad i=1,\ldots,n.$$
As $\beta_1,\ldots,\beta_n$ are the same roots in a different order, there exist a permutation $\gamma \in S_n$ such that
$$\beta_i = \alpha_{\gamma(i)},\qquad i=1,\ldots,n.$$
This numbering of the roots is associate to the isomorphisme $\phi_2$. If $\tau_2 = \phi_2(\sigma)$, then 
$$\sigma(\beta_i) = \beta_{\tau_2(i)}, \qquad i=1,\ldots,n.$$
Therefore, for all $i=1,\ldots,n$,
\begin{align*}
\sigma(\alpha_{\gamma(i)} )&= \alpha_{\gamma(\tau_2(i))}\\
\sigma(\alpha_{\gamma(i)}) &= \alpha_{\tau_1(\gamma(i))}.
\end{align*}
Thus $\alpha_{\gamma(\tau_2(i))} = \alpha_{\tau_1(\gamma(i))}$ for all $i$. Since $i \mapsto \alpha_i$ is one-to-one,
$$\gamma(\tau_2(i)) = \tau_1(\gamma(i)),\qquad i=1,\ldots,n,$$
so
$$\gamma  \tau_2 = \tau_1 \gamma.$$
Therefore $\tau_2 = \gamma^{-1} \tau_1 \gamma$, so  $\phi_2(\sigma) = \hat{\gamma}(\phi_1(\sigma))$, for all $\sigma \in \Gal(L/F)$:
$$\phi_2 = \hat{\gamma} \circ \phi_1.$$
\item[(b)] Let $G$ the image of $\phi_1$ in $S_n$ : $G = \{\phi_1(\sigma)\ | \ \sigma \in \Gal(L/F)\} \subset S_n$. 

Similarly the image of $\phi_2$ is $G' = \{\phi_2(\sigma)\ | \ \sigma \in \Gal(L/F)\} \subset S_n$. 

Since $\phi_2(\sigma) = \gamma^{-1} \phi_1(\sigma) \gamma$ for all $\sigma \in \Gal(L/F)$,
$$G' = \gamma^{-1} G \gamma.$$
So, if we change the labels, then $G$ gets replaces with a conjugate subgroup.
\ee
\end{proof}

\paragraph{Ex. 13.1.2}

{\it Prove that $A_4$ is the only subgroup of $S_4$ with 12 elements.
}

\begin{proof}
Let $H$ a subgroup of $S_n$ such that $[S_n:H]=2$. Then $H$ is normal in $S_n$ (by Exercise 12.1.20). Thus $G/H \simeq \{1,-1\}$. So there exists a group homomorphism 
$$\varphi:S_n \to \{1,-1\}, \qquad \ker(\varphi) = H.$$

Any two transpositions $\tau_1 = (a\, b), \tau_2 = (c\ d)$ of $S_n$ are conjugate: if $\gamma = (a \ c)(b\  d)$, then $\tau_2 = \gamma \tau_1 \gamma^{-1}$ (even if $b=c$).

Since  $ \{1,-1\} \simeq \Z/2\Z$ is abelian,
\begin{align*}
\varphi(\tau_2)  &= \varphi(\gamma) \varphi(\tau_1) \varphi(\gamma)^{-1}\\
&=  \varphi(\gamma) \varphi(\gamma)^{-1} \varphi(\tau_1) \\
&= \varphi(\tau_1)
\end{align*}
So $\tau_1, \tau_2 \in H$, or  $\tau_1, \tau_2 \in S_n  \setminus H$.

If $\tau_1, \tau_2$ are in $S_n\setminus H$, then $\varphi(\tau_1\tau_2) = \varphi(\tau_1)\varphi(\tau_2) = (-1)\times (-1) = 1$, so $\tau_1 \tau_2 \in H$. In both cases 
$\tau_1 \tau_2 \in H$. 

Sine every permutation $\sigma$ of $A_n$ is the product of an even number of transpositions, $\sigma \in H$, so $A_n \subset H$. As $|A_n| = |H | = n!/2$, $H = A_n$.

$A_n$ is the only subgroup of $S_n$ with $n!/2$ elements.
\end{proof}

\paragraph{Ex. 13.1.3}

{\it Explain carefully why (13.6) follows from Exercise 9 of section 2.4.
}

\begin{proof}
By definition, 
$$y_1 = x_1x_2+x_3x_4, \qquad y_2 = x_1x_3+x_2x_4,\qquad y_3 = x_1x_4+x_2x_3.$$
By Exercise 2.4.9, we know that
$$\Delta(\theta) = (y_1-y_2)^2(y_1-y_3)^2(y_2-y_3)^2
=[(x_1-x_4)(x_2-x_3)(x_1-x_3)(x_2-x_4)(x_1-x_2)(x_3-x_4)]^2
=\Delta
$$
As the evaluation is a ring homomorphism, if we applied to this equality in $F[x_1,x_2,x_3,x_4]$ the evaluation defined by $x_1 \mapsto \alpha_1,\ldots,x_4\mapsto \alpha_4$, we obtain
that the roots
$$\beta_1 = \alpha_1\alpha_2+\alpha_3\alpha_4, \qquad \beta_2 = \alpha_1\alpha_3+\alpha_2\alpha_4,\qquad \beta_3 = \alpha_1\alpha_4+\alpha_2\alpha_3,$$
are the images of $Y_1,y_2,y_3$ and satisfy
\begin{align*}
\Delta(\theta_f) &= (\beta_1-\beta_2)^2(\beta_1-\beta_3)^2(\beta_2-\beta_3)^2\\
&=[(\alpha_1-\alpha_4)(\alpha_2-\alpha_3)(\alpha_1-\alpha_3)(\alpha_2-\alpha_4)(\alpha_1-\alpha_2)(\alpha_3-\alpha_4)]^2\\
&=\Delta(f)
\end{align*}
\end{proof}

\paragraph{Ex. 13.1.4}

{\it Use Example 7.3.4 from Chapter 7 to show that (13.8) gives all subgroups of $\langle (1\,3\,2\,4), (1\,2)\rangle$ of order 4 or 8.
}

\begin{proof}
We obtain all subgroups of $D_8  \simeq \langle \sigma,\tau \rangle$, where $\sigma =  (1\,3\,2\,4), \tau = (1\,2)$, in Exercise 7.3.3

\begin{center}
\begin{tikzpicture}
    \node (n0) at (7,6) {$\{e\}$};
    \node (n1) at (1,4) {$\langle \tau \rangle$};
    \node (n2) at (4,4) {$\langle  \sigma^2 \tau \rangle$};
    \node (n3) at (7,4) {$\langle\sigma^2\rangle$};
    \node (n4) at (10,4) {$\langle  \sigma \tau\rangle$};
    \node (n5) at (13,4) {$\langle \sigma^3 \tau\rangle$};
    \node (n6) at (4,2) {$\langle \sigma^2, \tau \rangle$};
    \node (n7) at (7,2) {$\langle \sigma \rangle$};
    \node (n8) at (10,2) {$\langle \sigma^2, \sigma  \tau  \rangle$};
    \node (n9) at (7,0) {$G = \langle \sigma,\tau\rangle$};
    \draw[->] (n0) edge (n3) edge (n2) edge (n1) edge (n4) edge (n5);
    \draw[<-] (n6) edge (n1) edge (n2) edge (n3);
    \draw[<-] (n8) edge (n3) edge (n4) edge (n5);
    \draw[->] (n3) edge (n7);
    \draw[<-] (n9) edge (n6) edge (n7) edge (n8);
\end{tikzpicture}
\end{center}

If $G$ is a subgroup of order 4 or 8, then $G$ is one of the four groups
$$\langle \sigma^2, \tau \rangle, \quad \langle \sigma \rangle,\quad \langle \sigma^2, \sigma  \tau  \rangle, \quad \langle \sigma,\tau\rangle,$$
Moreover $\sigma^2 = (1\,2)(3\,4)$ and $\sigma \tau = (1\,4)(2\,3)$, so 
$$\langle \sigma^2, \tau \rangle = \langle (1\,2)(3\,4), (1\,2) \rangle =  \langle (3\,4), (1\,2) \rangle,$$
and
$$\langle \sigma^2, \sigma  \tau  \rangle = \langle (1\,2)(3\,4),  (1\,4)(2\,3) \rangle =  \langle (1\,2)(3\,4),  (1\,3)(2\,4) \rangle$$
is the group of double transpositions $\{(), (1\,2)(3\,4), (1\,4)(2\,3), (1\,3)(2\,4)\}$.

Therefore $G$ is one of the four groups given in the text
$$\langle (1\,2),(3\,4) \rangle,\qquad  \langle (1\,2)(3\,4),  (1\,3)(2\,4) \rangle,\qquad  \langle (1\,3\,2\,4) \rangle, \qquad \langle (1\,3\,2\,4), (1\,2)  \rangle.$$
\end{proof}

\paragraph{Ex. 13.1.5}

{\it Let $F$ be a field of characteristic $\ne 2$, and let $g \in F[x]$ be a monic cubic polynomial that has a root in $F$. Prove that $g$ splits completely over $F$ if and only if $\Delta(g) \in F^2$.
}

\begin{proof}
Let $g = (x-\alpha_1)(x-\alpha_2)(x-\alpha_3)$, where $\alpha_1,\alpha_2,\alpha_3$ lie in some splitting field of $F$, and $\alpha_1 \in F$.
\be
\item[$\bullet$] if $g$ splits completely over $F$, then $\alpha_1,\alpha_2,\alpha_3$ lie in $F$, therefore 

${\delta = (\alpha_1 - \alpha_2)(\alpha_1-\alpha_3)(\alpha_2-\alpha_3) \in F}$, so
$\Delta = \delta^2 \in F^2$.
\item[$\bullet$] Conversely, if $\Delta \in F^2$,then $\Delta = a^2, \ a \in F $, so $\delta =\pm a\in F$.
Since $\alpha_1 \in F$, the Euclidean division of $g(x)$ by $x-\alpha_1 \in F[x]$ gives 
$$g(x) = (x-\alpha_1)(x^2+px+q), \qquad p,q \in F.$$
$\alpha_2+\alpha_3 = -p \in F, \alpha_2\alpha_3 = q \in F$, so
$$(\alpha_1-\alpha_2)(\alpha_1 - \alpha_3) = \alpha_1^2  + p \alpha +q \in F.$$
If $\alpha_1 = \alpha_2$ or $\alpha_1 = \alpha_3$, then $\Delta(f) = 0 \in F^2$. 

In the other case, $(\alpha_1-\alpha_2)(\alpha_1 - \alpha_3) \ne 0$, so
$$\alpha_2 - \alpha_3 = \delta [(\alpha_1-\alpha_2)(\alpha_1 - \alpha_3)]^{-1} \in F.$$
Since $\alpha_2 + \alpha_3 \in F$, and $\alpha_2 - \alpha_3 \in F$, and since the characteristic of $F$ is not 2,
$$\alpha_2 = \frac{1}{2}[(\alpha_2 + \alpha_3) +(\alpha_2-\alpha_3)] \in F, \alpha_3 = \frac{1}{2}[(\alpha_2 + \alpha_3) -(\alpha_2-\alpha_3)] \in F.$$
Therefore $g = (x-\alpha_1)(x-\alpha_2)(x-\alpha_3)$ splits completely over $F$.
\ee
\end{proof}

\paragraph{Ex. 13.1.6}

{\it This exercise is concerned with the proof of part (c) of Theorem 13.1.1. Let $f(x) = x^4-c_1x^3+c_2x^2-c_3x+c_4$ as in the theorem.
\be
\item[(a)] Suppose that $f$ has roots $\alpha_1,\alpha_2,\alpha_3,\alpha_4$ such that $\alpha_1+\alpha_2 - \alpha_3-\alpha_4 = \alpha_1\alpha_2 - \alpha_3 \alpha_4 = 0$. Prove that $f$ is not separable.
\item[(b)] Let $\beta$ be a root of the resolvent $\theta_f(y)$. Use part (a) to prove that $4\beta+c_1^2-4c_2$ and $\beta^2 - 4 c_4$ can't both vanish when $f$ is separable.
\item[(c)] Suppose that $4\beta+c_1^2-4c_2^2=0$ in part (c) of Theorem 13.1.1. Prove carefully that $G$ is conjugate to $\langle (1\,3\,2\,4),(1\,2) \rangle$ if and only if $\Delta(f)(\beta^2 - 4 c_4) \not \in (F^*)^2$.
\ee
}

\begin{proof}
\be
\item[(a)] If  $\alpha_1+\alpha_2 - \alpha_3-\alpha_4 = \alpha_1\alpha_2 - \alpha_3 \alpha_4 = 0$, then 
\begin{align*}
s &:= \alpha_1+\alpha_2 = \alpha_3+\alpha_4\\
p &:= \alpha_1\alpha_2 = \alpha_3\alpha_4
\end{align*}
Thus $x^2-sx+p = (x-\alpha_1)(x - \alpha_2) = (x-\alpha_3)(x-\alpha_4)$, therefore
$$\{\alpha_1,\alpha_2\} = \{\alpha_3,\alpha_4.$$
Since $\alpha_3 = \alpha_1$ or $\alpha_3 = \alpha_2$, $f$ is not separable.

\item[(b)] If $\beta$ is a root of the resolvent $\theta_f$, we can relabel the roots of $f$ so that $\beta = \alpha_1\alpha_2 + \alpha_3\alpha_4$ and
$$4 \beta + c_1^2-4c_2 = (\alpha_1+\alpha_2 - \alpha_3 - \alpha_4)^2.$$
Since $\beta^2 - 4c_4 = (\alpha_1\alpha_2 - \alpha_3\alpha_4)^2$, if $4\beta+c_1^2-4c_2$ and $\beta^2 - 4 c_4$ both vanish, then $\alpha_1+\alpha_2 - \alpha_2 - \alpha_4 =0$ and $\alpha_1\alpha_2 - \alpha_3\alpha_4=0$. Then by part (a) $f$ is not separable.

Therefore $4\beta+c_1^2-4c_2$ and $\beta^2 - 4 c_4$ can't both vanish when $f$ is separable.

\item[(c)] Suppose that $4\beta + c_1^2-4c_2^2 = 0$ in part (c) of Theorem 13.1.1, where $\theta_f(y)$ has a unique root $\beta$ in $F$. Therefore $\theta_f(y) = (y-\beta)(y-\beta')(y-\beta'')$, where $\beta' \not \in F, \beta'' \not \in F$.

If $\theta_f$ was not separable, then $\beta' = \beta''$, and $\theta_f(t) = (y-\beta)(y-\beta')^2 \in F[y], \beta \in F$, thus $(y-\beta'^2) = y^2 - 2 \beta' y + \beta'^2 \in F[y]$, which implies that 
$2\beta' \in F$. 

Since the characteristic of $F$ is not 2, $\beta' \in F$. This is a contradiction, so $\theta_f$ is separable. Since the discriminant of $\theta_f$ and $f$ are equal, $f$ is separable.

Then by part (b), $\beta^2 - 4 c_4 \ne 0$, and since $f$ is separable, $\Delta(f) \ne 0$, so
$$\Delta(f) (\beta^2 - 4 c_4) \ne 0.$$
We know that $G = \langle (1\,3\,2\,4)\rangle$ or $G = \langle (1\,3\,2\,4),(1,2)\rangle$.
\be
\item[$\bullet$] Suppose that $G = \langle (1\,3\,2\,4)\rangle$. Then $\Gal(L/F) = \langle \sigma\rangle$, where $\sigma$ corresponds to $(1\,3\,2\,4)$.
We choose
 $$\sqrt{\Delta(f) (\beta^2 - 4 c_4)} = \sqrt{\Delta(f)} (\alpha_1\alpha_2 - \alpha_3  \alpha_4).$$
Since $(1\,3\,2\,4) = (1\,3)(3\,2)(2\,4) \not \in A_4$, $\sigma ( \sqrt{\Delta(f)} )= - \sqrt{\Delta(f)}$, and
$$\sigma (\alpha_1\alpha_2 - \alpha_3  \alpha_4) = \alpha_3  \alpha_4 - \alpha_2 \alpha_1 = - (\alpha_1\alpha_2 - \alpha_3  \alpha_4).$$
Therefore $\sigma$ fixes $\sqrt{\Delta(f) (\beta^2 - 4 c_4)}$, so $\sqrt{\Delta(f) (\beta^2 - 4 c_4)} \in F^*$, and $$\Delta(f) (\beta^2 - 4 c_4) \in (F^*)^2.$$
\item[$\bullet$] Suppose that  $G = \langle (1\,3\,2\,4),(1,2)\rangle$. Then $\Gal(L/F) = \langle \sigma, \tau \rangle$, where $\tau$ corresponds to $(1\,2)$.
$\tau ( \sqrt{\Delta(f)} )= - \sqrt{\Delta(f)}$ and $\tau (\alpha_1\alpha_2 - \alpha_3  \alpha_4) = \alpha_2\alpha_1 - \alpha_3  \alpha_4 = \alpha_1\alpha_2 - \alpha_3  \alpha_4$, so
$\tau(\sqrt{\Delta(f) (\beta^2 - 4 c_4)}) = -\sqrt{\Delta(f) (\beta^2 - 4 c_4)}$. Since the characteristic is not 2, $\sqrt{\Delta(f) (\beta^2 - 4 c_4)} \not \in F$, so 
$$\Delta(f) (\beta^2 - 4 c_4) \in (F^*)^2.$$
Therefore $G$ is conjugate to $\langle (1\,3\,2\,4),(1\,2) \rangle$ if and only if $\Delta(f)(\beta^2 - 4 c_4) \not \in (F^*)^2$.
\ee
\ee
\end{proof}

\paragraph{Ex. 13.1.7}

{\it In Exercise 18 of section 12.1 you found the roots of $f = x^4+2x^2-4x+2 \in \Q[x]$ using the formula developed in that section. At the end of the exercise, we said that "this quartic is especially simple". Justify this assertion using Theorem 13.1.1
}

\begin{proof}
By Exercise 12.1.18,
$$\theta_f(y) = y^3 - 2y^2-8y = y(y-4)(y+2).$$
Since $\theta_f(y)$ splits completely over $F$, by Theorem 13.1.1,
$$G = \langle(1\,2)(3\,4),(1\,3)(2\,4) \rangle \simeq \Z/2\Z \times \Z/2\Z.$$

(This result was already proved in Exercise 12.1.18, since the splitting field of $f$ is $\Q(i,\sqrt{2})$.)
\end{proof}

\paragraph{Ex. 13.1.8}

{\it In Example 10.3.10, we showed that the roots of $f=7m^4 - 16 m^3 -21 m^2 + 8m+4 \in \Q[m]$ can be constructed using origami. Show that the splitting field of $f$ is an extension of $\Q$ of degree 24. By the results of Section 10.1, it follows that the roots of $f$ are not constructible with straightedge and compass, since $24$ is not a power of $2$.
}

\begin{proof}
The discriminant of $g = \frac{1}{7} f$  is 
$$\Delta(g) = \frac{174446784}{117649} = 2^6\cdot 3^6\cdot 3739\cdot 7^{-6},$$
so $\Delta(g)$ is not a square in $\Q$.

The Ferrari resolvent is
$$\theta_f(y) = y^{3} + 3 y^{2} - \frac{240}{49} y - \frac{3824}{343}.$$

and $$7^3 \theta_f(y) = 343 y^{3} + 1029 y^{2} - 1680 y - 3824$$
has no root in $\Q$, so is irreducible over $\Q$.

By theorem 13.1.1, $G = S_4$.
Therefore the splitting field $L$ of $f$ has degree
$$[L:\Q] = |G| = 24.$$

\bigskip

Sage instructions :
\begin{verbatim}
var('m')
R.<m> = QQ[m]
f= 7*m^4-16*m^3-21*m^2+8*m+4
g=f/7
d=g.discriminant()
d.factor()
\end{verbatim}
$$2^{6} \cdot 3^{6} \cdot 7^{-6} \cdot 3739$$
\begin{verbatim}
l = f.coefficients(sparse=False);
c1 = -l[3]/l[4]; c2 = l[2]/l[4];c3 = -l[1]/l[4]; c4 = l[0]/l[4];
theta_f = y^3 -c2*y^2 +(c1*c3-4*c4)*y - c3^2-c1^2*c4 + 4*c2*c4;
\end{verbatim}
$$y^{3} + 3 y^{2} - \frac{240}{49} y - \frac{3824}{343}$$
\begin{verbatim}
theta_f.is_irreducible()
\end{verbatim}
$$\text{True}$$
\end{proof}

\paragraph{Ex. 13.1.9}

{\it As in Example 13.1.3, let $f = x^4+ax^3+bx^2+ax+1 \in F[x]$, and let $\alpha$ be a root of $f$ in some splitting field of $f$ over $F$. Show that $\alpha^{-1}$ is also a root of $f$, and then use (13.5) to conclude that 2 is a root of the resolvent $\theta_f(y)$.
}

\begin{proof}
If $\alpha$ is a root of $f$ in some splitting field $L$ of $F$, then $ \alpha^4+a\alpha^3+b\alpha^2+a\alpha+1 = 0$. If we divide by $\alpha^4$, we obtain
$1 + a \alpha^{-1} + b \alpha^{-2} + a\alpha^{-3} + \alpha^{-4}$, so $f(\alpha^{-1})=0$.
Note that
\begin{align*}
x^4+ax^3+bx^2+ax+1 &= x^2\left[ \left(x^2+\frac{1}{x^2} \right) + a \left( x + \frac{1}{x} \right) + b \right]\\
&=x^2\left[ \left(x+\frac{1}{x} \right)^2 + a \left( x + \frac{1}{x} \right) + b -2 \right]\\
\end{align*}
As $0$ is not a root of $f$, the roots of $f$ are the roots of  $z = x+\frac{1}{x}$, where $z$ is a root of $z^2+az+b-2$, so the roots of $f$ are the roots of the two polynomials
$$x^2-z_1x+1,\qquad x^2 -z_2x+1,$$
where $z_1,z_2$ are the roots in $L$ of
$$z^2+az+b-2.$$
If we relabel the roots so that $\alpha_1,\alpha_2$ are the roots of $x^2-z_1x+1$, and $\alpha_3,\alpha_4$ the roots of $x^2 -z_2x+1$, then $\alpha_1\alpha_2 = 1, \alpha_3\alpha_4 = 1$
therefore $\beta_1 = \alpha_1\alpha_2+ \alpha_3\alpha_4 = 2$ is a root of the Ferrari resolvent $\theta_f(y)$.
\end{proof}

\paragraph{Ex. 13.1.10}

{\it As in Example 13.1.4, let $f=x^4+bx^2+d \in F[x]$, where $d \not \in F^2$. Compute $\Delta(f)$ and $\theta_f(y)$.
}

\begin{proof}
The discriminant of $f$ is
$$\Delta(f) =  16 b^{4} d - 128 b^{2} d^{2} + 256 d^{3} = 16\  d\,  ( b^{2} - 4 d)^{2}.$$
The Ferrari resolvent is
$$\theta_f(y) = y^3-by^2-4dy+4bd = (y-b)(y^2-4d).$$

Sage instructions:
\begin{verbatim}
R.<x,b,d> = QQ[]
f=x^4+b*x^2+d
c1 = 0; c2 = b; c3 = 0; c4 = d;
theta_f = x^3 - c2*x^2 + (c1*c3-4*c4)*x - c3^2-c1^2*c4 + 4*c2*c4;
factor(theta_f)
\end{verbatim}
$$(- x + b) \cdot (- x^{2} + 4 d)$$
\begin{verbatim}
Delta = theta_f.discriminant(x)
factor(Delta)
\end{verbatim}
$$	
\left(16\right) \cdot d \cdot (- b^{2} + 4 d)^{2}
$$
Thus $\theta_f(y) = (y-b)(y - 2\sqrt{d})(y+2\sqrt{d})$ has a unique root if $F$ if $d \not \in F^2$, and the discriminant is not a square in $F^2$.
\end{proof}

{\it In Example 13.1.7 we showed that if $f=x^4+ax^3+bx^2+ax+1 \in \Z[x]$ is irreducible over $\Q$, then its Galois group is $\Z/2\Z \times \Z/2\Z$ if and only if there is $c \in \Q$ such that $4a^2+c^2 = (b+2)^2$.
\be
\item[(a)] Show that $c\in \Z$, and use the irreducibility of $f$ to prove that $c\ne 0$. Hence we may assume that $c>0$, so that $(2a,c,b+2)$ is a Pythagorean triple.
\item[(b)] Show that $3^2+4^2 = 5^2, 5^2+12^2 = 13^2, 7^2 + 24^2 = 25^2$, and $8^2 + 15^2 = 17^2$ give two examples of polynomials with $\Z/2\Z \times \Z/2\Z$ as Galois group (two of the triples give reducible polynomials).
\ee
}

\begin{proof}
\be
\item[(a)]  $c \in \Q$ is such that $c^2 = n \in \Z$. Write $c =a/b, b>0, a\wedge b = 1$. Then $a^2 = n b^2$. If $b\ne 1$, there is a prime $p$ such that $p\mid b$. But then $p\mid a^2$, thus $p\mid a$, in contradiction with $a\wedge b = 1$. So $c \in \Z$.

If $c = 0$, then $(b+2)^2 = 4a^2$, so $b+2 =   2\varepsilon a$, $ b = -2+ 2 \varepsilon a$ , where $\varepsilon = \pm 1$.

In Exercise 9, we saw that
$$f=  x^4+ax^3+bx^2+ax+1 = (x^2-z_1x+1)(x^2 - z_2x+ 1),$$
where $z_1,z_2$  are the roots of $ z^2+az+b-2.$
Here $b = -2+2\varepsilon a$, so $z_1,z_2$ are the roots of $$z^2 +az -4+2\varepsilon a = (z+a-2\varepsilon)(z+2\varepsilon),$$
so
$$z_1 = -a+2\varepsilon \in \Z,\qquad  z_2 = -2 \varepsilon \in \Z,$$
so $f$ is not irreducible over $\Q$, in contradiction with the hypothesis.
We have proved that $c\ne 0$ if $f$ is irreducible, and so $(2a,c,b+2)$ is a Pythagorean triple.

\item[(b)] 
$3^2+4^2 = 5^2$ gives $a=2, b =3$, and $f = x^4+2x^3+3x^2+2x+1 = (x^2+x+1)^2$ is not irreducible.

$5^2+12^2 = 13^2$ gives $a=6, b = 11$, and $f =  x^4+6x^3+11x^2+6x+1 = (x^2+3x+1)^2$ is not irreducible.

$7^2 + 24^2 = 25^2$ gives $a=12,b=23$, and $f= x^4 + 12x^3+23x^2+12x+1$ which is irreducible. So the Galois group of 
$$f= x^4 + 12x^3+23x^2+12x+1$$
is isomorphic to $\Z/2\Z \times \Z/2\Z$.

Verification with Sage:
\begin{verbatim}
R.<x> = QQ[]
f= x^4 + 12*x^3 + 23*x^2 + 12*x + 1
f.is_irreducible()
\end{verbatim}
\begin{center}
True
\end{center}
\begin{verbatim}
G = f.galois_group()
G.gens()
\end{verbatim}
$$ [(1,2)(3,4), (1,4)(2,3)]$$
\begin{verbatim}
G.structure_description()
\end{verbatim}
$$C2 \times C2$$
\ee
\end{proof}

$8^2 + 15^2 = 17^2$ gives $a=4, b = 15$, and $f = x^4 + 4 x^3 + 15 x^2 + 4x+1$, which is irreducible.
The Galois group of 
$$f = x^4 + 4 x^3 + 15 x^2 + 4x+1$$
is isomorphic to $\Z/2\Z \times \Z/2\Z$.

Note: the polynomial associate to $7^2 + 24^2 = 25^2$ is 
\begin{align*}
f&= x^4 + 12x^3+23x^2+12x+1\\
&= (x^2+6x+1)^2 - 15x^2\\
&= (x^2+(6+\sqrt{15}) x +1) (x^2 + (6-\sqrt{15}) x +1)
\end{align*}
The discriminant of the first factor is $\Delta_1 = 47 + 12 \sqrt{15}$ and the discriminant of the second is $\Delta_2 = 47 - 12 \sqrt{15}$.
Since
$$\left(\sqrt{47 + 12 \sqrt{15}}\right)\left(\sqrt{47 -  12 \sqrt{15}}\right) = \sqrt{47^2 - 144 \times 15} = \sqrt{49} = 7 \in \Q^*),$$
the splitting field of $f$ over $\Q$ is $\Q\left (\sqrt{47 + 12 \sqrt{15}}\right)$, which is a quadratic extension of a quadratic extension. The minimal polynomial of $a = \sqrt{47 + 12 \sqrt{15}}$ is $x^4-94x^2+49$, whose Galois group is also $\Z/2\Z \times \Z/2\Z$ (here $d=49$ is a square).

\paragraph{Ex. 13.1.12}

{\it This exercise is concerned with the proof of Proposition 13.1.5.
\be
\item[(a)] Prove (13.12).
\item[(b)] Prove that the two polynomials $h_1$ and $h_2$ defined in the proof of the proposition factor as $h_1 = (y-(\alpha_1+\alpha_2))(y-(\alpha_3+\alpha_4))$ and $h_2 = (y - \alpha_1\alpha_2)(y-\alpha_3\alpha_4)$.
\ee
}

\begin{proof}
\be
\item[(a)] Let $g = y^2+Ay+B \in F[y]$ and let $F \subset F(\sqrt{a}),\ a \in F$ be a quadratic extension.

If $\Delta(g) = 0$ then $a\Delta(g) = 0 \in F^2$. Suppose now that $g$ is irreducible over $F$.
\be
\item[$\bullet$] Suppose that $g$ splits completely over $F(\sqrt{a})$, so
$$g = (y-y_1)(y-y_2),\qquad y_1,y_2\in F(\sqrt{a}).$$
Then $\Delta(g) = (y_1-y_2)^2 = A^2 - 4B \in F$. We choose $\sqrt{\Delta(g)} = y_1-y_2 \in F(\sqrt{a})$.
 As $\deg(g) =2$, $g$ is irreducible over $f$, therefore the roots of $g$
\begin{align*}
y_1 &= \frac{1}{2}((y_1+y_2) + (y_1-y_2)) = \frac{1}{2} \left(-A - \sqrt{\Delta(g)}\right),\\
 y_2 &= \frac{1}{2}((y_1+y_2) - (y_1-y_2 ))= \frac{1}{2} \left (-A + \sqrt{\Delta(g)}\right)
 \end{align*}
are not in $F$, which is equivalent to
$$\sqrt{\Delta(g)} \not \in F.$$
Since $ \sqrt{\Delta(g)} \in F(\sqrt{a})$, and $\sqrt{\Delta(g)} \not \in F$,
$$\sqrt{\Delta(g)} = u + v\sqrt{a},\qquad u,v \in F,\qquad v\ne 0.$$
Therefore 
\begin{align*}
u^2 &= \left(\sqrt{\Delta(g)}  - v\sqrt{a}\right)^2\\
&=\Delta(g) + a v^2 - 2v\sqrt{a}\sqrt{\Delta(g)} 
\end{align*}
Since $v\ne 0$, and $\mathrm{char}(F) \ne 2$,
$$\sqrt{a} \sqrt{\Delta(g)}  = \frac{\Delta(g) + a v^2-u^2}{2v} \in F,$$
so 
$$a\Delta(g) \in F^2.$$
\item[$\bullet$] Conversely, suppose that $a\Delta(g) \in F^2$. Here $a\ne 0$ since $F(\sqrt{a})$ is a quadratic extension of $F$.
There exists $w \in F$ such that $a\Delta(g) =w^2$. 

We choose $\sqrt{\Delta(g)}$ such that
  $$\sqrt{\Delta(g)} =  \frac{w}{\sqrt{a}}  = \frac{w}{a} \sqrt{a} \in F(\sqrt{a}).$$
  Then 
  \begin{align*}
y_1 &= \frac{1}{2}((y_1+y_2) + (y_1-y_2)) = \frac{1}{2} \left(-A - \sqrt{\Delta(g)}\right),\\
 y_2 &= \frac{1}{2}((y_1+y_2) - (y_1-y_2 ))= \frac{1}{2} \left (-A + \sqrt{\Delta(g)}\right)
 \end{align*}
 are in $F(\sqrt{a})$, so $g = (y-y_1)(y-y_2)$ splits completely over $F(\sqrt{a})$.
 
 Finally, if $\Delta(g) = 0$, $g = (y-y_0)^2$, where $y_0 = -A/2 \in F$, splits completely over $F$, a fortiori over $F(\sqrt{a})$.
\ee
Conclusion: 

Let $g = y^2 + Ay+B$ and $F(\sqrt{a})$ a quadratic extension of $F$, with $\mathrm{char}(F) \ne 2$. 
If $\Delta(g)=0$, or if $g$ is irreducible over $F$, then
$$ g \text{ splits completely over } F(\sqrt{a}) \iff a\Delta(g) \in F^2.$$
\item[(b)]
\begin{align*}
&(y-(\alpha_1+\alpha_2))(y-(\alpha_3+\alpha_4))\\
&= y^2 -(\alpha_1+\alpha_2+\alpha_3+\alpha_4)y + (\alpha_1\alpha_3+\alpha_1\alpha_4+\alpha_2\alpha_3+\alpha_2\alpha_4)\\
&= y^2 - c_1y + (\alpha_1\alpha_2 + \alpha_1\alpha_3 + \alpha_1 \alpha_4 + \alpha_2 \alpha_3+\alpha_2\alpha_4+\alpha_3\alpha_4) - (\alpha_1\alpha_2+\alpha_3\alpha_4)\\
&= y^2 -c_1 y +c_2 - \beta
\end{align*}
so
$$h_1 =  y^2 -c_1 y +c_2 - \beta = (y-(\alpha_1+\alpha_2))(y-(\alpha_3+\alpha_4)).$$
Similarly
\begin{align*}
&(y-\alpha_1\alpha_2)(y-\alpha_3\alpha_4)\\
&=y^2 - (\alpha_1\alpha_2+\alpha_3\alpha_4) y +\alpha_1\alpha_2\alpha_3\alpha_4\\
&=y^2 - \beta y +c_4
\end{align*}
so
$$h_2 = y^2 - \beta y +c_4 = (y-\alpha_1\alpha_2)(y-\alpha_3\alpha_4).$$
\ee
\end{proof}

\paragraph{Ex. 13.1.13}

{\it Suppose that $f \in F[x]$ satisfies the hypothesis of part (c) of Theorem 13.1.1, and let $\alpha$ be a root of $f$. Prove that $G \simeq \Z/4\Z$ if $f$ splits completely over $F(\alpha)$, and $G\simeq D_8$ otherwise. This gives a version of part (c) that doesn't use resolvents. Since we can factor over extension fields by Section 4.2, this method is useful in practice.
}

\begin{proof}
With the hypothesis of part (c), $\Delta(f) \not \in F^2$, so $\Delta(f) \ne 0$ and $f$ is separable.

\be
 \item[$\bullet$] If $G \simeq \Z/4\Z$, then 
$G = \langle \sigma \rangle \subset S_4$,  where $\sigma$ corresponds to $\tilde{\sigma} \in \Gal(L/F)$. Write $G_\alpha = \mathrm{Stab}_G(\alpha)$. Since $f$ is irreducible,  $4 = |{\cal O}_\alpha| = (G: G_\alpha)$, so $G_{\alpha} = \{e\}$. Therefore $\tilde{\sigma}^i \ne\tilde{\sigma}^j$ if $1\leq i < j \leq 4$.   So  $\tilde{\sigma}(\alpha_1) = \alpha_3,\tilde{\sigma}(\alpha_3) = \alpha_2,\tilde{\sigma}(\alpha_2) = \alpha_4$ are the four distinct roots of $f$, and $\sigma = (1\,3\,2\,4)$.
$$f = (x-\alpha_1)(x-\alpha_3)(x-\alpha_2)(x-\alpha_4) = (x-\alpha)(x-\tilde{\sigma}(\alpha))(x-\tilde{\sigma}^2(\alpha))(x-\tilde{\sigma}^3(\alpha)).$$

As  $\Delta(f) \not \in F^2$,  $F(\sqrt{\Delta})$ is a quadratic extension of $F$.

Since the only subgroup of $G$ are $\{e\} \subset H =\langle \sigma^2 \rangle \subset G = \langle \sigma \rangle$, by the Galois correspondence, the only intermediate fields of $F\subset L$ are $F \subset F(\sqrt{\Delta})\subset L$, and the fixed field of $H = \langle \sigma^2 \rangle$ is $L_H =  F(\sqrt{\Delta})$.

If $F(\alpha) \subset F(\sqrt{\Delta})$, then $\alpha \in F(\sqrt{\Delta}) = L_H$, therefore $\sigma^2(\alpha) = \alpha$, and so $\alpha_2 = \alpha_1$, in contradiction with the separability of $f$. Hence $F(\alpha) \not \subset F(\sqrt{\Delta})$, so $$F(\alpha) = L = F(\alpha_1,\alpha_2,\alpha_3,\alpha_4).$$

Then $f$ splits completely over $F(\alpha)$.


 \item[$\bullet$] If $G \not \simeq \Z/4\Z$, then by Theorem 13.1.1, $G \simeq D_8$. Therefore $[L : F] = |G| = 8$, and $[F(\alpha) : F] = \deg(f) = 4$, which implies $F(\alpha) \ne L = F(\alpha_1,\alpha_2,\alpha_3,\alpha_4)$.
 Therefore one on the root $\alpha_i$ is not in $F(\alpha)$, and so $f =(x-\alpha_1)(x-\alpha_2)(x-\alpha_3)(x-\alpha_4)$ doesn't splits completely over $F(\alpha)$.
\ee



Conclusion. Let $f$ be a quadratic polynomial, and let $\alpha$ be a root of $f$.

 If $\Delta(f) \not \in F^2$ and $\theta_f(y)$ is reducible over $F$, then
\begin{align*}
f \text { splits completely over } F(\alpha) &\iff \Gal_F(f) \simeq \Z/4\Z,\\
f \text { doesn't split completely over } F(\alpha) &\iff \Gal_F(f) \simeq D_8.
\end{align*}
\end{proof}

Example 1: $f = x^4-12x^2+18$ over $\Q$.
\begin{verbatim}
R.<x> = QQ[]
f = x^4-12*x^2 + 18
print(f.is_irreducible())
factor(f.discriminant()), f.discriminant().is_square()
\end{verbatim}
$$\text{True}$$
$$(2^{11} \cdot 3^6, \text{False}).$$
\begin{verbatim}
l = f.coefficients(sparse=False);
c1 = -l[3]/l[4]; c2 = l[2]/l[4];c3 = -l[1]/l[4]; c4 = l[0]/l[4];
S.<y> = QQ[]
theta_f = y^3 -c2*y^2 +(c1*c3-4*c4)*y - c3^2-c1^2*c4 + 4*c2*c4;
factor(theta_f)
\end{verbatim}
$$(y + 12) \cdot (y^{2} - 72)$$
\begin{verbatim}
K.<a>= NumberField(f)
S.<x> = K[]
f = x^4-12*x^2 + 18
factor(f)
\end{verbatim}
$$(x - a) \cdot (x + a) \cdot (x - \frac{1}{3} a^{3} + 3 a) \cdot (x + \frac{1}{3} a^{3} - 3 a)$$

These results prove that the Galois group of  $f = x^4-12x^2+18$ over $\Q$ is isomorphic to $\Z/4\Z$.

\bigskip

Example 2: $f = x^4 - 2$ over $\Q$.
\begin{verbatim}
R.<x> = QQ[]
f = x^4-2
print(f.is_irreducible())
factor(f.discriminant()), f.discriminant().is_square()
\end{verbatim}
$$\text{True}$$
$$(-1\cdot 2^{11}, \text{False})$$
\begin{verbatim}
l = f.coefficients(sparse=False);
c1 = -l[3]/l[4]; c2 = l[2]/l[4];c3 = -l[1]/l[4]; c4 = l[0]/l[4];
S.<y> = QQ[]
theta_f = y^3 -c2*y^2 +(c1*c3-4*c4)*y - c3^2-c1^2*c4 + 4*c2*c4;
factor(theta_f)
\end{verbatim}
$$y \cdot (y^{2} + 8)$$
\begin{verbatim}
K.<a>= NumberField(f)
S.<x> = K[]
f = x^4-2
factor(f)
\end{verbatim}
$$(x - a) \cdot (x + a) \cdot (x^{2} + a^{2})$$
Thus the Galois group of $x^4 - 2$ over $\Q$ is $D_8$.

\bigskip

Example 3: $f = x^4-18x^2+9$ over $\Q$.
\begin{verbatim}
R.<x> = QQ[]
f = x^4-18*x^2 + 9
print(f.is_irreducible())
factor(f.discriminant()), f.discriminant().is_square()
\end{verbatim}
$$\text{True}$$
$$(2^{14}\cdot 3^6, \text{True})$$
\begin{verbatim}
l = f.coefficients(sparse=False);
c1 = -l[3]/l[4]; c2 = l[2]/l[4];c3 = -l[1]/l[4]; c4 = l[0]/l[4];
S.<y> = QQ[]
theta_f = y^3 -c2*y^2 +(c1*c3-4*c4)*y - c3^2-c1^2*c4 + 4*c2*c4;
factor(theta_f)
\end{verbatim}
$$	
(y - 6) \cdot (y + 6) \cdot (y + 18)
$$
The Galois group of $f = x^4-18x^2+9$ over $\Q$ is isomorphic to $\Z/2\Z \times \Z/2\Z$.

\paragraph{Ex. 13.1.14}

{\it Use Theorem 13.1.1 to compute the Galois groups of the following polynomials in $\Q[x]$:
\be
\item[(a)] $x^4+4x+2$.
\item[(b)] $x^4 + 8x+12$.
\item[(c)] $x^4+1$.
\item[(d)] $x^4+x^3+x^2 + x+1$.
\item[(e)] $ x^4-2$.
\ee
}

\begin{proof}
\be
\item[(a)] $f = x^4+4x+2$.

$\Delta(f) = -2^8\cdot 19$ is  not a square in $\Q$, and $\theta_f(y) = y^3 -8y - 16$ is irreducible over $\Q$, so $\Gal_\Q(f) \simeq S_4$ (part (a) of Theorem 13.1.11).


\item[(b)] $f = x^4 + 8x+12$.

$\Delta(f) = 2^{12}\cdot 3^4$ is  a square in $\Q$, and $\theta_f(y) = y^3 -48y - 64$ is irreducible over $\Q$, so $\Gal_\Q(f) \simeq S_4$ (part (a) of Theorem 13.1.11).

\item[(c)] $f = x^4+1$.

$\Delta(f) = 2^8$ is a square in $\Q$ and $\theta_f(y) = y(y-2)(y+2)$ splits completely over $\Q$, so $\Gal_\Q(f) \simeq \Z/2\Z \times \Z/2\Z$  (part (b) of Theorem 13.1.11).

\item[(d)] $f = x^4 + x^3+x^2+x+1$.

$\Delta(f) = 5^3$ is not a square, and $\theta_f(y) = (y-2)(y^2+y+1)$ has a unique root in $\Q$, so part (c) of Theorem 13.1.1 applies.
Let $\zeta$ a root of $f$. Then
$$f = (x-\zeta)(x-\zeta^2)(x-\zeta^3)(x-\zeta^4)$$
splits completely over $\Q(\zeta)$. By Exercise 13,
$$G \simeq \Z/4\Z.$$
(we know already this result, since $f = \Phi_4$.)

\item[(e)] $f = x^4-2$.

By Exercise 13, Example 2, $\Delta(f) = -2^{11}$ is not a square, and $\theta_f(y) = y(y^2+8)$ has a unique root in $\Q$. Moreover if $a = \sqrt[4]{2}$,
$$f = (x-a)(x+a)(x^2+a^2)$$ doesn't splits completely over $\Q$, so
$$G \simeq D_8.$$
\ee
\end{proof}

\paragraph{Ex. 13.1.15}

{\it In the situation of Theorem 13.1.1, assume that $\theta_f(y)$ has a root in $F$. In the proof of the theorem, we used (13.5) and (13.7) to show that $G$ is conjugate to a subgroup of $D_8$. Show that the weaker assertion that $|G| = 4$ or $8$ can be proved directly from (12.17).
}

\begin{proof}
By (12.17), the roots of the quartic $f =x^4 -c_1x^3 + c_2 x^2 - c_3x + c_4$ are
$$\alpha = \frac{1}{4}\left ( c_1 + \varepsilon_1 \sqrt{ 4y_1 + c_1^2 - 4 c_2} + \varepsilon_2 \sqrt{ 4y_2 + c_1^2 - 4 c_2} + \varepsilon_3 \sqrt{ 4y_3 + c_1^2 - 4 c_2}\right ),$$
where $y_1,y_2,y_3$ are the roots of the Ferrari resolvent 
$$\theta_f(y) = y^3 - c_2y^2+(c_1c_3-4c_4) y - c_3^2 -c_1^2c_4 + 4 c_2c_4,$$
and the $\varepsilon_i = \pm 1$ are chosen so that the product of the radicals $t_i = + \varepsilon_i \sqrt{ 4y_i + c_1^2 - 4 c_2}$ is
$$t_1t_2t_3 = c_1^3 - 4 c_1 c_2 + 8c_3.$$
Let $L = F(\alpha_1,\alpha_2,\alpha_3,\alpha_4)$ the splitting field of $F$.

Here $\theta_f(y)$ has a root in $F$, say $y_1$. Thus
$$\theta_f(y) =(y-y_1)g(y),$$
where $g(y) = y^2 + a y +b \in F[y].$
Therefore the roots $y_2,y_3$ of $g$ are in $F(\sqrt{\delta})$, where $\delta = a^2 - 4b \in F$ is the discriminant of $g$.
Moreover $t_1 = \alpha_1+\alpha_2-\alpha_3-\alpha_4  = \sqrt{ 4y_1 + c_1^2 - 4 c_2} \in L$, and similarly $t_2,t_3 \in L$, so $F(t_1,t_2,t_3) \subset L$, and by (12.17), $L \subset F(t_1,t_2,t_3)$, therefore
$$L = F(t_1,t_2,t_3) = F\left(\sqrt{ 4y_1 + c_1^2 - 4 c_2}, \sqrt{ 4y_2 + c_1^2 - 4 c_2},\sqrt{ 4y_3 + c_1^2 - 4 c_2}\right).$$
There is at most one $t_i$ equal to $0$. Indeed, if $t_1 = t_2 = 0$ (for instance), then $y_1 = y_2$ and $\theta_f$ and $f$ would not be separable, in contradiction with $\Delta(f) \ne 0$ in part (c) of Theorem 13.1.1. So we can choose the numbering such that $t_1t_2 \ne 0$ (perhaps $t_3 = 0$). Since $t_1t_2t_3 = c_1^3 - 4 c_1 c_2 + 8c_3 \in F$, $t_3 \in F(t_1,t_2)$, so
$$L = F(t_1,t_2,t_3) = F(t_1,t_2) =F\left(\sqrt{ 4y_1 + c_1^2 - 4 c_2}, \sqrt{ 4y_2 + c_1^2 - 4 c_2}\right).$$
Note $t_i^2 = 4y_i + C_1^2 - 4 c_2 \in L$, so $y_i \in L,\  i = 1,2,3$, so  $\sqrt{\delta} = y_2- y_3 \in L$, therefore $L(\sqrt{\delta}) = L$.
Consider the chain of inclusions
$$F \subset F(\sqrt{4y_1+c_1^2-4c_2}) \subset  F(\sqrt{4y_1+c_1^2-4c_2}, \sqrt{\delta}) \subset F(\sqrt{4y_1+c_1^2-4c_2}, \sqrt{\delta},\sqrt{4y_2+c_1^2-4c_2} ) =L.$$
 Since $4y_1+c_1^2-4c_2 \in F, \delta \in F$ and $4y_2+c_1^2-4c_2 \in F(\sqrt{\delta})$), the degree of each extension is 1 or 2, so 
$$[L:F] \mid 8.$$
Moreover $L \supset F(\alpha_1)$, and the minimal polynomial of $\alpha_1$ is 4, so $$[L:F] \geq [F(\alpha_1):F] = \deg(f) = 4.$$ Since $|G| = [L:F]$,
$$|G| = 4 \text{ or } |G| = 8.$$
\end{proof}

\paragraph{Ex. 13.1.16}

{\it Consider the subgroups $\langle (1\,2),(3\,4) \rangle$ and $\langle (1\,2)(3\,4),(1\,3)(2\,4) \rangle$ of $S_4$.
\be
\item[(a)] Prove that these subgroups are isomorphic but not conjugate. This shows that when classifying subgroups of a given group, it can happen that nonconjugate subgroups can be isomorphic as abstract groups.
\item[(b)] Explain why the subgroup $\langle (1\,2),(3\,4) \rangle$ isn't mentioned in Theorems 13.1.1 and 13.1.6.
\ee
}

\begin{proof}
\be
\item[(a)]
\begin{align*}
H_1 &= \langle (1\,2),(3\,4) \rangle = \{(), (1\,2),(3\,4), (1\,2) (3\,4)\}, \\
 H_2 &= \langle (1\,2)(3\,4),(1\,3)(2\,4) \rangle = \{ (), (1\,2)(3\,4),(1\,3)(2\,4), (1\,4)(2\,3)\}
 \end{align*}
  are both isomorphic to the Klein's group $\Z/2\Z \times \Z/2\Z$.
  
Every conjugate of $(1\,2)(3\,4)$ by $\sigma \in S_4$ is $(\sigma(1)\, \sigma(2)) (\sigma(3) \, \sigma(4))$, so is not in $H_1$. The groups $H_1,H_2$ are not conjugate. 
\item[(b)] $H_1 = \langle (1\,2),(3\,4) \rangle$ is not a transitive subgroup of $S_4$ (the orbit of 1 is $\{1,2\}$), so isn't mentioned in Theorems 13.1.1 and 13.1.6.
\ee
\end{proof}

\end{document}